\PassOptionsToPackage{unicode=true}{hyperref} % options for packages loaded elsewhere
\PassOptionsToPackage{hyphens}{url}
%
\documentclass[oneside,9pt,french,]{extbook} % cjns1989 - 27112019 - added the oneside option: so that the text jumps left & right when reading on a tablet/ereader
\usepackage{lmodern}
\usepackage{amssymb,amsmath}
\usepackage{ifxetex,ifluatex}
\usepackage{fixltx2e} % provides \textsubscript
\ifnum 0\ifxetex 1\fi\ifluatex 1\fi=0 % if pdftex
  \usepackage[T1]{fontenc}
  \usepackage[utf8]{inputenc}
  \usepackage{textcomp} % provides euro and other symbols
\else % if luatex or xelatex
  \usepackage{unicode-math}
  \defaultfontfeatures{Ligatures=TeX,Scale=MatchLowercase}
%   \setmainfont[]{EBGaramond-Regular}
    \setmainfont[Numbers={OldStyle,Proportional}]{EBGaramond-Regular}      % cjns1989 - 20191129 - old style numbers 
\fi
% use upquote if available, for straight quotes in verbatim environments
\IfFileExists{upquote.sty}{\usepackage{upquote}}{}
% use microtype if available
\IfFileExists{microtype.sty}{%
\usepackage[]{microtype}
\UseMicrotypeSet[protrusion]{basicmath} % disable protrusion for tt fonts
}{}
\usepackage{hyperref}
\hypersetup{
            pdftitle={SAINT-SIMON},
            pdfauthor={Mémoires II},
            pdfborder={0 0 0},
            breaklinks=true}
\urlstyle{same}  % don't use monospace font for urls
\usepackage[papersize={4.80 in, 6.40  in},left=.5 in,right=.5 in]{geometry}
\setlength{\emergencystretch}{3em}  % prevent overfull lines
\providecommand{\tightlist}{%
  \setlength{\itemsep}{0pt}\setlength{\parskip}{0pt}}
\setcounter{secnumdepth}{0}

% set default figure placement to htbp
\makeatletter
\def\fps@figure{htbp}
\makeatother

\usepackage{ragged2e}
\usepackage{epigraph}
\renewcommand{\textflush}{flushepinormal}

\usepackage{indentfirst}
\usepackage{relsize}

\usepackage{fancyhdr}
\pagestyle{fancy}
\fancyhf{}
\fancyhead[R]{\thepage}
\renewcommand{\headrulewidth}{0pt}
\usepackage{quoting}
\usepackage{ragged2e}

\newlength\mylen
\settowidth\mylen{...................}

\usepackage{stackengine}
\usepackage{graphicx}
\def\asterism{\par\vspace{1em}{\centering\scalebox{.9}{%
  \stackon[-0.6pt]{\bfseries*~*}{\bfseries*}}\par}\vspace{.8em}\par}

\usepackage{titlesec}
\titleformat{\chapter}[display]
  {\normalfont\bfseries\filcenter}{}{0pt}{\Large}
\titleformat{\section}[display]
  {\normalfont\bfseries\filcenter}{}{0pt}{\Large}
\titleformat{\subsection}[display]
  {\normalfont\bfseries\filcenter}{}{0pt}{\Large}

\setcounter{secnumdepth}{1}
\ifnum 0\ifxetex 1\fi\ifluatex 1\fi=0 % if pdftex
  \usepackage[shorthands=off,main=french]{babel}
\else
  % load polyglossia as late as possible as it *could* call bidi if RTL lang (e.g. Hebrew or Arabic)
%   \usepackage{polyglossia}
%   \setmainlanguage[]{french}
%   \usepackage[french]{babel} % cjns1989 - 1.43 version of polyglossia on this system does not allow disabling the autospacing feature
\fi

\title{SAINT-SIMON}
\author{Mémoires II}
\date{}

\begin{document}
\maketitle

\hypertarget{chapitre-premier.}{%
\chapter{CHAPITRE PREMIER.}\label{chapitre-premier.}}

1697

~

{\textsc{Mort étrange de Charles XI, roi de Suède.}} {\textsc{- Sa
tyrannie.}} {\textsc{- Son palais brûlé.}} {\textsc{- Princes Sobieski
s'en retournent sans recevoir le collier du Saint-Esprit.}} {\textsc{-
Conduite désapprouvée de l'abbé de Polignac en Pologne.}} {\textsc{-
Abbé de Châteauneuf y va la rectifier.}} {\textsc{- Froideur et plus du
prince de Conti pour la Pologne.}} {\textsc{- Plénipotentiaires à Delft
et à la Haye.}} {\textsc{- Distribution des armées.}} {\textsc{- M. de
Chartres, prince du sang, et M. du Maine ne servent plus.}} {\textsc{-
Ath pris par le maréchal Catinat.}} {\textsc{- Siège et prise de
Barcelone par le duc de Vendôme, qui est fait vice-roi de Catalogue.}}
{\textsc{- J'arrive à l'armée du maréchal de Choiseul, qui passe le
Rhin.}} {\textsc{- Belle retraite du maréchal de Choiseul.}} {\textsc{-
Inondations générales.}} {\textsc{- Beau projet du maréchal de Choiseul
avorté par ordre de la cour, qui fait repasser le Rhin à l'armée}}

~

Charles XI, roi de Suède, mourut à quarante-deux ans, le 15 avril de
cette année, à Stockholm. Il était de la maison palatine, et son père,
le célèbre Charles Gustave, en faveur duquel la reine Christine fut
obligée d'abdiquer, était fils de Catherine, sœur de ce grand Gustave
Adolphe, le conquérant de l'Allemagne, tous deux enfants de ce duc de
Sudermanie qui usurpa la Suède sur Sigismond, roi de Pologne, fils de
son frère Jean III, roi de Suède. Charles XI succéda à son père en 1660,
n'ayant que cinq ans, sous la tutelle d'Éléonore d'Holstein sa mère, et
avant qu'il eût vingt-cinq ans, il gagna plusieurs batailles en
personne, et d'autres grands avantages sur les Danois.

Il en sut profiter dès 1680 contre son pays\,: il s'affranchit de tout
ce qui bridait l'autorité royale, parvint au pouvoir arbitraire, et,
incontinent après qu'il l'eut affermi, le tourna en tyrannie. Il abolit
les états généraux et anéantit le sénat desquels il tenait toute son
autorité nouvelle, et s'appliqua avec trop de succès à la destruction
radicale de toute l'ancienne et grande noblesse, à laquelle il substitua
des gens de rien. Il ruina tous les seigneurs et les maisons même qui,
sous les deux célèbres Gustave, son père et celui de Christine, avaient
le plus grandement servi sa couronne de leurs conseils et de leurs bras,
et qui, dans le penchant de la Suède après la mort du grand Gustave
Adolphe, l'avaient le plus fortement soutenue, et s'étaient acquis le
plus de réputation en Europe. Il établit une chambre de révisions qui
fit rapporter non seulement toutes les gratifications et les grâces
reçues depuis l'avènement du grand Gustave Adolphe à la couronne, mais
les intérêts qu'elle en estima et tous les fruits, et qui confisqua tous
les biens sans miséricorde. Les plus grands et les plus riches tombèrent
dans la dernière misère\,; grand nombre emporta ce qu'il put dans les
pays étrangers, et tout ce qu'il y avait en Suède de noble et de
considérable demeura écrasé.

Le genre obscur et cruel de la longue maladie dont il mourut a fait
douter entre la main de Dieu vengeresse et le poison. Jusqu'après sa
mort, son corps ne fut pas à couvert de la punition en ce monde\,; le
feu prit au palais où il était encore exposé en parade. Ce fut avec
grande peine qu'on le sauva des flammes qui consumèrent tout le palais
de Stockholm. Il mourut avec l'honneur d'avoir été accepté pour
médiateur de la paix qui se traitait. Ce fut en sa faveur que le roi
tint si ferme en celle de Nimègue en 1679, pour lui faire restituer les
provinces qu'il avait perdues. Enfin c'est le père de Charles XII qui
depuis a fait tant de bruit en Europe et achevé de ruiner la Suède. La
mère de ce dernier était fille de Frédéric III, roi de Danemark, morte
dès 1693\,; et la reine sa grand'mère fut encore une fois régente.

Les princes Alexandre et Constantin Sobieski se lassèrent d'un incognito
qui ne leur donnait rien ici, et qui marquait seulement qu'ils n'y
pouvaient obtenir les distinctions dont ils s'étaient flattés. Cette
raison les fit renoncer à recevoir ici l'ordre du Saint-Esprit. On y
était fort mécontent de la reine leur mère. Ils prirent le parti de s'en
aller et de dire qu'ils voulaient arriver en Pologne avant l'élection\,:
ils prirent ainsi congé du roi, et s'en allèrent vers la mi-avril.

Les nouvelles de ce pays commençaient à n'être plus si favorables. On
apprit avec étonnement que l'abbé de Polignac s'était beaucoup trop
avancé et, entre autres promesses, s'était engagé d'accorder que le
prince de Conti prendrait à ses dépens Caminieck occupé par les Turcs,
et qu'il ferait cette conquête avant son couronnement, sans quoi son
élection demeurerait nulle. Un particulier, quelque grand et riche et
appuyé qu'il fût, ne pouvait pas se flatter de suffire à cette dépense,
et de faire dépendre la validité de l'élection du succès de cette
entreprise. C'était exposer la fortune d'un prince du sang, non
seulement à l'incertitude des hasards d'un grand siège, mais à toutes
les trahisons de ceux qui se trouveraient intéressés à le faire échouer
par leur engagement contre l'élection de ce prince. On en fut si choqué
à la cour, qu'on envoya Ferval en Pologne pour voir plus clair à ces
avances de l'abbé de Polignac, essayer de raccommoder ce qu'il avait
gâté, et donner des nouvelles plus nettes et plus désintéressées de
toute cette négociation. Peu après arriva un gentilhomme de la part du
cardinal Radziewski, archevêque de Gnesne, qui était à la tète du parti
du prince de Conti, et qui, comme primat de Pologne, était à la tête de
la république pendant l'interrègne. Le compte qu'il rendit, et la
commission dont il était chargé pour le roi et pour ce prince, donnèrent
beaucoup d'espérances, mais peu d'opinion de la conduite de l'abbé de
Polignac, qui, parfaitement bien avec la reine de Pologne, s'était
brouillé avec elle jusqu'aux éclats et à l'indécence, tellement qu'il
fut jugé à propos d'envoyer l'abbé de Châteauneuf lui servir
d'évangéliste, et qui porta à l'abbé de Polignac des ordres très précis
de ne rien faire que de concert avec lui. Il était frère de notre
ambassadeur à Constantinople. C'étaient deux Savoyards, tous deux gens
de beaucoup d'esprit et de belles-lettres, et tous deux fort capables
d'affaires, l'aîné avec plus de manège, l'autre avec encore plus de fond
et de sens\,; et on prit le parti d'attendre qu'il se fût bien mis au
fait de tout en Pologne, et d'en être informé par lui, avant que de
s'embarquer plus avant.

M. le prince de Conti était fort éloigné de désirer le succès d'une
élévation à laquelle il n'avait jamais pensé. Il allait jusqu'à le
craindre. Il était prince du sang, et quoique malvoulu du roi, il
jouissait de l'estime et de l'affection publique\,; il profitait encore
de la compassion de sa situation délaissée et de son espèce de disgrâce,
du parallèle qu'on faisait entre lui si nu et M. du Maine si comblé, de
la préférence sur lui de M. de Vendôme pour le commandement de l'armée,
et de l'indignation qui en naissait. Élevé avec Monseigneur, extrêmement
bien avec lui et dans toute sa privance, il comptait sur le
dédommagement le plus flatteur et le plus durable sous son règne\,;
enfin il était passionnément amoureux de M\textsuperscript{me} la
Duchesse\,: elle était charmante, et son esprit autant que sa figure.
Quoique M. le Duc fût fort étrange, et étrangement jaloux, M. le prince
de Conti ne laissait pas d'être parfaitement heureux. Par ce recoin
secret, il tenait de plus en plus à Monseigneur, qui commençait fort à
s'amuser de M\textsuperscript{me} la Duchesse, laquelle avait su lier
sourdement avec M\textsuperscript{lle} Choin. C'en était trop pour que
le brillant d'une couronne pût prévaloir sur let horreurs de s'expatrier
pour jamais\,; aussi parut-il extrêmement froid dans toute cette
affaire, très attentif à en faire peser toutes les difficultés, et si
lent à la suivre, qu'on s'aperçut aisément de toute sa répugnance.

Après quelques difficultés et quelques délais sur les passeports des
plénipotentiaires du roi pour la paix, ils arrivèrent, et incontinent
après Harlay et Crécy qui étaient à Paris partirent, et ils se
brouillèrent dès Lille. Le Normand, fermier général en ce département, y
était, qui fournit de bons chevaux à Crécy, son ami, et ne donna que des
colliers et des charrettes à l'autre qui, au lieu de ne s'en prendre
qu'à la sottise du fermier général, s'emporta contre son collègue. Il
écrivit à la cour des plaintes amères. Le Normand fut blâmé, Harlay
encore plus, qui, sur les réponses sèches qu'il reçut, se hâta de se
raccommoder avec Crécy. À Courtrai ils apprirent que les
plénipotentiaires des alliés avaient le caractère d'ambassadeurs, et
qu'ils se préparaient à leur faire beaucoup de chicanes sur le
cérémonial, parce qu'ils ne l'avaient point. Ils dépêchèrent donc un
courrier là-dessus qu'ils attendirent à Courtrai, et qui leur apporta le
caractère d'ambassadeurs\,; c'est ce qui fut cause qu'ils ne reçurent
que des civilités, mais aucuns honneurs sur toute la frontière
française, et que celle des ennemis leur en rendit de forts grands,
ainsi que le dedans de leur pays. Ils arrivèrent à Delft, où ils
trouvèrent Callières. Ceux des alliés et de Suède étaient à la Haye, à
quatre lieues d'eux\,; et à demi-lieue de Delft, le château de Ryswick
au prince d'Orange où ils devaient tous se trouver pour traiter. On
l'avait ouvert par divers côtés, afin que chacun pût entrer et sortir
par le sien, et s'asseoir vis-à-vis de son entrée autour d'une table
ronde pour éviter toute dispute de rang et de compétence. Force jeunes
gens de robe et de Paris étaient allés à la suite des nôtres. Harlay y
avait mené son fils qui avait beaucoup d'esprit et encore plus de
débauche et de folie, et qui fit là toutes les extravagances les plus
outrées et les plus continuelles, et dont plusieurs pouvaient avoir des
suites fâcheuses et embarrassantes même, sans que le père part t y
donner la plus légère attention.

La disposition des armées fut la même que l'année précédente, mais les
princes ne servirent point. Le roi en était convenu avec Monsieur pour
M. le duc de Chartres, et avec M. le Prince pour M. le Duc et M. le
prince de Conti, qui se chargea de le leur dire. Le roi à la fin prit ce
parti par le contraste de M. de Vendôme qui commandait une armée, et par
ce qui s'était passé en Flandre de M. du Maine qui, ayant fait encore
une campagne depuis, en fut dispensé pour toujours. M. le comte de
Toulouse, qui n'avait pas en soi la même raison, commanda la cavalerie
dans l'armée du maréchal de Boufflers, et chacun partit pour les
frontières. Le maréchal Catinat qui n'avait plus d'occupation en Italie
eut une armée en Flandre, avec laquelle il ouvrit la campagne par le
siège d'Ath qui était mal pourvu, et se défendit mollement\,; la place
se rendit le 7 juin, et le chevalier de Tessé en eut le gouvernement.

M. de Vendôme était parti pour la Catalogne avec l'ordre exprès de faire
le siège de Barcelone\,; le comte d'Estrées, vice-amiral en survivance
de son père, y amena la flotte au commencement de juin avec les galères
que commandait sous lui le bailli de Noailles, leur lieutenant général,
et avec ces forces navales ferma le port. Pimentel, qui avait défendu
Charleroi, et qui l'avait rendu en 1693 au maréchal de Villeroy,
commandait dans Barcelone. Le marquis de La Corzana, mestre de camp
général de la Catalogne, s'y était jeté, et le prince de Hesse-Darmstadt
commandait au Montjoui qui en est comme la citadelle, quoiqu'un peu
séparé de la ville. Ils avaient huit mille hommes d'infanterie de
troupes réglées, quelque cavalerie et le reste \emph{somettants}, qui
sont des milices fort aguerries, et le tout ensemble faisait vingt-cinq
mille hommes. Nous avions soixante pièces de batterie et vingt-huit
mortiers. Dehors étaient don François de Velasco, vice-roi de Catalogne
et le marquis de Grigny, général de la cavalerie avec une petite armée
et force miquelets. La place était plus qu'abondamment fournie de tout
et conserva une libre communication par un côté avec le vice-roi pour
pouvoir être rafraîchie.

M. de Vendôme n'avait point assez de troupes pour l'investir
entièrement, ni pour avoir assez de postes de proche en proche dans ses
derrières pour contenir les miquelets\,; tellement qu'il ne put tirer
ses subsistances que par le secours de la mer. Les troupes de l'armée
navale mirent pied à terre et servirent au siège, les chefs d'escadre
comme maréchaux de camp et le bailli de Noailles comme lieutenant
général. Le comte d'Estrées demeura sur la flotte. Outre ces
difficultés, les chaleurs étaient excessives. Il y eut beaucoup
d'actions très vives et très belles\,; le prince de Birkenfeld, à qui
son père avait donné le régiment d'infanterie d'Alsace, à la tête duquel
il était devenu lieutenant général, s'y distingua extrêmement, et
tellement de l'aveu de tout le monde, que le roi ne voulut pas attendre
la fin du siège à le faire brigadier, et récompenser le temps qu'il
avait perdu capitaine de cavalerie. Le duc de Lesdiguières y fit ses
premières armes d'une manière fort brillante. Les comtes de Mailly et de
Montendre, et le fils aîné du grand prévôt s'y signalèrent fort aussi.

La contrescarpe emportée, M. de Vendôme eut axis que la nuit du 15 au 16
juillet les assiégés devaient faire une grande sortie, et en même temps
le vice-roi avec toutes ses troupes attaquer le camp. Là-dessus M. de
Vendôme marcha au vice-roi, la nuit du 14 au 15, dont il trouva l'armée
partagée en deux camps\,; il en attaqua un, et fit attaquer l'autre par
d'Usson. Aucun des deux ne résista presque\,; ils furent surpris, et
tout prit la fuite, et le vice-roi même tout en chemise. Les deux camps
furent pillés, et pendant ce pillage quelque cavalerie ennemie prit le
temps de se former et de venir tomber sur les pillards, mais on avait
prévu cet inconvénient, et cette cavalerie fut défaite\,; on leur tua ou
prit huit cents hommes et beaucoup d'officiers. Le secrétaire et la
cassette du vice-roi furent pris avec ses papiers, et cinq mille pièces
de quatre pistoles. Par cette action l'armée ennemie fut entièrement
dissipée et hors d'état de rafraîchir la place ni de montrer de troupes
nulle part. On ne songea plus qu'à presser le siège. Il y eut encore
beaucoup d'actions fort vives. Enfin les mines ayant fait tout l'effet
qu'on en avait espéré, et l'assaut prêt à donner, M. de Vendôme envoya
Barbezières leur parler. Pimentel s'approcha de lui. Il y eut des
propositions sur l'état où la place se trouvait réduite qui produisirent
quelques allées et venues. Enfin ils entrèrent le 5 août en
capitulation, qui ne fut conclue que le 8. Elle fut telle que le
méritaient de si braves gens qui, par leur belle défense, s'étaient
montrés vrais Espagnols et dignes de l'être. On leur accorda trente
pièces de canon, quatre mortiers, des chariots couverts tant qu'ils
voulurent, et la plus honorable composition, et à la ville tous ses
privilèges, excepté l'inquisition que M. de Vendôme ne voulut pas
souffrir. Ils s'étoient fait un point d'honneur de ne battre point la
chamade. Il périt beaucoup de monde de part et d'autre à ce siège, mais
personne de marque. Le vice-roi, don François de Velasco, fut mandé à
Madrid pour rendre compte de sa conduite, et La Corzana fut fait
vice-roi. Le Montjoui se rendit par la même capitulation de la place,
sans avoir été attaqué.

Chemerault arriva le 15 août à Versailles, où Barbezieux ne se trouva
point, avec cette agréable nouvelle. Saint-Pouange le mena au roi, et
Laparat, qui, comme principal ingénieur, avait conduit le siège où il
avait été légèrement blessé, vint après rendre compte du détail de ce
qui s'y était passé. Lui et Chemerault étaient brigadiers. Le roi donna
douze mille livres à Chemerault et les fit tous deux maréchaux de camp,
et avec eux M. de Liancourt qui servait en Flandre, et ne s'attendait à
rien moins. Ce fut une galanterie que le roi fit à M. de La
Rochefoucauld. Il y eut suspension d'armes en Catalogne jusqu'au 1er
septembre. Le Llobregat servit de barrière pour la séparation des
Français et des Espagnols. Nous eûmes bien neuf mille hommes tués ou
blessés, parmi lesquels six cents officiers\,; les ennemis y perdirent
six mille hommes. Coigny, lieutenant général, et Nanelas sous lui,
furent mis pour commander dans Barcelone. Pimentel, qui l'avait
défendue, eut du roi d'Espagne un titre de Castille, et prit le nom de
marquis de La Floride. M. de Vendôme, quelques jours après, y fut reçu
vice-roi en grande cérémonie. Le présent en pareille occasion est de
cinquante mille écus.

J'arrivai à Landau sur la fin de mai, deux jours avant l'assemblée de
l'armée. Ce fut à Lempsheim où le marquis de Chamilly demeura avec une
partie de l'infanterie\,; le marquis d'Huxelles alla avec l'autre à
Spire, et le maréchal de Choiseul, avec une brigade d'infanterie et
toute la cavalerie, s'avança à Eppenheim pour la commodité des
fourrages, où on fit la réjouissance de la prise d'Ath.

Pendant qu'on subsistait ainsi tranquillement, tantôt dans un camp,
tantôt dans un autre, suivant l'abondance, le maréchal n'était pas sans
inquiétude que le prince Louis de Bade n'en voulût à Fribourg. Ce
soupçon et le remuement de leurs bateaux qui l'empêcha de s'avancer
davantage dans le Palatinat, quoique fort court de fourrages, le fit
songer à passer le Rhin. Il le proposa à la cour, et il en reçut la
permission, en même temps que Locmaria le joignit avec neuf escadrons et
dix bataillons du Luxembourg. Le maréchal tint son dessein secret,
partit dans sa chaise, suivi du duc de La Ferté, du comte du Bourg, de
Mélac et de Praslin, d'une brigade de cavalerie et d'une de dragons, et
s'en alla au fort Louis, où il arriva le dernier juin. Il y fut joint
par la cavalerie la plus à portée, puis par toute son infanterie, le
marquis d'Huxelles resté avec presque rien à Spire. Cependant il se hâta
d'occuper les bois et les défilés de Stolhofen pour pouvoir déboucher
par là. Le marquis de Renti, avec toute la cavalerie, arriva le 3
juillet au fort Louis, où il passa le Rhin, et le même jour l'artillerie
et les vivres joignirent aussi le maréchal de Choiseul assez près de la
tête des chaussées. De toutes parts l'ordre et l'extrême diligence de
l'exécution furent admirables. J'allai en arrivant voir le maréchal, qui
ne m'en avait dit qu'un mot léger à Ostoven, et qui m'en fit excuse sur
ce qu'il n'avait confié son projet qu'à ceux-là uniquement dont il ne se
pouvait passer pour l'exécution, dans la crainte que le prince de Bade
ne portât quelques troupes dans la plaine de Stolhofen\,; ce qui lui
était bien aisé, et ce qui aurait empêché le passage du Rhin.

Pour tromper mieux M. de Bade, le marquis d'Huxelles, qui n'avait à
Spire que les troupes que Locmaria avait amenées, et qu'il y avait
attendues absolument seul pendant vingt-quatre heures, fit passer le
Rhin sur le pont de Philippsbourg à quelques troupes, et à force
trompettes, cymbales et tambours, et persuada ainsi à l'ennemi que toute
l'armée était là, ce qui le retint à trois lieues à Bruchsall où il
était campé. Le maréchal, cependant, alla le 4 mettre son centre et son
quartier général à Niederbühl, sa droite à Cupenheim, et sa gauche à
Rastadt, la rivière de Murg coulant le long de la tête de son camp. Les
bords de son côté en étaient hauts, et de l'autre ils étaient bas. On en
rompit tous les gués, on retrancha bien la droite, on fit des redoutes,
et de ces hauteurs on voyait toute la plaine au delà de la Murg. On
accommoda bien Rastadt, et on prit toutes les précautions nécessaires
pour bien assurer le camp.

Le prince de Bade, enfin détrompé, vint le 7 se mettre à Muckensturm, à
demi-lieue de notre quartier général\,; de là, à la Murg. Au delà d'elle
il y avait une assez grande plaine, toute remplie de fourrages. On
aurait pu l'enlever le 6\,; mais on aima mieux laisser reposer l'armée,
et le 7 l'arrivée du prince de Bade empêcha d'y plus penser\,; mais le 8
la débandade fut générale, quelque chose qu'on pût faire\,; tout courut
fourrager cette plaine jusqu'entre les vedettes des ennemis, et à
l'entière merci de leurs gardes et de leur camp. Ces débandés furent
plus heureux que sages\,; leur extrême témérité fut leur salut. Les
ennemis n'imaginèrent jamais que ce fût désobéissance et extravagance\,:
ils la prirent pour un piège qu'on leur tendait\,; jamais pas un d'eux
ne branla. Tout le fourrage revint en abondance\,; il n'y eut pas un
cheval de perdu, ni un homme à dire ni blessé. Je ne pense pas que
jamais folie ait été en même temps et si générale et si heureuse.

Après qu'on se fut bien accommodé dans ce camp, il se trouva que les
convois qu'on tirait du fort Louis étaient incommodes et périlleux. On
jeta donc à trois lieues du quartier général un pont de bateaux sur le
Rhin, à l'endroit d'une île qui était séparée de notre bord par un bras
étroit. Le chemin du pont au camp était couvert d'un marais\,; mais ce
marais, cru impraticable, se le trouva si peu que nos convois suivirent
toujours leur premier chemin, et que ce pont ne fut qu'une inquiétude de
plus, que les ennemis ne vinssent le brûler de notre bord à l'île, ce
qui eu fit ôter les trois premiers bateaux toutes les nuits. Il servit
seulement à l'abondance du camp par le commerce avec les paysans
d'Alsace\,; et La Bretesche, lieutenant général, fut chargé de tout ce
côté-là. Chamilly fit un grand fourrage du côté de la montagne. Au
retour il trouva force hussards soutenus par Vaubonne avec des troupes.
Il y eut une petite action\,; Vaubonne fut chassé l'épée dans les reins
jusqu'à un petit ruisseau, qui, avec les approches de la nuit, le
délivra de la poursuite. Praslin s'y distingua fort\,; il y eut assez de
gens des ennemis tués, et fort peu des nôtres.

M. le maréchal de Choiseul demeura seize jours dans ce camp\,; les
fourrages vinrent à manquer tout à fait, il fallut songer à en sortir.
On défit le pont de bateaux, et tout aussitôt le bruit se répandit qu'on
allait décamper. Pour l'apaiser, Saint-Frémont fut détaché le 18 avec
presque tous les caissons de l'armée, sous prétexte d'aller quérir un
grand convoi au fort Louis. En effet il revint le même jour avec
beaucoup de ces mêmes caissons. Cela trompa et fit croire qu'on
séjournerait encore quelque temps. Le maréchal m'avait confié son
dessein. Notre camp était disposé de manière que les ennemis le voyaient
en entier, excepté quelques endroits interrompus par des avances de
haies et de bois, et les deux brigades de cavalerie qui fermaient la
gauche de la seconde ligne de dix-neuf escadrons, Horn et Ligondez dont
j'étais, et deux régiments de dragons qui couvraient ce flanc\,; mais la
gauche entière de la première ligne, qui était devant nous, était vue en
plein. Le 19 juillet, sur les onze heures du matin, toute l'armée eut
ordre de charger les gros bagages, une heure après les menus, avec
défense de détendre et de rien remuer\,: à deux heures après midi, nos
deux brigades et les dragons nos voisins, que les ennemis ne voyaient
pas, comme je viens de l'expliquer, reçurent ordre de détendre et de
marcher sur-le-champ sans bruit. La Bretesche, lieutenant général, et
Montgomery, maréchal de camp, officiers généraux de la seconde ligne de
cette aile, vinrent la prendre et la menèrent au delà des bois par
lesquels nous étions arrivés, passer la nuit dans la plaine de
Stolhofen, et cependant les gros et menus bagages, l'artillerie inutile
et tous les caissons filèrent entre le Rhin et nous. La Bretesche avait
défenses expresses de branler, quelque combat qu'il entendit. La raison
en était qu'il restait assez de troupes pour combattre dans un lieu
aussi étroit qu'était celui d'où on se retirait\,; qu'il fallait une
grosse escorte pour tous les bagages de l'armée\,; et qu'en cas de
malheur, nos troupes se seraient trouvées toutes franches et en bon
ordre dans la plaine, pour recevoir et soutenir tout ce qui déboucherait
les bois venant de notre camp.

Sur les six heures du soir, le maréchal monta sur cette hauteur
retranchée de sa droite à laquelle il avait fait travailler exprès tout
le jour, et disposa toute son affaire avec tant de justesse, qu'avec le
signal d'un bâton levé en l'air avec du blanc au bout de distance en
distance, ce ne fut qu'une même chose que détendre, charger, monter à
cheval, marcher, et quoique au petit pas, perdre les ennemis de vue.
Comme il ne restait nulle sorte d'équipage au camp, et que tout était
sellé et bridé, cette grande armée disparut en un moment, en plein jour,
aux yeux des ennemis. L'armée marcha sur deux colonnes. Le régiment
colonel général de cavalerie fit l'arrière-garde de la gauche avec du
canon, et le prince de Talmont ensuite avec les gardes ordinaires\,;
enfin, un détachement de cavalerie, sous un lieutenant-colonel qui était
commandé tous les jours à Rastadt. Le bonhomme Lafréselière, lieutenant
général, conduisait cette arrière-garde. Le maréchal fit celle de la
droite avec la gendarmerie et quelques détachements derrière elle, et
Chamaraude fit avec tous les grenadiers de l'armée l'arrière-garde de
tout. Montgon, qui par son poste devait être avec nous, obtint du
maréchal de demeurer auprès de lui. Avant la nuit noire, presque toute
l'armée avait débouché tous les bois et était entrée dans la plaine de
Stolhofen. Ceux des généraux impériaux qui se trouvèrent à la promenade
accoururent de toutes parts sur les bords de la Murg pour voir ce
décampement, mais il fut si prompt qu'il ne leur donna pas loisir de
faire la moindre contenance d'inquiéter cette retraite, l'une des plus
belles qu'on ait vues.

Somières, capitaine de cavalerie au régiment de La Feuillade, avait été
pris à ce fourrage du marquis de Chamilly dont j'ai parlé, et fut
renvoyé quelques jours après cette retraite. Il rapporta au maréchal de
Choiseul, en ma présence, que les Impériaux, fondés sur ce convoi de
Saint-Frémont, ne crurent point que notre armée marchât de quelques
jours\,; que le 19 juillet, jour de cette belle retraite, le prince
Louis de Bade rentrait de la promenade avec le duc de Lorraine, et
venait de mettre pied à terre, lorsqu'on le vint avertir à toutes jambes
que nous décampions\,; qu'il répondit que cela n'était pas possible,
fondé sur ce que lui-même venait de voir un instant auparavant
travailler encore sur cette hauteur de notre droite\,; qu'en même temps
il lui vint un second avis semblable qui le fit aussitôt remonter à
cheval et courir aux bords de la Murg, où ce capitaine prisonnier le
suivit. Ils ne virent que l'arrière-garde se dérober à leurs yeux, ce
qui remplit tellement le prince de Bade d'étonnement et d'admiration
qu'il demanda à ce qui l'accompagnait s'ils avaient jamais rien vu de
pareil, et il ajouta que pour lui il n'avait pas cru jusqu'alors qu'une
armée si considérable et si nombreuse pût disparaître ainsi en un
instant. Cette retraite en effet fut honorable et hardie, et en même
temps sûre. Elle se fit en plein jour, mais si promptement que les
ennemis n'en purent tirer aucun avantage\,; et, quoique en plein jour,
si proche de la nuit, que l'obscurité la favorisa presque autant que si
on l'eût faite dans les ténèbres. Elle fut fière, belle, bien entendue,
savante, et digne enfin d'un général qui avait si bien appris sous les
plus grands maîtres.

Sa gloire en cette occasion eût été sans regret, sans un accident qui
arriva. Blansac menait une colonne d'infanterie et fut surpris de la
nuit dans les bois. Un petit parti qu'il avait sur son aile entendit
quelque cavalerie marcher fort près de soi. Ce peu de cavalerie était
des Impériaux égarés, qui, reconnaissant le péril où ils se trouvaient,
au lieu de répondre au qui-vive, se dirent entre eux, en allemand\,:
«\,Sauvons-nous.\,» Il n'en fallut pas davantage pour leur attirer une
décharge du petit parti français, à laquelle ils répondirent à coups de
pistolet. Ce bruit fit faire, sans le commandement de personne, une
décharge de ce côté-là à toute la colonne d'infanterie, et Blansac,
voulant s'avancer pour savoir ce que ce pouvait être, en essuya une
seconde. II eut le bonheur de n'en être point blessé, mais cinq pauvres
capitaines furent tués et quelques subalternes blessés.

L'armée ne fut pourtant point troublée par cette escopetterie, et passa
la nuit auprès de nos deux brigades dans la plaine de Stolhofen, comme
chacun se trouva. Le lendemain 20, dès le matin, elle en passa le
défilé, et rampa, la droite et le quartier général à Lichtenau, la
gauche peu éloignée de Stolhofen, l'abbaye de Schwartzals vers le
centre, un gros ruisseau à la tête du camp, et le Rhin à trois quarts de
lieue derrière nous. Nous y demeurâmes dix ou douze jours pour voir ce
que deviendrait le prince Louis de Bade, qui demeura dans son même camp
de Muckensturm. De là nous primes celui de Lings, puis celui de
Wilstedt, si proche du fort de Kehl qu'il rendit notre pont de bateaux
inutile, et que par le pont de Strasbourg la communication était libre,
sans escorte, et continuelle entre l'armée et cette place.

Le comte du Bourg fut chargé là d'un grand fourrage\,; ce qui, joint à
quelques autres bagatelles, brouilla le marquis de Renti avec le
maréchal de Choiseul, son beau-frère. Renti était un très galant homme,
vaillant et homme de bien, mais avec cela épineux à l'excès. Le maréchal
s'était fait une autre affaire avec Revel. C'était à lui à être de jour
celui de la retraite et par conséquent à faire celle des deux
arrière-gardes où le maréchal n'était pas. Le bonhomme Lafréselière, que
toute l'armée aimait et honorait, et qui le méritait, était lieutenant
général aussi, et son retour tombait immédiatement avant celui de Revel.
Il était aussi lieutenant général de l'artillerie, il la commandait, et
par là il ne pouvait prendre jour de lieutenant général dans l'armée, ni
marcher à son tour qu'une fois dans la campagne. Il voulut prendre sa
bisque d'être de jour à la retraite\,; le maréchal, qui l'aimait et qui
comptait sur sa capacité, décida en sa faveur, et Revel fut outré. Du
camp de Wilstedt nous allâmes prendre celui d'Offenbourg.

Cet été ne fut que pluies universelles, et débordements partout qui
interrompaient le commerce. Paris et ses environs furent inondés, à ce
qu'on nous mandait, et ce que nous éprouvions ne nous donnait pas de
peine à le croire. Cela durait depuis deux mois, et notre général en
gémissait dans l'impatience d'exécuter un projet qu'il avait fait
approuver à la cour\,: c'était d'aller attaquer des retranchements faits
dès le commencement de la guerre pour garder les gorges qui sont à
l'entrée de la Franconie, de la Souabe et de la Bavière. Schwartz et un
comte de Furstemberg gardaient ces lignes, qui, par leur étendue,
étaient difficiles à conserver. Le maréchal de Choiseul mourait d'envie
de s'y faire un passage dans des pays abondants et qui depuis bien des
années n'avaient point souffert des guerres, d'essayer à y prendre des
quartiers d'hiver et de baller un pont d'une structure particulière qui
se jetait en peu d'heures sur le Rhin, et qui, étant toujours là tout
prêt, inquiétait toutes les campagnes sitôt que notre armée descendait
le Rhin.

Le marquis d'Huxelles était resté à Spire, d'où il n'avait bougé avec
Locmaria et les troupes que ce dernier avait amenées, depuis que l'armée
avait passé le Rhin, et ce passage n'avait point été de son goût.
Pendant que le maréchal méditait l'exécution de son projet, et que le
temps commençait à lui faire espérer la possibilité de l'entreprendre,
Huxelles fut averti par des paysans que les ennemis faisaient un pont à
Guermersheim. C'est l'homme du monde qui aimait le moins les
entreprises, et qui craignait le plus de se commettre. Sans approfondir
davantage, il se retira à Guermersheim, entre Philippsbourg et Landau,
en donna avis au maréchal de Choiseul, et, ne se croyant pas encore là
en sûreté, s'alla mettre à Lauterbourg. Ce ne fut pas tout. Il dépêcha
un courrier à Barbezieux à qui il communiqua toute sa peur pour
l'Alsace. Le maréchal reçut l'avis du marquis d'Huxelles avec dépit,
parce qu'il jugea la terreur panique, ou que le passage pouvait être
empêché par ce que d'Huxelles avait de troupes et qu'il venait d'être
fortifié par un corps que Mélac lui avait amené\,; mais sa colère fut
extrême lorsque toute sa disposition faite pour marcher aux
retranchements le lendemain, et jusqu'à la munition distribuée aux
troupes, il lui arriva sur le midi un courrier de Barbezieux, avec un
ordre positif du roi de repasser le Rhin sur-le-champ, toutes choses,
toutes raisons et toutes représentations cessantes, et sans délai d'un
moment. Le maréchal qui m'avait confié son projet me fit les plaintes
les plus amères, à moi et aux généraux qui étaient du secret. Il ne
douta pas que cet ordre ne lui eût été attiré par le marquis d'Huxelles,
sur lequel, tout sage et tout mesuré qu'il était, il s'échappa entre
Lafréselière, du Bourg, Praslin et moi. Il se voyait arracher sa gloire
et une exécution dont l'importance influait si fort sur la paix qui se
traitait, ou si elle ne se concluait pas, sur toute la suite de la
guerre\,; mais il fallut obéir, et sans que le prince Louis de Bade eût
songé à passer le Rhin, il nous fallut le repasser le lendemain sur le
pont de Strasbourg à travers des eaux et des fanges inconcevables.

Je passai un jour entier dans la ville avec cinq ou six de mes amis, à
nous reposer dans la maison de M. Rosen, qu'il me prêtait toutes les
campagnes. Il y eut quelques petites escarmouches à l'arrière-garde, que
Villars, qui n'était chargé de rien, fit tout ce qu'il put pour tourner
en combat où il n'avait rien à perdre, et pouvait gagner de l'honneur,
parce que rien ne roulait sur lui, et il fut enragé d'en être empêché
par La Bretesche qui était de jour, et par Bartillac, lieutenant général
de l'aile, qui avaient les plus expresses défenses du maréchal de
laisser rien engager. L'armée campa sous Strasbourg, sans entrer dans la
ville, puis traversa l'Alsace par lignes et par brigades, le plus
légèrement qu'il se put, et s'alla remettre en front de bandière à
Musbach qui était le camp du prince Louis de Bade, l'année précédente,
lorsque notre armée était dans le Spirebach. Je l'y laisserai reposer,
pour parler de l'affaire de Pologne et de M. le prince de Conti, dont
nous apprîmes l'élection à Niederbühl. Comme nous y étions tout proche
des ennemis, le maréchal de Choiseul eut la politesse d'envoyer un
trompette au prince Louis de Bade, pour l'en avertir et qu'il ne fût pas
surpris de la réjouissance que l'armée en devait faire le soir.

\hypertarget{chapitre-ii.}{%
\chapter{CHAPITRE II.}\label{chapitre-ii.}}

1697

~

{\textsc{Affaires de Pologne.}} {\textsc{- Le roi déclare l'élection du
prince de Conti, qui refuse modestement le rang de roi de Pologne.}}
{\textsc{- Départ du prince de Conti, conduit par mer par le célèbre
Jean Bart.}} {\textsc{- Mouvements divers sur ce départ.}} {\textsc{-
Électeur de Saxe couronné à Cracovie.}} {\textsc{- Prince de Conti
arrive à la rade de Dantzig\,; est peu accueilli\,; la ville contre lui,
et n'ose mettre pied à terre.}} {\textsc{- Retour du prince de Conti qui
voit à Copenhague le roi de Danemark incognito.}} {\textsc{- Hardie
expédition de Pointis à Carthagène.}} {\textsc{- Situation du maréchal
de Choiseul et du prince Louis de Bade, qui prend Eberbourg.}}
{\textsc{- Suspension d'armes sur le Rhin.}} {\textsc{- Curieux
sortilège.}} {\textsc{- Portland et ses conférences avec le maréchal de
Boufflers à la tête des armées.}} {\textsc{- Paix signée à Ryswick.}}
{\textsc{- Attention du roi pour le roi et la reine d'Angleterre.}}
{\textsc{- Haine personnelle du roi et du prince d'Orange, et sa cause}}

~

L'abbé de Châteauneuf, arrivant en Pologne, trouva le prince Jacques
réuni à la reine sa mère, et l'abbé de Polignac déclamant contre elle et
contre tous les siens, sans aucun ménagement\,: qu'à bout d'espérance
pour aucun de ses fils, elle s'était liée au parti de l'empereur, qui
faute d'argent avait abandonné le duc de Lorraine, et portait
ouvertement l'électeur de Saxe, qui était devenu le seul compétiteur du
prince de Conti. Cet électeur avait fait abjuration entre les mains du
duc de Saxe-Zeitz, évêque de Javarin, qui était passionné Autrichien\,;
il promit cent douze millions, l'entretien de beaucoup de troupes, et
surtout d'infanterie, dont le besoin était le plus grand pour reprendre
Caminieck, et il offrit de rejoindre la Silésie à la Pologne, et de se
charger du consentement de l'empereur en le dédommageant par
démembrement d'une partie de ses propres États. Il s'assura de l'appui
des Moscovites et de n'être point troublé par les rois du Nord\,; et
avec cela, il gagna l'évêque de Cujavie, quelques autres évêques,
Jablonowski, grand général, le petit général de la couronne et le petit
général de Lituanie, avec quelques autres sénateurs et d'autres moindres
seigneurs qui lui acquirent quatre palatinats. L'espérance du cardinalat
lui dévoua Davia, nonce du pape, sous prétexte du grand intérêt de la
religion à y réunir un puissant électeur, chef né des protestants
d'Allemagne, et leur protecteur en titre\,; et tout cela se fit le plus
secrètement qu'il se put. Sa partie faite, il marcha avec ses troupes en
Silésie, sous prétexte d'aller joindre l'armée impériale en Hongrie, et
d'en prendre le commandement, et s'approcha fort près des frontières de
Pologne.

D'autre part, le cardinal Radziewski, chef de la république pendant
l'interrègne, comme primat du royaume par son archevêché de Gnesne, le
prince Sapiéha, grand général de Lituanie, Bielinski, maréchal de la
diète de l'élection, étaient à la tête du parti du prince de Conti, avec
presque tous les sénateurs, les officiers de la couronne, l'armée à la
tête de laquelle le grand veneur de la couronne s'était mis en l'absence
des deux généraux, et vingt-huit palatinats.

L'élection commença le 27 juin, et s'acheva le même jour. L'évêque, le
grand général Jablonowski, le petit général Potoski et leurs partisans,
appuyés des palatinats de Cracovie, Cujavie, Siradie et Masovie,
s'élevèrent contre, et l'évêque de Cracovie montra l'acte d'abjuration
de l'électeur de Saxe, signé de l'évêque de Javarin, que le nonce Davia
affirma être sa véritable signature\,; {[}ils{]} élurent ce prince
contre toutes les formes, les lois et le droit du primat. L'évêque de
Cujavie proclama l'électeur de Saxe roi de Pologne et grand-duc de
Lituanie dans le champ de l'élection, et y entonna le \emph{Te Deum} que
les siens chantèrent tout de suite. Le primat, de son côté, à la tête
des siens et des vingt-huit autres palatinats, proclama le prince de
Conti. Le prince Radziwil, voyant ce désordre, crut pouvoir ramener le
palatinat de Masovie, où il avait quantité de vassaux, et marcha droit à
lui. On lui cria qu'on le tuerait s'il s'avançait davantage\,; mais au
lieu de s'intimider, il se hâta, et, saisissant l'enseigne plantée à
leur tête, leur cria qu'il fallait donc le tuer ou le suivre, et tous le
suivirent. Il marcha donc avec cette foule de sénateurs et de nonces à
Varsovie, avec le primat, qui entra dans la cathédrale de Saint-Jean
(car Varsovie est du diocèse de Posnanie), chanta le \emph{Te Deum}, et
lit tirer le canon dans l'arsenal, suivant les règles, les lois et les
formes.

Galleran, secrétaire de l'abbé de Polignac, arriva le jeudi 11 juillet
de bonne heure à Marly, avec cette bonne nouvelle\,; le roi la tint
secrète, et envoya à Monseigneur et à M. le prince de Conti, que le
courrier du roi trouva revenant de Meudon à Marly. Après la promenade où
M. le prince de Conti l'alla trouver, et qui s'acheva sans parler de
Pologne, le roi, rentré chez M\textsuperscript{me} de Maintenon, y fit
appeler Torcy, et envoya chercher le prince de Conti, qui se jeta à ses
genoux. Il y avait par le courrier de l'abbé de Polignac une lettre de
lui et une de l'abbé de Châteauneuf, toutes deux fort courtes, qui le
traitaient de roi, avec le dessus à Sa Majesté Polonaise. Le roi, après
avoir félicité le prince de Conti et reçu ses remerciements, voulut
aussi le traiter en roi de Pologne\,; mais ce prince le supplia
d'attendre que son élection fût plus certaine et hors de toute crainte
de revers, pour n'être point embarrassé de lui, si, contre toute
espérance, il arrivait quelque révolution en faveur de l'électeur de
Saxe. Cette modestie, qui venait de désir, fut fort louée\,; le roi y
consentit, et ne laissa pas de vouloir rendre la nouvelle publique. Il
sortit donc de la chambre de M\textsuperscript{me} de Maintenon dans le
grand cabinet, où il y avait beaucoup de dames de celles qui avaient la
privance d'y entrer, à qui le roi dit en leur montrant le prince de
Conti\,: «\,Je vous amène un roi. » Aussitôt la nouvelle se répandit
partout\,; le prince de Conti fut étouffé de compliments, et il alla à
Saint-Germain la dire au roi et à la reine d'Angleterre, à qui le roi le
manda aussi par le duc de La Trémoille, et l'envoya en même temps dire
aussi à Monsieur à Saint-Cloud.

L'électrice de Brandebourg, zélée protestante, ne sut des desseins et
des démarches de l'électeur que ce qu'il ne put cacher. Elle l'y
traversa dans tout ce qu'elle en put apprendre, et lorsqu'elle sut qu'il
s'était fait catholique, le jour de la Trinité, elle en fut outrée au
point qu'elle s'en blessa, et en accoucha. Elle dépêcha au marquis de
Brandebourg-Culbach, son père, de venir en Saxe pour en prendre
l'administration, ce qu'il eut la sagesse de ne pas faire. L'électeur
l'avait donnée, en son absence, au mari de la princesse de Furstemberg,
que nous avons ici, et qui est catholique. Il en prit donc le
gouvernement\,; mais l'électrice ne voulut jamais souffrir qu'il fit
célébrer la messe à Dresde. Pendant ce contraste domestique, l'électeur
s'était avancé tout auprès de Cracovie avec cinq ou six mille hommes de
ses troupes et force Polonais de son parti.

Malgré cela, celui du prince de Conti tenait bon, et il en arriva le 30
août un courrier, avec des nouvelles qui furent la matière des
résolutions prises le même jour et le lendemain, et d'une longue
audience que le roi donna le surlendemain matin dimanche, 1er septembre,
dans son cabinet à Versailles, à M. le prince de Conti, avant la messe.
Il en sortit les larmes aux yeux, et on sut incontinent après qu'il s'en
allait en Pologne. Il pria le roi de ne point traiter
M\textsuperscript{me} la princesse de Conti en reine jusqu'à ce qu'il
eut nouvelle de son couronnement, pour éviter tout embarras en cas que
l'affaire échouât et qu'il fût obligé de revenir. Le roi lui donna deux
millions comptant, et quatre cent mille livres à emporter avec lui, et
cent mille francs pour son équipage, outre toutes les remises faites en
Pologne, que Samuel Bernard s'était chargé d'y faire payer, tant de
l'argent du roi, que de celui de M. le prince de Conti. Ce prince passa
le lundi, en partie à Paris, et le mardi, 3 septembre, en partit le
soir, pour Dunkerque. Le célèbre Jean Bart répondit de le mener
heureusement, malgré la flotte ennemie qui était devant ce port, et tint
parole.

On vit des mouvements bien différents dans cette grande séparation. Le
roi, ravi de se voir glorieusement délivré d'un prince à qui il n'avait
jamais pardonné le voyage de Hongrie, beaucoup moins l'éclat de son
mérite et l'applaudissement général que jusque dans sa cour et sous ses
yeux il n'avait pu émousser par l'empressement même de lui plaire et la
terreur de s'attirer son indignation, ne pouvait cacher sa joie et son
empressement de le voir éloigné pour toujours. On distinguait aisément
ce sentiment particulier de celui du faible avantage d'avoir un prince
de son sang à la tête d'une nation qui figurait peu parmi les autres du
Nord, et qui laissait encore moins figurer son roi. Tout voulait le
prince de Conti à la tête de nos armées. Cet événement était au roi
l'importunité d'un désir et d'un jugement si universel, à son fils
bien-aimé un si fâcheux contraste, et le délivrait du seul de sa maison,
dont la pureté du sang ne fût point flétrie par le mélange de la
bâtardise, et qui en même temps était l'unique dont l'entière nudité
excitait le murmure, pour n'en rien dire de plus, contre les immenses
établissements de ceux qui étaient nés dans l'obscurité légale, et de
ceux encore qui, étant du sang des rois, n'étaient revêtus qu'à titre de
leurs mariages avec les enfants naturels.

M\textsuperscript{me} la princesse de Conti, qui sentait le poids qui
accablait un mari qu'elle aimait et dont elle partageait la fortune,
parut transportée de joie de se voir sur le point de régner. M. le
Prince, plus sensible encore à la gloire d'une couronne pour un gendre
qu'il estimait et qu'il ne se pouvait empêcher d'aimer, cachait sous
cette couverture la joie du repos de sa famille, et M. le Duc nageait
entre la rage de la jalousie d'un mérite si supérieur et récompensé
comme tel par un choix si flatteur, et la satisfaction de se voir à
l'abri du sentiment journalier des pointes de ce mérite, et d'autres
encore plus sensibles à un mari de son humeur. Qui fut à plaindre\,? Ce
fut M\textsuperscript{me} la Duchesse. Elle aimait, elle était aimée,
elle ne pouvait douter qu'elle ne le fût plus que l'éclat d'une
couronne. Il fallait prendre part à une gloire si proche, à la joie du
roi, à celle de sa famille qui l'observait dans tous les moments, qui
voyait clair, mais qui ne put mordre sur les bienséances. Monseigneur
fut un peu touché, mais au bout, aise de la joie d'autrui, son apathie
ne fut point émue. M. du Maine, transporté au fond de l'âme d'une
délivrance si grande et si peu espérée, prit le visage et la contenance
qu'il voulut et qu'il jugea la plus convenable, et le public demeura
partagé entre la douleur de la perte de sis délices, et la joie de les
voir couronnées. Monsieur et M. son fils furent assez aises.
M\textsuperscript{me} de Maintenon triomphait dans ses réduits\,; et les
armées, n'espérant plus de le voir à leur tête, s'affligèrent moins
qu'il fût tout à fait perdu pour elles, qu'elles ne prirent de part au
royal établissement où il était appelé. Pour lui, noyé dans la douleur
la plus profonde, à bout d'obstacles, de difficultés, de délais, il faut
avouer qu'il soutint mal un si brillant choix, et qu'il ne put cacher ni
son désir ni son espérance qu'à la fin il ne réussirait pas.

Il était encore à Paris lorsque le roi reçut un courrier du primat, qui
pressait son départ, dont Torcy lui alla porter les lettres qui le
traitaient de roi. Enfin, il partit de Paris le mardi 3, à onze heures
du soir\,; il répandit deux mille louis par les chemins, d'une malle mal
fermée, dont une partie fut rapportée à Paris, à l'hôtel de Conti. Il
arriva le jeudi après midi à Dunkerque, où tout l'argent qui lui était
destiné l'attendait. Le vent contraire fit qu'il ne s'embarqua que le
vendredi au soir, sur cinq frégates, avec cinquante personnes seulement
pour sa suite. Le chevalier de Sillery, son premier écuyer, frère de
Puysieux et de l'évêque de Soissons, le suivit, et avant de partir,
épousa une M\textsuperscript{lle} Bigot, riche et de beaucoup d'esprit,
avec qui il vivait depuis fort longtemps.

M. le prince de Conti trouva neuf gros vaisseaux ennemis à l'embouchure
de la Meuse, qui l'attendaient au passage. Un vent forcé les empêcha de
l'atteindre, quoiqu'ils y fissent tous leurs efforts\,; cependant le roi
reçut des nouvelles de plus en plus favorables de l'abbé de Polignac de
l'assemblée de la noblesse à Varsovie. Cet ambassadeur attendait le
prince de Conti avec une grande confiance\,; il avait été quarante-cinq
jours sans recevoir aucune lettre d'ici. La reine de Pologne, retirée à
Dantzig et logée chez le maître de la poste, les interceptait toutes, et
à la fin, pour se moquer de l'abbé de Polignac, lui en envoya toutes les
enveloppes. Le prince de Conti passa le Sund sans obstacle, le roi de
Danemark ayant voulu demeurer neutre. Il était avec la reine le 15 aux
fenêtres du château de Cronenbourg à le voir passer : Bart, qui savait
que ce château ne rend point le salut, hésita s'il le ferait, et le
donna pourtant de tout son canon\,; le château répondit de tout le sien,
à cause du prince, qui fit redoubler un second salut, sur ce qu'il
apprit de quelques bâtiments légers qui s'étaient approchés de ses
frégates, que le roi et la reine de Danemark le regardaient passer. Le
17, il se trouva à la rade de Copenhague, où le comte de Guldenlew qui
avait été en France, et plusieurs seigneurs le vinrent saluer, que
Bonrepos, ambassadeur de France en Danemark, lui présenta.

Pendant ce voyage, l'électeur de Saxe ne perdit pas son temps. Le primat
lui avait écrit pour le supplier de ne point troubler leur liberté, et
de vouloir bien se retirer de Pologne, puisque le prince de Conti était
élu et proclamé suivant les lois. L'assemblée de la noblesse de Varsovie
avait établi une garde auprès du corps du feu roi pour empêcher qu'on ne
l'enlevât et qu'on ne le portât à Cracovie, où il est d'usage que la
pompe funèbre et le couronnement du successeur se fassent dans la même
cérémonie. L'électeur jugea que tout dépendait de la force et de la
promptitude\,: il reçut dans un château royal près de Cracovie l'hommage
des principaux de son parti, qui lui firent jurer les \emph{pacta
conventa} qu'ils avaient dressés, lui firent livrer le château de
Cracovie, et l'y menèrent loger. Dans ce château sont gardés la couronne
et tous les ornements royaux dont il s'empara, après avoir fait enfoncer
les portes du lieu où ils étaient. Ensuite, on dressa un catafalque dans
l'église de Cracovie, comme si le corps du feu roi y eût été présent\,;
on y fit les mêmes obsèques, et en même temps, l'évêque de Cujavie,
assisté de quelques autres, couronna l'électeur de Saxe, en présence des
principaux, et d'une multitude de son parti. Le primat, contre les
droits duquel l'évêque de Cujavie attentait en tant de façons, aussi
bien que contre toutes les lois du royaume, publia un long manifeste
contre lui et contre tous les partisans de Saxe, et en même temps des
universaux (circulaires) pour convoquer les petites diètes préparatoires
à la diète générale qui devait décider sur la double élection.

Incontinent après, c'est-à-dire le 25 septembre, le prince de Conti
arriva à la rade de Dantzig, où l'abbé de Châteauneuf qui l'attendait
alla le saluer. La ville s'était déclarée saxonne, et ne fit faire aucun
compliment au prince de Conti. Peu de Polonais, et encore moins de
marque, l'allèrent saluer à bord. Il y demeura à attendre l'ambassade
dont on le flattait, à la tête de laquelle le prince Lubomirski devait
être, et les troupes que le prince Sapiéha lui devait mener. Cependant
ceux de Dantzig refusèrent des vivres à nos frégates, et n'en voulurent
laisser aucune dans leur port. À la fin, l'ambassade de la république
vint saluer le prince de Conti sur sa frégate, l'évêque de Plosko à la
tête, Lubomirski était avec la partie de l'armée de la couronne qui
tenait pour le prince de Conti, que force Polonais vinrent saluer, et
parmi eux Primiski, échanson de la couronne, fort déclaré pour ce parti.
L'évêque de Plosko donna un grand repas au prince de Conti, près de
l'abbaye d'Oliva, avec tout ce qu'il y eut là de plus distingué des
Polonais. Ils burent à la santé de leur roi, qui, n'acceptant pas encore
ce titre, leur fit raison à la liberté de la république. Marége, qui
était à M. le prince de Conti, gentilhomme gascon, et que son esprit et
ses saillies avaient fort mêlé avec tout le monde, relevait à peine
d'une grande maladie, lorsqu'il s'embarqua avec son maître. Il était à
ce repas, où on but à la Polonaise. Il en fut fort pressé, et se
défendait du mieux qu'il pouvait. M. le prince de Conti vint à son
secours, et l'excusa sur ce qu'il était malade\,; mais ces Polonais,
qui, pour se faire entendre, parlaient tous latin, et fort mauvais
latin, ne se payèrent point de cette excuse, et, le forçant à boire,
s'écrièrent en furie\,: \emph{Bibat et moriatur\,!} Marége, qui était
fort plaisant et aussi fort colère, n'en sortait point quand il le
contait à son retour, et faisait beaucoup rire ceux qui lui en
entendaient faire le récit.

Cependant les lettres de nos deux abbés faisaient tout espérer, et celle
du prince de Conti tout craindre. Il trouvait que dix millions ne
l'acquitteraient pas des promesses que l'abbé de Polignac avait faites.
C'était là-dessus que l'abbé comptait, et ceux qu'il avait engagés par
là voulaient voir des espèces à bon escient, avant de se comporter de
même. Cela arrêta tout court le prince Sapiéha et l'armée de Lituanie
qui devait venir joindre le prince de Conti, qui demeurait toujours en
rade et à bord, bien résolu de ne mettre pied à terre que lorsqu'il
verrait des troupes à portée et prêtes à le recevoir\,; mais au lieu
d'armée, qui ne fit pas une seule marche vers lui, il ne vit que des
Polonais avides qui le pressaient d'acquitter les promesses immenses que
l'abbé de Polignac leur avait faites. Le désir de réussir dans cette
grande affaire, dont il espérait la pourpre, l'avait aveuglé, et tiré de
lui des engagements impossibles, de sorte que, trompé le premier en
tout, il trompa le roi et le prince de Conti.

Quoique le primat tînt bon avec un parti et des troupes cantonnées dans
son château de Lowitz, le manque de vivres, les glaces très prochaines
sur ces mers, ni corps d'armée, ni corps de noblesse en aucun mouvement
pour venir recevoir M. le prince de Conti, force déserteurs
considérables, faute d'acquitter les promesses de l'abbé de Polignac\,;
c'en était plus qu'il ne fallait pour persuader le retour à un candidat
plus empressé que n'était M. le prince de Conti, qui pour soi et pour la
France faisait un triste et humiliant personnage, accueilli de personne,
aboyé de tous, et n'osant mettre pied à terre dans un parage ennemi qui
lui refusait des vivres, et ne voulait laisser approcher aucun de ses
bâtiments. Il manda donc au roi sa résolution et ses raisons. Le roi les
loua tout haut à M. le Prince, et envoya Torcy faire compliment de sa
part à M\textsuperscript{me} la princesse de Conti sur sa douleur de ce
qu'elle ne serait point reine et sur le plaisir de revoir bientôt M. le
prince de Conti. On a vu plus haut ce qu'il en fallait croire de cette
joie du roi\,; et en même temps il envoya ordre aux abbés de Polignac et
de Châteauneuf de revenir. Un détachement de trois mille chevaux saxons
vint secrètement autour de l'abbaye d'Oliva pour enlever M. le prince de
Conti, espérant qu'il aurait mis pied à terre. L'abbé de Polignac s'en
sauva à grand'peine, et vendu par ceux de Dantzig y perdit tout son
équipage.

Bart mit à la voile le 6 novembre et ne put sortir de la rade de Dantzig
que le 8\,; il prit, chemin faisant, cinq vaisseaux de Dantzig. Celui de
M. le prince de Conti ayant touché le 15 sur un banc près de Copenhague,
il y passa sur une chaloupe, et y coucha chez M. de Guldenlew. Il vit
après le roi de Danemark incognito, sous le nom de comte d'Alais. Il se
rembarqua le 19, laissant les cinq vaisseaux de Dantzig en dépôt au roi
de Danemark. Il arriva le 10 décembre à Nieuport\,; il y débarqua, parce
que la paix était faite, pour achever son voyage par terre\,; et le
jeudi au soir 12, il arriva à Paris, où il se trouva plus à son gré
qu'il n'eût fait roi à Varsovie. Le lendemain matin, vendredi 13, il
salua le roi qui le reçut à merveille, au fond bien fâché de le revoir.
Il essuya un mauvais temps continuel en ce retour, et ne vit point le
primat. Primiski, dont il se loue le plus, lui dit sur son vaisseau\,:
que s'il avait su qu'il songent à venir, il serait accouru en France
pour l'en empêcher, tant il y avait peu d'apparence de succès.

Ce prince, qui n'avait pu cacher sa douleur à son départ, ne put
empêcher, à son retour, qu'on ne démêlât son contentement extrême. Il
trouva que Mgr le duc de Bourgogne venait d'épouser la princesse de
Savoie. L'abbé de Polignac reçut en chemin ordre d'aller droit en son
abbaye de Bonport, près du Pont-de-l'Arche en Normandie, sans approcher
de la cour ni de Paris\,; et l'abbé de Châteauneuf reçut en même temps
un pareil ordre d'exil. Le prince de Conti, tout mesuré qu'il est, se
plaignit hautement de l'abbé de Polignac\,; il lui pardonnera
difficilement la peur qu'il lui a donnée. J'ai voulu achever tout de
suite tout ce qui regarde ce triste mais illustre voyage\,; il faut
maintenant revenir sur nos pas.

Pointis, chef d'escadre, se rendit célèbre par son entreprise sur
Carthagène. Il prit en passant des flibustiers à file de Saint-Domingue,
dont Ducasse, qui avait été longtemps avec eux, était devenu gouverneur
à force de mérite. Avec ce secours il alla attaquer Carthagène qui ne
s'y attendait pas et se défendit fort mal. Il la pilla, et outre neuf
millions en argent ou en barre, ce qui y fut pris en pierreries et en
argenterie est inconcevable. Cette expédition, qui a tout à fait l'air
d'un roman, fut conduite avec un jugement, et dans l'exécution avec une
présence d'esprit égale à la valeur\,; les flibustiers eurent grand
débat avec Pointis pour leur part, de la plus grande partie de laquelle
ils se prétendaient fraudés. Comme ils virent qu'il se moquait d'eux,
ils retournèrent tout court à Carthagène, la pillèrent de nouveau, y
firent un riche butin, et y trouvèrent encore beaucoup d'argent, puis
envoyèrent ici Galifet, lieutenant de roi de Saint-Domingue, qui était à
l'expédition, porter les plaintes de Ducasse et les leurs. Pointis fut
poursuivi par vingt-deux vaisseaux anglais à qui il échappa. Ils prirent
quelques bâtiments flibustiers, sur lesquels il n'y avait presque rien,
et le vaisseau de Pointis qui servait d'hôpital, où il n'y avait que des
malades et quelques pestiférés. Galifet arriva à Versailles le 20 août,
et presque en même temps Pointis à Brest, malgré six vaisseaux anglais
qui l'attendaient à l'entrée. Il salua le roi à Fontainebleau, le 27
septembre, de qui il fut très bien reçu et fort loué. Il sauva toute sa
prise, et présenta au roi une émeraude grosse comme le poing, et se
justifia fort contre Ducasse et les flibustiers. Peu de jours après il
fut fait lieutenant général, et je pense qu'il s'est mis en état
d'achever sa vie fort à son aise.

J'ai laissé M. le maréchal de Choiseul au camp de Musbach qui s'avança à
Odernheim. Le prince Louis de Bade avait passé le Rhin à Mayence presque
en même temps que nous à Strasbourg, et il était à Creutznach sur la
Nave où il s'était retranché. La Nave est une rivière guéable partout,
mais assez large, fort rapide, avec de l'eau jusqu'au poitrail des
chevaux, et quelquefois plus dans son milieu\,; son lit plein de gros
cailloux roulants et glissants, fort incommodes\,; les bords du côté du
Hundsrück élevés et escarpés. Ceux du côté de l'Alsace sont plats.
Creutznach est un peu élevé, il est des deux côtés avec un pont qui les
joint et qui enfile directement la rivière. Le prince Louis avait bien
fortifié le côté de Creutznach du côté de l'Alsace, tenait toute cette
petite ville et son pont, et avait son armée le long de la Nave qui
coulait à sa tête, et ses flancs couverts chacun d'un ruisseau. En cette
posture, il fouettait de son camp tout ce qui pouvait s'approcher de la
rivière, qui l'était elle-même du pont, et avec la hauteur de son côté
voyait fort loin du nôtre. Il demeura ainsi tranquillement plusieurs
jours, amassant quantité de fourrages du Hundsrück par ses derrières, et
toutes les provisions et munitions de Mayence par un pont de bateaux
qu'il jeta à trois ou quatre lieues de Bingen, où aussi il établit ses
fours. Dès qu'il eut tout à souhait, il attaqua le château d'Eberbourg
par un détachement de son armée qui se relevait tous les jours.
Eberbourg est un pigeonnier sur une pointe de rochers, à demi-lieue de
Creutznach dans la montagne. Sa situation ni celle du pays ne
demandaient point d'investiture, ni plus d'une attaque, de manière que
les Impériaux faisaient ce petit siège en pantoufles.

Le maréchal de Choiseul s'était approché d'eux, et le bruit de leur
canon était une musique piquante à entendre. De secourir ce château rien
ne le permettait d'attaquer le prince Louis, posté comme je viens de le
représenter, parut entièrement impossible\,; restait un troisième parti,
c'était de s'aller placer sur une hauteur au deçà de la Nave qui
commandait leur attaque, et la faire cesser par nos batteries, mais en
même temps il se trouva qu'il n'y avait pas pour trois jours de
fourrages, après quoi il faudrait se retirer. Ce dernier parti n'allait
donc qu'à leur faire suspendre, trois jours durant, leur siége pour leur
laisser après toute liberté, et par cela même fut jugé ridicule. Ce que
le maréchal de Choiseul fit de mieux, fut d'assembler tous les officiers
généraux, de leur exposer l'état des choses, et de les obliger tous à
dire leur avis l'un après l'autre tout haut et devant tous. Par ce moyen
il coupa court à tous les propos qui pourraient se tenir et s'écrire,
parce que chacun parlant tout haut devant tant de témoins, il n'y avait
plus de porte de derrière, et c'était ce que le maréchal s'était
proposé. Dans cette espèce de conseil de guerre, chacun se regarda, et
fut bien étonné d'avoir à dire si publiquement qu'il ne pût se dédire ou
déguiser ce qu'il aurait dit. Aucun ne fut d'avis d'attaquer le prince
Louis, sans exception, aucun ne fut d'avis d'aller sur cette hauteur
pour ne faire que suspendre l'attaque d'Eberbourg et se retirer trois
jours après, excepté Villars tout seul dont tous se moquèrent. Il fut
donc résolu de se retirer quand il n'y aurait plus de fourrage\,; et
cependant d'Arcy se rendit avec tous les honneurs de la guerre, excepté
du canon, et fut traité par le prince Louis avec toutes sortes de
politesses et les louanges que méritait sa valeur et sa belle défense.
Il avait été capitaine dans Picardie\footnote{Le régiment de Picardie,
  dont il s'agit ici, avait été le premier régiment organisé\,; il
  datait du règne d'Henri II.}. Nous gagnâmes donc en deux marches le
camp de Marcksheim, où, logé un peu à part avec quatre ou cinq de mes
amis, je me délassai de la querelle des officiers généraux, dont je
n'avais cessé d'être fatigué, surtout depuis que ce courrier du cabinet
nous eut fait repasser le Rhin.

Ce fut en ce camp que nous reçûmes, par un courrier du cabinet, la
nouvelle de la paix signée à Ryswick, excepté avec l'empereur et
l'empire, mais la suspension d'armes avec eux. M. le maréchal de
Choiseul envoya aussitôt un trompette à M. le prince Louis de Bade, et
lui manda l'ordre qu'il venait de recevoir. Le prince Louis caressa fort
le trompette, et manda au maréchal qu'il avait le même avis de la Haye
par M. Straatman, un des ambassadeurs de l'empereur, mais qu'il n'avait
encore aucun ordre de Vienne\,; ce qui n'empêcherait pas d'observer la
suspension en attendant. Il défendit aussitôt tous actes d'hostilités et
rappela tous les hussards et tous les partis qui étaient dehors. J'eus
grande envie de prendre cette occasion d'aller voir Mayence\,; plusieurs
y furent, mais je n'en pus jamais obtenir la permission du maréchal. Il
se tint toujours à dire que j'étais trop marqué, et que, tout général
d'armée qu'il était, il n'y avait que le roi qui pût permettre à un duc
et pair de sortir du royaume. Nous n'en fûmes pas longtemps à portée,
mais ma curiosité n'en fut pas moins dépitée. La suspension était
jusqu'au 1er novembre, et laissait à cette armée liberté de subsister en
attendant en pays ennemi. Comme par la suspension il n'y avait rien à
craindre, et que les fourrages manquaient absolument, l'armée alla
cantonner dans le pays de la Sarre, et le maréchal prit son quartier
général aux Deux-Ponts. Un comte de Nassau-Hautveiller, voisin de là, y
amena de fort bons chiens courants pour le lièvre, et cette honnêteté
nous fit grand plaisir. L'ennui nous faisait faire des promenades à pied
de trois ou quatre lieues, et nous gagna à tel point plusieurs que nous
étions, qu'il nous persuada une vraie équipée.

Du Bourg, lors maréchal de camp et directeur de la cavalerie, faisait
ses revues par les quartiers. Il nous conta plusieurs choses sérieuses
d'une femme possédée, qu'il avait apprises en passant à l'abbaye de
Metloch, à deux lieues de Sarrelouis, qu'on y exorcisait, et que
pourtant il n'avait point vue. Sur cela nous partîmes sept ou huit,
moitié relais, moitié poste, pour faire dix-huit lieues. Au sortir de
Sarrelouis, nous trouvâmes des gens qui en venaient, qui nous assurèrent
que cette possédée n'était rien moins, mais ou une espèce de folle, ou
une pauvre créature qui cherchait à se faire nourrir. La honte, et le
courrier qui portait les quartiers d'hiver et l'ordre de la séparation
de l'armée, que nous avions rencontré à deux lieues de là, nous
empêchèrent de pousser plus loin. Nous tournâmes bride et revînmes tout
de suite aux Deux-Ponts, où je trouvai le courrier du cabinet couché
dans le lit de mon valet de chambre. Dès le lendemain matin je pris
congé de M. le maréchal de Choiseul et je partis pour Paris.

Cette sottise me fait souvenir d'une histoire si extraordinaire et de
telle nature, que, pour ne la pas oublier et pour n'en pas allonger ces
Mémoires, je la mettrai parmi les Pièces\footnote{Voy., sur ces pièces
  justificatives qui n'ont pas été retrouvées. t. Ier. page 437, note.},
et ma raison la voici\,: j'avais lié une grande amitié dans les
mousquetaires avec le marquis de Rochechouart-Faudoas, qui y était
aussi, quoique de plusieurs années plus âgé que moi. C'était un homme de
valeur, d'excellente compagnie, et de beaucoup d'esprit, de sens, de
discernement et de savoir. Il était riche et paresseux\,; il ne trouva
pas les portes ouvertes pour s'avancer dans le service aussi promptement
qu'il eût voulu, il se dépita contre M. de Barbezieux, et quitta.
M\textsuperscript{me} la duchesse de Mortemart et moi le voulûmes marier
à une fille de M. le duc de Chevreuse qui épousa ensuite M. de Lévi. M.
de Chevreuse en mourait d'envie, mais il ne finissait pas aisément une
affaire. M. de Rochechouart s'en lassa, et il épousa une Chabannes,
riche, fille du marquis de Corton. Il ne vécut pas longtemps avec elle,
n'en eut point d'enfants, et mourut chez lui près de Toulouse fort
brusquement. Sa femme en fut si touchée qu'elle se fit religieuse aux
Bénédictines de Montargis, où elle vécut très saintement. J'ai voulu
expliquer quel était le marquis de Rochechouart, parce qu'il a été
témoin oculaire de l'histoire dont il s'agit, qu'il vint tout droit à
Paris du lieu près de Toulouse où il en eut le spectacle, et me la conta
en arrivant. C'était en carême 1696\,; je lui en fis tant de scrupule,
qu'il alla au grand pénitencier. Par la dernière lettre que j'ai reçue
de lui, de chez lui où il était retourné en automne la même année, il me
mandait que la même histoire, interrompue à sa vue la première fois,
recommençait dans le même lieu avec le savant et l'homme de Pampelune,
et que dans peu de jours il m'en ferait savoir le succès définitif.
C'est en ce point qu'il mourut, et je n'en ai pu apprendre de nouvelles,
parce que, ayant promis le secret du nom de son ami et du lieu où cela
se passait, il ne me voulut jamais nommer ni l'un ni l'autre. Ce marquis
de Rochechouart fut une vraie perte, et je le regrette encore tous les
jours.

La campagne se passa fort tranquillement en Flandre depuis la prise
d'Ath par le maréchal Catinat. Il n'y fut partout question que de
s'observer et de subsister. La paix cependant se traitait fort lentement
à Ryswick, où il s'était perdu beaucoup de temps en cérémonial et en
communications de pouvoirs. Les Hollandais, qui voulaient la paix, s'en
laissaient et plus encore le prince d'Orange, qui avait beaucoup perdu
en Angleterre, et ne tirait pas du parlement ce qu'il voulait. Son grand
point était d'être reconnu roi d'Angleterre par la France, et, s'il
pouvait, d'obliger le roi à faire sortir de son royaume le roi Jacques
d'Angleterre et sa famille. L'empereur, fort embarrassé de sa guerre de
Hongrie, des révoltes de cette année, des avantages considérables que
les Turcs y avaient remportés, ne voulait point de paix sur la mauvaise
bouche. Il retenait l'Espagne par cette raison, dans l'espérance
d'événements qui le missent en meilleure posture et lui procurassent des
conditions plus avantageuses. Tout cela arrêtait la paix. Le prince
d'Orange, bien informé du désir extrême que le roi avait de la faire,
jugea en devoir profiter pour tirer meilleur parti de l'opiniâtreté de
la maison d'Autriche, et, sans avoir l'air de l'abandonner, après en
avoir reçu une si utile protection contre les Stuarts et les catholiques
pour son usurpation, faire une paix particulière, en stipulant pour
cette maison, si elle voulait y entrer, sinon conclure pour l'Angleterre
et la Hollande, et s'en sauver en alléguant que cette république dont il
recevait ses principaux secours et de laquelle il était bien connu qu'il
était maître plus que souverain, et l'Angleterre dont il ne l'était pas
tant à beaucoup près, quoique roi, lui avaient forcé la main, et que
tout ce qu'il avait pu, dans une presse si peu volontaire, avait été de
prendre soin, autant qu'il avait pu, de mettre à couvert les intérêts de
l'empereur et de l'Espagne. Suivant cette idée, qu'il fit adopter
secrètement aux Hollandais, Portland, par son ordre, fit demander tout à
la fin de juin une conférence au maréchal de Boufflers, à la tête de
leurs armées.

Portland était Hollandais, s'appelait Bentinck, avait été beau et
parfaitement bien fait, et en conservait encore des restes\,; il avait
été nourri page du prince d'Orange. Il s'était personnellement attaché à
lui. Le prince d'Orange lui trouva de l'esprit, du sens, de l'entregent
et propre à l'employer en beaucoup de choses. Il en fit son plus cher
favori, et lui communiquait ses secrets, autant qu'un homme aussi
profond et aussi caché que l'était le prince d'Orange en était capable.
Bentinck discret, secret, poli aux autres, fidèle à son maître, adroit
en affaires, le servit très utilement. Il eut la première confiance du
projet et de l'exécution de la révolution d'Angleterre\,; il y
accompagna le prince d'Orange, l'y servit bien\,; il en fut fait comte
de Portland, chevalier de la Jarretière et fut comblé de biens\,; il
servait de lieutenant général dans son armée. Il avait eu commerce avec
le maréchal de Boufflers, à sa sortie de Namur, et pendant qu'il fut
arrêté.

Le prince d'Orange n'ignorait ni le caractère ni le degré de confiance
et de faveur auprès du roi, des généraux de ses armées. Il aima mieux
traiter avec un homme droit, franc et libéral, tel qu'était Boufflers,
qu'avec l'emphase, les grands airs et la vanité du maréchal de Villeroy.
Il ne craignit pas plus l'esprit et les lumières de l'un que de l'autre,
et il comprit que ce qui se passerait par eux irait droit au roi et
reviendrait de même du roi à eux, mais que par Boufflers ce serait avec
plus de précision et de sûreté, parce qu'il n'y ajouterait rien du sien,
ni à informer le roi, ni à donner ses réponses. Boufflers répondit à un
gentilhomme du pays chargé de cette proposition de Portland, qu'il en
écrirait au roi par un courrier exprès, et ce courrier lui apporta fort
promptement l'ordre d'accorder la conférence, et d'écouter ce qu'on lui
voudrait dire. Elle se tint presqu'à la tête des gardes avancées de
l'armée du maréchal de Boufflers. Il y mena peu de suite, Portland
encore moins, qui ne s'approchèrent point, et demeurèrent à cheval
chacune de son côté. Le maréchal et Portland s'avancèrent seuls avec
quatre ou cinq personnes, et, après les premiers compliments, mirent
pied à terre seuls et à distance de n'être point entendus. Ils
conférèrent ainsi debout, en se promenant quelques pas. Il y en eut
trois de la sorte dans le mois de juillet, après la première de la fin
de juin. La dernière de ces quatre fut plus nombreuse en
accompagnements, et les suites se mêlèrent et se parlèrent avec force
civilités, comme ne doutant plus de la paix. Les ministres de l'empereur
en irent des plaintes à ceux d'Angleterre à la Haye qui furent
froidement reçues. À chaque conférence le maréchal de Boufflers en
rendait compte par un courrier. La cinquième se tint, le 1er août, au
moulin de Zenick, entre les deux armées. Portland y fit présent de trois
beaux chevaux anglais au maréchal de Boufflers, d'un au duc de Guiche,
beau-frère du maréchal, et d'un autre à Pracomtal, lieutenant général,
gendre de Montchevreuil, et extrêmement bien avec le maréchal de
Boufflers qu'ils avaient suivi à cette conférence. La sixième fut
extrêmement longue, et la dernière se tint dans une maison de Notre-Dame
de Halle que Portland avait fait meubler et où il avait fait porter de
quoi écrire. Lui et le maréchal furent enfermés longtemps dans une
chambre, pendant que leur suite, pied à terre, nombreuse de part et
d'autre et mêlée ensemble, fit la conversation d'une manière polie et
fort amiable, comme ne doutant plus de la paix.

En effet ces conférences la pressèrent. Les ministres des alliés eurent
peur que le maréchal de Boufflers et Portland ne vinssent au point de
conclusion pour l'Angleterre, et que la Hollande n'y fût entraînée\,; et
la prise de Barcelone fut un nouvel aiguillon qui rendit effectif et
sérieux à Ryswick ce qui jusqu'alors n'avait été qu'un indécent
pelotage. Je ne m'embarquerai pas ici dans le récit de cette paix. Elle
aura vraisemblablement le sort de toutes les précédentes\,: des acteurs
et des spectateurs curieux et instruits en écriront la forme et le
fond\,; je me contenterai de dire que tout le monde convint après, que
les alliés n'eurent Luxembourg que de la grâce de M. d'Harlay, qui,
malgré ses deux collègues, trancha du premier, quoique les deux autres
aient beaucoup souffert de ses avis et de ses manières, et qu'ils aient
eu la sagesse de n'en venir jamais à aucune brouillerie. J'ai ouï
assurer ce fait souvent à Callières, qui ne s'en pouvait consoler.
L'empereur et l'empire à leur ordinaire ne voulurent pas signer avec les
autres, mais autant valut, et leur paix se fit ensuite telle qu'elle
avait été projetée à Ryswick.

La première nouvelle qu'on eut de sa signature fut par un aide de camp
du maréchal de Boufflers qui arriva le dimanche 22 septembre à
Fontainebleau, dépêché par ce maréchal, sur ce que l'électeur de Bavière
lui avait mandé que la paix avait été signée à Ryswick le vendredi
précédent à minuit. Le lendemain matin il y arriva un autre courrier du
même maréchal, accompagnant jusque-là celui que l'électeur envoyait
porter la même nouvelle en Espagne, et à quatre heures après midi du
même lundi, un autre de don Bernard-François de Quiros, premier
ambassadeur plénipotentiaire d'Espagne, pour y porter la même nouvelle.
M. de Bavière eut la petitesse de faire écrire pour prier qu'on amusât
ce courrier de l'ambassadeur, pour donner moyen au sien d'arriver avant
lui à Madrid\,; et le plaisant est qu'on avait beau jeu à l'amuser, il
n'avait pas un sou pour payer sa poste ni pour vivre, et le roi lui fit
donner de l'argent. On sut par lui qu'il était six heures du matin du
samedi quand la paix fut signée. Enfin, le jeudi 26 septembre, Celi,
fils d'Harlay, arriva à cinq heures du matin à Fontainebleau, après
s'être amusé en chemin avec une fille qu'il trouva à son gré et du vin
qui lui parut bon. Il avait fait toutes les sottises et toutes les
impertinences dont un jeune fou et fort débauché et parfaitement gâté
par son père s'était pu aviser, dont plusieurs même avaient été fort
loin et importantes, qu'il couronna par ce beau délai\,; ainsi il
n'apprit rien de nouveau.

Le roi et la reine d'Angleterre étaient à Fontainebleau, à qui la
reconnaissance du prince d'Orange fut bien amère, mais ils en
connaissaient la nécessité pour avoir la paix, et savaient bien aussi
que cet article ne l'était guère moins au roi qu'à eux-mêmes, dont
j'expliquerai tout présentement la raison. Ils se consolèrent comme ils
purent, et parurent même fort obligés au roi, qui tint également ferme à
ne vouloir pas souffrir qu'ils sortissent de France, ni qu'ils
quittassent le séjour de Saint-Germain. Ces deux points avaient été
vivement demandés\,; le dernier surtout dans l'impossibilité d'obtenir
l'autre, tant à Ryswick que dans les conférences par Portland. Le roi
eut l'attention de dire à Torcy, sur le point de la signature, que si le
courrier qui en apporterait la nouvelle arrivait, un ou plusieurs, l'un
après l'autre, il ne lui vint point dire, s'il était alors avec le roi
et la reine d'Angleterre, et il défendit aux musiciens de chanter rien
qui eût rapport à la paix, jusqu'au départ de la cour d'Angleterre.

On sut en même temps que le prince Eugène avait gagné une bataille
considérable en Hongrie, qui y rétablit fort les affaires et la
réputation de l'empereur, mais dont, faute d'argent, il ne put profiter,
comme il eût été aisé de faire grandement. Pour achever de suite cette
matière de la paix, les ratifications étant échangées, elle fut publiée
le 22 octobre, à Paris, avec l'Angleterre et la Hollande, et huit jours
après avec l'Espagne. Celi, qui était retourné, arriva à Versailles le 2
novembre portant la nouvelle de la signature de la paix avec l'empereur
et presque tout l'empire\,; quelques protestants faisaient encore
difficulté de la signer, sur ce que le roi insistait que la religion
catholique fût conservée dans les pays à eux rendus\,; et à la fin ils y
passèrent. Celi, malgré sa conduite, eut douze mille livres de
gratification. On peut juger que les \emph{Te Deum} et les harangues de
tous les corps furent la suite de cette paix, dans lesquelles il fut
bien répété que le roi avait bien voulu la donner à l'Europe. Retournons
maintenant à beaucoup de choses laissées en arrière, pour n'avoir pas
voulu interrompre le voyage de M. le prince de Conti, et la conclusion
de la paix. Ajoutons-y, auparavant de finir la guerre, que pendant la
campagne, vers le fort des conférences du maréchal de Boufflers, M. le
comte de Toulouse fut fait seul lieutenant général.

Je m'aperçois que j'oublie de tenir parole sur les raisons particulières
qui rendaient au roi la reconnaissance du prince d'Orange pour roi
d'Angleterre si amère\,; les voici\,: le roi était bien éloigné, quand
il eut des bâtards, des pensées qui, par degrés, crûrent toujours en lui
pour leur élévation. La princesse de Conti, dont la naissance était la
moins odieuse, était aussi la première\,; le roi la crut magnifiquement
mariée au prince d'Orange, et la lui fit proposer, dans un temps où ses
prospérités et son nom dans l'Europe lui persuadaient que cela serait
reçu comme le plus grand honneur et le plus grand avantage. Il se
trompa\,: le prince d'Orange était fils d'une fille du roi d'Angleterre,
Charles Ier, et sa grand'mère était fille de l'électeur de Brandebourg.
Il s'en souvint avec tant de hauteur qu'il répondit nettement que les
princes d'Orange étaient accoutumés à épouser des filles légitimes des
grands rois, et non pas leurs bâtardes. Ce mot entra si profondément
dans le cœur du roi qu'il ne l'oublia jamais, et qu'il prit à tâche, et
souvent contre son plus palpable intérêt, de montrer combien
l'indignation qu'il en avait conçue était entrée profondément en son
âme.

Il n'y eut rien d'omis de la part du prince d'Orange pour l'effacer\,:
respects, soumissions, offices, patience dans les injures et les
traverses personnelles, redoublement d'efforts, tout fut rejeté avec
mépris. Les ministres du roi en Hollande eurent toujours un ordre exprès
de traverser ce prince, non seulement dans les affaires d'État, mais
dans toutes les particulières et personnelles\,; de soulever tout ce
qu'ils pourraient de gens des villes contre lui, de répandre de l'argent
pour faire élire aux magistratures les personnes qui lui étaient les
plus opposées, de protéger ouvertement ceux qui étaient déclarés contre
lui, de ne le point voir\,; en un mot, de lui faire tout le mal et
toutes les malhonnêtetés dont ils pourraient s'aviser. Jamais le prince,
jusqu'à l'entrée de cette guerre, ne cessa, et publiquement, et par des
voies plus sourdes, d'apaiser cette colère\,; jamais le roi ne s'en
relâcha. Enfin, désespérant d'obtenir de rentrer dans les bonnes grâces
du roi, et dans l'espérance de sa prochaine invasion de l'Angleterre, et
de l'effet de la formidable ligue qu'il avait formée contre la France,
il dit tout haut qu'il avait toute sa vie inutilement travaillé à
obtenir les bontés du roi, mais qu'il espérait du moins être plus
heureux à mériter son estime. On peut juger ensuite quel triomphe ce fut
pour lui que de forcer le roi à le reconnaître roi d'Angleterre, et tout
ce que cette reconnaissance coûta au roi.

\hypertarget{chapitre-iii.}{%
\chapter{CHAPITRE III.}\label{chapitre-iii.}}

1697

~

{\textsc{M. d'Orléans, cardinal.}} {\textsc{- Mort du célèbre Santeuil,
du baron de Beauvais, de La Chaise, de la duchesse de La Feuillade, du
duc de Duras.}} {\textsc{- Époque des ducs-maréchaux de France de porter
le nom de maréchal.}} {\textsc{- Retraite volontaire de Pelletier,
ministre d'État.}} {\textsc{- Sa fortune et sa famille.}} {\textsc{- Les
postes à M. de Pomponne.}} {\textsc{- Maximes du roi contre un premier
ministre et de ne mettre jamais aucun ecclésiastique dans le conseil.}}
{\textsc{- Emplois au dehors.}} {\textsc{- Bonrepos et sa fortune.}}
{\textsc{- Des Alleurs.}} {\textsc{- Du Héron.}} {\textsc{- Prince de
Hesse-Darmstadt fait grand d'Espagne, et pourquoi.}} {\textsc{-
Singulière retraite d'Aubigné, frère de M\textsuperscript{me} de
Maintenon.}} {\textsc{- Cour et vie particulière de la princesse.}}
{\textsc{- Tracasserie avec la duchesse du Lude.}} {\textsc{-
Préparatifs du mariage de Mgr le duc de Bourgogne.}} {\textsc{- Goût du
roi pour la magnificence de sa cour.}} {\textsc{- Ses égards}}

~

Il {[}le roi{]} était au conseil à Marly le mardi dernier juillet,
lorsque le courrier du cardinal de Janson arriva apportant la promotion
des couronnes, et une heure après, celui du pape avec la calotte pour
l'évêque d'Orléans, qui au sortir du conseil la lui présenta et la reçut
baissé fort bas, de ses mains sur sa tête, avec beaucoup d'amitié. Pour
achever de suite, le cardinal de Janson arriva le 8 septembre à
Versailles, et avec lui l'abbé de Barrière, camérier du pape, avec la
barrette du cardinal de Coislin, à qui le roi la donna le lendemain à sa
messe. Quelques jours après, étant au lever du roi, il lui demanda si on
le verrait à cette heure avec des habits d'invention\,: «\,Moi, sire,
dit le nouveau cardinal, je me souviendrai toujours que je suis prêtre
avant que d'être cardinal.\,» Il tint parole\,: il ne changea rien à la
simplicité de sa maison et de sa table, il ne porta jamais que des
soutanelles de drap ou d'étoffes légères, sans soie, et n'eut de rouge
sur lui que sa calotte et le ruban de son chapeau. Le roi, qui s'en
doutait bien, loua fort sa réponse, et encore plus sa conduite qui le
mit de plus en plus en vénération.

M. le Duc tint cette année les états de Bourgogne, en la place de M. le
Prince, son père, qui n'y voulut pas aller. Il y donna un grand exemple
de l'amitié des princes, et une belle leçon à ceux qui la recherchent.
Santeuil, chanoine régulier de Saint-Victor, a été trop connu dans la
république des lettres et dans le monde, pour que je m'amuse à m'étendre
sur lui. C'était le plus grand poète latin qui ait paru depuis plusieurs
siècles\,; plein d'esprit, de feu, de caprices les plus plaisants, qui
le rendaient d'excellente compagnie\,; bon convive surtout, aimant le
vin et la bonne chère, mais sans débauche, quoique cela fût fort déplacé
dans un homme de son état, et qui avec un esprit et des talents aussi
peu propres au cloître, était pourtant au fond aussi bon religieux
qu'avec un tel esprit il pouvait l'être. M. le Prince l'avait presque
toujours à Chantilly quand il y allait\,; M. le Duc le mettait de toutes
ses parties\,; en un mot, princes et princesses, c'était de toute la
maison de Condé à qui l'aimait le mieux, et des assauts continuels avec
lui de pièces d'esprit en prose et en vers, et de toutes sortes
d'amusements, de badinages et de plaisanteries, et il y avait bien des
années que cela durait. M. le Duc voulut l'emmener à Dijon. Santeuil
s'en excusa, allégua tout ce qu'il put\,; il fallut obéir, et le voilà
chez M. le Duc établi pour le temps des états. C'étaient tous les soirs
des soupers que M. le Duc donnait ou recevait, et toujours Santeuil à sa
suite qui faisait tout le plaisir de la table. Un soir que M. le Duc
soupait chez lui, il se divertit à pousser Santeuil de vin de Champagne,
et de gaieté en gaieté, il trouva plaisant de verser sa tabatière pleine
de tabac d'Espagne dans un grand verre de vin et de le faire boire à
Santeuil pour voir ce qui en arriverait. Il ne fut pas longtemps à en
être éclairci\,: les vomissements et la fièvre le prirent, et en deux
fois vingt-quatre heures, le malheureux mourut dans des douleurs de
damné, mais dans les sentiments d'une grande pénitence, avec lesquels il
reçut les sacrements et édifia autant qu'il fut regretté d'une compagnie
peu portée à l'édification, mais qui détesta une si cruelle expérience.

D'autres morts suivirent de près\,: le baron de Beauvais d'apoplexie,
duquel j'ai parlé ailleurs, que le roi regretta.

La Chaise, capitaine de la porte, et frère du P. de La Chaise, qui,
d'écuyer de l'archevêque de Lyon dont il commandait l'équipage de
chasse, lui fit cette fortune. Ils ne s'y oublièrent ni l'un ni l'autre,
tous deux firent toujours une profession ouverte de respect et
d'attachement pour MM. de Villeroy, et La Chaise n'évitait point de
parler de l'archevêque de Lyon et de ses chasses. C'était un grand
échalas, prodigieux en hauteur, et si mince, qu'on croyait toujours
qu'il allait rompre, très bon et honnête homme\,; il mourut en revenant
de Bourbon, et son fils eut aussitôt sa charge, et deux jours après, le
roi écrivit de sa main au P. de La Chaise qu'il donnait à son neveu cent
mille écus de brevet de retenue, qui était aussi un fort honnête garçon.

La duchesse de La Feuillade mourut à Paris fort jeune, de la poitrine,
et ce fut dommage de toutes façons, et il n'y eut que son mari qui ne
s'en soucia guère. Il avait toujours très mal vécu avec elle,
quoiqu'elle ne méritât rien moins, et avec un parfait mépris pour sa
famille qui avait toujours fait merveilles pour lui. Il répondit une
fois assez plaisamment à quelqu'un qui voulait parler à son beau-père,
et qui lui demanda ce qu'il faisait, qu'il était à éplucher de la salade
avec ses commis\,: en effet Châteauneuf n'avait aucun département que
des provinces\footnote{Les quatre secrétaires d'État, sous l'ancienne
  monarchie, se partageaient les provinces, parce qu'il n'y avait pas
  alors de ministère de l'intérieur. Voici le tableau des provinces qui
  dépendaient, en 1781, des divers secrétaires d'État, d'après Guyot
  (\emph{Traité des Offices}, livre I, chap.~LXXIX) 1° le secrétaire
  d'État, chargé des affaires étrangères, avait, dans son département,
  la Guyenne, la Gascogne, la Normandie, la Champagne, la principauté de
  Dombes, le Berry\,; 2° du secrétaire d'État de la maison du roi
  dépendaient la ville et généralité de Paris, le Languedoc, la
  Provence, la Bourgogne, la Bresse, le Bugey, le Valromey, le pays de
  Gex, la Bretagne, le comté de Foix, la Navarre, le Béarn, le Bigorre,
  la Picardie, le Boulonnais, la Touraine, le Bourbonnais, l'Auvergne,
  le Nivernais, la Marche, le Limousin, l'Orléanais, le Poitou, l'Aunis
  et la Saintonge\,; 3° les ports de mer et les colonies relevaient du
  ministre de la marine\,; 4° le secrétaire d'État de la guerre avait
  dans son département, les Trois-Évêchés (Toul, Metz et Verdun), la
  Lorraine, l'Artois, la Flandre, l'Alsace, la Franche-Comté, le
  Dauphiné, le Roussillon et l'île de Corse. Le secrétaire d'État de la
  maison du roi (c'était le département de Châteauneuf dont parle
  Saint-Simon) était, comme on le voit, celui des ministres qui avait
  dans ses attributions le plus grand nombre de provinces.}. Les
huguenots étaient le département particulier de sa charge de secrétaire
d'État, qui la rendait importante lorsqu'ils faisaient un corps armé
avec lequel il fallait compter, mais depuis la révocation de l'édit de
Nantes, cette charge de secrétaire d'État était à peu près nulle, et
Châteauneuf de son génie et de sa personne existait encore moins, s'il
se pouvait. La Feuillade n'eut point d'enfants de ce mariage et n'avait
guère cherché à en avoir.

Le duc de Duras mourut de la petite vérole, et de beaucoup d'autres, en
Flandre pendant la campagne. Il était brigadier de cavalerie avec
distinction, et l'affaire de sa survivance de capitaine des gardes du
corps était comme faite, ce qui augmenta fort la douleur de sa famille.
Sa mère, sœur du duc de Ventadour, ne s'en est jamais consolée\,; elle
l'aimait uniquement. C'était un homme bien fait et d'une beauté
singulière\,; le vin et les débauches l'avaient fort changé et rendu
goutteux. C'était un fort honnête homme et fort aimé, brave, doux et
voulant faire, mais sans aucun esprit. Son père l'avait assez
étrangement marié de tous points\,; il lui céda sa dignité en le
mariant, le fit appeler le duc de Duras, et prit le nom de maréchal de
Duras. Jusqu'alors on ne l'avait jamais appelé que le duc de Duras, et
c'est le premier duc-maréchal de France qui, par le défaut de terres à
porter divers noms, et pour la distinction de l'un et de l'autre, se
soit fait appeler maréchal. Jamais on n'a dit que le duc de Navailles,
le duc de Vivonne, etc. Depuis, cet exemple a été suivi par la même
convenance, et peu à peu a quelquefois prévalu sans cette raison. Le duc
de Duras ne laissa que deux filles\,; il n'avait qu'un frère beaucoup
plus jeune que lui à qui le roi donna son régiment.

Le même jour de la nouvelle de cette mort, qui était un mercredi, 18
septembre, à Versailles, veille du départ du roi pour Fontainebleau, M.
Pelletier, ministre d'État, prit congé du roi à la fin du conseil, et,
sans en avoir parlé à qui que ce fût qu'au roi, monta tout de suite en
carrosse, et se retira en sa maison de Villeneuve-le-Roi. Il avait passé
par les charges de conseiller au parlement et de président en la
quatrième chambre des enquêtes\,; après il fut prévôt des marchands, et
fit à Paris ce quai près de la Grève qui porte encore son nom. De là il
devint conseiller d'État, qui est le débouché ordinaire des prévôts des
marchands. Il fut connu de M. Le Tellier et de M. de Louvois qui le
servirent pour ces places, et entra tellement dans leur confiance qu'il
devint l'arbitre des affaires de leur famille et des débats particuliers
du père et du fils, qu'ils eurent toujours le bon esprit de cacher sous
le dernier secret. À la mort de M. Colbert, MM. Le Tellier et de
Louvois, qui savaient ce que leur avait coûté un habile contrôleur
général leur ennemi, mirent tout leur crédit à faire donner cette place
à Pelletier qui la craignit plus qu'il n'en eut de joie. Il y fut
parfaitement reconnaissant pour ses bienfaiteurs. Après leur mort, il
continua d'être l'arbitre des affaires de leur famille, à laquelle il
demeura parfaitement attaché, et vécut toujours avec M. de Barbezieux
dans une sorte de dépendance.

C'était un homme fort sage et fort modéré, fort doux et obligeant, très
modeste et d'une conscience timorée\,; d'ailleurs fort pédant et fort
court de génie. Il y a un mot du premier maréchal de Villeroy sur lui
admirable. Il donnait un matin petite direction chez lui\,; tout ce qui
la devait composer était arrivé, et on attendait M. Pelletier qui était
contrôleur général. Enfin, las d'attendre, le maréchal envoya chez lui,
et à Versailles où ils étaient, il n'y avait pas loin. On vint dire au
maréchal de Villeroy qu'apparemment M. Pelletier avait oublié la petite
direction\footnote{Voy. sur le conseil de finances appelé \emph{petite
  direction}, la note à la fin du t. Ier, p. 445 et 446.}, et qu'il
était allé courre le lièvre. «\,Par\ldots. répondit le maréchal en
colère, avec son ton de fausset, nous avons vu M. Colbert qui n'en
courait pas tant et qui en prenait davantage.\,» On rit et on commença
la petite direction. Lorsque ce contrôleur général vit venir la guerre
de 1688, la confiance intime qui était entre M. de Louvois et lui en fit
prévoir toutes les suites. C'était à lui à en porter tous le poids par
les fonds extraordinaires, et ce poids l'épouvanta tellement, qu'il ne
cessa d'importuner le roi jusqu'à ce qu'il lui permit de quitter sa
place de contrôleur général. Outre que ce n'était pas le compte de M. de
Louvois, qui avait repris alors le premier crédit, le roi y eut une
grande peine. Il aimait et il estimait Pelletier, il se souvenait
toujours des embarras qu'il avait essuyés des divisions de MM. de
Louvois et Colbert, il en était à l'abri entre Louvois et Pelletier, et
à la veille d'une grande guerre ce lui était un grand soulagement.
N'ayant pu venir à bout de vaincre le contrôleur général après plusieurs
mois de dispute, cette même convenance engagea le roi à lui proposer
Pelletier de Sousy, son frère et intendant des finances, pour contrôleur
général. Celui-ci avait bien plus de lumières et de monde, mais son
frère ne crut pas le devoir exposer aux tentations d'une place qu'il ne
tient qu'à celui qui la remplit de rendre aussi lucrative qu'il veut, et
il supplia le roi de n'y point penser. Le roi, plus plein d'estime
encore par cette action pour Pelletier, mais plus embarrassé du choix,
voulut qu'il le fît lui-même, et il proposa Pontchartrain dont j'aurai
lieu de parler ailleurs, et qui fut contrôleur général. Pelletier
demeura simple ministre d'État\,; et comme, hors de se trouver au
conseil, il n'avait aucune fonction, il demeura peu compté par le
courtisan, qui l'appela le ministre Claude. Ils se souvenaient encore de
celui de Charenton, et en effet Pelletier s'appelait Claude.

Il eut l'administration des postes à la mort de M. de Louvois, et le roi
le traita toujours avec tant de confiance et d'amitié qu'à une maladie
que le chancelier Boucherat avait eue l'année précédente, il s'était
laissé assez entendre à Pelletier pour que celui-ci pût compter d'être
son successeur. Pelletier était droit et vraiment homme de bien. Il fit
ses réflexions. Il avait toujours eu dessein de mettre un intervalle
entre la vie et la mort, et il comprit qu'un chancelier ne pouvait plus
se retirer. Boucherat, plus qu'octogénaire, tombait de jour en jour,
cela fit peur à Pelletier\,; il voulut prévenir la vacance. L'affaire de
la paix le retenait. Il ne trouvait pas séant de la laisser imparfaite,
mais dès qu'il la vit assurée à peu près il demanda son congé. Ce fut un
débat entre le roi et lui qui dura plus de deux mois\,; il ne l'arracha
qu'à grand'peine. Au moins le roi exigea qu'il le viendrait voir tous
les ans deux ou trois fois dans son cabinet par les derrières, et qu'il
conservât toutes ses pensions qui allaient à quatre-vingt mille livres
de rente. Il capitula il s'engagea à venir voir le roi comme il le
désirait\,; mais il se débattit tant sur les pensions qu'il n'en garda
que vingt mille livres pour lui et six mille livres pour son fils, à qui
il avait remis, il y avait déjà longtemps, une charge de président à
mortier, qu'il avait achetée étant contrôleur général. En entrant au
conseil, d'où il partit pour sa retraite, il tira le duc de Beauvilliers
dans une fenêtre et la lui confia. Il était fort son ami et de M. de
Pomponne qui ne la sut que par l'événement, et qui s'écria qu'il l'avait
prévenu. Mais Pomponne n'était pas en même situation\,; il était chargé
des affaires étrangères dont Torcy, son gendre, n'avait encore que le
nom.

La famille de M. Pelletier fut également surprise et affligée, mais elle
n'y perdit rien. Pelletier conserva tout son crédit, et fit plus pour
elle de sa retraite qu'il n'avait fait jusqu'alors à la cour\,; il ne
vit exactement personne à Villeneuve que sa plus étroite famille et
quelques gens de bien. Il passa l'hiver à Paris avec son fils dans sa
maison, et s'y élargit un peu davantage. Il était fort des amis de mon
père, et il voulut bien me voir. J'en fus ravi, et j'admirai moins la
sérénité tranquille et douce que je remarquai en lui, que son attention
à rompre tous discours sur sa retraite et qui sentit l'encens. Il avait
lors soixante-six ans et une santé parfaite de corps et d'esprit. Il
passa toujours les hivers à Paris, où je le voyais de temps en temps, et
toujours avec plaisir et respect pour sa vertu, et tout le reste de
l'année à Villeneuve, et soutint sa retraite avec une grande sagesse et
une grande piété. Il avait épousé une Fleuriau, veuve d'un président de
Fourcy, qu'il avait perdue il y avait longtemps. M\textsuperscript{me}
de Châteauneuf, femme du secrétaire d'État, était fille du premier
mariage de sa femme, laquelle avait un frère d'âge fort disproportionné
d'elle qui était M. d'Armenonville. M. Pelletier le fit travailler sous
lui, et lui procura une charge d'intendant des finances. Il a été fort
connu dans le monde, et j'aurai occasion d'en parler.

M. Pelletier, outre son fils aîné, en eut deux autres et deux filles
mariées à M. d'Argouges et à M. d'Aligre, maîtres des requêtes tous
deux. De sa retraite il fit le premier conseiller d'État, l'autre,
président à mortier, tous deux fort jeunes, et un de ses fils évêque
d'Angers, puis d'Orléans, quoique éborgné jeune d'une fusée à une
fenêtre de l'hôtel de ville au feu de la Saint-Jean. L'autre fut
supérieur des séminaires de Saint-Sulpice. C'était un cafard qui en
bannit la science et y mit tout en misérables minuties. Il usurpa du
crédit à force de molinisme et eut souvent part aux grâces
ecclésiastiques. Il était lourde dupe et dominait fort le clergé.
C'était un animal si plat et si glorieux qu'il disait quelquefois à ses
jeunes séminaristes qu'il malmenait pour des riens\,: «\,Mais vous
autres, à qui croyez-vous donc avoir affaire\,? savez-vous que je suis
fils d'un ministre d'État, et contrôleur général, et frère d'un évêque
et d'un président à mortier\,» et avec cela il croyait avoir tout dit.

Le roi ne remplit point la place de ministre et donna le soin des postes
à M. de Pomponne. Peu de jours après, voyant au conseil des dépêches de
Rome qui ne ressemblaient pas à celles qu'il avait accoutumé de recevoir
du cardinal de Janson, qui après sept ans de séjour fort utile ne
faisait qu'en arriver, le roi se mit sur ses louanges, et ajouta qu'il
regardait comme un vrai malheur de ne pouvoir pas le faire ministre.
Torcy, qui avait porté les dépêches, mais sans s'asseoir ni opiner
encore, crut faire sa cour de dire, entre haut et bas, qu'il n'y avait
personne plus propre que lui, et que dès qu'il avait le bonheur d'en
être estimé capable par le roi, il ne voyait pas ce qui pouvait
l'empêcher de l'être. Le roi, qui l'entendit, répondit que lorsqu'à la
mort du cardinal Mazarin il avait pris le timon de ses affaires, il
avait en grande connaissance de cause bien résolu de n'admettre jamais
aucun ecclésiastique dans son conseil, et moins encore les cardinaux que
les autres\,; qu'il s'en était bien trouvé, et qu'il ne changerait pas.
Il ajouta qu'il était bien vrai qu'outre la capacité, le cardinal de
Janson n'aurait pas les inconvénients des autres, mais que ce serait un
exemple\,; qu'il ne le voulait pas faire\,; ce qui ne l'empêchait pas de
regretter de ne l'y pouvoir faire entrer. Je l'ai su de Torcy même, et
longtemps auparavant, de M. de Beauvilliers et de M. de Pontchartrain
père.

Le comte de Portland fut destiné à l'ambassade de France, le comte de
Tallard à celle d'Angleterre, Bonrepos à celle de Hollande, qui fut
relevé en Danemark par le comte de Chamilly, neveu du lieutenant
général. Quelque temps après, Villars, commissaire général de la
cavalerie, fils du chevalier de l'ordre, fut choisi pour envoyé à
Vienne\,; Phélypeaux, maréchal de camp, à Cologne\,; des Alleurs, à
Berlin\,; du Héron, colonel de dragons, à Wolfenbuttel\,; d'Iberville, à
Mayence. Bonrepos se prétendait gentilhomme du pays de Foix\,; il avait
passé sa vie dans les bureaux de la marine. M. de Seignelay s'en servait
avec confiance, et quoique l'oncle et le neveu ne fussent pas toujours
d'accord, M. de Croissy lui donna aussi la sienne. Un traité de marine
et de commerce que, pendant la paix précédente, il alla faire en
Angleterre où il réussit fort bien, le fit connaître à Croissy. Il y
demeura longtemps à reprises, et en homme d'esprit et de sens, se
procurait l'occasion de faire des voyages à la cour, où il fit valoir
son travail. Cet emploi le décrassa. Il continua à travailler sous M. de
Seignelay, puis sous M. de Pontchartrain, mais non plus sur le pied de
premier commis. Il obtint permission d'acheter une charge de lecteur du
roi, pour en avoir les entrées et un logement à Versailles\,; il s'y
était fait des amis de ceux de M. de Seignelay, et d'autres encore. Il
était honnête homme et fort bien reçu dans les maisons les plus
distinguées de la cour. Tout cela l'aida à prendre un plus grand vol, et
il y réussit toujours dans ses ambassades. C'était un très petit homme,
gros, d'une figure assez ridicule, avec un accent désagréable, mais qui
parlait bien, et avec qui il y avait à apprendre, et même à s'amuser.
Quoiqu'il ne se fût pas donné pour un autre, il était sage et
respectueux\,; il avait fort gagné chez M. de Seignelay pendant la
prospérité de la marine\,; il était riche et entendu, fort honorable et
toutefois ménageait très bien son fait. Il était frère de d'Usson,
lieutenant général qui n'était pas sans mérite à la guerre, où il a
passé toute sa vie. Point mariés tous deux, ils prirent soin d'un fils
de leur frère aîné qui était demeuré dans son pays de Foix, et dont on
n'a jamais ouï parler. Ce neveu s'appelait Bonac, dont j'aurai occasion
de parler.

Des Alleurs était un Normand de fort peu de chose, fait à peindre, et de
grande mine, qui lui avait fort servi en sa jeunesse. Il avait été
longtemps capitaine aux gardes, et servit toute cette guerre de major
général à l'armée du Rhin\,; et il l'était excellent. À la longue, il
devint lieutenant général et grand-croix de Saint-Louis. C'était un
matois doux, respectueux, affable à tout le monde, et qui le connaissait
bien\,; il avait de la valeur et beaucoup d'esprit, du tour, de la
finesse, avec un air toujours simple et aisé. Il s'amouracha à
Strasbourg, où il était employé les hivers, de M\textsuperscript{lle} de
Lutzbourg, belle, bien faite, et de fort bonne maison, laquelle avait eu
plus d'un amant, et qui, n'ayant rien vaillant que beaucoup d'esprit et
d'adresse, voulut faire une fin comme les cochers, et fit si bien
qu'elle l'épousa.

Du Héron était aussi Normand et peu de chose, fort bien fait aussi, mais
d'une autre façon et bien plus jeune. C'était un très bon officier et un
des plus excellents sujets qu'on pût choisir à tous égards pour les
négociations, et avec cela doux, modeste, appliqué et fort honnête
homme. Iberville avait été dans les bureaux de M. de Croissy, d'où on le
prit pour Mayence. C'était encore un Normand et fort délié, et très
capable d'affaires. Des autres, j'aurai lieu d'en parler ailleurs ainsi
que de Puysieux, qui alla relever Amelot, conseiller d'État, en Suisse,
et d'Harcourt, en Espagne.

Ces emplois étrangers me font souvenir d'une anecdote étrangère qui
mérite bien de n'être pas oubliée. J'ai remarqué en parlant du siège et
de la prise de Barcelone par M. de Vendôme, que le prince de Darmstadt
commandait dans le Montjoui, qui en est comme la citadelle, quoiqu'un
peu séparé. Le fil de la narration m'a emporté ailleurs, il faut revenir
à ce prince. C'était un homme fort bien fait, de la maison de Hesse,
parent de la reine d'Espagne, de ces cadets qui n'ont rien, qui servent
où ils peuvent pour vivre, et qui vont cherchant fortune. On prétend
qu'à un premier voyage qu'il fit en effet en Espagne, il ne déplut pas à
la reine. Le reste de ce que je vais raconter, on le prétendit aussi, je
n'en puis fournir d'autres garants, mais je l'ai ouï prétendre à des
personnages qui n'étaient ni accusés ni en place de prétendre
légèrement. On prétendit donc que le même conseil de Vienne qui, par
raison d'État, ne se fit pas scrupule d'empoisonner la reine d'Espagne,
fille de Monsieur, parce qu'elle n'avait point d'enfants, et parce
qu'elle avait trop d'ascendant sur le cœur et sur l'esprit du roi son
mari, et qui fit exécuter ce crime par la comtesse de Soissons, réfugiée
en Espagne, sous la direction du comte de Mansfeld, ambassadeur de
l'empereur à Madrid, ne fut pas plus scrupuleux sur un autre point.

Il avait remarié le roi d'Espagne à la sœur de l'impératrice. C'était
une princesse grande, majestueuse, très bien faite, qui n'était pas sans
beauté et sans esprit, et qui, conduite par les ministres de l'empereur
et par le parti qu'il s'était de longue main formé à Madrid, prit un
grand crédit sur le roi d'Espagne. C'était bien une partie principale de
ce que le conseil de l'empereur s'était proposé\,; mais le plus
important manquait, c'était des enfants. Il en avait espéré de ce second
mariage, parce qu'il s'était leurré que l'empêchement venait de la
reine, dont ce conseil s'était défait. Ne pouvant plus se dissimuler au
bout de quelques années de ce second mariage que le roi d'Espagne ne
pouvait avoir d'enfants, ce même conseil eut recours au prince de
Darmstadt, et comme l'exécution n'était pas facile, et demandait des
occasions qui ne pouvaient être amenées que par un long temps, ils
l'engagèrent à s'attacher tout à fait au service d'Espagne, et
l'empereur et ses partisans l'appuyèrent de toutes leurs forces, non
seulement pour lui faire trouver tous les avantages qui pouvaient l'y
fixer, mais tous les moyens encore de pouvoir demeurer à la cour, qui
était tout leur but\,; c'est ce qui le fit gouverneur des armes en
Catalogne après la perte de Barcelone, et la paix faite, c'est ce qui, à
la fin de cette année, le fit faire grand d'Espagne à vie, pour qu'il
pût demeurer à la cour et s'y insinuer à loisir, pour venir à bout du
dessein de faire un enfant à la reine.

Les princes étrangers effectifs, c'est-à-dire souverains ou de maison
actuellement souveraine, beaucoup moins les prétendus et les factices,
ni ces seigneurs de francs-alleux qu'ils appellent un État, n'ont aucun
rang ni aucune sorte de distinction en Espagne. Les grands de toutes
classes ne font pas même aucune sorte de comparaison avec eux\,; ils y
sont comme la noblesse ordinaire. C'est pour cela que lorsque les
princes veulent s'attacher à cette cour, et que cette cour veut
elle-même les y attacher, elle les fait grands à vie, mais de première
classe, au moyen de quoi ils peuvent être de tout, et aller partout,
parce qu'à ce titre de grands, ils ont le premier rang partout, et un
rang parmi les autres grands, qui ne peut embarrasser leur chimère s'ils
en avaient quelqu'une, parce que ce rang est égal pour tous les grands
de même classe, et les traitements aussi, et tous très distingués
partout, et tous très réglés et très établis sans dispute, et qu'entre
les grands ils affectent de marcher comme ils se trouvent, et de n'avoir
aucune ancienneté parmi eux. Je ne dirai pas si la reine fut
inaccessible de fait ou de volonté, je ne dirai pas non plus si
elle-même, comme on l'a assuré, mais je crois sans le bien savoir, avait
un empêchement de devenir mère. Quoi qu'il en soit, M. de Darmstadt,
grand d'Espagne, s'établit et se familiarisa à la cour de Madrid, fut
des mieux avec le roi et la reine, arriva à des privances fort rares en
ce pays-là\,; sans aucun fruit qui pût mettre la succession de la
monarchie en sûreté contre les différentes prétentions, ni rassurer de
ce côté le politique conseil de Vienne.

Revenons maintenant en France voir un assez petit événement, mais tout à
fait singulier. M\textsuperscript{me} de Maintenon, dans ce prodige
incroyable d'élévation où sa bassesse était si miraculeusement parvenue,
ne laissait pas d'avoir ses peines\,; son frère n'était pas une des
moindres par ses incartades continuelles. On le nommait le comte
d'Aubigné\,; il n'avait jamais été que capitaine d'infanterie, et
parlait toujours de ses vieilles guerres comme un homme qui méritait
tout, et à qui on faisait le plus grand tort du monde de ne l'avoir pas
fait maréchal de France il y a longtemps\,; d'autres fois il disait
assez plaisamment qu'il avait pris son bâton en argent. Il faisait à
M\textsuperscript{me} de Maintenon des sorties épouvantables de ce
qu'elle ne le faisait pas duc et pair, et sur tout ce qui lui passait
par la tête, et ne se trouvait avoir rien que les gouvernements de
Béfort, puis d'Aigues-Mortes, après de Cognac qu'il garda avec celui de
Berry pour lequel il rendit Aigues-Mortes, et d'être chevalier de
l'ordre. Il courait les petites filles aux Tuileries et partout, en
entretenait toujours quelques-unes, et vivait le plus ordinairement avec
elles et leurs familles et des compagnies de leur portée où il mettait
beaucoup d'argent.

C'était un panier percé, fou à enfermer, mais plaisant avec de l'esprit
et des saillies et des reparties auxquelles on ne se pouvait attendre.
Avec cela bon homme et honnête homme, poli, et sans rien de ce que la
vanité de la situation de sa sœur eût pu mêler d'impertinent, mais
d'ailleurs il l'était à merveille, et c'était un plaisir qu'on avait
souvent avec lui de l'entendre sur les temps de Scarron, et de l'hôtel
d'Albret, quelquefois sur des temps antérieurs, et surtout ne se pas
contraindre sur les aventures et les galanteries de sa sœur, en faire le
parallèle avec sa dévotion et sa situation présente, et s'émerveiller
d'une si prodigieuse fortune. Avec le divertissant, il y avait beaucoup
d'embarrassant à écouter tous ces propos qu'on n'arrêtait pas où on
voulait, et qu'il ne faisait pas entre deux ou trois amis, mais à table
devant tout le monde, sur un banc des Tuileries, et fort librement
encore dans la galerie de Versailles, où il ne se contraignait pas non
plus qu'ailleurs de prendre un ton goguenard, et de dire très
ordinairement le beau-frère, lorsqu'il voulait parler du roi. J'ai
entendu tout cela plusieurs fois, surtout chez mon père où il venait
plus souvent qu'il ne désirait, et dîner aussi, et je riais souvent sous
cape de l'embarras extrême de mon père et de ma mère, qui fort souvent
ne savaient où se mettre.

Un homme de cette humeur, si peu capable de se refuser rien, et avec un
esprit et une plaisanterie à assener d'autant mieux les choses qu'il
n'en craignait pour soi ni le ridicule ni les suites sérieuses, était un
grand fardeau pour M\textsuperscript{me} de Maintenon. Dans un autre
genre elle n'était pas mieux en belle-sœur. C'était la fille d'un nommé
Picère, petit médecin, qui s'était fait procureur du roi de la ville de
Paris, qu'Aubigné avait épousée en 1678, que sa sœur était auprès des
enfants de M\textsuperscript{me} de Montespan, qui crut lui faire une
fortune par ce mariage. C'était une créature obscure, plus, s'il se
pouvait, que sa naissance, modeste, vertueuse, et qui, avec ce mari,
avait grand besoin de l'être\,; sotte à merveille, de mine tout à fait
basse, d'aucune sorte de mise, et qui embarrassait également
M\textsuperscript{me} de Maintenon à l'avoir avec elle et à ne l'avoir
pas. Jamais elle ne put en rien faire, et elle se réduisit à ne la voir
qu'en particulier. Des gens du monde, cette femme n'en voyait point, et
demeurait dans la crasse de quelques commères de son quartier. C'étaient
des plaintes trop fondées et fréquentes à M\textsuperscript{me} de
Maintenon sur son mari, à qui cette reine, partout ailleurs si absolue,
ne pouvait jamais faire entendre raison, et qui la malmenait très
souvent elle-même.

Enfin, à bout sur un frère si extravagant, elle fit tant par
Saint-Sulpice que, comme c'était un homme tout de sauts et de bonds, et
qui avait toujours besoin d'argent, on lui persuada de quitter ses
débauches, ses indécences et ses démêlés domestiques, de vivre à son
aise, sa dépense entière payée tous les mois, et sa poche de plus
garnie, et pour cela de se retirer dans une communauté qu'un M. Doyen
avait établie sous le clocher de Saint-Sulpice pour des gentilshommes,
ou soi-disant, qui vivaient là en commun dans une espèce de retraite et
d'exercices de piété, sous la direction de quelques prêtres de
Saint-Sulpice. M\textsuperscript{me} d'Aubigné, pour avoir la paix, et
plus encore, parce que M\textsuperscript{me} de Maintenon le voulut, se
retira dans une communauté, et disait tout bas à ses commères que cela
était bien dur, et qu'elle s'en serait fort bien passée. M. d'Aubigné ne
laissa ignorer à personne que sa sœur se moquait de lui de lui faire
accroire qu'il était dévot, qu'on l'assiégeait de prêtres, et qu'on le
ferait mourir chez ce M. Doyen. Il n'y tint pas longtemps sans retourner
aux filles, aux Tuileries, et partout où il put\,; mais on le rattrapa,
et on lui donna pour gardien un des plus plats prêtres de Saint-Sulpice,
qui le suivait partout comme son ombre et qui le désolait. Quelqu'un de
meilleur aloi n'eût pas pris un si sot emploi. Mais ce Madot n'avait
rien de meilleur à faire, et n'avait pas l'esprit de s'occuper ni même
de s'ennuyer. Il remboursait force sottises, mais il était payé pour
cela, et gagnait très bien son salaire par une assiduité dont il n'y
avait peut-être que lui qui pût être capable. M. d'Aubigné n'avait
qu'une fille unique dont M\textsuperscript{me} de Maintenon avait
toujours pris soin, qui ne quittait jamais son appartement partout, et
qu'elle élevait sous ses yeux comme sa propre fille.

J'arrivai à Paris avec la plupart de ce qui avait servi en Flandre et en
Allemagne, et j'allai tout aussitôt à Versailles où la cour ne faisait
guère qu'arriver de Fontainebleau. M\textsuperscript{me} de Saint-Simon
y avait été tout le voyage fort agréablement, et le roi me reçut avec
toute sorte de bontés. Je trouvai une petite tracasserie domestique, que
je ne dédaignerai pas de mettre ici, comme l'entrée à des choses plus
considérables, dont on aime à se souvenir des échelons, et qui
expliquera aussi la cour naissante de la princesse sur qui tout le monde
avait les yeux, parce qu'elle faisait déjà beaucoup l'amusement du roi
et de M\textsuperscript{me} de Maintenon. La cour ne la voyait que deux
fois la semaine à sa toilette. Elle était donc renfermée avec ses dames,
et le roi y en joignit quelques autres pour qu'elle ne vît pas toujours
les mêmes visages, et c'était une extrême faveur pour celles qui eurent
cette privance. Les duègnes furent les duchesses de Chevreuse, de
Beauvilliers et de Roquelaure, la princesse d'Harcourt et
M\textsuperscript{me} de Soubise\,; quatre entre deux âges, dont trois
comme nièces de la duchesse du Lude, qui furent les duchesses d'Uzès, de
Sully, et M\textsuperscript{me} de Boufflers, et M\textsuperscript{me}
de Beringhen, et deux autres duègnes qui, sans être mandées, avaient
liberté d'y aller tant qu'il leur plaisait\,: c'étaient aussi des
favorites, M\textsuperscript{me}s de Montchevreuil et d'Heudicourt\,;
les autres n'y venaient que mandées. Les jeunes étaient trois femmes de
secrétaires d'État, M\textsuperscript{me}s de Maurepas, de Barbezieux et
de Torcy, et trois filles qui ne paraissaient en nul autre lieu qu'en ce
particulier et chez leurs mères, M\textsuperscript{lle} de Chevreuse,
M\textsuperscript{lle} d'Ayen et M\textsuperscript{lle} d'Aubigné. Les
vieilles étaient peu mandées, et s'excusaient souvent, et c'était plutôt
une distinction qu'une compagnie\,; les autres étaient pour l'amusement
et surtout pour les promenades. Le roi et M\textsuperscript{me} de
Maintenon n'y voulaient rien que du plus trayé dans leur goût, et le
dessein était d'accoutumer ainsi la princesse par un petit nombre de
tous âges, et de la former par la conversation et les manières des
vieilles, et de la divertir par la compagnie des jeunes. On en demeura à
ce nombre. Plusieurs essayèrent d'être admises qui furent refusées,
entre autres la duchesse de Villeroy et les deux filles de M. le Grand,
dont la duchesse du Lude fut fort mortifiée.

La figure, la modestie, le maintien de M\textsuperscript{me} de
Saint-Simon avait plu au roi à Fontainebleau\,: il s'en était expliqué
plusieurs fois\,; cela donna lieu à la comtesse de Roucy, et ensuite à
la maréchale de Rochefort, amie de M\textsuperscript{me} la maréchale de
Lorges, de proposer à la duchesse du Lude de faire initier
M\textsuperscript{me} de Saint-Simon chez la princesse. La duchesse du
Lude, qui crut que la maréchale de Lorges les en avait priées, vint chez
elle. Elle y apprit d'elle-même qu'elle n'y avait point pensé, et n'en
avait jamais parlé à ces dames\,; sur quoi la duchesse du Lude lui conta
le refus que je viens de dire, et l'assura qu'elle faisait sagement de
n'avoir point d'empressement pour cela. Il arriva que la comtesse de
Mailly, amie intime de M\textsuperscript{me} la maréchale de Lorges, et
moi ami de famille et parent des Mailly, et ami intime de l'abbé de
Mailly, son beau-frère, qui devint peu après archevêque d'Arles, avait
parlé de M\textsuperscript{me} de Saint-Simon à M\textsuperscript{me} de
Maintenon sans que personne l'en eût priée\,; que M\textsuperscript{me}
de Maintenon avait répondu que cela devrait déjà être fait\,; que
c'étaient des personnes comme M\textsuperscript{me} de Saint-Simon qu'il
fallait approcher de la princesse, et lui ordonna de le dire de sa part
à la duchesse du Lude\,; cela s'était passé la veille. La comtesse de
Mailly n'en avait voulu rien dire que la duchesse du Lude ne le sût. Le
hasard fit qu'elle la rencontra comme elle remontait de chez la
maréchale de Lorges. La duchesse du Lude demeura fort étonnée de la
chose, après les personnes de faveur qui avaient été refusées, et très
piquée de la manière, parce qu'elle ne douta pas que la maréchale de
Lorges, sûre de son fait, ne se fût moquée d'elle\,; et voilà comme les
choses très apparentes se trouvent pourtant très fausses. Dès le
lendemain M\textsuperscript{me} de Saint-Simon fut mandée, et presque
tous les jours le reste du voyage de Fontainebleau, et depuis très
souvent, avec une jalousie de toutes les autres et de leurs familles
qu'il fallut laisser tomber.

Le roi, qui de plus en plus mettait ses complaisances en la princesse
qui surpassait son âge sans mesure en art, en soins, en grâces pour les
mériter, ne voulut pas perdre un jour au delà des douze ans pour faire
célébrer son mariage, et l'avait fixé au 7 décembre qui tombait à un
samedi. Il s'était expliqué qu'il serait bien aise que la cour y fût
magnifique, et lui-même, qui depuis longtemps ne portait plus que des
habits fort simples, en voulut des plus superbes. C'en fut assez pour
qu'il ne fût plus question de consulter sa bourse ni presque son état
pour tout ce qui n'était ni ecclésiastique ni de robe. Ce fut à qui se
surpasserait en richesse et en invention. L'or et l'argent suffirent à
peine. Les boutiques des marchands se vidèrent en très peu de jours\,;
en un mot, le luxe le plus effréné domina la cour et la ville, car la
fête eut une grande foule de spectateurs. Les choses allèrent à un point
que le roi se repentit d'y avoir donné lieu, et dit qu'il ne comprenait
pas comment il y avait des maris assez fous pour se laisser ruiner par
les habits de leurs femmes\,; il pouvait ajouter, et par les leurs\,;
mais la bride était lâchée, il n'était plus temps d'y remédier, et au
fond, je ne sais si le roi en eût été fort aise, car il se plut fort
pendant les fêtes à considérer tous les habits. On vit aisément combien
cette profusion de matières et ces recherches d'industrie lui
plaisaient, avec quelle satisfaction il loua les plus superbes et les
mieux entendus, et que le petit mot lâché de politique, il n'en parla
plus, et fut ravi qu'il n'eût pas pris.

Ce n'est pas la dernière fois que la même chose lui est arrivée. Il
aimait passionnément toute sorte de somptuosités à sa cour, et surtout
aux occasions marquées, et qui s'y serait tenu à ce qu'il avait dit lui
eût très mal fait sa cour. Il n'y avait donc pas moyen d'être sage parmi
tant de folies. Il fallut plusieurs habits\,; entre
M\textsuperscript{me} de Saint-Simon et moi, il nous en coûta vingt
mille livres. Les ouvriers manquèrent pour mettre tant de richesses en
œuvre. M\textsuperscript{me} la Duchesse s'avisa d'en envoyer enlever
par des hoquetons de chez le duc de Rohan. Le roi le sut, le trouva très
mauvais, et fit sur-le-champ renvoyer ces ouvriers à l'hôtel de Rohan\,;
et il faut remarquer que le duc de Rohan était un des hommes de France
que le roi aimait le moins, et pour lequel il se contraignait le moins
de le marquer. Il fit encore une autre chose bien honnête, et tout cela
montrait bien le désir que tout le monde fût au plus magnifique\,: il
choisit lui-même un dessin de broderie pour la princesse. Le brodeur lui
dit qu'il allait quitter tous ses ouvrages pour celui-là. Le roi ne le
voulut pas. Il lui commanda bien précisément d'achever premièrement tout
ce qu'il avait entrepris, et de ne travailler à celui qu'il choisissait
qu'ensuite, et il ajouta que s'il n'était pas fait à temps, la princesse
s'en passerait.

On publia que les fêtes dureraient jusqu'à Noël, mais elles furent
restreintes à deux bals, un opéra et un feu d'artifice, et de tout
l'hiver après il n'y eut plus de bals. Le roi, pour éviter toutes
disputes et toutes difficultés, supprima toutes cérémonies\,; il régla
qu'il n'y aurait point de fiançailles dans son cabinet, mais qu'elles se
feraient tout de suite avec le mariage à la chapelle pour éviter la
queue qui ne serait point portée en cérémonie, mais par l'exempt des
gardes du corps en service auprès de la princesse, tout comme il la
portait tous les jours, et que le poêle serait tenu par l'évêque nommé
de Metz, premier aumônier en survivance de son oncle, et par l'aumônier
du roi de quartier qui se trouverait de jour, et ce fut l'abbé Morel\,;
que Mgr le duc de Bourgogne donnerait seul la main à la princesse, tant
en allant qu'en revenant de la chapelle, et que, passé M. le Prince,
aucun prince ne signerait sur le livre du curé. On jugea que ce dernier
point fut décidé en faveur des bâtards, qui, étant du festin royal par
une grâce nouvelle qui avait commencé au mariage de M. le duc de
Chartres, et point de droit, puisqu'ils n'étaient pas princes du sang,
auraient eu un dégoût de ne signer pas aussi sur le registre, à quoi
aussi les princes du sang se seraient opposés\,; ainsi M. le Duc ne
signa point. M. le prince de Conti n'était pas encore arrivé. Le roi
avait aussi précédemment réglé qu'il ne recevrait point le serment des
officiers principaux de la maison de la princesse\,; qu'ils n'en
prêteraient point jusqu'après son mariage, et qu'alors elle les
recevront\,; ce qui fut exécuté ainsi. M\textsuperscript{me} de Verneuil
fut mandée au mariage, et eut la dernière place au festin royal, comme
cela s'était fait au mariage de M. le duc de Chartres et de M. du Maine,
mais elle n'y fut que le jour du mariage, et aussitôt après, elle s'en
retourna à Paris. Aucune dame assise ne se trouva pas à un de ces
festins, non pas même la duchesse du Lude. La duchesse d'Angoulême,
veuve du bâtard de Charles IX, n'y fut point mandée, comme elle ne
l'avait point été aux mariages de M. le duc de Chartres et de M. du
Maine, parce qu'elle n'avait pas le rang de princesse du sang.

\hypertarget{chapitre-iv.}{%
\chapter{CHAPITRE IV.}\label{chapitre-iv.}}

1697

~

{\textsc{Mariage de Mgr le duc de Bourgogne.}} {\textsc{- Mariage des
deux filles du comte de Tessé.}} {\textsc{- Fortune et fin singulière du
premier Varenne.}} {\textsc{- Prince de Vaudémont et sa fortune.}}
{\textsc{- M. de Lorraine rétabli demande Mademoiselle et perd sa
mère.}} {\textsc{- Abbé, depuis cardinal de Mailly, archevêque
d'Arles.}} {\textsc{- Abbé de Castries, aumônier ordinaire de
M\textsuperscript{me} la duchesse de Bourgogne.}} {\textsc{-
M\textsuperscript{me} Cantin, première femme de chambre de
M\textsuperscript{me} la duchesse de Bourgogne.}} {\textsc{- Fortune de
Lavienne.}} {\textsc{- Mauresse, religieuse à Moret, fort énigmatique.}}

~

Le samedi matin 7 décembre, toute la cour alla de bonne heure chez Mgr
le duc de Bourgogne, qui alla ensuite chez la princesse. Sa toilette
finissait, où il y avait peu de dames, la plupart étant allées à la
tribune ou sur les échafauds placés dans la chapelle pour voir la
cérémonie. Toute la maison royale avait déjà été chez la princesse, et
attendait chez le roi où les mariés arrivèrent un peu avant midi. Ils
trouvèrent le roi dans le salon qui, un moment après, se mit en chemin
de la chapelle. La marche et tout le reste se passa comme au mariage de
M. le duc de Chartres que j'ai décrit, excepté que le cardinal de
Coislin, en l'absence du cardinal de Bouillon grand aumônier, qui était
à Rome, commença par les fiançailles, après lesquelles chacun fit à
genoux une médiocre pause pour l'intervalle entre les fiançailles et le
mariage. Le cardinal dit une messe basse, après laquelle le roi et la
maison royale retourna comme elle était venue, et se mit tout de suite à
table. La duchesse du Lude et les duchesses et princesses qui se
trouvèrent en bas eurent leurs carreaux partout, et les ducs et princes
en arrière du roi. La duchesse du Lude, M\textsuperscript{me}s de
Mailly, Dangeau et Tessé, s'approchèrent de la princesse, pendant la
célébration des fiançailles et du mariage seulement, pendant laquelle
Dangeau et Tessé soutenaient par en haut son bas de robe. Les dames du
palais ne bougèrent de leurs places. Un courrier tout prêt à la porte de
la chapelle partit pour Turin au moment que le mariage fut célébré. La
journée se passa assez ennuyeusement. Sur les sept heures du soir le roi
et la reine d'Angleterre arrivèrent, que le roi avait été convier
quelques jours auparavant. Il tint le portique, et sur les huit heures
ils vinrent dans le salon du bout de la galerie joignant l'appartement
de M\textsuperscript{me} la duchesse de Bourgogne, d'où, malgré la
pluie, ils virent tirer un feu d'artifice sur la pièce des Suisses. On
soupa ensuite comme on avait dîné, le roi et la reine d'Angleterre de
plus, la reine entre les deux rois. En sortant de table on fut coucher
la mariée, de chez laquelle le roi fit sortir absolument tous les
hommes. Toutes les dames y demeurèrent, et la reine d'Angleterre donna
la chemise que la duchesse du Lude lui présenta. Mgr le duc de Bourgogne
se déshabilla dans l'antichambre au milieu de toute la cour, assis sur
un ployant. Le roi y était avec tous les princes. Le roi d'Angleterre
donna la chemise qui lui fut présentée par le duc de Beauvilliers.

Dès que M\textsuperscript{me} la duchesse de Bourgogne fut au lit, Mgr
le duc de Bourgogne entra, et se mit dans le lit à sa droite en présence
des rois et de toute la cour, et aussitôt après le roi et la reine
d'Angleterre s'en allèrent\,; le roi s'alla coucher, et tout le monde
sortit de la chambre nuptiale, excepté Monseigneur, les dames de la
princesse, et le duc de Beauvilliers qui demeura toujours au chevet du
lit du côté de son pupille, et la duchesse du Lude de l'autre\,;
Monseigneur y demeura un quart d'heure avec eux à causer, sans quoi ils
eussent été assez empêchés de leurs personnes\,; ensuite il fit relever
M. son fils, et auparavant lui fit embrasser la princesse malgré
l'opposition de la duchesse du Lude. Il se trouva qu'elle n'avait pas
tort. Le roi le trouva mauvais, et dit qu'il ne voulait pas que son
petit-fils baisât le bout du doigt à sa femme jusqu'à ce qu'ils fussent
tout à fait ensemble. Il se rhabilla dans l'antichambre à cause du
froid, et s'alla coucher chez lui à l'ordinaire. Le petit duc de Berry,
gaillard et résolu, trouva bien mauvaise la docilité de M. son frère, et
assura qu'il serait demeuré au lit.

Le dimanche il y eut cercle chez M\textsuperscript{me} la duchesse de
Bourgogne. Le feu roi qui les avait vu tenir avec beaucoup de dignité à
la reine sa mère, et les avait vus tomber sur la fin de
M\textsuperscript{me} la Dauphine-Bavière \footnote{Marie-Anne-Christine-Victoire
  de Bavière, mariée le 28 janvier 1680 à Louis de France, Dauphin,
  morte le 20 avril 1690.}, voulut les rétablir. Ce premier fut
magnifique par le prodigieux nombre de dames assises en cercle et
d'autres debout derrière les tabourets et d'hommes derrière ces dames,
et la beauté des habits. Il commença à six heures\,; le roi y vint à la
fin, et mena toutes les dames dans le salon près de la chapelle, où
elles trouvèrent une belle collation, puis à la musique, après quoi il
tint le portique. À neuf heures il conduisit M. et M\textsuperscript{me}
la duchesse de Bourgogne chez cette princesse, et tout fut fini pour la
journée\,; elle continua à vivre comme avant d'être mariée, mais Mgr le
duc de Bourgogne alla tous les jours chez elle, où les dames eurent
ordre de ne les laisser jamais seuls, et souvent ils soupaient tête à
tête chez M\textsuperscript{me} de Maintenon. Le mercredi 11 décembre le
roi vint, sur les six heures, chez M\textsuperscript{me} la duchesse de
Bourgogne, où il y avait grosse cour. Il y attendit le roi et la reine
d'Angleterre, puis entrèrent dans la galerie pleine d'échafauds et
superbement ornée pour le bal. La tête y tourna au duc d'Aumont qui se
mêla de toutes ces fêtes, à la place du duc de Beauvilliers qui était en
année, mais qui ne les put ordonner à cause de ses fonctions auprès des
enfants de France. Ce fut donc une foule et un désordre dont le roi même
fut accablé. Monsieur fut battu et foulé dans la presse\,; on peut juger
ce que devinrent les autres. Plus de place\,; tout de force et de
nécessité\,; on se fourrait où on pouvait. Cela dépara toute la fête. Il
y eut un branle, et juste ce qu'il fallut de princes et de princesses du
sang, avec M. le comte de Toulouse, pour le mener. Voici ce qui dansa,
outre ces princes et princesses, de dames\,; d'hommes beaucoup
davantage.

Les duchesses de

Mesdames de

Sully\,;

Villequier\,;

Saint-Simon\,;

Châtillon, sa sœur\,;

Albret\,;

Tonnerre\,;

Luxembourg\,;

La Porte\,;

Villeroy\,;

Dangeau,

Lauzun\,;

La Vieuville\,;

Roquelaure\,;

Goesbriant\,;

M\textsuperscript{lle} d'Elbœuf\,;

Barbezieux\,;

M\textsuperscript{lle} d'Armagnac\,;

Montgon.

La princesse d'Espinoy.

Mesdemoiselles de

Menetou, fille de la duchesse de La Ferté\,;

Tourpes, fille de la maréchale d'Estrées\,;

Fürstemberg, nièce du cardinal de Fürstemberg\,;

Melun, sœur du prince d'Espinoy\,;

Solre-Croy, fille du comte de Solre, chevalier de l'ordre (le prince et
la princesse d'Espinoy m'avaient prié de la mener)\,;

Trois filles d'honneur de Madame\,;

Rebénac, fille du frère de M. de Feuquières, depuis
M\textsuperscript{me} de Souvré\,;

Lussan, fille de la dame d'honneur de M\textsuperscript{me} la
Princesse. M. de Lussan, son père, était chevalier de l'ordre, et
premier gentilhomme de la chambre de M. le Prince.

Sur les neuf heures, on porta sur des tables à la main une grande
collation devant la reine, les rois et tout autour du bal, et sur les
dix heures et demie on alla souper. Les princes du sang n'y furent plus
admis, il n'y eut que les princesses du sang avec la famille royale. Il
n'y eut rien jusqu'au samedi 14 décembre que fut le second bal. M.
d'Aumont y eut sa revanche. Tout y fut dans le plus grand ordre du
monde. À sept heures, le roi, le roi et la reine d'Angleterre, la
famille royale, les princes du sang, les danseurs seulement en hommes,
et toutes les dames vinrent chez M\textsuperscript{me} la duchesse de
Bourgogne, d'où ils entrèrent dans la galerie, et ce bal fut admirable
et tout entier en habits qui n'avaient pas encore paru. Le roi trouva
celui de M\textsuperscript{me} de Saint-Simon si à son gré qu'il se
tourna à M. le maréchal de Lorges en quartier de capitaine des gardes,
derrière lui, et lui donna le prix sur tous les autres. Mgr le duc de
Bourgogne se trouva libre à prendre à ce bal, après avoir rendu, ce qui
ne s'était pas trouvé à l'autre, et prit la duchesse de Sully. Il se
trouva encore libre une seconde fois, et prit M\textsuperscript{lle}
d'Armagnac. M. le prince de Conti venait d'arriver\,; il fut au bal,
mais il ne voulut pas danser. On servit, comme l'autre fois, une grande
collation, et, un peu après minuit, on alla faire \emph{media noche}, où
les princes du sang ne furent point encore, après lequel le roi et la
reine d'Angleterre s'en allèrent. M\textsuperscript{me} de Maintenon ne
parut à rien, sinon aux deux bals qu'elle vit commencer assise derrière
la reine d'Angleterre, et ne fut qu'une demi-heure à chacun. Le mardi 17
décembre, toute la cour alla sur les quatre heures à Trianon où on joua
jusqu'à l'arrivée du roi et de la reine d'Angleterre. Le roi les mena
dans une tribune où on montait sur la salle de la comédie de chez
M\textsuperscript{me} de Maintenon, qui y monta aussi avec Monseigneur
et M\textsuperscript{me} la duchesse de Bourgogne, ses dames et celles
de la reine. Monseigneur, Monsieur, Madame et tout le reste de la cour
était en bas dans la salle. L'opéra d'\emph{Issé} de Destouches, fort
beau, y fut très bien joué\,; l'opéra fini, chacun s'en retourna, et par
ce spectacle finirent toutes les fêtes du mariage.

M. de Vendôme, voyant la trêve en Catalogne et la paix assurée, avait
demandé et obtenu son congé de bonne heure, mais il n'avait fait que
saluer le roi, et s'en était allé à Anet se mettre sans façon et sans
mystère entre les mains des chirurgiens. Il en avait un pressant besoin,
mais ils le manquèrent. Sa naissance devenue si à la mode et les succès
de Catalogne lui avaient donné une audace qui ne fit depuis que croître.
Il reparut à la cour le jour du dernier bal, et fut très bien reçu du
roi, et par conséquent de toute la cour.

Tessé avait marié, l'année précédente, sa fille aînée à La Varenne,
moyennant la lieutenance générale d'Anjou, qui était dans sa famille
depuis Henri IV, qui la donna avec la Flèche à ce La Varenne si connu
dans tous les Mémoires de ces temps-là pour avoir eu l'esprit et
l'adresse de devenir une espèce de personnage, de marmiton, puis de
cuisinier, enfin de portemanteau d'Henri IV qu'il servait dans ses
plaisirs, et qu'il servit depuis dans ses affaires. Ce fut lui qui eut
la principale part au retour des jésuites en France, et à ce magnifique
établissement qu'ils ont à la Flèche dont il partagea la seigneurie avec
eux. Il s'y retira, à la mort d'Henri IV, très riche et vieux, et y
vécut fort à son aise. C'était beaucoup la mode des oiseaux en ce
temps-là, et il s'amusait fort à voler. Une pie s'étant relaissée un
jour dans un arbre, on ne pouvait l'en faire sortir à coups de pierres
et de bâton\,; le vieux La Varenne et tous les chasseurs étaient autour
de l'arbre à tâcher de l'en faire partir, lorsque la pie, importunée de
tout ce bruit, se mit à crier de toute sa force \emph{au Maquereau}, et
le répéta sans fin. La Varenne, qui devait toute sa fortune à ce métier,
se mit tout d'un coup dans la tête que, par un miracle, comme le
reproche que fit l'âne de Balaam à ce faux prophète, la pie lui
reprochait ses péchés. Il en fut si troublé qu'il ne put s'empêcher de
le montrer, puis, agité de plus en plus, de le dire à la compagnie\,;
elle en rit d'abord, mais, voyant ce bonhomme changer beaucoup, puis se
trouver mal, on tâcha de lui faire entendre que cette pie avait
apparemment appris, à parler dans quelque village voisin et à dire cette
sottise, et qu'elle s'était échappée, et s'était trouvée là. Il n'y
avait en effet pas autre chose à en croire, mais La Varenne ne put
jamais en être persuadé. Il fallut du pied de l'arbre le ramener chez
lui\,; il y arriva avec la fièvre et toujours frappé de cette folle
persuasion\,; rien ne put le remettre, et il mourut en très peu de
jours. C'est l'aïeul paternel de tous ces La Varenne. Tessé avait une
autre fille fort jolie dont il fit le mariage avec Maulevrier, qui avait
quitté le petit collet lorsque son frère fut tué dans Namur\,; il était
fils de Maulevrier frère de M. Colbert et chevalier de l'ordre, qui
mourut de douleur de n'avoir pas été maréchal de France, comme je l'ai
raconté. Celui-ci avait le régiment de Navarre, et me donnera lieu de
parler de lui.

En même temps presque que le prince de Darmstadt s'établit en Espagne,
comme je l'ai expliqué, il fut fait vice-roi de Catalogne par l'intrigue
des serviteurs de l'empereur et l'appui de la reine d'Espagne, et par
les mêmes chemins le prince de Vaudemont fut fait gouverneur général du
Milanais. C'est un personnage sur lequel il faut s'arrêter, et dont je
parlerai plus d'une fois dans les suites. Il était bâtard de Charles IV,
duc de Lorraine, gendre du duc d'Elbœuf et beau-frère du comte de
Lislebonne, frère du même duc d'Elbœuf. Il l'était aussi du duc de La
Rochefoucauld. Détaillons tout ceci.

On connaît encore trop la vie et les diverses fortunes de Charles IV,
duc de Lorraine, pour parler de son génie et des extrémités où il le
jeta. Ami de tous les partis, fidèle à aucun, souvent dépouillé de ses
États, et tantôt les abdiquant, puis les reprenant, tantôt en France
avec les rebelles, puis à la cour, tantôt à la tête de ses troupes sans
feu ni lieu qu'il faisait subsister aux dépens d'autrui, et y vivant
lui-même, d'autres fois au service de la France, puis de l'empereur,
après de l'Espagne, souvent à Bruxelles, enfin enlevé et conduit
prisonnier en Espagne\,; toujours marié, et jamais avec la duchesse
Nicole, héritière de Lorraine, sa cousine germaine, fille aînée d'Henri,
duc de Lorraine, frère aîné de son père, qu'il avait épousée en 1621,
dont il n'eut point d'enfants et qu'il perdit en janvier 1657, ni avec
Marie, fille unique de Charles, comte d'Apremont, qu'il épousa en 1665,
et dont il n'eut point d'enfants encore, et qu'il laissa veuve en
septembre 1675 qu'il mourut.

Charles IV était frère aîné du prince François qui fut cardinal, et qui,
voyant le duc son frère sans enfants, quitta le chapeau pour épouser
Claude-Françoise, seconde et dernière fille du duc Henri de Lorraine,
frère aîné de son père, en sorte que les deux frères épousèrent les deux
sœurs pour conserver par elles le duché de Lorraine qui, à défaut de
Nicole, l'aîné sans enfants, tombait à sa sœur Claude-Françoise, épouse
du prince François. Ces princes étaient frères de la seconde femme de
Gaston, duc d'Orléans, dont Louis XIII ne voulut jamais reconnaître le
mariage clandestin, laquelle fut mère de M\textsuperscript{me} la
grande-duchesse, mère du dernier grand-duc de Toscane et de
M\textsuperscript{me} de Guise, mortes de nos jours.

Du mariage du prince François, qui avait été cardinal, vint ce grand
capitaine qui n'a jamais joui du duché de Lorraine, qui épousa la reine
douairière de Pologne, sœur de l'empereur, et qui acquit tant de
réputation à la tête des armées de l'empereur et de l'empire. Il laissa
un fils qui fut rétabli par la paix de Ryswick, à qui nous allons voir
faire hommage au roi du duché de Bar et épouser Mademoiselle. Cette
généalogie expliquée, rapprochons-nous de ce qui m'y a fait écarter.

Charles IV, marié depuis longtemps à la duchesse Nicole, était à
Bruxelles, amoureux de M\textsuperscript{me} de Cantecroix. Il aposta un
courrier qui lui apporta la nouvelle de la mort de la duchesse Nicole.
Il en donna part dans Bruxelles, prit le grand deuil, et quatorze jours
après épousa Béatrix de Cusance, veuve du comte de Cantecroix, dans
Besançon aux Minimes, arrivant de Bruxelles, en avril 1637, et en donna
aussi part à toute la ville. Bientôt après la fourbe fut découverte, et
on apprit de tous côtés que la duchesse Nicole était pleine de vie et de
santé et n'avait seulement pas été malade. M\textsuperscript{me} de
Cantecroix, qui n'en avait pas été la dupe, fit tout comme si elle l'eût
été, mais elle était grosse\,; elle s'apaisa. Ils continuèrent de
réputer la duchesse Nicole pour morte, et de vivre ensemble à la face du
monde comme étant effectivement mariés, sans qu'il eût jamais été
question de dissoudre le mariage de la duchesse Nicole, ni devant ni
après, laquelle se réfugia à Paris. Le duc Charles eut donc de ce beau
mariage prétendu par lui tout seul une fille d'abord, puis un fils,
parfaitement bâtards l'un et l'autre, et universellement regardés comme
tels. Ces deux enfants tinrent tout de leur père. Il maria la fille en
octobre 1660 au comte de Lislebonne, frère puîné du duc d'Elbœuf, dont
elle n'a eu que quatre enfants qui aient vécu. Le prince de Commercy,
qui servit toujours l'empereur et le prince Paul, tué à Neerwinden, dont
j'eus le régiment, comme je l'ai dit en son temps, tous deux point
mariés, et deux filles, M\textsuperscript{lle} de Lislebonne qui ne l'a
point été non plus, et M\textsuperscript{lle} de Commercy qui épousa en
1691 le prince d'Espinoy, qui sont deux personnes dont j'aurai souvent
occasion de parler.

Le fils est M. de Vaudemont dont il s'agit. Charles IV l'éleva auprès de
lui, et, comme il le prétendait toujours légitime, il le fit appeler le
prince de Vaudemont, et le nom lui en est demeuré. La sœur et le frère
sont pourtant nés du vivant de la duchesse Nicole, qui mourut à Paris
longtemps après la naissance de l'une et de l'autre, en février 1657. M.
de Vaudemont fut un des hommes des mieux faits de son temps. Un beau
visage et grande mine, des yeux beaux et fort vifs, pleins de feu et
d'esprit, aussi en avait-il infiniment, soutenu d'autant de fourbe,
d'intrigue et de manège qu'en avait son père. Il le suivit partout dès
sa jeunesse, dans toutes ses guerres, et en apprit bien le métier. Il le
suivit aussi à Paris, où sa galanterie fit du bruit à la cour. Il y lia
amitié avec le marquis, depuis maréchal de Villeroy, et avec plusieurs
seigneurs distingués et qui approchaient plus du roi, surtout avec ceux
de la maison de Lorraine dont il captait fort la bienveillance. Son père
le maria à Bar, en avril 1669, à une fille du duc d'Elbœuf, frère aîné
de M. de Lislebonne et de sa première femme qui était Lannoy, et mère en
premières noces de la femme du duc de La Rochefoucauld qui toute sa vie
fut si bien avec le roi.

La liaison du duc Charles avec les Espagnols, et ses séjours en
Franche-Comté qui lors était à eux, et à Bruxelles, attacha M. de
Vaudemont à leur service, et la catastrophe de son père ne put l'en
séparer parce qu'il y espéra des emplois dont il ne pouvait se flatter
ailleurs. Dix ans de guerre contre l'Espagne donnèrent occasion au
prince de Vaudemont d'employer tous ses talents pour s'avancer, et il
les employa utilement. La nouvelle liaison d'intérêt de l'Espagne avec
la Hollande et le voisinage des Pays-Bas y forma des liaisons dont
Vaudemont sut profiter. Il sut s'insinuer auprès du prince d'Orange, et
peu à peu devint de ses amis jusqu'à être admis dans sa confidence. Il
fit un voyage en Espagne chargé de diverses commissions secrètes. Il
trouva cette cour dans le désespoir de ses pertes, fort animée contre la
personne du roi. Le sang quoique illégitime qui coulait dans ses veines
ni la liaison intime en laquelle il était parvenu auprès du prince
d'Orange ne lui avait pas appris à l'aimer. Il n'avait rien à en
attendre\,: il se lâcha donc en courtisan à Madrid contre la personne du
roi avec une hardiesse égale à l'indécence. Retournant en Flandre il
voulut voir l'Italie, et il s'arrêta à Rome, où il s'insinua tant qu'il
put parmi la faction espagnole, et pour lui plaire en usa sur le roi
comme il avait fait à Madrid. Ce qui avait été méprisé et tenu pour
ignoré d'abord ne put plus l'être sur un théâtre tel que Rome, qui est
la patrie commune de toutes les nations catholiques. Les serviteurs du
roi s'offensèrent d'une insolence si publique et si soutenue, et en
écrivirent, de façon que le roi fit prier le roi d'Espagne de mettre
ordre à une conduite si éloignée du respect qui en tout temps est dû aux
têtes couronnées, ou de n'être pas surpris s'il faisait traiter et
chasser de Rome M. de Vaudemont comme il le méritait. Cette démarche
finit la scène que M. de Vaudemont donnait avec tant de licence, et les
mêmes partisans d'Autriche qui l'y soutenaient furent les plus ardents à
le faire disparaître. Il regagna donc les Pays-Bas par le Tyrol et
l'Allemagne, avec ce nouveau mérite envers l'Espagne et l'empereur,
auquel le prince d'Orange ne fut pas le moins sensible, par cette haine
personnelle du roi qu'il ne pouvait émousser, ni M. de Lorraine
indifférent par la situation où le roi continuait à le tenir, bien qu'il
ne se soit jamais échappé en la moindre chose à l'égard du roi. Il se
faisait honneur, au contraire, de lui porter un profond respect, et de
supporter avec silence et toujours avec sagesse l'état auquel sa
puissance l'avait réduit\,; mais au fond de l'âme, les héros se sentent
de l'humanité, et il ne voulut rien moins que du mal à M. de Vaudemont
de cette conduite, quoique lui-même fût bien éloigné de la tenir.
Vaudemont était son cousin germain bâtard, et M. de Lorraine était lors
dans l'apogée de sa gloire et de son autorité dans le conseil et dans la
cour de l'empereur.

Tout concourut donc après ce départ précipité de Rome à faire marcher M.
de Vaudemont à pas de géant. La Toison d'or, grand d'Espagne, prince de
l'empire, capitaine général, tout lui fondit rapidement sur la tête, et
bientôt après le grand emploi de mestre de camp général\footnote{Cette
  dignité répondait à celle de colonel général de la cavalerie.}, et
enfin de gouverneur des armes aux Pays-Bas. Élevé de la sorte et payé à
proportion, il vécut avec splendeur, et comme il avait infiniment
d'esprit et d'adresse, il vint à bout d'émousser l'envie, et de se faire
presque autant aimer que considérer par son crédit et respecter par ses
emplois. C'était un homme affable, prévenant, obligeant, attentif à
plaire et à servir, et qui ambitionnait l'amour du bourgeois et de
l'artisan à proportion autant que des personnes les plus distinguées.
L'oisiveté de la paix lui fit recourir les bonnes fortunes, où il ne fut
pas heureux. Il le fut encore moins en habiles gens, qui pensèrent le
tuer dans le grand remède. Je lui ai ouï conter, non pas cela, mais
qu'étant tombé dans l'état où en effet ce remède l'avait mis, qu'il
disait être un rhumatisme goutteux universel qui le tint des années
entières sans aucun usage de ses bras ni de ses jambes, un empirique, à
qui à bout de remèdes il se livra, l'avait rétabli comme il était, et
mis en état de monter à cheval. Il marchait peu et difficilement,
s'asseyait et se levait avec peine, mais pourtant sans être
nécessairement aidé en toutes ses actions, n'avait plus d'os aux doigts
des mains qui étaient comme entortillés les uns sur les autres. Avec
cela une très bonne santé, la tête parfaite, nul véritable régime de
nécessité ni pour le manger ni pour veiller, la taille comme il l'avait
toujours eue, c'est-à-dire la plus belle du monde et fort haute, les
jambes seulement tout d'une venue, et le plus grand air et la plus
grande mine du monde, douce, majestueuse, spirituelle au dernier point.
Je me suis étendu sur ces bagatelles pour des raisons qui se verront
dans la suite.

La guerre de 1688 arrivée, le prince, qui voulait être maître des
troupes d'Espagne, mit tout son crédit à élever son ami au commandement
des armées. Des emplois qu'il avait jusque-là, il n'y avait plus qu'un
pas à faire. Le prince de Waldeck qui les commandait était vieux, on lit
en sorte qu'il se retirât, et que M. de Vaudemont fût mis en sa place
sous l'électeur de Bavière, et en chef en son absence. La paix
s'avançant, le prince d'Orange se fit une véritable affaire de procurer
le gouvernement du Milanais à Vaudemont. Il y fit entrer l'empereur qui
mit en mouvement tous ses serviteurs en Espagne et à la reine, et M. de
Vaudemont se trouva placé dans le plus grand et le plus brillant emploi
de la monarchie d'Espagne par la protection du nouveau roi d'Angleterre
et de l'empereur. Je le répète, tout ce détail est important à retenir
pour ce qui se trouvera dans les suites.

Par la paix de Ryswick, M. de Lorraine fut rétabli avec les mêmes
conditions que son père n'avait pas voulu admettre, et qui l'empêchèrent
toute sa vie d'y rentrer, et en même temps son mariage fut arrêté avec
Mademoiselle\,; sur quoi quelqu'un dit assez plaisamment de la feue
reine d'Espagne, de M\textsuperscript{me} de Savoie et de celle-ci, que,
de ses trois filles, Monsieur en avait marié une à la cour, une autre à
la ville, et la dernière à la campagne. Couronges, qui avait été
gouverneur de M. de Lorraine, qui était le principal de son conseil et
grand maître de sa maison, vint tout à la fin de cette année en faire la
demande, premièrement au roi, puis à Monsieur. La duchesse de Lorraine
sa mère venait de mourir. Elle était reine douairière de Pologne en
premières noces sans enfants, et sœur de l'empereur\,; on l'appelait la
reine-duchesse.

L'année finit par la nomination des bénéfices\,: l'abbé de Mailly,
aumônier du roi, et qui était fort de mes amis, eut l'archevêché
d'Arles. Sa mère l'avait fait prêtre à coups de bâton, et l'avait laissé
mourir de faim longues années à Saint-Victor. Elle en avait fait autant
à un autre de ses fils, qui, plus docile, s'était fait religieux de
Saint-Victor. C'était un homme de bien, à qui le mariage de son frère
avec la nièce de M\textsuperscript{me} de Maintenon valut l'évêché de
Lavaur. Ce même mariage fit enfin mon ami archevêque d'Arles, qui
n'avait de sa vie eu d'autre vocation que celle de sa mère, qui ne
s'était pas contraint pour l'étude, et d'ailleurs ce qu'il avait fallu
pour ne se pas perdre. Arles lui plut fort par le voisinage de Rome. Le
cardinalat est une maladie bien commune, et qui prend les gens de bonne
heure.

Le roi acheva enfin de nommer la maison de M\textsuperscript{me} la
duchesse de Bourgogne, et l'abbé de Castries, neveu du cardinal Bonzi et
beau-frère de la dame d'atours de M\textsuperscript{me} la duchesse de
Chartres, obtint la charge d'aumônier ordinaire. C'était un homme
extrêmement aimable dans la société, que le roi s'était capricié de ne
point faire évêque, dont aussi il n'avait pas trop pris le chemin. Il
était fort honnête homme, et avait beaucoup d'amis. Intimement lié avec
son frère et sa belle-sœur, et logeant avec eux, il voulut ne les point
quitter, demeurer honnêtement à la cour, et avoir un logement.

Cela me fait souvenir que j'ai oublié une bagatelle qui ne l'est rien
moins chez ces princesses. C'est de parler de la première femme de
chambre de M\textsuperscript{me} la duchesse de Bourgogne. Le roi
choisit M\textsuperscript{me} Cantin, bien faite, polie, fort à sa
place, douce, obligeante, et sachant fort le monde. Elle était femme de
Cantin et belle-sœur de Lavienne. Ce Lavienne, qui avait fait plus d'un
métier, était devenu baigneur, et si à la mode, que le roi, du temps de
ses amours, s'allait baigner et parfumer chez lui\,; car jamais homme
n'aima tant les odeurs, et ne les craignit tant après, à force d'en
avoir abusé. On prétendait que le roi, qui n'avait pas de quoi fournir à
tout ce qu'il désirait, avait trouvé chez Lavienne des confortatifs qui
l'avaient rendu plus content de lui-même, et que cela, joint à la
protection de M\textsuperscript{me} de Montespan, le fit enfin premier
valet de chambre. Il conserva toute sa vie la confiance du roi. On en a
vu un trait sur l'aventure de M. du Maine en Flandre, et de la gazette
de Hollande. Lavienne, qui avait passé sa vie avec les plus grands
seigneurs, n'avait jamais pu apprendre le moins du monde à vivre.
C'était un gros homme, noir, frais, de bonne mine, qui gardait encore sa
moustache comme le vieux Villars, rustre, très volontiers brutal, pair
et compagnon avec tout le monde, et ce qui est plaisant, parce qu'il
n'en savait pas davantage (car il n'était point glorieux, et n'avait
d'impertinent que l'écorce), honnête homme, ni méchant ni malfaisant,
même bon homme et serviable. Il avait poussé son frère Cantin qu'il
avait fait barbier du roi, puis premier valet de garde-robe. Celui-ci
était un bon homme qui se tenait obscurément dans son état, et qu'on ne
voyait jamais qu'en fonction auprès du roi.

À propos de confiance du roi et de ses domestiques intimes, il faut
réparer un autre oubli. On fut étonné à Fontainebleau cette année qu'à
peine la princesse (car elle ne fut mariée qu'au retour) y fut
arrivée\,; que M\textsuperscript{me} de Maintenon la fit aller à un
petit couvent borgne de Moret où le lieu ne pouvait l'amuser, ni aucune
des religieuses dont il n'y en avait pas une de connue. Elle y retourna
plusieurs fois pendant le voyage, et cela réveilla la curiosité et les
bruits. M\textsuperscript{me} de Maintenon y allait souvent de
Fontainebleau, et à la fin on s'y était accoutumé. Dans ce couvent était
professe une Mauresse inconnue à tout le monde, et qu'on ne montrait à
personne. Bontems, premier valet de chambre et gouverneur de Versailles,
dont j'ai parlé, par qui les choses du secret domestique du roi
passaient de tout temps, l'y avait mise toute jeune, avait payé une dot
qui ne se disait point, et de plus continuait une grosse pension tous
les ans. Il prenait exactement soin qu'elle eût son nécessaire, et tout
ce qui peut passer pour abondance à une religieuse, et que tout ce
qu'elle pouvait désirer de toute espèce de douceurs lui fût fourni. La
feue reine y allait souvent de Fontainebleau, et prenait grand soin du
bien-être du couvent, et M\textsuperscript{me} de Maintenon après elle.
Ni l'une ni l'autre ne prenaient pas un soin direct de cette Mauresse
qui pût se remarquer, mais elles n'y étaient pas moins attentives. Elles
ne la voyaient pas toutes les fois qu'elles y allaient, mais souvent
pourtant, et avec une grande attention à sa santé, à sa conduite et à
celle de la supérieure à son égard. Monseigneur y a été quelquefois, et
les princes ses enfants une ou deux fois, et tous ont demandé et vu la
Mauresse avec bonté. Elle était là avec plus de considération que la
personne la plus connue et la plus distinguée, et se prévalait fort des
soins qu'on prenait d'elle et du mystère qu'on en faisait, et
quoiqu'elle vécût régulièrement, on s'apercevait bien que la vocation
avait été aidée. Il lui échappa une fois, entendant Monseigneur chasser
dans la forêt, de dire négligemment\,: «\,C'est mon frère qui chasse.\,»
On prétendait qu'elle était fille du roi et de la reine, que sa couleur
l'avait fait cacher et disparaître et publier que la reine avait fait
une fausse couche, et beaucoup de gens de la cour en étaient persuadés.
Quoi qu'il en soit, la chose est demeurée une énigme.

\hypertarget{chapitre-v.}{%
\chapter{CHAPITRE V.}\label{chapitre-v.}}

1698

~

{\textsc{Année 1698. Éclat et accommodement de l'archevêque de Reims et
des jésuites.}} {\textsc{- Deux lourdes sottises de Sainctot,
introducteur des ambassadeurs.}} {\textsc{- Mensonge d'une tapisserie du
roi, etc., réformé.}} {\textsc{- Dispute de rang entre
M\textsuperscript{me}s d'Elbœuf et de Lislebonne.}} {\textsc{- Mort du
P. de Chevigny.}} {\textsc{- Mort de la duchesse de Berwick.}}
{\textsc{- Mariage de M. de Lévi et de M\textsuperscript{lle} de
Chevreuse.}} {\textsc{- Mariage du comte d'Estrées et d'une fille du duc
de Noailles, faite dame du palais, avec la marquise de Lévi.}}
{\textsc{- Mariage de Mortagne et de M\textsuperscript{me} de Quintin.}}
{\textsc{- Bissy, évêque de Toul, depuis cardinal, refuse l'archevêché
de Bordeaux.}} {\textsc{- Vaïni, chevalier de l'ordre.}} {\textsc{-
Chevaliers du Saint-Esprit romain en 1675.}} {\textsc{- L'ordre renvoyé
en 1688 par le duc de Bracciano.}} {\textsc{- Électeur de Saxe
pleinement roi de Pologne.}} {\textsc{- Mort de M. d'Hanovre.}}
{\textsc{- Obrecht va à Ratisbonne pour les affaires de Madame avec
l'électeur palatin.}}

~

L'année commença par l'accommodement que le premier président fit par
ordre du roi des jésuites avec l'archevêque de Reims. Ce prélat, à
l'occasion d'une ordonnance qu'il avait faite sur la fin de l'année
dernière dans son diocèse, s'y était exprimé sur la doctrine et sur la
morale d'une manière qui déplut aux jésuites. Ils essayèrent de faire en
sorte que l'archevêque s'expliquât d'une manière publique qui les mit
hors d'intérêt. C'est ce qu'il ne voulut point faire, tellement que ces
pères, peu accoutumés à trouver de la résistance nulle part, et à
dominer les prélats les plus considérables, tout au moins à en être
ménagés avec beaucoup de circonspection, éclatèrent contre celui-ci par
un écrit qui ne le ménageait pas, mais qui, à tout hasard, les laissait
libres, parce qu'il parut sans nom d'auteur. L'archevêque en porta ses
plaintes au roi avec tant de menaces, que l'écrit fut supprimé autant
qu'il le put être, et l'imprimeur sévèrement châtié. Cela ne contenta
pas l'archevêque\,; ses menaces continuèrent. Les jésuites, déjà
mortifiés de ce qui venait d'arriver, se servirent de la porte de
derrière qu'ils s'étaient ménagée, et protestèrent qu'ils ignoraient
l'auteur de l'écrit. Avec une humiliation pour eux si nouvelle, ils
espérèrent tout de leur crédit auprès du roi, et que l'archevêque à son
tour se trouverait heureux de leur désaveu\,; mais il se trouva qu'ils
avaient affaire à un homme qui ne les aimait, ni ne les craignait, ni ne
les ménageait\,; qui dans le fond avait raison\,; que son siège, ses
richesses, son neveu, et sa doctrine rendaient considérables\,; qui
était personnellement fort bien et dans la familiarité du roi\,; qui
était soutenu par MM. de Paris, de Meaux, et même par M. de Chartres,
les prélats alors les plus en faveur, et avec qui il s'était comme
enrôlé contre M. de Cambrai. Les jésuites ne purent donc rien obtenir,
sinon que le roi parlerait à M. de Reims pour qu'il ne les poussât point
à bout par des écrits, et une interdiction dans son diocèse, mais qu'il
voulait qu'il fût content, et qu'il chargerait le premier président de
cette affaire.

Elle fut bientôt finie. L'archevêque n'osa pousser les choses à bout, et
voulut faire sa cour, et les jésuites, au désespoir de s'être embourbés
avec trop de confiance, ne cherchaient qu'à sortir de ce mauvais pas.
Cela finit donc, de l'avis du premier président, par une visite à
l'archevêque du provincial et des trois supérieurs des trois maisons de
Paris, qui sans lui parler plus de son ordonnance, ne lui demandèrent
autre chose que de vouloir être persuadé de la sincérité de leurs
respects, et de la protestation qu'ils lui faisaient qu'aucun des leurs
n'était capable d'avoir fait l'écrit dont il avait lieu de se plaindre,
qu'il avait paru sans qu'ils en eussent eu la moindre connaissance, et
qu'ils l'improuvaient de tout leur cœur, en le suppliant de les honorer
du retour de sa bienveillance. L'archevêque les reçut et leur répondit
assez cavalièrement. Ils ne s'en aimèrent pas mieux, mais de part et
d'autre, ils n'osèrent plus s'escarmoucher.

Sainctot, introducteur des ambassadeurs, fit faire une sottise à la
duchesse du Lude, qui pensa devenir embarrassante. Ferreiro, chevalier
de l'Annonciade, et ambassadeur de Savoie, allant à une audience de
cérémonie chez M\textsuperscript{me} la duchesse de Bourgogne, Sainctot
dit à la duchesse du Lude qu'elle devait aller le recevoir dans
l'antichambre avec toutes les dames du palais. Celles-ci, jalouses de
n'être point sous la charge de la dame d'honneur, ne l'y voulurent point
accompagner\,; la duchesse du Lude allégua qu'elle ne se souvenait point
d'avoir vu les autres dames d'honneur de la reine, ni de
M\textsuperscript{me} la Dauphine, aller recevoir les ambassadeurs.
Sainctot lui maintint que cela se devait, et l'entraîna à le faire. Le
roi le trouva mauvais, et lava la tête le jour même à Sainctot\,; mais
l'embarras fut qu'aucun autre ambassadeur ne voulut prendre cette même
audience sans recevoir le même honneur. On eut toutes les peines du
monde à leur faire entendre raison sur une nouveauté faite par une
ignorance qui ne pouvait tourner en usage et en règle, et ce ne fut
qu'après une longue négociation et des courriers dépêchés à leurs
maîtres et revenus plus d'une fois qu'ils se contentèrent chacun d'un
écrit signé de Torcy, portant attestation que cela ne s'était jamais
pratiqué pour aucun ambassadeur, que ce qui s'était passé à l'égard de
Ferreiro était une ignorance, et que cette faute ne se commettrait plus.
Avec cet écrit, ils prirent leur audience, la duchesse du Lude ne
bougeant de sa place, auprès et en arrière de M\textsuperscript{me} la
duchesse de Bourgogne.

À quelque temps de là, le même Sainctot en fit bien une autre.
Heemskerke, ambassadeur de Hollande, avait amené sa femme et sa fille.
Sa femme eut son audience publique de M\textsuperscript{me} la duchesse
de Bourgogne, assise au milieu du cercle, à la droite de la duchesse du
Lude, chacune sur leur tabouret, comme c'est l'usage. En arrivant, reçue
en dedans de la porte par la dame d'honneur, elle la mena par la main à
M\textsuperscript{me} la duchesse de Bourgogne, à qui elle baisa le bas
de la robe, et dont tout de suite elle fut baisée, comme cela est de
droit pour toutes les femmes titrées. En même temps, elle présenta sa
fille, qui l'avait suivie avec Sainctot, dont c'est la charge. La fille
baisa le bas de la robe, et tout aussitôt se présenta pour être baisée.
M\textsuperscript{me} la duchesse de Bourgogne étonnée hésite, la
duchesse du Lude fait signe de la tête que non\,; Sainctot n'en fait pas
à deux fois, et hardiment pousse la fille de la main, et dit à
M\textsuperscript{me} la duchesse de Bourgogne\,: «\,Baisez, Madame,
cela est dû.\,» À cela (et le tout fut fait en un tour de main),
M\textsuperscript{me} la duchesse de Bourgogne, jeune, toute neuve,
embarrassée de faire un affront, eut plus tôt fait de déférer à
Sainctot, et sur sa périlleuse parole la baisa. Tout le cercle en
murmura tout haut, et femmes assises, et dames debout, et courtisans. Le
roi, qui survient toujours à ces sortes d'audiences pour faire l'honneur
à l'ambassadrice de la saluer et ne la recevoir point chez lui, n'en sut
rien dans cette foule. Au partir de là, l'ambassadrice alla chez Madame.
Même cérémonie et même entreprise pour la fille. Madame, qui en avait
reçu tant et plus en sa vie, voyant la fille approcher son minois, se
recula très brusquement. Sainctot lui dit que M\textsuperscript{me} la
duchesse de Bourgogne lui venait de faire l'honneur de la baiser.
«\,Tant pis\,! répondit Madame fort haut, c'est une sottise que vous lui
avez fait faire, que je ne suivrai pas.\,» Cela fit grand bruit\,; le
roi ne tarda pas à le savoir. Sur-le-champ, il envoya chercher Sainctot,
et lui dit qu'il ne savait qui le tenait de ne le pas chasser et lui
ôter sa charge\,; et de là lui lava la tête d'une manière plus fâcheuse
qu'il ne lui était ordinaire quand il réprimandait. De ceci, les
ambassadeurs ne s'en émurent point\,: leur caractère qui se communique à
leurs femmes, parce que mari et femme ne sont qu'un, ne va pas jusqu'à
leurs enfants, et ils ne prétendirent rien là-dessus. Ce Sainctot était
un homme qui faisait ce qu'il voulait, et favorisait qui il lui
plaisait, au hasard d'être grondé si le cas y échéait\,; ce qui
n'arrivait guère par l'ignorance et le peu de cas qu'il s'introduisit de
faire des cérémonies.

Cela me fait souvenir d'une friponnerie insigne qu'il fit étant maître
des cérémonies, charge qu'il vendit pour acheter celle d'introducteur
des ambassadeurs, et que je découvris par le plus grand hasard du monde.
Je ne ferai point ici une digression de la célèbre affaire des Corses à
Rome et du duc de Créqui, ambassadeur de France, et du traité de Pise
qui la termina en 1664, qui sont choses connues de tout le monde. Par ce
traité, entre autres articles, il fut réglé que la satisfaction convenue
et mise par écrit serait faite au roi, et lue par le cardinal Chigi,
neveu du pape et envoyé exprès légat \emph{a latere}, en présence des
grands du royaume. L'audience s'allant donner dans peu de jours, le roi
envoya le grand maître des cérémonies avertir de sa part tous les ducs
de s'y trouver. Les ducs demandèrent d'y être couverts. La reine mère,
qui de tout temps favorisait les princes étrangers, par amitié pour la
comtesse d'Harcourt et la duchesse d'Épernon sa sœur qui en avaient le
rang, et qui de tout temps avaient été ses favorites, crut faire
beaucoup pour eux que faire décider que personne en cette audience ne
serait couvert que le légat seul. Cela ne faisait rien à Monsieur ni aux
princes du sang, qui ne s'y trouvèrent pas, parce que le légat eut un
fauteuil, dans lequel il fit sa lecture et son compliment, et que
Monsieur même n'aurait pu avoir un tabouret. Les comtes de Soissons et
d'Harcourt, nommés pour mener le légat à l'audience, demandèrent à en
être excusés puisqu'ils ne se couvriraient point. Ils furent refusés,
ils le menèrent, demeurèrent tête nue à toute l'audience, et le
ramenèrent. Ces faits n'ont jamais été contestés par les princes ni par
personne.

Étant allé un matin faire ma cour au roi à Meudon, où il était libre aux
courtisans d'aller, le hasard fit qu'après le lever du roi, j'allai
m'asseoir dans une pièce par où le roi allait passer pour aller à la
messe, qu'on appelait la chambre de Madame. Justement la tapisserie qui
fut faite de cette audience avec les visages au naturel était tendue
dans cette chambre. Je remarquai que les deux comtes de Soissons et
d'Harcourt y étaient représentés couverts. Je me récriai sur cette
faute. Chamlay, assis auprès de moi, répondit que MM. de Savoie et de
Lorraine étaient couverts aux audiences. J'en convins, mais je lui
appris la différence de celle-ci. Je sentis ou la ruse des princes de
s'être dédommagés pour l'avenir par une tapisserie subsistante, ou la
sottise de ceux qui l'avaient faite. J'en parlai aux ducs de Chaulnes,
encore alors en pleine santé, de Chevreuse, de Coislin, qui avaient été
à cette audience, et à d'autres encore. M. de Luxembourg, qui vivait et
qui s'y était trouvé, et qui avec MM. de Chaumes et de Coislin s'était
le plus remué lors de cette audience, entra dans cette méprise. Ils
parlèrent à Sainctot qui était lors maître des cérémonies. Il convint
tout d'abord qu'il était vrai que les deux comtes étaient demeurés
découverts, et à toute l'audience, et que le légat seul y fut couvert.
Ces messieurs lui proposèrent de faire une note sur son registre du
mensonge de la tapisserie. Il renifla, et fit ce qu'il put pour leur
persuader que cela n'était pas nécessaire, et on va voir pourquoi\,;
mais comme il vit qu'ils s'échauffaient, et qu'ils parlaient de le
demander au roi, il n'osa plus résister. Ils allèrent donc avec lui chez
Desgranges, maître des cérémonies. Il montra le registre, mais il se
trouva qu'il ne portait pas un mot qu'il y eût quelqu'un de couvert ou
non, d'où il résultait que les deux comtes l'avaient été, puisque,
l'étant toujours, la différence de ne l'être pas cette fois-là valait
bien la peine d'être exprimée. Ces messieurs ne purent s'empêcher de
montrer à Sainctot qu'ils sentaient vivement son infidélité\,; lui aux
excuses de la négligence et bien honteux. Il écrivit à la marge tout ce
que ces messieurs lui dictèrent sur la tapisserie, et le signa\,; mais
cela fit que ces messieurs ne s'en contentèrent pas, et qu'ils se firent
donner chacun un certificat par Sainctot, et de la vérité du fait, et du
mensonge de la tapisserie, et du silence du registre, et de ce qui y
avait été mis en marge. Il les fournit dès le lendemain avec force
compliments, et se tint heureux qu'on n'en fît pas plus de bruit. Et
voilà comment les rangs sont entre les mains de gens de peu qui s'en
croient les maîtres, et qui se croient en droit de faire plaisir à qui
il leur plaît aux dépens de vérité et de justice. C'est une contagion
qui a passé depuis aux grands maîtres et aux maîtres des cérémonies, et
même à ceux du Saint-Esprit. Blainville, beau-frère de M. de Chevreuse,
qui n'était pas duc en 1664, mais qui était à la cour, et fils du duc de
Luynes, qui agit lors avec les autres, était grand maître des
cérémonies, charge qu'il avait eue de M. de Rhodes\,; ainsi il ne fut
question que du registre de Sainctot.

La révérence en mante, que les dames de Lorraine vinrent faire au roi
sur la mort de la reine-duchesse, mère de M. de Lorraine, fit schisme
entre elles. M\textsuperscript{me} de Lislebonne, par sa bâtardise
cousine germaine du père de M. de Lorraine, prétendit comme la plus
proche marcher la première, et par conséquent, M\textsuperscript{lle} de
Lislebonne et M\textsuperscript{lle} d'Espinoy ses filles immédiatement
après elle. M\textsuperscript{me} d'Elbœuf, veuve de l'aîné de la maison
de Lorraine en France, s'en moqua et l'emporta, de sorte que
M\textsuperscript{me} de Lislebonne ni ses filles n'y voulurent pas
aller. M\textsuperscript{me} de Valentinois n'y fut point non plus. Je
ne sais ce qu'on lui mit dans la tête.

Le P. de Chevigny de l'Oratoire mourut en ce temps-ci. C'était un
gentilhomme de bon lieu, qui avait servi longtemps avec réputation, et
connu du roi. M. de Turenne l'aimait fort, et tous les généraux de ces
temps-là l'estimaient. Cela l'avait fort mis dans le grand monde. Dieu
le toucha et il se fit prêtre, se mit dans l'Oratoire, et le servit
d'aussi bonne foi et d'aussi bon cœur qu'il avait servi le roi et le
monde. Il conserva d'illustres amis dans sa retraite dont il ne sortait
presque jamais. Il se trouva fort mêlé et lié avec tous les fameux
jansénistes, et en butte à leurs persécuteurs. C'était un homme droit,
franc, vrai, et d'une vertu simple, unie, militaire, mais grande, fidèle
à Dieu, à ses amis et au parti qu'il croyait le meilleur. Cela
embarrassa les pères de l'Oratoire. Il était ami intime de M. et de
M\textsuperscript{me} de Liancourt. Sans quitter l'Oratoire il se retira
avec eux, et après leur mort, M. de La Rochefoucauld, tout ignorant et
tout courtisan qu'il était, mais qui avait un extrême respect pour la
mémoire de M. et de M\textsuperscript{me} de Liancourt le pria tant de
demeurer à Liancourt qu'il s'y fixa. Quand il y venait compagnie avec M.
de La Rochefoucauld, on ne le voyait point, que M. de La Rochefoucauld,
M. le maréchal de Lorges et quelques amis très particuliers, et quand le
roi y passait, il se tapissait dans un grenier. À un de ces voyages du
roi, je ne sais qui en parla. Le roi le voulut voir. Le P. de Chevigny
en fut surpris, car le jansénisme l'avait fort barbouillé auprès de lui.
Il fallut pourtant obéir, et M. de La Rochefoucauld l'amena. Le roi lui
fit toutes sortes d'honnêtetés, et causa longtemps avec lui de ses
anciennes guerres, puis de sa retraite. Le P. de Chevigny fut fort
respectueux et mesuré, et point embarrassé. Ce fut à qui le verrait.
Jamais il ne fut si aise qu'après que tout ce monde fut parti. Ce
château, du temps de M. et de M\textsuperscript{me} de Liancourt, était
le rendez-vous et l'asile des principaux jansénistes. Il le fut bien
encore après. M. de La Rochefoucauld, qui les y avait tous vus, les aima
toujours. Ce qui en restait y allait voir le P. de Chevigny. Il y mourut
saintement comme il y avait vécu, sans cesse appliqué à la prière, à
l'étude et à toutes sortes de bonnes œuvres, et toujours gaiement et
avec liberté. M. de La Rochefoucauld, M. de Duras, M. le maréchal de
Lorges en furent fort affligés, et grand nombre d'autres personnes.

Le duc de Berwick perdit en même temps une très aimable femme, qu'il
avait épousée par amour, et qui avait très bien réussi à la cour et à
Saint-Germain. Elle était fille de milord Lucan, tué à Neerwinden,
lieutenant général et capitaine des gardes du roi Jacques. Elle était à
la première fleur de son âge, belle, touchante, faite à peindre, une
nymphe. Elle mourut de consomption à Montpellier, où son mari l'avait
menée pour la guérir par ce changement d'air. Elle lui laissa un fils.

Deux mariages amusèrent la cour au commencement de cette année. Celui de
M\textsuperscript{lle} de Chevreuse avec le marquis de Lévi, qui en eut
la lieutenance générale de Bourbonnais de son père où ils avaient leurs
biens. C'était un jeune homme bien fait, tout militaire et fort
débauché, qui n'avait jamais eu la plus légère teinture d'éducation, et
qui, avec cela, avait de l'esprit, de la valeur, de l'honneur et
beaucoup d'envie de faire. Son père était un homme de beaucoup d'esprit,
sans aucunes mœurs, retiré chez lui, et fort obscur à Paris quand il y
venait\,; la mère une joueuse sans fin et partout, avare à l'excès, et
faite et mise comme une porteuse d'eau. Tout cela cadrait mal avec les
mœurs et le génie de M. et de M\textsuperscript{me} de Chevreuse. La
légèreté de la dot et une naissance susceptible de tout les
déterminèrent, avec une place de dame du palais qui attendait
M\textsuperscript{lle} de Chevreuse. Quand il fallut dresser le contrat
de mariage, dont toutes les conditions étaient convenues, on fut arrêté
sur le nom de baptême du marquis de Lévi. A. et M\textsuperscript{me} de
Charlus se le demandèrent l'un à l'autre. Il se trouva qu'il n'en avait
point\,; de là on douta s'il avait été baptisé. Tous trois l'ignoraient.
Ils s'avisèrent que sa nourrice vivait encore, et qu'elle était à Paris.
Ce fut elle qu'ils consultèrent. Elle leur apprit que, portant leur
enfant avec eux en Bourbonnais pour le faire tenir au vieux marquis de
Lévi son grand-père, M. Colbert, évêque d'Auxerre, chez qui ils
couchèrent, en peine du voyage d'un enfant si tendre sans baptême et
n'ayant pu leur persuader de le laisser ondoyer, avait le matin, avant
qu'ils fussent éveillés, envoyé chercher la nourrice et l'enfant, et
l'avait ondoyé dans sa chapelle, et les laissa partir après sans leur en
avoir parlé\,; qu'arrivés en Bourbonnais, le baptême se remit plusieurs
fois par divers contretemps, et que lorsqu'elle quitta la maison il
n'avait pas été fait, dont elle ne s'était pas mise en peine parce
qu'elle le savait ondoyé. Ce trait est si étrange que je le mets ici
pour la curiosité, et parce qu'il sert plus que tout à caractériser des
gens qui en sont capables. Il fallut donc en même jour faire au marquis
de Lévi les cérémonies du baptême, lui faire faire sa première
confession et sa première communion, et le soir à minuit, le marier à
Paris à l'hôtel de Luynes.

Deux jours après, le comte d'Estrées épousa M\textsuperscript{lle}
d'Ayen. Une vieille bourgeoise qui s'appelait M\textsuperscript{lle} de
Toisy, riche et sans enfants, qui voyait bonne compagnie et fort
au-dessus d'elle, amie du cardinal d'Estrées et fort ménagée par
M\textsuperscript{me} de Noailles, donna une grande partie de la
médiocre dot, et le cardinal d'Estrées, qui voyait la faveur des
Noailles, et qui en espérait tout, acheva de sa bourse d'aplanir
l'affaire. Il les maria et dit la messe à minuit dans la chapelle de
Versailles, M\textsuperscript{me} la duchesse de Bourgogne et grand
monde aux tribunes, et force conviés en bas, et la noce se fit chez M.
de Noailles. Le lendemain la nouvelle marquise de Lévi et la nouvelle
comtesse d'Estrées furent déclarées dames du palais.

Il s'en fit un troisième à Paris assez ridicule, de Mortagne avec
M\textsuperscript{me} de Quintin. Elle et Montgomery, inspecteur de
cavalerie dont j'ai parlé, étaient enfants des deux frères. Elle avait
épousé le comte de Quintin qui était un Goyon, de même maison que MM. de
Matignon, qui était fils du marquis de La Moussaye et d'une fille du
maréchal de Bouillon, laquelle était sœur de la duchesse de La
Trémoille, de M\textsuperscript{me}s de Roucy et de Duras, et des ducs
de Bouillon et maréchal de Turenne, tous huguenots. M. de La Moussaye
avait acheté la belle terre de Quintin en Bretagne du duc de La
Trémoille son beau-frère, dont son fils porta le nom, qui était frère
aîné de M. de La Moussaye, lieutenant général et attaché à M. le Prince,
dans le parti duquel il mourut gouverneur de Stenay sans avoir été
marié. M\textsuperscript{me} de Quintin avait été fort jolie,
parfaitement bien faite, fort du monde, veuve de bonne heure sans
enfants, riche de ses reprises et de trente mille livres de rente que M.
le maréchal de Lorges lui faisait sa vie durant pour partie de
l'acquisition de Quintin qu'il avait faite de son mari. En cet état et
avec beaucoup d'esprit, elle vit la meilleure compagnie de la cour, et
comme elle avait l'esprit galant et impérieux, elle devint une manière
de fée qui dominait sur les soupirants sans se laisser toucher le bout
du doigt qu'à bonnes enseignes, et de là, sur tout ce qui venait chez
elle, toutefois avec jugement, et se fit une cour où on était en respect
comme à la véritable, et aussi touché d'un regard et d'un mot qu'elle
adressait. Elle avait un bon souper tous les soirs. Les grandes dames la
voyaient comme les grands seigneurs. Elle s'était mise sur le pied de ne
sortir jamais de chez elle, et de se lever de sa chaise pour fort peu de
gens. Monsieur y allait\,; elle était la reine de Saint-Cloud, où elle
n'allait qu'en bateau, et encore par grâce, et n'y faisait que ce qu'il
lui plaisait. Elle y avait apprivoisé jusqu'à Madame qui l'allait voir
aussi. M\textsuperscript{me} de Bouillon, autre reine de Paris, elle
l'avait subjuguée, l'avait souvent chez elle, et le duc et le cardinal
de Bouillon.

Le comte d'Auvergne fut longues années son esclave. M. de La Feuillade y
venait deux fois la semaine souper de Versailles, et retournait au
coucher du roi\,; et c'était une farce de la voir partager ses grâces
entre lui et le comte d'Auvergne, qui rampait devant elle, malgré sa
roguerie, et mourait à petit feu des airs et des préférences de l'autre.
Le comte de Fiesque qui, avec beaucoup d'esprit, était une manière de
cynique fort plaisant quelquefois, impatienté de cette fée, lui fit une
chanson et mettre un matin sur sa porte en grosses lettres, comme les
affiches d'indulgences aux églises\,: \emph{Impertinence plénière}. Peu
à peu la compagnie se mêla\,; le jeu prit un peu plus\,; l'avarice
diminua la bonne chère. La Feuillade avait enfin expulsé le comte
d'Auvergne, puis était mort. Le tribunal existait encore, et la décision
souveraine sur tout ce qui se passait, mais il ne florissait plus tant.
Mortagne, qui depuis vingt ans en était amoureux, et qui s'était fait la
justice de n'oser le montrer que par une assiduité pleine de respect, et
surtout de silence, parmi une si brillante cour, espéra alors que le
moment était venu de couronner sa patience. Il osa soupirer tout haut et
déclarer sa persévérance. Il était riche et capitaine de gendarmerie\,;
de l'honneur, de la valeur, de la politesse, avec un esprit doux et
médiocre. La fée fut touchée d'un amour si respectueux, si fidèle, si
constant. Elle était vieille et devenue infirme\,; elle couronna son
amour et l'épousa. Mortagne n'était rien\,; son nom était Collin. Il
était des Pays-Bas voisins de celui de Liège. Son père ou son grand-père
était homme d'affaires de la maison de Mortagne qui était ruinée. Il s'y
était enrichi, en avait acheté les terres, et celui-ci en portait le
nom. Il n'était rien moins que beau ni jeune\,; bien fait, mais un peu
gras\,; engoncé et fort rouge. Pas un de ses valets ne l'avait vu sans
perruque, ni s'habiller ou se déshabiller, d'où on jugeait qu'il avait
sur lui quelque chose qu'il ne voulait pas montrer. Ce mariage surprit
tout le monde, qui trouva Mortagne encore plus fou qu'elle de l'avoir
fait. Cela leur diminua à tous deux l'estime et la considération du
monde. La maison de M\textsuperscript{me} de Mortagne tomba fort\,; ils
s'en consolèrent par l'abondance et par filer ensemble le parfait amour.

La mort de l'archevêque de Bordeaux de la maison d'Anglure, frère de
Bourlemont, qui avait été auditeur de rote, fit donner cet archevêché à
Bissy, évêque de Toul, qui, grand courtisan de Saint-Sulpice, avait
tellement capté l'évêque de Chartres, qu'il l'avait fort prôné à
M\textsuperscript{me} de Maintenon et au roi. Bissy, qu'on verra dans la
suite faire une si grande fortune, ne crut pas le siège de Bordeaux
propre à l'en approcher. Il en voulait un plus voisin de la cour, d'où
il pût intriguer à son aise, et non pas se confiner à Bordeaux, et se
fit un honneur auprès de ses dupes de ne vouloir pas quitter sa première
épouse pauvre et d'un gouvernement fort étendu, pour être archevêque
d'un beau siège et dans une grande ville. Toul, en attendant mieux,
convenait plus à ses vues, et il y demeura. Bordeaux fut donné à Besons,
évêque d'Aire, qui le remplit fort dignement. Son frère aîné était
intendant de la province, et venait d'être fait conseiller d'État.
C'était un des intendants du royaume des plus accrédités.

Le cardinal de Bouillon donna en même temps la dernière marque de son
crédit. Sa princerie était sa folie dominante. Il en avait usurpé à Rome
tous les avantages qu'il avait pu. Il y prétendait l'\emph{altesse
éminentissime}, qu'il se faisait donner partout par ses valets\,;
personne autre à Rome ne voulut tâter de cette nouveauté. Il ne se
rebuta point. Il trouva un gentilhomme romain fort à simple tonsure,
qui, avec de l'argent, s'était fait faire prince par le pape\,; et ces
princes de pape sont à Rome même fort peu de chose. De sa personne, il
était encore moins, mais bien fait, voyant les dames et avec de
l'ambition. Il s'était attaché au cardinal de Bouillon en ses précédents
voyages\,; en celui-ci il s'y attacha de plus en plus. Le cardinal lui
fit grande montre de son crédit, et lui laissa entrevoir l'ordre par sa
protection\,; c'en fut assez pour obtenir de lui l'\emph{altesse
éminentissime}, et tout aussitôt voilà toutes les dépêches du cardinal
de Bouillon remplies de la convenance d'envoyer l'ordre à quelque baron
romain, qui fit honneur à la France par son attachement, et qui servit
bien ses ministres par ses avis et par son crédit, comme de temps en
temps on en avait toujours honoré quelqu'un. Il vanta ensuite la
naissance, l'esprit, la considération et le crédit de Vaïni à Rome, et
les services qu'on en pourrait tirer, et fit tant enfin que le roi lui
envoya l'ordre, c'est-à-dire le nomma à la Chandeleur, avec la
permission, dès qu'il aurait fait ses preuves, de le porter, en
attendant qu'il reçût le collier.

Si Vaïni en fut transporté d'aise, le cardinal de Bouillon le fut encore
plus\,; mais tout Rome en fut étrangement scandalisé. Cette cour l'avait
supporté dans le duc Lanti par son alliance pontificale, et parce qu'il
était beau-frère du duc de Bracciano, le premier laïque de Rome sans
dispute d'aucun, parce qu'il était plus vieux que le connétable Colonne,
et qu'entre ces deux, incontestablement les premiers et avec de grandes
distinctions très établies au-dessus de tous autres, ils ne se précèdent
l'un l'autre que par l'âge. Le duc de Bracciano avait longtemps porté le
collier de l'ordre du Saint-Esprit, et c'étaient des Ursins, des
Colonne, des Sforce qui l'avaient eu, bien différents en tout de Vaïni.

Je dis que le duc de Bracciano l'avait porté longtemps. M. de Nevers,
par commission du roi, le lui avait donné à Rome, en septembre 1675, et
le même jour au duc Sforce, veuf d'une Colonne, et lors gendre de
M\textsuperscript{me} de Thianges, sœur de M\textsuperscript{me} de
Montespan (et sa femme est la duchesse Sforce qu'après sa mort nous
avons tant vue à la cour), et au prince de Sonnino qui était Colonne
fils du connétable. Tout cela n'était point des Vaïni. Lors de l'éclat
entre Innocent XI et le roi pour les franchises du quartier des
ambassadeurs à Rome, et que M. de Lavardin l'était en 1688, que ce pape
ne voulut jamais voir et qu'il excommunia, le duc de Bracciano renvoya
au roi le collier de son ordre, quoique marié à une Françoise, depuis la
célèbre princesse des Ursins, et prit la Toison d'or du roi d'Espagne\,;
c'est le premier depuis l'institution de l'ordre du Saint-Esprit qui
l'ait renvoyé.

Parlant des pays étrangers, il est temps de dire que l'électeur de Saxe,
de plus en plus établi en Pologne, s'était réconcilié presque tous les
grands qui s'étaient opposés à lui, et le primat même, qui enfin l'avait
reconnu. Il était à Varsovie, et toutes les puissances de l'Europe
l'avaient félicité comme roi de Pologne. Le nonce Davia l'avait fort
utilement servi à Rome, mais tous ces exemples ne purent encore rien sur
le roi qui ne pouvait voir le prince de Conti, sans un grand déplaisir
de n'avoir pu s'en défaire honnêtement par une couronne.

Madame, qui pleurait tous ses parents selon le degré de parenté, comme
les autres en portent le deuil, fut très affligée de la mort du nouvel
et premier électeur d'Hanovre. Il avait épousé Sophie, fille d'une fille
du malheureux roi d'Angleterre, Charles Ier, et de l'électeur palatin
qui se fit roi de Bohême, et qui perdit ses États et mourut proscrit.
Quoique Madame n'eût jamais guère vu cette tante, elle lui écrivait
fidèlement des volumes deux et trois fois la semaine, depuis qu'elle
était en France. Le roi l'alla voir sur cette mort. Ses affaires ne
finissaient point avec l'électeur palatin, qui avait à payer, et qui
différait toujours sur toutes sortes de prétextes. Le roi voulut envoyer
pour cela à Ratisbonne Crécy qui entendait bien les affaires
d'Allemagne\,; mais celle-ci était une affaire de droit et un procès
dont Crécy aima mieux se débarrasser sur un autre, et il proposa Obrecht
qui y fut envoyé. C'était le préteur royal de Strasbourg, un génie fort
au-dessus de son état, et l'homme d'Allemagne qui en possédait le mieux
les lois et les coutumes. M. de Louvois le sut gagner\,; et lui sut
mettre les troupes du roi dans Strasbourg en pleine paix, sans coup
férir, qui nous est demeuré depuis.

\hypertarget{chapitre-vi.}{%
\chapter{CHAPITRE VI.}\label{chapitre-vi.}}

1698

~

{\textsc{Le czar et ses voyages.}} {\textsc{- Saint-Albans envoyé, et
Portland ambassadeur d'Angleterre, à Paris.}} {\textsc{- Premiers
princes de Parme et de Toscane incognito en France\,; le dernier
distingué.}} {\textsc{- Distraction du cardinal d'Estrées.}} {\textsc{-
M\textsuperscript{lle}s de Soissons enlevées, et à Bruxelles.}}
{\textsc{- Le comte de Soissons errant.}} {\textsc{- Abbé de Caudelet
fait et défait évêque de Poitiers.}} {\textsc{- Mort du président Talon
et sa dépouille.}} {\textsc{- Mort de M\textsuperscript{me} de
Sillery.}} {\textsc{- Mort de Villars, chevalier de l'ordre\,; pourquoi
dit Orondat.}} {\textsc{- Castries, chevalier d'honneur de
M\textsuperscript{me} la duchesse de Chartres.}} {\textsc{- Mort de
Brienne.}} {\textsc{- Mort du duc de Bracciano.}}

~

Le czar\footnote{Pierre le Grand, souverain de Russie, de 1689 à 1725.}
avait déjà commencé ses voyages. Il a tant et si justement fait de bruit
dans le monde, que je serai succinct sur un prince si grand et si connu,
et qui le sera sans doute de la postérité la plus reculée, pour avoir
rendu redoutable à toute l'Europe, et mêlé nécessairement à l'avenir
dans les affaires de toute cette partie du monde, une cour qui n'en
avait jamais été une, et une nation méprisée et entièrement ignorée pour
sa barbarie. Ce prince était en Hollande à apprendre lui-même et à
pratiquer la construction des vaisseaux. Bien qu'incognito, suivant sa
pointe, et ne voulant point s'incommoder de sa grandeur ni de personne,
il se faisait pourtant tout rendre, mais à sa mode et à sa façon.

Il trouva sourdement mauvais que l'Angleterre ne s'était pas assez
pressée de lui envoyer une ambassade dans ce proche voisinage, d'autant
que, sans se commettre, il avait fort envie de lier avec elle pour le
commerce. Enfin l'ambassade arriva\,: il différa de lui donner audience,
puis donna le jour et l'heure, mais à bord d'un gros vaisseau Hollandais
qu'il devait aller examiner. Il y avait deux ambassadeurs qui trouvèrent
le lieu sauvage, mais il fallut bien y passer. Ce fut bien pis quand ils
furent arrivés à bord. Le czar leur fit dire qu'il était à la hune, et
que c'était là qu'il les verrait. Les ambassadeurs qui n'avaient pas le
pied assez marin pour hasarder les échelles de corde, s'excusèrent d'y
monter\,; le czar insista, et les ambassadeurs fort troublés d'une
proposition si étrange et si opiniâtre\,; à la fin, à quelques réponses
brusques aux derniers messages, ils sentirent bien qu'il fallait sauter
ce fâcheux bâton, et ils montèrent. Dans ce terrain si serré et si fort
au milieu des airs, le czar les reçut avec la même majesté que s'il eût
été sur son trône\,; il écouta la harangue, répondit obligeamment pour
le roi et la nation, puis se moqua de la peur qui était peinte sur le
visage des ambassadeurs, et leur fit sentir en riant que c'était la
punition d'être arrivés auprès de lui trop tard.

Le roi Guillaume, de son côté, avait déjà compris les grandes qualités
de ce prince, et fit de sa part tout ce qu'il put pour être bien avec
lui. Tant fut procédé entre eux qu'enfin le czar, curieux de tout voir
et de tout apprendre, passa en Angleterre, toujours incognito, mais à sa
façon. Il y fut reçu en monarque qu'on veut gagner, et, après avoir bien
satisfait ses vues, repassa en Hollande. Il avait dessein d'aller à
Venise et à Rome et dans toute l'Italie, surtout de voir le roi et la
France. Il fit sonder le roi là-dessus, et le czar fut mortifié de ce
que le roi déclina honnêtement sa visite, de laquelle il ne voulut point
s'embarrasser. Peu après en avoir perdu l'espérance, il se résolut de
voyager en Allemagne, et d'aller jusqu'à Vienne. L'empereur le reçut à
la Favorite, accompagné seulement de deux de ses grands officiers, et le
czar du seul général Le Fort, qui lui servait d'interprète et à la suite
duquel il paraissait être comme de l'ambassadeur de Moscovie. Il monta
par l'escalier secret, et trouva l'empereur à la porte de son
antichambre la plus éloignée de la chambre. Après les premiers
compliments l'empereur se couvrit. Le czar voulut demeurer découvert à
cause de l'incognito\,; ce qui fit découvrir l'empereur. Au bout de
trois semaines, le czar fut averti d'une grande conspiration en
Moscovie, et partit précipitamment pour s'y rendre. En passant en
Pologne il en vit le roi, et ce fut là que furent jetés les premiers
fondements de leur amitié et de leur alliance.

En arrivant chez lui, il trouva la conspiration fort étendue, et sa
propre sœur à la tête. Il l'avait toujours fort aimée et bien traitée,
mais il ne l'avait point mariée. La nation en gros était outrée de ce
qu'il lui avait fait couper sa barbe, rogné ses habits longs, été force
coutumes barbares, et de ce qu'il mettait des étrangers dans les
premières places et dans sa confiance\,; et pour cela il s'était formé
une grande conspiration qui était sur le point d'éclater par une
révolution. Il pardonna à sa sœur qu'il mit en prison, et fit pendre aux
grilles de ses fenêtres les principaux coupables, tant qu'il en put
tenir par jour. J'ai écrit de suite ce qui le regarde pour cette année,
pour ne pas sautiller sans cesse d'une matière à l'autre. C'est ce que
je vais faire par même raison sur celle qui va suivre.

Le roi d'Angleterre était au comble de satisfaction de se voir enfin
reconnu par le roi, et paisible sur ce trône\,; mais un usurpateur n'est
jamais tranquille et content. Il était blessé du séjour du roi légitime
et de sa famille à Saint-Germain. C'était trop à portée du roi, et trop
près d'Angleterre pour le laisser sans inquiétude. Il avait fait tous
ses efforts, tant à Ryswick que dans les conférences de Portland et du
maréchal de Boufflers, pour obtenir leur sortie du royaume, tout au
moins leur éloignement de la cour. Il avait trouvé le roi inflexible\,;
il voulut essayer tout, et voir si, n'en faisant plus une condition,
puisqu'il avait passé carrière, et comblant le roi de prévenances et de
respects, il ne pourrait pas obtenir ce fruit de ses souplesses. Dans
cette vue il envoya le duc de Saint-Albans, chevalier de la Jarretière,
complimenter le roi sur le mariage de Mgr le duc de Bourgogne. Il ne
pouvait choisir un homme plus marqué pour une simple commission\,; on
fut surpris même qu'il l'eût acceptée. Il était bâtard de Charles II,
frère aîné du roi Jacques II, et c'était bien encore là une raison pour
Saint-Albans de s'en excuser. Il voulut même prétendre quelques
distinctions, mais on tint poliment ferme à ne le traiter que comme un
simple envoyé d'Angleterre. Les ducs de ce pays-là n'ont aucun rang ici,
non plus que ceux d'ici en Angleterre. Le roi avait fait la duchesse de
Portsmouth et le duc de Richemont, son fils, duc et duchesse à brevet,
et accordé un tabouret de grâce en passant à la duchesse de Cleveland,
maîtresse de Charles Il, son ami. La duchesse de La Force, retirée en
Angleterre pour la religion, et, avant elle, la duchesse Mazarin,
fugitive de son mari, et fixée en Angleterre, y avaient obtenu le rang
des duchesses\,; mais ce sont des grâces particulières qui ne tirent
point à conséquence pour le général.

Ce duc de Saint-Albans fut le précurseur du comte de Portland, à
l'arrivée duquel il prit congé. J'ai déjà assez parlé de ce favori pour
n'avoir pas besoin d'y rien ajouter. Les mêmes raisons qui l'avaient
fait choisir pour conférer avec le maréchal de Boufflers le firent
préférer à tout autre pour cette ambassade. On n'en pouvait nommer un
plus distingué. Sa suite fut nombreuse et superbe, et sa dépense
extrêmement magnifique en table, en chevaux, en livrées, en équipages,
en meubles, en habits, en vaisselle et en tout, et avec une recherche et
une délicatesse exquise. Tout arriva presque au même temps, parce que le
comte vint de Calais dans son carrosse à journées, et reçut partout
toutes sortes d'honneurs militaires et civils. Il était en chemin
lorsque le feu prit à White-Hall, le plus vaste et le plus vilain palais
de l'Europe, qui fut presque entièrement brûlé, et qui n'a pas été
rétabli depuis, de sorte que les rois se sont logés et assez mal au
palais de Saint-James. Portland eut sa première audience particulière du
roi, le 4 février, et fut quatre mois en France. Il arriva avant que
Tallard fût parti, ni aucun autre de la part du roi, pour Londres.
Portland parut avec un éclat personnel, une politesse, un air de monde
et de cour, une galanterie et des grâces qui surprirent. Avec cela,
beaucoup de dignité, même de hauteur, mais avec discernement, et un
jugement prompt, sans rien de hasardé. Les Français qui courent à la
nouveauté, au bon accueil, à la bonne chère, à la magnificence, en
furent charmés. Il se les attira, mais avec choix, et en homme instruit
de notre cour, et qui ne voulait que bonne compagnie et distinguée.
Bientôt il devint à la mode de le voir, de lui donner des fêtes, et de
recevoir de lui des festins. Ce qui est étonnant, c'est que le roi, qui
au fond n'était que plus outré contre le roi Guillaume, y donna lieu
lui-même, en faisant pour cet ambassadeur ce qui n'a jamais été fait
pour aucun autre. Aussi fit toute la cour pour lui à l'envi\,: peut-être
le roi voulut-il compenser par là le chagrin qu'il eut en arrivant de
voir, dès le premier jour, sa véritable mission échouée.

Dès la première fois qu'il vit Torcy avant d'aller à Versailles, il lui
parla du renvoi, à tout le moins, de l'éloignement du roi Jacques et de
sa famille. Torcy sagement n'en fit point à deux fois, et lui barra tout
aussitôt la veine. Il lui répondit que ce point, tant de fois proposé
dans ses conférences avec le maréchal de Boufflers, et sous tant de
diverses formes débattu à Ryswick, avait été constamment et nettement
rejeté partout\,; que c'était une chose réglée et entièrement finie\,;
qu'il savait que le roi, non seulement ne se laisserait jamais entamer
là-dessus le moins du monde, mais qu'il serait extrêmement blessé d'en
ouïr parler davantage\,; qu'il pouvait l'assurer de la disposition du
roi à correspondre en tout, avec toutes sortes de soins, à la liaison
qui se formait entre lui et le roi d'Angleterre, et personnellement à le
traiter lui avec toutes sortes de distinctions\,; qu'un mot dit par lui
sur Saint-Germain serait capable de gâter de si utiles dispositions, et
de rendre son ambassade triste et languissante\,; et que, s'il était
capable de lui donner un conseil, c'était celui de ne rien gâter, et de
ne pas dire un seul mot au roi, ni davantage à aucun de ses ministres,
sur un point convenu, et sur lequel le roi avait pris son parti.
Portland le crut, et s'en trouva bien\,; mais on verra bientôt que ce ne
fut pas sans dépit, et le roi approuva extrêmement que Torcy lui eût dès
l'abord fermé la bouche sur cet article. On prit un grand soin de faire
en sorte qu'aucun Anglais de Saint-Germain ne se trouvât à Versailles ni
à Paris, à aucune portée de ceux de l'ambassadeur, et cela fut très
exactement exécuté.

Portland fit un trait au milieu de son séjour qui donna fort à penser,
mais qu'il soutint avec audace sans faire semblant de s'apercevoir qu'on
l'eût même remarqué. Vaudemont passait des Pays-Bas à Milan, sans
approcher de la cour. Soit affaires, soit galanterie pour l'ami intime
de son maître qu'il voulut ménager, il partit de Paris, et s'en alla à
Notre-Dame-de-Liesse, auprès de Laon, voir Vaudemont qui y passait. Le
marquis de Bedmar passa bientôt après d'Espagne aux Pays-Bas, pour y
remplir la place qu'y avait Vaudemont de gouverneur des armes. Il
n'avait pas les mêmes exclusions personnelles que Vaudemont avait
méritées. Il vint à Paris et à la cour, où Monsieur à cause de la feue
reine sa fille le présenta au roi, de qui il fut fort bien reçu.
Portland suivit Monseigneur à la chasse. Deux fois il alla de Paris à
Meudon pour courre le loup, et toutes les deux fois Monseigneur le
retint à souper avec lui. Le roi lui donna un soir le bougeoir à son
coucher, qui est une faveur qui ne se fait qu'aux gens les plus
considérables et que le roi veut distinguer. Rarement les ambassadeurs
se familiarisent à faire leur cour à ces heures, et s'il y en vient, il
n'arrive presque jamais qu'ils reçoivent cet agrément.

Celui-ci prit son audience de congé le 20 mai, comblé de tous les
honneurs, de toutes les fêtes, de tous les empressements possibles. Le
maréchal de Villeroy eut ordre du roi de le mener voir Marly, et de lui
en faire les honneurs. Il voulut voir tout ce qu'il y a de curieux et
surtout Fontainebleau, dont il fut plus content que d'aucune autre
maison royale. Quoiqu'il eût pris congé, il alla faire sa cour au roi,
qui prenait médecine. Le roi le fit entrer après l'avoir prise, ce qui
était une distinction fort grande, et pour la combler, il le fit entrer
dans le balustre de son lit, où jamais étranger, de quelque rang et de
quelque caractère qu'il fût, n'était entré, à l'exception de l'audience
de cérémonie des ambassadeurs. Au sortir de là Portland alla trouver
Monseigneur à la chasse qui le ramena pour la troisième fois souper avec
lui à Meudon. Le grand prieur s'y mit au-dessus de lui avec quelque
affectation, dont l'autre, quoique ayant pris congé, s'offensa fort, et
le lendemain matin alla fièrement dire au roi que s'il avait donné le
rang de princes du sang à MM. de Vendôme, il ne leur disputerait pas,
mais que, s'ils ne l'avaient pas, il croyait que le grand prieur devait
avoir pour lui les honnêtetés qu'il n'avait pas eues. Le roi lui
répondit qu'il n'avait point donné ce rang à MM. de Vendôme, et qu'il
manderait à Monseigneur qui était encore à Meudon de faire que cela
n'arrivât plus. Monsieur lui voulut faire voir Saint-Cloud lui-même.
Madame exprès n'y alla pas, et Monsieur lui donna un grand repas où
Monseigneur se trouva et grande compagnie. Ce fut encore là un honneur
fort distingué.

Mais parmi tant de fleurs, il ne laissa pas d'essuyer quelques épines,
et de sentir la présence du légitime roi d'Angleterre en France. Il
était allé une autre fois à Meudon pour suivre Monseigneur à la chasse.
On allait partir et Portland se bottait, lorsque Monseigneur fut averti
que le roi d'Angleterre se trouverait au rendez-vous. À l'instant il le
manda à Portland, et qu'il le priait de remettre à une autre fois. Il
fallut se débotter et revenir tout de suite à Paris.

Il était grand chasseur. Soit envie de voir faire la meute du roi, soit
surprise de ne recevoir aucune autre civilité du duc de La Rochefoucauld
que la simple révérence lorsqu'ils se rencontraient, il dit et répéta
souvent qu'il mourait d'envie de chasser avec les chiens du roi. Il le
dit tant et devant tant de gens qu'il jugea impossible que cela ne fût
revenu à M. de La Rochefoucauld, et cependant sans aucune suite. Lassé
de cette obscurité il la voulut percer, et au sortir d'un lever du roi
aborda franchement le grand veneur, et lui dit son désir. L'autre ne
s'en embarrassa point. Il lui répondit assez sèchement qu'à la vérité il
avait l'honneur d'être grand veneur, mais qu'il ne disposait point des
chasses\,; que c'était le roi d'Angleterre dont il prenait les ordres\,;
qu'il y venait très souvent, mais qu'il ne savait jamais qu'au moment de
partir quand il ne venait pas au rendez-vous, et tout de suite la
révérence, et laissa là Portland dans un grand dépit, et toutefois sans
se pouvoir plaindre. M. de La Rochefoucauld fut le seul grand seigneur
distingué de la cour qui n'approcha jamais Portland. Ce qu'il lui
répondit était pure générosité pour le roi d'Angleterre. Ce prince, à la
vérité, disposait quand il voulait de la meute du roi, mais il y avait
bien des temps qu'il ne chassait point, et jamais à toutes les chasses.
Il ne tenait donc qu'à M. de La Rochefoucauld d'en donner à Portland
tant qu'il aurait voulu, à coup sûr, mais piqué de la prostitution
publique à la vue de la cour de Saint-Germain, il ne put se refuser
cette mortification au triomphant ambassadeur de l'usurpateur qui avait
attaché à son char jusqu'à M. de Lauzun, malgré ses engagements et son
attachement au roi et à la reine d'Angleterre, et sans y pouvoir gagner
que de la honte, pour suivre la mode et croire faire sa cour au roi.

Enfin Portland, comblé en toutes les manières possibles, se résolut au
départ. La faveur naissante du duc d'Albemarle l'inquiétait et le hâta.
M. le Prince le pria de passer à Chantilly, et il lui donna une fête
magnifique avec ce goût exquis qui, en ce genre, est l'apanage
particulier aux Condé. De là Portland continua son chemin par la
Flandre\,; non seulement il eut la permission du roi d'y voir toutes les
places qu'il voudrait, mais il le fit accompagner par des ingénieurs
avec ordre de les lui bien montrer. Il fut reçu partout avec les plus
grands honneurs, et eut toujours un capitaine et cinquante hommes de
garde. Le bout d'un si brillant voyage fut de trouver à sa cour un jeune
et nouveau compétiteur qui prit bientôt le dessus, et qui ne lui laissa
que les restes de l'ancienne confiance, et le regret d'une absence qui
l'avait laissé établir. Sur son départ de Paris, il avait affecté de
répandre que tant que le roi Jacques serait à Saint-Germain la reine
d'Angleterre ne serait point payée du douaire qui lui avait été accordé
à la paix, et il tint parole.

Avant de quitter les étrangers, je ferai une courte mention du voyage
que vinrent faire en France, les premiers mois de cette année, le frère
du duc de Parme qui y fut incognito, et quelque temps après le prince
Gaston, second fils du grand-duc, par la singularité qu'ils furent tous
deux les deux derniers ducs de Parme et de Toscane. Ce dernier garda
aussi l'incognito, mais ce nonobstant le roi voulut le distinguer, et
qu'il baisât M\textsuperscript{me} la duchesse de Bourgogne\,; il était
fils de M\textsuperscript{me} la grande-duchesse, cousine germaine du
roi, et la vit fort, tant qu'il fut à Paris. Le roi prit même quelque
soin de sa conduite. Il chargea Albergotti, à cause du pays, de se tenir
presque toujours auprès de lui, et de prendre garde à lui faire voir
bonne compagnie. Il demeura peu en ce pays-ci, d'où il passa en
Allemagne chez la princesse de Saxe-Lauenbourg, son épouse, avec
laquelle il se brouilla depuis à ne se jamais revoir. Le frère du duc de
Parme demeura presque toute l'année.

Je me souviens qu'à Fontainebleau, où on se donne plus qu'ailleurs de
grands repas les uns aux autres, le cardinal d'Estrées, logé à la
chancellerie, lui en voulut donner un où il pria beaucoup de gens
distingués de la cour. Il me pria aussi, et j'y trouvai de plus ce qu'il
avait lors de sa plus proche famille, pour lui aider à faire les
honneurs au prince de Parme\,; mais il arriva que nous fîmes le festin
sans lui. Le cardinal qui allant et venant avait prié depuis plusieurs
jours les gens qu'il voulut à mesure qu'il les avait rencontrés, n'avait
oublié que le prince de Parme. Le matin du repas le souvenir lui en
vint\,; il demanda quel de ses gens l'avait été inviter de sa part, il
se trouva qu'il n'en avait chargé aucun. Il y envoya vitement, mais il
arriva que le prince de Parme était engagé et pour plusieurs jours. On
plaisanta beaucoup le cardinal pendant le repas de cette rare
distraction. Il en avait souvent de pareilles.

Le roi, à la prière de M. de Savoie, envoya enlever
M\textsuperscript{lle} de Carignan par un lieutenant de ses gardes du
corps à l'hôtel de Soissons, qui la mena aux Filles de Sainte-Marie dans
un carrosse de l'ambassadeur de Savoie. En même temps l'électeur de
Bavière en fit autant à Bruxelles, où il fit conduire dans un couvent
M\textsuperscript{lle} de Soissons de chez sa mère. Leur conduite était
depuis longtemps tellement indécente, et leur débauche si prostituée que
M. de Savoie ne put plus supporter ce qu'il en apprenait. Quelque temps
après il envoya une dame de Savoie ici où M\textsuperscript{lle} de
Soissons se devait rendre, pour les conduire toutes deux dans ses États
où il comptait de les resserrer fort dans un couvent, mais à la fin
elles obtinrent, l'une de retourner chez sa mère à Bruxelles, l'autre de
l'y aller trouver d'ici. Pendant ce temps-là le comte de Soissons, leur
frère aîné, qui était sorti d'ici depuis quelques années, quoique comblé
des grâces et des bontés du roi, continuait à courir l'Europe pour
chercher du service et du pain. On n'en avait voulu, ni en Angleterre,
ni en Allemagne, ni à Venise. Il s'en alla chercher fortune en Espagne,
qu'il n'y trouva non plus qu'ailleurs. Il eut peine à obtenir permission
de passer à Turin, où M. de Savoie ne le voulait point voir. Sa femme y
était dans un couvent, fort pauvre et fort retirée.

L'évêque de Poitiers était mort au commencement de cette année. Il avait
été longtemps prêtre de l'Oratoire sous le nom de P. Saillans, et il
était de ces Baglioni qui ont tant figuré dans les guerres d'Italie. Ses
sermons l'avaient fait évêque de Tréguier, où il avait appris le
bas-breton pour pouvoir entendre et prêcher les peuples de ce diocèse.
De là il passa à Poitiers. C'était un excellent évêque, qui venait peu à
Paris. Il ressemblait parfaitement à tous les portraits de saint
François de Sales. J'en fus très fâché\,; il était ami intime de mon
père et de ma mère. Son évêché fut donné à Pâques à l'abbé de Caudelet.
C'était un bon gentilhomme de Bretagne, frère d'un capitaine aux gardes,
fort estropié, et qui avait bien servi. Ils étaient parents de la
maréchale de Créqui, et souvent chez elle. L'envie de lui voir un si bel
évêché et la rage de n'en avoir point firent aller au P. de La Chaise
les plus noires calomnies contre l'abbé de Caudelet qui avait toujours
passé pour un fort honnête homme et de très bonnes mœurs, et qui l'était
en effet, et entre autres impostures, qu'il avait passé au jeu tout le
vendredi saint, veille du jour de sa nomination à Poitiers. La vérité
était qu'ayant assisté à tous les offices de la journée, il alla sur le
soir voir la maréchale de Créqui qui était seule et fatiguée des
dévotions. Elle aimait à jouer\,; elle proposa à l'abbé de l'amuser une
heure au piquet. Il le fit par complaisance, fit collation avec elle et
puis se retira. Cela fut bien vérifié ensuite. Le P. de La Chaise,
épouvanté de ce qu'il recevait sur son compte, le dit au roi qui lui ôta
sur-le-champ Poitiers. L'éclat fut grand\,: le pauvre abbé, accablé de
l'affront, se cacha longtemps, puis fut trouvé dans la Chartreuse de
Rouen, où, sans prendre l'habit, il vécut longtemps comme les chartreux.
Au bout de quelques années il s'en alla en Bretagne, où il a passé le
reste de sa vie dans la même solitude et dans la même piété, sans s'en
être dérangé un moment, ni {[}avoir{]} jamais fait la moindre démarche
pour avoir quoi que ce soit.

Son frère cependant éclaircit la scélératesse, et prouva si nettement la
fausseté de tous les allégués, que le P. de La Chaise, qui était bon et
droit, fit tout ce qu'il put pour obtenir un gros évêché à l'abbé de
Caudelet\,; mais le roi tint ferme, jusque-là qu'ils en eurent des
prises lui et son confesseur, à qui il reprocha qu'il était trop bon, et
l'autre, au roi, qu'il était trop dur et qu'il ne revenait jamais. Il ne
se rebuta point, et tant qu'il a vécu, il a souvent fait de nouveaux
efforts, mais tous aussi inutiles.

On sut aussi qui était le faux délateur, et qui avait fait et envoyé ces
calomnies atroces. C'était l'abbé de La Châtre, frère du gendre du
marquis de Lavardin. Il était aumônier du roi depuis longtemps, et il
enrageait de n'être point évêque, et contre tous ceux qui le devenaient.
C'était un homme qui ne manquait pas d'esprit, mais pointu, désagréable,
pointilleux, fort ignorant parce qu'il n'avait jamais voulu rien faire,
et si perdu de mœurs, que je lui vis dire la messe à la chapelle un
mercredi des cendres, après avoir passé la nuit masqué au bal, faisant
et disant les dernières ordures, à ce que vit et entendit M. de La
Vrillière devant qui il se démasqua, et qui me le conta le lendemain
matin une demi-heure avant que je le rencontrasse habillé allant à
l'autel. D'autres aventures l'avaient déjà perdu auprès du roi pour être
évêque. Il était fort connu et fort méprisé. Il ne porta pas loin le
châtiment de son dernier crime, et la vengeance du pauvre abbé de
Caudelet qui fut plaint de tout le monde.

Le président Talon alla aussi en l'autre monde voir s'il est permis de
souffler le froid et le chaud comme M. de Luxembourg le lui avait fait
faire. Lamoignon eut sa charge de président à mortier, et Portail eut la
sienne d'avocat général où il brilla plus que lui, et s'y fit beaucoup
de réputation d'éloquence et d'équité. Ce n'est pas qu'il ne fût fils de
notre rapporteur, plus que très favorable à M. de Luxembourg, mais il
faut dire la vérité.

M\textsuperscript{me} de Sillery mourut à Liancourt, où elle était
retirée depuis un grand nombre d'années. Elle était sœur du père de M.
de La Rochefoucauld, qui avait tant figuré avec M\textsuperscript{me} de
Longueville dans le parti de M. le Prince, et qui eut tant d'esprit et
d'amis. Sa sœur en avait aussi beaucoup, mais rien vaillant\,; ce qui
fit son mariage. Elle se trouva mal mariée, et ne parut point à la cour.
M. de Sillery avait aussi beaucoup d'esprit, mais nulle conduite, et se
ruina en fils de ministre, sans guerre ni cour. Il ne laissait pas
d'être fort dans le monde et désiré par la bonne compagnie. Il allait à
pied partout faute d'équipage, et ne bougeait de l'hôtel de La
Rochefoucauld ou de Liancourt avec sa femme, qui s'y retira dans le
désordre de ses affaires, longtemps avant la mort de son mari. Elle
était fort considérée de ses neveux, et assistée de tout. Puysieux,
qu'on vient de voir ambassadeur en Suisse, le chevalier de Sillery,
écuyer de M. le prince de Conti, et l'évêque de Soissons, étaient ses
enfants. Sillery, leur père, était petit-fils du chancelier de Sillery
et fils de Puysieux, secrétaire d'État, chassé avec le chancelier dès
1640, et mort en disgrâce, et de cette fameuse M\textsuperscript{me} de
Puysieux si bien avec la reine mère, si comptée et si impérieuse avec le
monde, et qui mangea à belles dents, pour s'amuser, pour cinquante mille
écus de point de Gênes à ses manchettes et à ses collets, qui était lors
la grande mode. Elle était Étampes, et commença la ruine de son fils.

Le vieux Villars mourut en même temps à Paris en deux jours, à plus de
quatre-vingts ans. J'aurais assez parlé de lui lorsqu'il fut chevalier
d'honneur de M\textsuperscript{me} la duchesse de Chartres à son
mariage, si je ne me souvenais à cette heure de l'origine de son nom
d'Orondat, qu'on lui donnait toujours, et qui ne lui déplaisait pas. La
voici\,: la comtesse de Fiesque, si intime de Mademoiselle, avait amené
de Normandie avec elle M\textsuperscript{lle} d'Outrelaise, et la
logeait chez elle. C'était une fille de beaucoup d'esprit, qui se fit
beaucoup d'amis qui l'appelèrent la Divine, nom qu'elle communiqua
depuis à M\textsuperscript{me} de Frontenac, avec qui elle alla demeurer
depuis à l'Arsenal, et avec qui elle passa inséparablement sa vie, autre
personne d'esprit et d'empire, et de toutes les bonnes compagnies de son
temps. On ne les appelait que les Divines. Pour en revenir donc à
l'Orondat, M\textsuperscript{me} de Choisy, autre personne du grand
monde, alla voir la comtesse de Fiesque, et y trouva grande compagnie.
L'envie de pisser la prit\,; elle dit qu'elle allait monter en haut chez
la Divine, qui était M\textsuperscript{lle} d'Outrelaise. Elle monte
brusquement, y trouve M\textsuperscript{lle} de Bellefonds, tante
paternelle du maréchal, jeune et extrêmement jolie, et voit un homme qui
se sauve et qu'elle ne put connaître. La figure de cet homme
parfaitement bien fait la frappa tant, que, de retour à la compagnie, et
contant son aventure, elle dit que ce ne pouvait être qu'Orondat. La
plupart de la compagnie savait que Villars était en haut, où il était
allé voir M\textsuperscript{lle} de Bellefonds dont il était fort
amoureux, qui n'avait rien, et qu'il épousa fort peu après. Ils rirent
fort de l'aventure et de l'Orondat. Maintenant qu'on s'est heureusement
défait de la lecture des romans, il faut dire qu'Orondat est un
personnage du \emph{Cyrus}, célèbre par sa taille et sa bonne mine, qui
charmait toutes les héroïnes de ce roman alors fort à la mode.
M\textsuperscript{me} la duchesse d'Orléans souhaita fort que M. de
Castries, mari de sa dame d'atours, eût la place qu'avait Villars auprès
d'elle\,; Monsieur, qui a toujours fort aimé M\textsuperscript{me} de
Montespan, y consentit, et M. du Maine acheva l'affaire auprès du roi.

Quelque temps après mourut M. de Brienne, l'homme de la plus grande
espérance de son temps en son genre, le plus savant, et qui possédait à
fond toutes les langues savantes et celles de l'Europe. Il eut de très
bonne heure la survivance de son père, qui avait eu la charge de
secrétaire d'État du département des affaires étrangères, lorsque
Chavigny fut chassé. Loménie qui voulait rendre son fils capable de la
bien exercer, et qui n'avait que seize ou dix-sept ans, l'envoya voyager
en Italie, en Allemagne, en Pologne, et par tout le Nord jusqu'en
Laponie. Il brilla fort, et profita encore plus dans tous ces pays, où
il conversa avec les ministres et ce qu'il y trouva de gens plus
considérables, et en rapporta une excellente relation latine. Revenu à
la cour, il y réussit admirablement, et dans son ministère, jusqu'en
1664 qu'il perdit sa femme, fille de ce même Chavigny, et sœur de M. de
Troyes, de la retraite duquel j'ai parlé, de la maréchale Clérembault,
etc. Il l'avait épousée quatre ans après la mort de Chavigny. Il fut
tellement affligé de cette perte, que rien ne put le retenir. Il se jeta
dans les pères de l'Oratoire et s'y fit prêtre. Dans les suites il s'en
repentit. Il écrivit des lettres, des élégies, des sonnets beaux et
pleins d'esprit, et tenta tout ce qu'il put pour rentrer à la cour et en
charge. Cela ne lui réussit pas\,; la tête se troubla, il sortit de sa
retraite et se remit à voyager. Il lui échappa beaucoup de messéances à
son état passé et à celui qu'il avait embrassé depuis. On le fit revenir
en France, où, bientôt après, on l'enferma dans l'abbaye de
Château-Landon. Sa folie ne l'empêcha pas d'y écrire beaucoup de poésies
latines et françaises, parfaitement belles et fort touchantes, sur ses
malheurs\footnote{Louis-Henri de Loménie, comte de Brienne, a laissé des
  Mémoires sur les règnes de Louis XIII et de Louis XIV. Ils ont été
  publiés en 1828, par M. François Barrière (Paris, Ponthieu. 2
  vol.~in-8°).}. Il laissa un fils qui est aussi mort enfermé, et deux
filles. Sa sœur et sa fille aînée épousèrent MM. de Gamaches, père et
fils\,; et l'autre fille, M. de Poigny-Angennes\,; ainsi ont fini les
Loménie, M. de Lyonne eut la charge de M. de Brienne. Sa famille a
encore moins duré, et n'a pas fini plus heureusement\,; tel est
d'ordinaire le sort des ministres.

En même temps mourut à Rome le duc de Bracciano, à soixante-dix-huit
ans, dont tout le mérite consista en sa naissance et en ses grands
biens. C'était, comme je l'ai dit, le premier laïque de Rome, grand
d'Espagne, prince du Soglio du pape, et chef de la maison des Ursins. La
sœur de son père était la fameuse duchesse de Montmorency, qui, après la
mort tragique de son mari en 1632, se retira à Moulins, où elle se fit
fille de Sainte-Marie.

\hypertarget{chapitre-vii.}{%
\chapter{CHAPITRE VII.}\label{chapitre-vii.}}

1698

~

{\textsc{Duchesse de Bracciano, ses premières aventures\,; prend le nom
de princesse des Ursins.}} {\textsc{- Étrange et hardie tentative du
cardinal de Bouillon de faire l'abbé d'Auvergne cardinal.}} {\textsc{-
Mariage de Souvré avec M\textsuperscript{lle} de Rebénac\,; du vieux
Seissac avec une sœur du duc de Chevreuse\,; du comte d'Ayen avec
M\textsuperscript{lle} d'Aubigné.}} {\textsc{- Le roi paye les dettes de
M. de La Rochefoucauld.}} {\textsc{- Mort de l'abbé de Marsillac.}}
{\textsc{- Le roi prend le deuil d'un enfant de M. le prince de Conti,
et pourquoi.}} {\textsc{- Mort de Fervaques\,; sa dépouille et son
testament.}} {\textsc{- Duc de Lesdiguières accommodé, par ordre du roi,
par le maréchal de Duras seul, son beau-père, avec Lambert.}} {\textsc{-
M. de Lorraine en Lorraine, où le duc d'Elbœuf revient mal avec lui.}}
{\textsc{- Camp de Compiègne résolu et déclaré.}}

~

M. de Bracciano, veuf d'une Ludovisio sans enfants, épousa, en février
1675, Anne-Marie de La Trémoille, fille de M. de Noirmoutiers, qui
figura assez dans les troubles de la minorité de Louis XIV pour se faire
faire duc à brevet. Elle avait épousé Blaise de Talleyrand, qui se
faisait appeler le prince de Chalais, et qui fut de ce fameux duel
contre MM. de La Frette, où le frère aîné du duc de Beauvilliers fut
tué, et qui fit sortir les autres du royaume. M\textsuperscript{me} de
Chalais alla joindre son mari en Espagne, d'où ils passèrent en Italie.
Elle alla toujours devant à Rome, où la mort empêcha son mari de l'aller
trouver. Elle était jeune, belle, de beaucoup d'esprit avec beaucoup de
monde, de grâces et de langage\,; elle eut recours à Rome aux cardinaux
de Bouillon et d'Estrées, qui en prirent soin en faveur du nom et de la
nation, et bientôt après pour des raisons plus touchantes. Le désir de
la retenir à Rome, où ils étaient pour du temps, leur fit naître celui
de l'y établir. Elle n'avait point d'enfants, et presque point de bien.
Ils écrivirent à la cour qu'un homme de la considération dont était à
Rome le duc de Bracciano était bon à acquérir au roi, et que le moyen de
le lui attacher était de lui faire épouser M\textsuperscript{me} de
Chalais. La pensée fut approuvée et suivie. M. de Bracciano, tonnelé par
les deux cardinaux, se persuada qu'il était amoureux de
M\textsuperscript{me} de Chalais\,; il n'avait point d'enfants, le
mariage se fit, et la même année il fut fait chevalier de l'ordre.
M\textsuperscript{me} de Bracciano étala tout son esprit et tous ses
charmes à Rome, et fit bientôt du palais des Ursins une espèce de cour
où se rassemblait tout ce qu'il y avait de plus grand et de meilleure
compagnie en hommes et en femmes\,: c'était la mode d'y aller, et être
sur un pied de distinction d'y être reçu. Le mari cependant était compté
pour peu de chose. Le ménage ne fut pas toujours concordant, mais sans
brouillerie ouverte\,: ils furent quelquefois bien aises de se séparer.
C'est ce qui donna lieu à la duchesse de Bracciano de faire deux voyages
en France, au dernier desquels elle passa quatre ou cinq ans. C'est
celui où je la connus, et où je puis dire que je fis avec elle une
amitié particulière à l'occasion de celle qui était entre elle et ma
mère, dès son précédent voyage. Elle deviendra bientôt un personnage si
grandement singulier, que je me suis volontiers étendu sur elle.

Le cardinal de Bouillon, qui était lors à Rome en grande splendeur, lui
rendit le service d'empêcher, par l'autorité du pape, que les créanciers
très nombreux ne fussent reçus à mettre le scellé. M. de Bracciano
n'avait point d'enfants\,; sa femme, depuis son retour, l'avait tout à
fait regagné. Il l'avait faite, comme il est permis à Rome, sa légataire
universelle, et ses meubles, son argenterie, ses bijoux et ses
pierreries étaient infinis. Il n'y eut donc que ses terres qui purent
servir à payer les dettes. Don Livio Odescalchi, neveu d'Innocent XI,
extraordinairement riche, acheta pour près de deux millions le duché de
Bracciano, mais avec la condition expresse que M\textsuperscript{me} de
Bracciano en quitterait le nom, et c'est ce qui lui fit prendre celui de
princesse des Ursins, sous lequel elle est devenue fameuse.

Le cardinal de Bouillon, après ce service rendu pour le scellé, se
brouilla avec elle, mais aux couteaux tirés, et ne se sont jamais revus.
M\textsuperscript{me} de Bracciano, car elle en portait encore le nom,
prétendit tendre son palais de violet par un privilège particulier aux
aînés de la maison Ursine. Le cardinal de Bouillon, lors sous-doyen du
sacré collège, prit l'affirmative pour la faire tendre en noir, et avec
tant d'aigreur et de hauteur, que c'en a été pour le reste de leur vie.
Il en eut avec cela le dégoût tout entier\,; le pape le condamna et
donna gain de cause à M\textsuperscript{me} de Bracciano, qui ne tarda
pas à le rendre au cardinal de Bouillon.

Il venait de faire, par le pape, son neveu, l'abbé d'Auvergne, grand
prévôt du chapitre de Strasbourg, et lui-même s'en fit faire chanoine.
Il commençait, sans s'en apercevoir encore, à n'être plus si bien à la
cour. L'affaire de M. de Cambrai s'examinait fort sérieusement à Rome.
Il y avait ses gens, et ses antagonistes les leurs, avec le jeune abbé
Bossuet, neveu de M. de Meaux qui prit cette occasion de le former et de
le faire connaître. Le cardinal de Bouillon était de la congrégation où
cette affaire se jugeait\,; il se contint dans les commencements, et se
contenta de toutes les voies sourdes par lesquelles il put servir un
ami, auquel il avait de si puissants intérêts, comme je les ai expliqués
en leur temps\,; mais peu à peu le pied lui glissa, et ses manèges, que
MM. de Paris, de Meaux et de Chartres avaient tant de raisons de ne pas
cacher au roi, lui furent clairement démontrés. Le parti fut pris de
n'en pas faire semblant pour en découvrir davantage, et le mettre après,
à coup sûr, hors de combat pour la défense de son ami, et en user
cependant avec lui, du côté de la cour, avec toutes les apparences de la
distinction et de la confiance ordinaire.

Il était dans cette position lorsqu'il imagina un trait qui commença et
qui avança bien sa perte. L'empereur n'avait point de serviteur plus
zélé ni plus attaché entre les princes de l'empire que le duc de
Saxe-Zeitz, évêque de Javarin, et travaillait à Rome depuis assez
longtemps à le faire cardinal seul, et hors le temps de la promotion des
couronnes. Par la même raison, le roi s'y opposait de toutes ses forces,
et en avait fait mettre un article exprès dans les instructions du
cardinal de Bouillon. Vint l'abjuration de l'électeur de Saxe entre les
mains de l'évêque de Javarin pour se rendre éligible en Pologne.
L'évêque passa pour l'avoir converti. L'empereur fit sonner le plus haut
qu'il put à Rome le service d'avoir ramené à sa communion un électeur de
l'empire, chef et protecteur né de tous les protestants d'Allemagne, et
renouvela d'ardeur et d'instances à cette occasion pour la promotion de
l'évêque. Cette conjoncture parut d'autant plus favorable au cardinal de
Bouillon qu'il voyait le pape fort incliné à accorder à l'empereur sa
demande, et que le pape traitait le cardinal de Bouillon avec beaucoup
de ménagement\,; il crut donc qu'il n'y avait pas un moment à perdre
pour en profiter.

Il écrivit au roi tout ce qu'il put de plus exagéré sur les engagements
du pape à l'empereur, et sur la promotion de l'évêque de Javarin comme
instante, que, dans cette extrémité, tout ce qu'il avait pu faire pour
parer l'affront de voir donner un cardinal seul et \emph{motu proprio}
aux instances de l'empereur, malgré toutes celles du roi, avait été de
trouver moyen que la France en eût un en même temps\,; qu'il avait eu
toutes les peines imaginables à y réussir, mais à condition que ce
Français serait choisi par le pape, et que, pour éviter qu'il n'en prît
quelqu'un qui ne fût pas agréable au roi, il avait fait effort de tout
son crédit auprès du pape pour lui en faire accepter un, le plus attaché
au roi, et qui pût être en état et en âge de le servir longtemps\,; que
c'était l'abbé d'Auvergne, excepté lequel, le pape lui avait déclaré
qu'il n'en ferait aucun autre. Il joignit à cela tout ce qu'il crut
capable de faire avaler au roi, comme un service aussi adroit que
signalé, un mensonge qui pouvait passer pour unique en son genre. En
même temps, il dit au pape tout ce qu'il put pour lui persuader que,
dans la presse et le désir où il était de contenter l'empereur, il
croyait avoir obtenu de la bonté et de l'amitié dont le roi voulait bien
l'honorer, le plus grand point qu'il eût pu se proposer pour tirer Sa
Sainteté de la situation forcée où elle se trouvait, qui était de faire
condescendre le roi à la promotion de l'évêque de Javarin, en faisant en
même temps un Français, chose où, jusque-là, on n'avait pu parvenir à
amener le roi, mais qu'en même temps Sa Majesté n'y voulait consentir
que pour son neveu l'abbé d'Auvergne\,; que c'était tout ce qu'il avait
pu tirer du roi, et qu'il croyait par là avoir rendu un grand service au
roi et au pape, en le mettant en état de satisfaire l'empereur sans se
brouiller avec le roi, en faisant à la fois l'évêque de Javarin et
l'abbé d'Auvergne.

Il arriva, pour le malheur du cardinal de Bouillon, qu'un hameçon si
adroitement préparé n'eut pas l'effet qu'il s'était promis de sa
hardiesse. Le pape, qui, par les offices pressants qu'il recevait
d'ailleurs que du cardinal de Bouillon, de la part du roi, contre M. de
Cambrai, et qui était en même temps bien informé de la conduite de ce
cardinal tout en faveur du même prélat, quoique l'homme du roi à Rome,
ne pouvait ajuster deux choses si contradictoires. Il soupçonna de la
profondeur dans l'arrangement du discours et de la proposition du
cardinal de Bouillon, et surtout dans l'empressement qui lui échappa de
brusquer la promotion de l'évêque et de l'abbé, et cela lui fit prendre
le parti d'attendre d'ailleurs des nouvelles de France. D'autre part, le
roi fut surpris au dernier point de la dépêche du cardinal de Bouillon,
et comme il n'avait eu que trop d'occasions en sa vie de le connaître,
il ne douta point qu'il n'eût suggéré au pape un expédient si flatteur à
la vanité des Bouillon, mais si destructif de l'intérêt et des ordres du
roi contre la promotion de l'évêque de Javarin. Il entra en colère, et
en même temps en crainte que cette promotion se précipitât, et il fit
dépêcher un courrier au cardinal de Bouillon, par lequel, sans entrer en
aucun raisonnement, il réitéra ses ordres contre la promotion de
l'évêque de Javarin, et ajouta en même temps que, si contre toute
attente, et malgré toute représentation, le pape se déterminait à passer
outre, il s'opposait à ce qu'aucun Français, et particulièrement l'abbé
d'Auvergne, fût fait cardinal, à qui il défendait de l'accepter même
s'il était fait, sous peine de désobéissance. Outre cette dépêche au
cardinal de Bouillon, le courrier était chargé d'une autre portant mêmes
ordres au principal agent des évêques opposés à M. de Cambrai, avec
commandement de plus de l'aller tout sur-le-champ montrer au pape\,; ce
qui fut exécuté. Le pape alors se sut bon gré des soupçons qui l'avaient
fait différer, et le cardinal de Bouillon pensa mourir de honte, de
dépit et de rage. Le pape qui, en effet, était pressé de faire l'évêque
de Javarin, ne l'était pas au point où le cardinal de Bouillon l'avait
mandé pour faire agréer l'expédient qu'il avançait, et qui, plus
français en son âme qu'impérial, voyait l'extrême répugnance du roi pour
cet évêque, temporisa si bien qu'il mourut sans le faire cardinal, et
manifesta de plus en plus par cette conduite l'audacieux mensonge du
cardinal de Bouillon, que ce pape avait fait mander aussi au roi.

Trois mariages se suivirent de près à la fin de février et au
commencement de mars. Souvré, frère de Barbezieux et maître de la
garde-robe du roi, épousa la fille unique du feu marquis de Rebénac,
frère du marquis de Feuquières, à condition d'en porter les armes, et de
prendre le nom de Pas, dans les actes, qui est celui de MM. de
Feuquières. Il en eut beaucoup de biens, et la lieutenance générale du
gouvernement de Béarn et basse Navarre. Rebénac était fort honnête
homme, et fort employé et distingué dans les négociations.

Le vieux Seissac épousa la dernière sœur du second lit du duc de
Chevreuse, jeune et jolie, qui, avec peu de bien, le voulut malgré la
disproportion d'âge, dans l'espérance d'être bientôt veuve, et de jouir
des grands avantages de son contrat de mariage. C'était un homme de
grande qualité et de beaucoup d'esprit, que démentaient toutes les
qualités de l'âme. Il avait eu la charge de maître de la garde-robe du
roi de M. de Guitry, lorsque le roi fit pour lui la nouvelle charge de
grand maître de la garde-robe. Seissac était fort riche, fort gascon,
gros joueur et beaucoup du grand monde, mais peu estimé, et on se
défiait fort de son adresse au jeu.

Le roi, dans ces temps-là, jouait aussi fort gros jeu et c'était le
brelan qui était à la mode. Un soir que Seissac était de la partie du
roi, M. de Louvois vint lui parler à l'oreille. Un moment après le roi
donna son jeu à M. de Lorges, à qui il dit de le tenir, et de continuer
pour lui jusqu'à ce qu'il fuit revenu, et s'en alla dans son cabinet
avec M. de Louvois\,; dans cet intervalle Seissac fit une tenue à M. de
Lorges, qu'il jugea contre toutes les règles du jeu, puis un va-tout
qu'il gagna ne portant quasi rien. Le coup était fort gros. Le soir M.
de Lorges se crut oblige d'avertir le roi de ce qui s'était passé. Le
roi fit arrêter sans bruit le garçon bleu qui tenait le panier des
cartes et le cartier. Les cartes se trouvèrent pipées, et le cartier,
pour avoir grâce, avoua que c'était Seissac qui les lui avait fait
faire, et l'avait mis de part avec lui.

Le lendemain Seissac eut ordre de se défaire de sa charge et de s'en
aller chez lui. Au bout de quelques années il obtint la permission
d'aller en Angleterre. Il y joua plusieurs années, et gagna extrêmement.
À son retour il eut liberté de se tenir où il voudrait, hors de se
présenter devant le roi. Il s'établit à Paris où il tint grand jeu chez
lui. Après, Monsieur, à qui tout était bon pour le jeu, demanda
permission au roi pour que Seissac pût jouer avec lui à Paris et à
Saint-Cloud. Monseigneur, à la prière de Monsieur, obtint la même
permission pour Meudon, et de l'un à l'autre ces deux princes se le
firent accorder pour jouer à Versailles et de là à Marly, où, sur le
pied de joueur, il était à la fin de presque tous les voyages. C'était
un homme très singulier, qui comptait le mépris et les avanies pour
rien, et qui avait encore la fantaisie de ne porter le deuil de
personne. Il disait que cela l'attristait et n'était bon à rien, et le
soutint ainsi de ses plus proches toute sa vie. Ils le lui rendirent\,;
car lorsqu'il mourut, M. de Chevreuse ni pas un parent ne portèrent le
deuil de lui. Son nom, maintenant éteint, était Castelnau, non pas des
Castelnau du maréchal de France, mais il portait celui de
Clermont-Lodève, d'une héritière de cette maison anciennement éteinte,
qui en avait apporté les biens dans la sienne.

Le troisième mariage fut plus brillant et mieux assorti pour les âges.
Ce fut celui du comte d'Ayen avec M\textsuperscript{lle} d'Aubigné. Le
roi avait eu grande envie de la faire épouser au prince de Marsillac,
petit-fils de M. de La Rochefoucauld. Lui et M\textsuperscript{me} de
Maintenon ne s'aimaient point, et ne s'étaient jamais aimés. Il avait
été toujours fort bien avec M\textsuperscript{me} de Montespan, et
surtout avec M\textsuperscript{me} de Thianges dont il aimait encore les
enfants. Le roi s'en apercevait\,; il ne laissait pas de désirer que
cela fût autrement entre eux. Comme ils n'avaient jamais été brouillés,
et qu'ils n'avaient aucun rapport ensemble, l'embarras était la façon de
les mettre sur un autre pied, d'autant qu'il n'y avait rien à
l'extérieur, et qu'ils en savaient trop tous deux pour s'attaquer, et
n'avoir pas tous les ménagements possibles. M. de La Rochefoucauld, à
qui le roi en parla, n'y consentit que par respect et complaisance.
M\textsuperscript{me} de Maintenon, qui avait ses raisons pour un autre
choix, répondit au roi froidement. Tant de glaces des deux côtés
rebutèrent le roi qui n'en parla plus que faiblement à
M\textsuperscript{me} de Maintenon, pour lui demander à qui elle pouvait
donner la préférence sur un homme de la naissance, des biens et des
charges qu'aurait le prince de Marsillac. Elle lui proposa le comte
d'Ayen. À son tour le roi ne répondit pas comme M\textsuperscript{me} de
Maintenon l'eût désiré. Il n'aimait point M\textsuperscript{me} de
Noailles\,; elle avait trop d'esprit pour lui, et trop entrante et trop
intrigante\,; c'était la mettre dans leur sanctuaire intime, et le roi
avait peine à s'y résoudre. M\textsuperscript{me} de Maintenon qui se
voulait entièrement attacher M. de Paris, et à l'appui de l'affaire de
M. de Cambrai se frayer un chemin d'avoir part aux affaires de l'Église
et aux bénéfices surtout, qu'elle n'avait jamais pu entamer au P. de La
Chaise, tourna si bien le roi qui aimait M. de Noailles, et à le
rassurer sur ce qu'elle écarterait M\textsuperscript{me} de Noailles de
leurs particuliers, que le mariage fut agréé et tout aussitôt conclu.

M\textsuperscript{me} de Maintenon assura six cent mille livres sur son
bien après elle\,; elle en avait beaucoup plus, et point d'autre
héritière. Le roi donna trois cent mille livres comptant, cinq cent
mille livres sur l'hôtel de ville, pour cent mille livres de pierreries,
avec les survivances du gouvernement de Roussillon, Perpignan, etc., de
M. de Noailles de trente-huit mille livres de rente au soleil, et de
celui de Berry de M. d'Aubigné de trente mille livres de rente, et sur
le tout une place de dame du palais. La déclaration s'en fit le mardi 11
mars. Le lendemain M\textsuperscript{me} de Maintenon se mit sur son lit
au sortir de table, et les portes furent ouvertes aux compliments de
toute la cour. M\textsuperscript{me} la duchesse de Bourgogne, tout
habillée, y passa la journée tenant M\textsuperscript{lle} d'Aubigné
auprès d'elle, et faisant les honneurs comme une particulière chez une
autre. On peut juger si personne s'en dispensa, à commencer par
Monseigneur. On y accourut de Paris, et Monsieur qui y était vint
exprès. Le mardi, dernier mars, ils furent fiancés le soir à la
chapelle, M\textsuperscript{me} la duchesse de Bourgogne et toute la
cour aux tribunes, et la noce en bas. Tout ce qui en était avait vu le
roi chez M\textsuperscript{me} de Maintenon avant son souper. Le
lendemain tard dans la matinée, M\textsuperscript{me} de Maintenon vint
avec toute la noce à la paroisse, où M. de Paris dit la messe et les
maria, d'où ils allèrent tous dîner chez M. de Noailles, dans
l'appartement de M. le comte de Toulouse, qu'il lui avait prêté.
L'après-dînée, M\textsuperscript{me} de Maintenon, sur son lit, et la
comtesse d'Ayen, sur un autre dans une autre pièce joignante, reçurent
encore toute la cour. On s'y portait, tant la foule y était grande, mais
la foule du plus distingué. Le soir on soupa chez M\textsuperscript{me}
de Maintenon avec elle et M\textsuperscript{me} la duchesse de
Bourgogne, et les hommes dans une autre chambre. Après souper, on coucha
les mariés dans le même appartement. Le roi donna la chemise au comte
d'Ayen, et M\textsuperscript{me} la duchesse de Bourgogne à la mariée.
Le roi les vit au lit avec toute la noce, il tira lui-même leur rideau,
et leur dit pour bonsoir qu'il leur donnait à chacun huit mille livres
de pension. Le roi en même temps paya les dettes de M. de La
Rochefoucauld, qui se montaient fort haut\,; ainsi il ne perdit pas tout
au mariage de M\textsuperscript{lle} d'Aubigné, auquel j'ai oublié de
remarquer que M. et M\textsuperscript{me} d'Aubigné se trouvèrent et
furent à tout.

La joie de M. de La Rochefoucauld fut un peu troublée par la perte qu'il
fit de son frère l'abbé de Marsillac\,; je dis un peu, parce que
l'amitié n'était pas bien vive quoique bienséante. L'esprit, le bon
sens, le goût de la bonifie compagnie et la considération dégagée de
celle de la naissance, de la faveur et des places, étaient devenus dans
cette famille un apanage de cadets. Celui-ci et le feu chevalier de La
Rochefoucauld son frère, qui étaient tendrement unis, avaient pleinement
joui de ces avantages et de la douceur de beaucoup d'amis particuliers
dont ils furent fort regrettés. Ils étaient fort goutteux, et on ne les
voyait jamais à la cour. Ceux qui ont vu M. de La Rochefoucauld père
prétendaient que l'abbé de Marsillac en faisait fort souvenir dans ses
manières et dans la conversation. Ce même apanage se maintint dans la
deuxième génération M. de Liancourt le recueillit tout entier, et il ne
passa plus outre. Les abbayes de l'abbé de Marsillac furent sur-le-champ
données à l'abbé de La Rochefoucauld, qui en avait déjà beaucoup. Il
était oncle paternel de M. de La Rochefoucauld, et toutefois de son même
âge. Il aimait tant la chasse, que le nom d'abbé Tayaut lui en était
demeuré. M. de La Rochefoucauld, à qui dans leur temps de misère il
avait donné tout le sien, l'aimait avec une extrême tendresse et une
grande considération. Il le logeait et l'avait toujours partout avec lui
à la cour. C'était le meilleur homme, mais le plus court et le plus
simple qui fût sur terre, et de la meilleure santé. Ni lui ni l'abbé de
Marsillac n'étaient point dans les ordres.

M. le prince de Conti perdit son fils, le prince de La Roche-sur-Yon,
qui n'avait que quatre ans. Le roi en prit le deuil en noir. Il ne
portait point le deuil des enfants au-dessous de sept ans, et on ne
l'avait pas porté de ceux de lui et de la reine, mais il avait voulu
faire cet honneur là à M. du Maine pour un des siens, et n'osa pas après
cela ne le pas prendre de ceux des princes du sang. Il alla voir M. le
prince et M\textsuperscript{me} la princesse de Conti. Les princes du
sang s'y trouvèrent, et le reconduisirent jusque chez
M\textsuperscript{me} de Maintenon.

Fervaques mourut en ce même temps, en revenant de Bourbon. C'était un
vieux garçon, honnête homme, toujours galant, qui n'avait jamais été
marié, et qui avait acheté, il y a longtemps, du grand prévôt, le
gouvernement du Maine et du Perche qui vaut quatorze mille livres de
rente. Il était riche, quoique frère cadet de Bullion. Leur mère était
sœur aînée de la maréchale de La Mothe, qui vint demander au roi le
gouvernement pour Bullion, qui en offrait deux cent mille livres pour
celui qu'il lui plairait gratifier. Sur-le-champ il l'accorda, donna à
la maréchale douze mille livres d'augmentation de pension, et fit mander
à Rosen qui était à Paris qu'il lui donnait les deux cent mille livres
de Bullion. La paix lui avait fait perdre une assez bonne confiscation
que le roi lui avait donnée.

Il se trouva un testament de Fervaques par lequel, entre autres legs, il
donnait à la duchesse de Ventadour la jouissance, sa vie durant, d'une
terre de quatorze mille livres de rente, et malgré ces legs il revenait
fort gros à Bullion. Il avait été conseiller au parlement de Metz, après
avoir éprouvé à un siège qu'il n'était pas propre à la guerre, sans
avoir pourtant rien fait de malhonnête. On s'aperçut à un repas à la
tranchée qu'il ne mangeait point\,; on l'en pressa, il répondit
plaisamment qu'il ne mangeait jamais qu'il ne fût sûr de la digestion.
Il avoua franchement sa peur sans la témoigner autrement que par ses
paroles. Il quitta à la fin de la campagne et n'en fut pas moins estimé.

Son père était fils de Bullion, surintendant des finances et président à
mortier. Il fut président à mortier en survivante, et se laissa
persuader d'en donner la démission pour une place de conseiller
d'honneur et la charge de greffier de l'ordre. Son fils dont je parlais
tout à l'heure ne prit une charge de conseiller au parlement de Metz
qu'en passant. Il acheta la charge de prévôt de Paris, à l'ombre de
laquelle il reprit l'épée, et parut ainsi dans le monde et à Versailles.
Sa femme, qui, était une Rouillé, sœur de la marquise de Noailles puis
duchesse de Richelieu, enrageait de voir sa sœur femme de qualité. Elle
et son mari, sous prétexte de rendre des devoirs à la maréchale de La
Mothe et à la duchesse de Ventadour, sa fille, de chez qui ils ne
bougeaient, se fourraient tant qu'ils pouvaient partout.
M\textsuperscript{me} de Bullion était altière, glorieuse, impérieuse,
et ne supportait qu'avec peine d'être à la cour, parce qu'elle y voulait
aller, sans parvenir à être de la cour. De bien meilleures qu'elles ne
songeaient pas à manger ni à entrer dans les carrosses. Enfin, après de
longues douleurs, elle offrit si gros à M\textsuperscript{me} de
Ventadour, dame d'honneur de Madame, pour entrer dans son carrosse, que,
tentée de la somme, elle le dit franchement à Monsieur et à Madame qui,
par considération pour elle, y consentirent. M\textsuperscript{me} de
Bullion entra donc ainsi dans le carrosse de Madame, et soupa une fois
avec elle et Monsieur à Saint-Cloud, dont elle pensa mourir de joie,
mais elle en demeura là, et le roi n'en voulut jamais ouïr parler pour
manger, ni pour les carrosses de M\textsuperscript{me} la Dauphine. Un
gouvernement de province, quelque petit qu'il fût, était donc bien peu
de convenance à Bullion\,; et si son frère l'avait eu, au moins avait-il
servi, été capitaine d'une des compagnies de gendarmerie de la reine, et
n'avait jamais été de robe. Bullion et sa femme devaient donc tout à la
maréchale de La Mothe, et à M\textsuperscript{me} de Ventadour chez
lesquelles ils passaient leur vie. Malgré cela, M\textsuperscript{me} de
Bullion, aussi avare que riche et glorieuse, et c'est beaucoup dire, et
qui traitait son mari comme un petit garçon, lui fit attaquer le
testament de son frère, et faire un procès directement à
M\textsuperscript{me} de Ventadour sur l'usufruit que Fervaques lui
avait laissé. Cette infamie, et faite le lendemain du gouvernement du
Maine et du Perche, souleva contre elle et la cour et la ville à n'oser
plus se montrer nulle part. Elle soutint la gageure, se brouilla avec
ses protectrices, et perdit son procès avec toutes les sauces et avec
une acclamation générale. Question fut après de se raccommoder, et de
sortir par là de la sorte d'excommunication générale où elle était
tombée avec tout le monde. Cela dura quelques mois. À force de
soumissions qui lui coûtèrent bien cher, M\textsuperscript{me} de
Ventadour fut assez borine pour lui pardonner, et peu à peu il n'y parut
plus.

Le duc de Lesdiguières, qui était fort jeune et fort doux, et qui ne
tarda pas à montrer qu'il était aussi fort brave, eut quelques paroles
en sortant de la comédie avec Lambert, colonel d'infanterie, jeune homme
très suffisant, qui voulut porter ses plaintes aux maréchaux de France,
et qui ne savait apparemment pas que les ducs ne les reconnaissent
point. Le roi le sut, et ordonna à M. de Duras, beau-père de M. de
Lesdiguières, d'accommoder seul cette affaire, qui n'alla pas plus loin.

M. de Lorraine arriva à Strasbourg allant en Lorraine. Le marquis
d'Huxelles, commandant d'Alsace, l'y reçut moins comme un duc de
Lorraine qu'en neveu du roi qu'il allait être. M. d'Elbœuf se hâta de
l'aller voir. Il tint en revenant des propos peu mesurés qui revinrent
et déplurent fort à M. de Lorraine. Il en fut embarrassé, et essaya de
s'en justifier auprès du roi, à qui cela ne faisait pas grand'chose.
Quelque temps après il voulut retourner en Lorraine pour montrer qu'il
était bien en ce pays-là, malgré ce qui s'en était débité. Il n'osa
pourtant s'y hasarder sans en parler au roi, qui ne le lui conseilla
pas. C'était un homme audacieux et qui ne voulait pas avoir le démenti
d'un voyage qu'il avait annoncé\,; mais il l'eut tout du long. M. de
Lorraine, qui en fut averti, en fit parler au roi qui au conseil fit
succéder la défense, et M. d'Elbœuf demeura tout court. Bouzols,
beau-frère de Torcy, fut complimenter de la part du roi M. de Lorraine à
son arrivée.

Le roi désormais en pleine paix voulut étonner l'Europe par une montre
de sa puissance qu'elle croyait avoir épuisée par une guerre aussi
générale et aussi longue, et en même temps se donner, et plus encore à
M\textsuperscript{me} de Maintenon, un superbe spectacle sous le nom de
Mgr le duc de Bourgogne. Ce fut donc sous le prétexte de lui faire voir
une image de la guerre, et de lui en donner les premières leçons, autant
qu'un temps de paix le pouvait permettre, qu'il déclara un camp à
Compiègne qui serait commandé par le maréchal de Boufflers sous ce jeune
prince. Les troupes qui en grand nombre le devaient composer furent
nommées, et les officiers généraux choisis pour y servir. Le roi fixa
aussi en même temps celui qu'il comptait d'aller à Compiègne, et fit
entendre qu'il serait bien aise d'y avoir une fort grosse cour. Je
remets au temps de ce voyage à en parler plus particulièrement.

\hypertarget{chapitre-viii.}{%
\chapter{CHAPITRE VIII.}\label{chapitre-viii.}}

1698

~

{\textsc{P. La Combe à la Bastille.}} {\textsc{- Orage contre les ducs
de Chevreuse et de Beauvilliers et les attachés à M. de Cambrai.}}
{\textsc{- Sainte magnanimité du duc de Beauvilliers.}} {\textsc{-
Grande et prodigieuse action de l'archevêque de Paris.}} {\textsc{-
Quatre domestiques principaux des enfants de France chassés et
remplacés, et le frère de M. de Cambrai cassé.}} {\textsc{- M. de Meaux
consulte M. de la Trappe sur M. de Cambrai, publie {[}sa lettre{]} à son
insu, et le brouille pour toujours avec cet archevêque et avec ses
amis.}} {\textsc{- Duchesse de Béthune, principale amie de
M\textsuperscript{me} Guyon.}} {\textsc{- Complaisance des ducs de
Chevreuse et de Beauvilliers pour moi sur M. de la Trappe.}} {\textsc{-
Plaisante et fort singulière aventure entre le duc de Charost et moi sur
M. de Cambrai et}}

~

M. de la Trappe. --- Caretti, empirique, devient grand seigneur.

Cependant l'affaire de M. de Cambrai était à la cour dans une grande
effervescence\,; les écrits de part et d'autre se multipliaient. Le P.
La Combe fut mis à la Bastille, duquel on publia qu'on découvrit
d'étranges choses. M\textsuperscript{me} de Maintenon avait levé le
masque, et conférait continuellement avec MM. de Paris, de Meaux et de
Chartres. Ce dernier ne pouvait pardonner à M. de Cambrai le projet bien
avéré de lui avoir voulu enlever M\textsuperscript{me} de Maintenon
jusque dans son retranchement de Saint-Cyr\,; et les Noailles, si
nouvellement unis à elle par leur mariage, avaient auprès d'elle les
grâces de la nouveauté auxquelles elle ne résistait jamais. Son dessein
de porter M. de Paris dans la confiance de la distribution des
bénéfices, pour énerver le P. de La Chaise qu'elle n'aimait ni sa
société, et de s'introduire dans ce nouveau crédit à l'appui de celui de
l'archevêque, lui faisait embrasser tout ce qui pouvait l'y porter, et
par conséquent une cause dont il était une des parties principales, et
la rendait ennemie de tout ce qui la pouvait contre-balancer auprès du
roi. Les ducs de Chevreuse et de Beauvilliers, et leurs femmes, tenaient
directement à lui par une faveur ancienne qui avait fait naître la
confiance, et qui était fondée sur l'estime et sur une continuelle
expérience de leur vertu. Cette habitude, qui jusqu'alors les avait
rendus les plus florissants et les plus considérés de la cour, avait
contenu l'envie.

Il était question d'un effort pour déprendre le roi d'eux.
M\textsuperscript{me} de Maintenon, entraînée par M. de Chartres, et
piquée de la conduite indépendante d'elle des deux ducs sur les
\emph{Maximes des saints}, que l'un avait corrigées chez l'imprimeur,
l'autre directement présentées au roi en particulier, consentait à leur
perte, et le duc de Noailles, qui songeait à s'assurer la dépouille de
M. de Beauvilliers, poussait incessamment à la roue. Il ne voulait pas
moins que la charge de gouverneur des enfants de France, celle de chef
du conseil des finances, et celle de ministre d'État. Il sentait que si
le roi pouvait se laisser persuader, sous prétexte du danger de la
doctrine et de la confiance, d'ôter ses petits-fils à Beauvilliers, il
n'était plus possible qu'il pût demeurer à la cour, et que, par
nécessité, les deux autres places seraient en même temps vacantes, et
que toutes trois ne pouvaient guère que le regarder, dans l'heureuse et
nouvelle position où il se trouvait. Les difficultés qui se
rencontraient et qui se multipliaient à Rome sur la condamnation de M.
de Cambrai, et la conduite qu'y tenait le cardinal de Bouillon, malgré
des ordres si contraires, aigrissait la cabale au dernier point, et
devint enfin le moyen qu'elle mit en œuvre pour culbuter les ducs de
Chevreuse et de Beauvilliers.

M\textsuperscript{me} de Maintenon la proposa au roi comme un moyen
auquel il était obligé en conscience, pour le succès de la bonne cause,
et ôter à la mauvaise les appuis qu'elle faisait valoir à Rome, où on ne
pouvait croire que, s'il était aussi convaincu qu'il voulait qu'on le
crut des opinions de MM. de Paris, de Meaux et de Chartres, contre celle
de M. de Cambrai, il ne laisserait pas le plus grand protecteur et le
plus déclaré de la dernière, dans les places de son conseil, beaucoup
moins dans celle de gouverneur de ses petits-fils, avec un nombre de
subalternes qu'il y avait mis, et qui étaient tous dans cette même
doctrine\,; que cette apparence si plausible, soutenue des démarches du
cardinal de Bouillon, donnait un poids à Rome, qui embarrassait le
pape\,; qu'il en répondrait devant Dieu s'il laissait plus longtemps un
si grand obstacle, et qu'il était temps de le renverser, et de montrer
au pape par cet exemple qu'il n'avait aucune sorte de ménagement à
garder.

Tout jeune que j'étais, je fus assez instruit pour tout craindre.
M\textsuperscript{me} de Maintenon était pleine jusqu'à répandre. Il lui
échappait des imprudences dans les particuliers\,; elle en lâchait à
M\textsuperscript{me} la duchesse de Bourgogne, et quelquefois devant
des dames du palais. Elle savait que la comtesse de Roucy n'avait jamais
pardonné à M. de Beauvilliers d'avoir été pour M. d'Ambres contre elle
dans un procès où il y allait de tout pour sa mère et pour elle, et
qu'elle gagna. L'orage grondait\,; les courtisans s'en aperçurent\,; les
envieux osèrent pour la première fois lever la tête\,:
M\textsuperscript{me} de Roucy, âpre à la vengeance, et plus encore à
faire bassement sa cour à M\textsuperscript{me} de Maintenon, ne perdait
point de moments particuliers, et en remportait toujours quelque chose,
et elle en triomphait assez pour avoir l'imprudence de me le confier,
quoiqu'elle n'ignorât pas ma liaison intime, tant la haine a
d'aveuglement. Je recueillais tout avec soin\,; je le conférais en
moi-même avec d'autres connaissances\,; j'en raisonnais avec Louville à
qui Pomponne, ami intime des deux ducs, se déplorait ouvertement, et
apprenait tout ce qu'il découvrait. Louville, à ma prière, avait plus
d'une fois parlé à M. de Beauvilliers\,; M. de Pomponne, de son côté, ne
s'y était pas oublié, et tout avait été inutile. Il ignorait ce dernier
et extrême danger\,; personne n'avait osé lui en montrer le détail\,; il
ne le voyait qu'en gros. Je me résolus donc à le lui faire toucher, et à
ne lui rien cacher de tout ce que j'avais découvert et que je viens
d'écrire.

J'allai donc le trouver, j'exécutai mon dessein dans toute son étendue,
et j'ajoutai, comme il était vrai, que le roi était fort ébranlé. Il
m'écouta sans m'interrompre et avec beaucoup d'attention. Après m'avoir
remercié avec tendresse, il m'avoua que lui, son beau-frère et leurs
femmes s'apercevaient depuis longtemps de l'entier changement de
M\textsuperscript{me} de Maintenon, de celui de la cour, et même de
l'entraînement du roi. J'en pris occasion de le presser d'avoir moins
d'attachement, au moins en apparence, pour ce qui l'exposait si fort, de
montrer plus de complaisance, et de parler au roi. Il fut
inébranlable\,: il me répondit sans la moindre émotion qu'à tout ce
qu'il lui revenait de plusieurs côtés, il ne doutait point qu'il ne fût
dans le péril que je venais de lui représenter\,; mais qu'il n'avait
jamais souhaité aucune place\,; que Dieu l'avait mis en celles où il
était\,; que quand il les lui voudrait ôter, il était tout prêt de les
lui remettre\,; qu'il n'y avait d'attachement que pour le bien qu'il y
pouvait faire\,; que, n'en pouvant plus procurer, il serait plus que
content de n'avoir plus de compte à en rendre à Dieu, et de n'avoir plus
qu'à le prier dans la retraite où il n'aurait à penser qu'à son salut\,;
que ses sentiments n'étaient point opiniâtreté\,; qu'il les croyait
bons, et que les pensant tels, il n'avait qu'à attendre la volonté de
Dieu, en paix et avec soumission, et se garder surtout de faire la
moindre chose qui pût lui donner du scrupule en mourant. Il m'embrassa
avec tendresse, et je m'en allai si pénétré de ces sentiments si
chrétiens, si élevés et si rares, que je n'en ai jamais oublié les
paroles, tant elles me frappèrent, et que si je les racontais à cent
fois différentes, je crois que je les redirais toutes et dans le même
arrangement que je les entendis.

Cependant l'orage arriva au point de maturité, et en même temps un autre
prodige. Les Noailles se servaient bien de M. de Paris pour persuader au
roi par conscience un éclat qui retentit jusqu'à Rome, et d'ôter
d'auprès des princes tout mauvais levain\,; mais ni le mari ni la femme
n'osèrent jamais lui confier leur but\,: il était trop homme de bien\,;
ils le connaissaient\,: ils auraient craint de lui égarer la bouche, et
Dieu permit qu'il en devint l'arbitre. Le roi, poussé par les trois
évêques sur le gros de l'affaire, et pressé en détail par
M\textsuperscript{me} de Maintenon qui, serrant la mesure, lui avait
proposé le duc de Noailles pour toutes places du duc de Beauvilliers, ne
tenait plus à ce dernier que par un filet d'ancienne estime et
d'habitude, qui cependant le retenait assez pour le peiner. Dans ce
tiraillement, il ne put se décider lui-même, et voulut consulter un des
trois prélats. Qu'il ne choisit pas l'évêque de Chartres, sa défiance
sur son attachement personnel à M\textsuperscript{me} de Maintenon, qui
le ferait penser tout comme elle, put aisément l'en détourner. Mais M.
de Meaux n'avait pas le même inconvénient à craindre. Il était accoutumé
à lui ouvrir son cœur sur ses pensées de conscience, et de son
domestique intérieur les plus secrètes. M. de Meaux avait conservé les
entrées et la confiance que lui avait données sa place de précepteur de
Monseigneur. Il avait été le seul témoin des différents combats, et à
différentes reprises, qui avaient séparé le roi de M\textsuperscript{me}
de Montespan. M. de Meaux seul en avait eu le secret, et y avait porté
tous les coups. Malgré tant d'avances, tant d'habitudes, tant d'estime,
on ne sait ce qui put l'exclure de la préférence de cette importante
consultation, et ce qui la fit donner à celui des trois qui portait son
exclusion naturelle par être frère de celui à qui, si M. de Beauvilliers
était perdu, toute sa dépouille était dès lors destinée. Néanmoins,
quoique de connaissance plus nouvelle même que M. de Chartres, puisqu'il
n'avait jamais approché du roi que depuis qu'il était archevêque de
Paris, ce fut lui que le roi préféra. Il se trouva dans ces temps où
l'impression de tout ce qui avait été dit au roi pour le faire
archevêque de Paris, et tout ce qu'il en avait remarqué depuis, l'avait
puissamment frappé d'une estime qui lui ouvrait le cœur pour tout ce qui
regardait la conscience, qu'il ne répandait alors plus volontiers que
dans son sein. Aucune réflexion sur ce qu'il était à M. de Noailles ne
le retint. Il lui fit sa consultation si entière, qu'elle alla jusqu'à
lui dire qu'en cas qu'il se défit de M. de Beauvilliers, c'était au duc
de Noailles à qui il s'était déterminé de donner toutes ses places.

Si M. de Paris y eût consenti, dans l'instant même, la perte de l'un et
l'élévation de l'autre était déclarée. Mais si la vertu et le
détachement de M. de Beauvilliers m'avaient pénétré d'admiration et de
surprise, celles de l'archevêque de Paris furent, s'il se peut, encore
plus admirables, puisqu'il y a peut-être moins à faire pour s'abandonner
humblement à la chute, et ne s'en vouloir garantir par rien de peur de
s'opposer à la volonté de Dieu, qu'il n'y a à prendre sur soi pour
conserver dans les plus grandes places le protecteur de son adversaire,
et d'une cause qu'on a si solennellement entrepris de faire condamner,
et devenir sciemment l'obstacle de la plus grande fortune d'un frère
avec qui on est parfaitement uni, et des établissements de sa maison les
plus éclatants et les plus solides. C'est là pourtant ce que sans
balancer fit l'archevêque de Paris. Il s'écria sur la pensée du roi
comme passant le but, lui représenta avec force la vertu, la candeur, la
droiture de M. de Beauvilliers, la sécurité où le roi devait être à son
égard pour ses petits-fils, le tort extrême que cette chute ferait à sa
réputation, {[}au point d'{]} attirer dans Rome un dangereux blâme à la
bonne cause, par celui qu'y encourraient ceux qui seraient si
naturellement soupçonnés de l'avoir opérée. Il conclut à ôter d'auprès
des princes quelques subalternes dont on n'était pas si sûr et dont la
disgrâce ferait voir à Rome la partialité et les soins du roi, sans
faire un éclat aussi préjudiciable et aussi scandaleux que serait celui
d'ôter M. de Beauvilliers.

Ce fut ce qui le sauva, et le roi en fut fort aise\,: le fond d'estime
et la force de l'habitude n'avait pu être arraché par tous les soins que
M\textsuperscript{me} de Maintenon avait pris d'en venir à bout, et par
elle-même et par tout ce qu'elle avait pu y employer d'ailleurs. Il en
fut de même à divers degrés du duc de Chevreuse que la chute de M. de
Beauvilliers eût entraîné, et que sa conservation raffermit\,; et le
roi, rassuré sur le point de la conscience par un homme en qui sur ce
point il avait mis sa confiance, et qui de plus s'y trouvait aussi
puissamment intéressé, respira et devint inaccessible aux coups qu'à
l'appui de cette affaire on voulut leur porter désormais. Mais l'orage
tomba sur les autres sans que M. de Beauvilliers trop suspect à leur
égard les pût sauver.

Ce fut pourtant avec lui-même que le roi décida leur disgrâce. Il fut
longtemps seul avec lui le matin du lundi 2 juin avant le conseil, et
l'après-dînée on sut que l'abbé de Beaumont, sous-précepteur\,; l'abbé
de Langeron, lecteur\,; Dupuis et L'Échelle, gentilshommes de la manche
de Mgr le duc de Bourgogne, étaient chassés sans aucune conservation
pécuniaire, et Fénelon, exempt des gardes du corps, cassé, sans autre
faute que le malheur d'être frère de M. de Cambrai. On apprit tout de
suite que M. de Beauvilliers avait ordre de présenter au roi un mémoire
des sujets qu'il croirait propres à remplir les quatre places auprès des
princes.

Rien ne marqua plus la rage de la cabale que Fénelon cassé, qui, par son
emploi, n'approchait point des princes, et dont la doctrine assurément
était nulle. Aussi M\textsuperscript{me} de Maintenon fut-elle outrée de
s'être vue toucher au but, pour n'avoir plus d'espérance contre des gens
qui, échappés de ce naufrage, ne pouvaient plus être attaqués, ni donner
sur eux aucune prise. Aussi ne leur pardonna-t-elle jamais\,; mais, en
habile femme, elle sut prendre son parti, ployer sous le joug du roi, et
vivre peu à peu, à l'extérieur au moins, honnêtement avec d'anciens
amis, puisqu'elle n'avait pu les perdre. M. de Noailles fut encore plus
outré qu'elle et fut longtemps en grand froid avec son frère.
M\textsuperscript{me} de Noailles n'en était pas moins affligée, mais
elle en savait trop pour ne pas sentir les conséquences de cette
brouillerie domestique. Elle mit donc tous ses soins d'abord pour
empêcher le plus qu'elle put qu'on ne s'en aperçût, ensuite pour les
raccommoder, à quoi il fallut bien que son mari en vînt. Le maréchal de
Villeroy, M. de La Rochefoucauld, un gros d'envieux qui chacun à sa
façon avait poussé à la roue, et qui, ravis de la chute des deux
beaux-frères, auraient encore été plus piqués d'en voir profiter M. de
Noailles, furent désolés d'un si grand coup manqué, et par leur
jalousie, et par leur espérance sur la dépouille. M\textsuperscript{me}
la duchesse de Bourgogne qui, à force de n'être occupée qu'à plaire au
roi et à M\textsuperscript{me} de Maintenon, prenait, en jeune personne,
toutes les impressions que lui donnait cette tante si factice, et qui ne
cachait pas toujours celles qu'elle avait prises, parut {[}avoir{]}
depuis cette époque un grand éloignement pour MM. et
M\textsuperscript{me}s de Chevreuse et de Beauvilliers, à travers tous
les ménagements que le goût du roi lui imposait, et plus encore l'amitié
tendre et toute l'intime confiance de Mgr le duc de Bourgogne pour eux.

Ce qui acheva d'ôter toute espérance à la cabale qui les avait voulu
perdre fut de voir deux jours après les quatre places vacantes chez les
princes remplies de quatre hommes proposés par M. de Beauvilliers, les
abbés Le Fèvre et Vittement, Puységur et Montriel. Vittement dut ce
choix à son mérite, et à la beauté de la harangue qu'il avait faite au
roi sur la paix, à la tête de l'Université dont il était alors recteur,
et qui fut universellement admirée. Louville conseilla au duc de
Beauvilliers les deux gentilshommes de la manche. Il avait été avec eux
dans le régiment d'infanterie du roi, capitaine. Puységur en était
lieutenant-colonel, et par là fort connu du roi. Il l'était extrêmement
de tout le monde parce qu'il avait été l'âme de toutes les campagnes de
M. de Luxembourg toute la dernière guerre. Outre ces fonctions de
maréchal des logis de l'armée qu'il faisait avec grande étendue et
grande supériorité, il soulageait M. de Luxembourg pour tous les autres
ordres de l'armée, il avait la principale part à ses projets de campagne
et à leur exécution, et la confiance en lui était telle que M. de
Luxembourg ne se cachait pas de ne rien penser et de ne rien faire pour
la guerre sans lui. Montriel, ancien capitaine au même régiment, était
fort attaché à Puységur, et tous deux fort amis de Louville et très
propres à cet emploi auprès d'un prince dont l'âge demandait désormais
plus d'application pour les choses du monde, et surtout de la guerre que
pour celles de l'étude.

En même temps que ces amis de M. de Cambrai furent chassés,
M\textsuperscript{me} Guyon fut transférée de Vincennes, où était le P.
La Combe, à la Bastille, et sur ce qu'on lui mit auprès d'elle deux
femmes pour la servir, peut-être pour l'espionner, on crut qu'elle était
là pour sa vie. Cet éclat ne laissa pas de porter fortement sur les ducs
de Chevreuse et de Beauvilliers, et sur leurs épouses. À Versailles où
ils vivaient fort peu avec le monde, cela ne parut guère. Mais le jeudi
suivant, octave de la Fête-Dieu, c'est-à-dire le quatrième jour après
l'éclat, le roi alla à Marly\,; ils essuyèrent une désertion presque
générale\,; M. de Beauvilliers, qui était en année, servait jusqu'au
dîner inclus, et le marquis de Gesvres achevait toujours les journées.

Tout était local. À Versailles, le service était précis et réglé, et les
grandes entrées attendaient dans les cabinets quand ils avaient à
attendre. À Marly, où le roi n'en avait que deux, et encore à peine,
nulle grande entrée n'y mettait le pied. Il fallait attendre dans la
chambre du roi, ou dans les salons, mêlé avec tout le courtisan, et
cette attente prenait une grande partie de la matinée, lorsqu'il n'y
avait pas conseil qui y était bien moins fréquent qu'à Versailles. Pour
les dames, les plus retirées partout ailleurs ne le pouvaient guère être
à Marly. Elles s'assemblaient pour le dîner, presque jusqu'au souper
elles demeuraient dans le salon, et par-ci par-là, les distinguées dans
la première pièce de l'appartement de M\textsuperscript{me} de
Maintenon, où elle ni le roi ne se tenaient pas, mais où elles le
voyaient passer plus à leur aise, et mieux remarquées.

M\textsuperscript{me}s de Chevreuse et de Beauvilliers, accoutumées à
voir l'élite des dames se ramasser autour d'elles partout, se trouvèrent
tout ce voyage-là et quelques autres ensuite fort esseulées. Personne ne
les approcha celui-ci, et si le hasard, ou quelque soin, en amenait
auprès d'elles, c'était sur des épines, et elles ne cherchaient qu'à se
dissiper, ce qui arrivait bientôt après. Cela parut bien nouveau et
assez amer aux deux sœurs\,; mais, semblables à leurs maris en vertus et
en bienséances, elles ne coururent après personne, se tinrent
tranquilles, virent sans dédain ce flux de la cour, mais sans paraître
embarrassées, reçurent bien le peu et le rare qui leur vint, mais sans
empressement, et à leur façon ordinaire, et surtout sans rien chercher,
et ne laissèrent pas de bien remarquer et distinguer les différentes
allures, et tous les degrés de crainte, de politique ou d'éloignement.
Leurs maris aussi courtisés, et encore plus environnés qu'elles,
éprouvèrent encore plus d'abandon, et ne s'en émurent pas davantage.
Tout cela eut un temps, et peu à peu, on se rapprocha d'eux et d'elles,
parce qu'on vit le roi les traiter avec la même distinction, et que la
même politique qui avait éloigné d'eux le gros du monde l'en rapprocha
dans les suites, et que l'envie, lasse de bouder inutilement, fit enfin
comme les autres.

Pendant ces dégoûts, La Reynie interrogea plusieurs fois
M\textsuperscript{me} Guyon et le P. La Combe. Il se répandit que ce
barnabite disait beaucoup, mais que M\textsuperscript{me} Guyon se
défendait avec beaucoup d'esprit et de réserve. Les écrits continuaient.
Le roi loua publiquement l'histoire de toute cette affaire, que M. de
Meaux lui avait présentée, et dit qu'il n'y avait pas avancé un mot qui
ne fût vrai. M. de Meaux était ce voyage-là fort brillant à Marly, et le
roi avait chargé le nonce d'envoyer ce livre au pape. Rome fut agitée de
tout cet éclat. L'affaire qui dormait un peu à la congrégation du
Saint-Office, où elle avait été renvoyée, reprit couleur, et couleur qui
commença à devenir fort louche pour M. de Cambrai.

Dans ces entrefaites, il arriva une chose qui ne laissa pas de
m'importuner. M. de Meaux était anciennement ami de M. de la Trappe\,:
il l'était allé voir quelquefois, et ils s'écrivaient de temps en
temps\,; ils s'aimaient et ils s'estimaient encore davantage. M. de
Meaux, dans les premières crises de la dispute, lui envoya ses premiers
écrits, ceux que M. de Cambrai publia d'abord, et en même temps les
\emph{Maximes des saints\,;} il le pria d'examiner ces différents
ouvrages, et, sans en faire un lui-même dont il n'avait ni le temps ni
la santé, de lui mander franchement, et en amitié, ce qu'il en pensait.
M. de la Trappe lut attentivement tout ce que M. de Meaux lui avait
envoyé. Tout savant et grand théologien qu'il fût, le livre des
\emph{Maximes des saints} l'étonna et le scandalisa beaucoup. Plus il
l'étudia, et plus ces mêmes sentiments le pénétrèrent. Il fallut enfin
répondre après avoir bien examiné. Il crut répondre en particulier et à
son ami\,; il compta qu'il n'écrivait qu'à lui, et que sa lettre ne
serait vue de personne. Il ne la mesura donc point comme on fait un
jugement, et il manda tout net à M. de Meaux, après une dissertation
fort courte, que si M. de Cambrai avait raison, il fallait briller
l'Évangile, et se plaindre de Jésus-Christ, qui n'était venu au monde
que pour nous tromper. La force terrible de cette expression était si
effrayante, que M. de Meaux la crut digne d'être montrée à
M\textsuperscript{me} de Maintenon, et M\textsuperscript{me} de
Maintenon, qui ne cherchait qu'à accabler M. de Cambrai de toutes les
autorités possibles, voulut absolument qu'on imprimât cette réponse de
M. de la Trappe à M. de Meaux.

On peut imaginer quel triomphe d'une part, et quels cris perçants de
l'autre. M. de Cambrai et ses amis n'eurent pas assez de voix ni de
plumes pour se plaindre, et pour tomber sur M. de la Trappe, qui de son
désert osait anathématiser un évêque, et juger de son autorité, et de la
manière la plus violente et la plus cruelle, une question qui était
déférée au pape, et qui était actuellement sous son examen. Ils en
firent même faire des reproches amers à M. de la Trappe\,; et de là,
éclatèrent contre lui.

M. de la Trappe fut très affligé de l'impression de sa lettre, et de se
voir sur la scène, au moment qu'il s'en était le moins défié. Il prit le
parti d'écrire une seconde lettre à M. de Meaux, et en même temps de la
publier. Il lui faisait des reproches, mais comme un ami, d'avoir
communiqué sa réponse sur sa dispute avec M. de Cambrai, qu'il lui avait
écrite avec ouverture de cœur, dans sa confiance accoutumée de leur
commerce de lettres, que celle-ci serait brûlée aussitôt qu'elle aurait
été lue\,; qu'il était affligé avec amertume de la peine qu'il apprenait
de toutes parts qu'elle causait à des personnes dont il avait toujours
particulièrement honoré les vertus, les places et les personnes, qu'il
l'était encore davantage du bruit qu'il lui revenait que faisait sa
réponse, lui qui depuis tant d'années ne cherchait qu'à être oublié, qui
dans aucun temps n'était entré dans aucune affaire de l'Église, et qui,
en les évitant toutes, ne s'était vu forcé, qu'avec un très grand
déplaisir, à se défendre sur des questions monastiques de son état qui
l'avaient conduit plus loin qu'il n'aurait voulu, mais qu'il n'avait pu
abandonner en conscience\,; qu'il était vrai que ce qu'il lui avait
mandé sur M. de Cambrai, il l'avait pensé, et qu'il le penserait
toujours\,; niais que, sans penser autrement ni chercher le moins du
monde à se déguiser, surtout sur des points de doctrine, où il se serait
tu s'il avait pu craindre de se voir imprimer, parce que son partage
était la retraite et le silence, ou, s'il avait été forcé à s'expliquer,
il l'aurait fait au moins dans des termes mesurés, convenables à être
publiés, et propres à répondre à sa vénération pour l'épiscopat, et en
particulier au respect qu'il avait pour la personne, la vertu et le
savoir de M. de Cambrai, et que l'entière différence de sentiment où il
était de lui ne devait pas altérer pour sa dignité dans l'Église, ni
pour sa personne. C'était là dire, ce semble, tout ce qu'il était
possible de plus satisfaisant, et c'était à M. de Meaux, et plus encore
à M\textsuperscript{me} de Maintenon, qu'il s'en fallait prendre, qui
avaient commis une si grande infidélité pour exciter tout ce fracas.
Mais M. de Cambrai et ses amis, à bout de colère contre leur
persécutrice, et d'écrits faits et à faire au fond contre M. de Meaux,
ne se contentèrent de rien, et ne le pardonnèrent de leur vie à M. de la
Trappe.

Il arrive quelquefois aux plus gens de bien de diviniser certaines
passions, et telle est la faiblesse de l'homme. J'étais passionnément
attaché à M. de la Trappe\,; je l'étais intimement à M. de Beauvilliers,
et fort à M. de Chevreuse\,; ils ne se cachaient de rien devant moi, et
quelquefois il leur échappait des amertumes sur M. de la Trappe, que
j'aurais voulu ne pas entendre. Je me souviens qu'ayant dîné en
particulier chez M. de Beauvilliers, il nous proposa à M. de Chevreuse,
au duc de Béthune et à moi, une promenade en carrosse autour du canal de
Fontainebleau. La duchesse de Béthune était la grande âme du petit
troupeau, l'amie de tous les temps de M\textsuperscript{me} Guyon, et
celle devant qui M. de Cambrai était en respect et en admiration, et
tous ses amis en vénération profonde. Le petit troupeau avait donc réuni
dans une liaison intime la fille de M. Fouquet et les filles de M.
Colbert\,; et le duc de Béthune, qui n'allait pas en ce genre à la
cheville du pied de sa femme, était, à cause d'elle, fort recueilli des
deux ducs et des deux duchesses. À peine fûmes-nous vers le canal, que
le bonhomme Béthune mit la conversation sur M. de la Trappe à propos de
M. de Cambrai, dont on parlait\,; les deux autres suivirent, et tous
trois se lâchèrent tant et si bien, qu'après avoir un peu répondu, puis
gardé le silence pour ne les pas exciter encore davantage, je sentis que
je ne pouvais plus supporter leurs propos. Je leur dis donc naïvement
que je sentais bien que ce n'était pas à moi, à mon âge, à exiger qu'ils
se tussent, mais qu'à tout âge on pouvait sortir d'un carrosse\,; que je
les assurais que je ne les en aimerais et ne les en verrais pas moins,
en ajoutant que c'était pour moi la dernière épreuve où mon attachement
pût être mis, mais que je leur demandais l'amitié d'avoir aussi égard à
rua faiblesse s'ils voulaient l'appeler ainsi, et de me mettre pied à
terre, après quoi ils diraient tout ce qu'ils voudraient en pleine
liberté. MM. de Chevreuse et de Beauvilliers sourirent. «\,Eh bien\,!
dirent-ils, nous avons raison, mais nous n'en parlerons plus,\,» et
firent taire le duc de Béthune, qui voulait toujours bavarder.
J'insistai, et sans fâcherie, à sortir pour les laisser à leur aise.
Jamais ils ne le voulurent souffrir, et ils eurent cette amitié pour moi
que jamais depuis je ne leur en ai ouï dire un mot. Pour le bonhomme
Béthune il n'était pas si maître de lui, mais comme aussi je ne m'en
contraignais pas comme pour les deux autres, je lui répondais de façon
que c'en était pour longtemps.

Encore ce mot pour sa singularité\,: le duc de Charost, son fils, ne
bougeait de chez moi, et était intimement de mes amis\,; il était aussi
un des premiers tenants du petit troupeau, et, comme tel, protégé des
ducs de Chevreuse et de Beauvilliers, qui nous avaient liés ensemble,
mais qui ne lui parlaient jamais de quoi que ce soit, que des affaires
de leur communion. Par même raison Charost était infatué à l'excès de M.
de Cambrai, et fort aliéné de M. de la Trappe. Nous badinions et
plaisantions fort ordinairement ensemble, et de temps en temps il se
licenciait avec moi sur M. de la Trappe. Je l'avertis plusieurs fois de
laisser ce chapitre, que tout autre je l'abandonnais à tout ce qu'il
voudrait dire, et en badinerais avec lui, mais que celui-là était plus
fort que moi, et que je le conjurais d'épargner ma patience et les
sorties que je ne pourrais retenir. Malgré ces avis très souvent
réitérés, il se mit sur ce chapitre à Marly dans la chambre de
M\textsuperscript{me} de Saint-Simon où nous avions dîné et où il
n'était resté que M\textsuperscript{me}s du Châtelet et de Nogaret avec
nous. Je parlai d'abord, je le fis souvenir après de ce que je lui avais
tant de fois répété\,; il poussa toujours sa pointe, et de propos en
propos de plaisanterie fort aigre, et où il ne se retenait plus, il me
lâcha avec un air de mépris pour M. de la Trappe que c'était mon
patriarche devant qui tout autre n'était rien. Ce mot enfin combla la
mesure. «\,Il est vrai, répondis-je d'un air animé, que ce l'est, mais
vous et moi avons chacun le nôtre, et la différence qu'il y a entre les
deux, c'est que le mien n'a jamais été repris de justice.\,» Il y avait
déjà longtemps que M. de Cambrai avait été condamné à Rome. À ce mot,
voilà Charost qui chancelle (nous étions debout), qui veut répondre, et
qui balbutie\,; la gorge s'enfle, les yeux lui sortent de la tête, et la
langue de la bouche. M\textsuperscript{me} de Nogaret s'écrie,
M\textsuperscript{me} du Châtelet saute à sa cravate qu'elle lui défait
et le col de sa chemise, M\textsuperscript{me} de Saint-Simon court à un
pot d'eau, lui en jette et tâche de l'asseoir et de lui en faire avaler.
Moi, immobile, je considérais le changement si subit qu'opère un excès
de colère et un comble d'infatuation, sans toutefois pouvoir être
mécontent de ma réponse. Il fut plus de trois ou quatre Paters à se
remettre, puis sa première parole fut que ce n'était rien, qu'il était
bien, et de remercier les dames. Alors je lui fis excuse, et le fis
souvenir que je le lui avais bien dit. Il voulut répondre, les dames
interrompirent. On parla de toute autre chose, et Charost se raccoûtra,
et s'en alla peu après. Nous n'en fûmes pas un instant moins bien ni
moins librement ensemble, et dès la même journée\,; mais ce que j'y
gagnai, c'est qu'il ne se commit jamais plus à quoi que ce soit sur M.
de la Trappe. Quand il fut sorti, les dames me grondèrent, et se mirent
toutes trois sur moi\,; je ne fis qu'en rire. Pour elles, elles ne
pouvaient revenir de l'étonnement et de l'effroi de ce qu'elles avaient
vu, et nous convînmes, pour la chose et pour l'amour de Charost, de n'en
parler à personne\,; et en effet, qui que ce soit ne l'a su.

Un événement singulier, que le grand-duc manda à Monsieur, surprit
extrêmement tout ce qui à Paris et à la cour avait connu Caretti.
C'était un Italien qui s'y était arrêté longtemps, et qui gagnait de
l'argent en faisant l'empirique. Ses remèdes eurent quelque succès. Les
médecins, jaloux à leur ordinaire, lui firent toutes sortes de
querelles, puis de tours, pour le faire échouer, et s'avantagèrent tant
qu'ils purent des mauvais succès qui lui arrivaient. Les meilleurs
remèdes et les plus habiles échouent à bien des maladies\,; à plus forte
raison ces sortes de gens qui donnent le même remède, tout au plus
déguisé, à toutes sortes de maux, et qui, à tout hasard, entreprennent
les plus désespérés, et des gens à l'agonie à qui les médecins ne
peuvent plus rien faire, dans l'espérance que, si ces malades viennent à
réchapper, on criera au miracle du remède, et qu'on courra après eux, et
que, s'ils ne réussissent pas, ils auront une excuse bien légitime par
l'extrémité que ces malades ont attendue avant de les appeler. Caretti
vécut ainsi assez longtemps, et n'avait d'autre subsistance que son
industrie. Il avait de l'esprit, du langage, de la conduite\,; il
réussit assez pour se mettre en quelque réputation. Caderousse, alors
fort du monde, et depuis longtemps désespéré de la poitrine, se mit
entre ses mains, et guérit parfaitement. Cela le mit sur un grand pied
qui fut soutenu par d'autres fort belles cures.

La plus singulière fut celle de M. de La Feuillade, abandonné
solennellement des médecins, qui le signèrent, et que Caretti ne voulut
pas entreprendre sans cette formalité. Il se mourait d'avoir depuis
quelque temps quitté une canule, qu'il portait depuis une grande
blessure qu'il avait eue autrefois à travers du corps. Caretti le guérit
parfaitement et en peu de temps. Il était fort cher pour ces sortes
d'entreprises, et faisait consigner gros.

Enrichi et en honneur, en dépit des médecins, et avec des amis
considérables, il se mit à faire l'homme de qualité, et à se dire de la
maison Caretti, héritier de la maison Savoli\,; que d'autres héritiers
plus puissants que son père lui avaient enlevé cette riche succession et
son propre bien, et l'avaient réduit à la misère et au métier qu'il
faisait pour vivre. On se moqua de lui et ses protecteurs mêmes\,;
personne n'en voulut rien croire\,; il le maintint toujours, et se
trouvant enfin assez à son aise, il dit qu'il s'en allait tâcher de
faire voir qu'il avait raison, et il obtint de Monsieur une
recommandation de sa personne et de ses intérêts pour le grand-duc. Il
fit, après, quelques voyages à Bruxelles et quelques cures aux Pays-Bas,
et repassa ici allant effectivement en Italie. Au bout de quatre ou cinq
ans, il gagna son procès à Florence, et le grand-duc manda à Monsieur
que sa naissance et son droit avaient été reconnus\,; qu'il lui avait
été adjugé cent mille livres de rente dans l'État ecclésiastique, et
qu'il croyait que le pape l'en allait faire mettre en possession. En
effet, cet empirique vécut encore longtemps grand seigneur.

La beauté de M\textsuperscript{me} de Soubise avait achevé ce que les
intrigues de la Fronde et la faveur de la fameuse duchesse de Chevreuse
et de sa belle-sœur\footnote{La duchesse de Montbazon n'était pas
  belle-sœur, mais belle-mère de la duchesse de Chevreuse. Voy. p.~149
  de ce volume.} (\emph{belle-mère}) la belle M\textsuperscript{me} de
Montbazon, avaient commencé. Je l'expliquerai le plus courtement qu'il
me sera possible.

\hypertarget{chapitre-ix.}{%
\chapter{CHAPITRE IX.}\label{chapitre-ix.}}

1698

~

{\textsc{Curiosités sur la maison de Rohan.}} {\textsc{- Ses grandes
alliances.}} {\textsc{- Juveigneurs, ou cadets de Rohan, décidés n'avoir
rien que de commun en tout et partout avec tous autres juveigneurs
nobles et libres de Bretagne.}} {\textsc{- Vicomtes de Rohan décidés à
alterner avec les comtes de Laval-Montfort jusqu'à ce que ces derniers
eussent la propriété du lieu de Vitré.}} {\textsc{- Le parlement, par
égards, non par rang, aux obsèques de l'archevêque de Lyon, fils du
maréchal de Gié.}} {\textsc{- M\textsuperscript{lle} de La Garnache\,;
son aventure\,; duchesse de Loudun à vie seulement.}} {\textsc{- Henri
de Rohan fait duc et pair\,; son mariage et celui de son unique
héritière\,; enfants de celle-ci.}} {\textsc{- Benjamin de Rohan, sieur
de Soubise, duc à brevet ou non vérifié.}} {\textsc{- M. de Sully
obtient un tabouret de grâce aux deux sœurs du duc de Rohan, son gendre,
non mariées.}} {\textsc{- Dispute de préséance au premier mariage de M.
Gaston, entre les duchesses d'Halluyn et de Rohan, décidée en faveur de
la première.}} {\textsc{- Louis, puis Hercule de Rohan, faits l'un après
l'autre ducs et pairs de Montbazon\,; famille de ce dernier}}

~

Jamais aucun de la maison de Rohan n'avait imaginé d'être prince\,;
jamais de souveraineté chez eux, jamais en Bretagne ni en France\,;
depuis qu'ils y furent venus sous Louis XI, aucune autre distinction que
celle des établissements que méritaient leurs grandes possessions de
terre, leurs hautes alliances et une naissance qui, sans avoir d'autre
origine que celle de la noblesse, ni avoir jamais été distinguée de ce
corps, était pourtant fort au-dessus de la noblesse ordinaire, et se
pouvait dire de la plus haute qualité. Ils avaient par leur baronnie le
second rang en Bretagne, et puis ils l'alternèrent avec les barons de
Vitré, mais cela n'influait point sur leurs cadets\,; quoique sortis de
plus d'une sœur des ducs de Bretagne, ils ne purent obtenir aucune
préférence sur les autres puînés nobles de Bretagne, et Alain VI,
vicomte de Rohan, fut obligé vers 1300 par Jean II, duc de Bretagne, de
reconnaître que, selon la coutume de cette province, tous les
juveigneurs\footnote{Fils cadets de maisons nobles, du latin
  \emph{juniores}.} de Rohan devaient être hommes liges\footnote{On
  appelait \emph{homme lige} un vassal lié (\emph{ligatus}) à son
  seigneur par des obligations plus étroites que les autres vassaux.
  Voy. notes à la fin du volume.} du duc de Bretagne, et qu'il avait
droit de retirer de leurs terres tous les émoluments et profits de fief
qu'il pouvoir retirer de celles de ses autres sujets libres. C'est ce
duc de Bretagne qui fut écrasé par la chute d'une muraille à Lyon, à
l'entrée du pape Clément V, où il accompagnait Philippe le Bel qui
l'avait fait duc et pair en 1297, et il mourut à Lyon le 18 novembre
1305, quatre jours après la chute de ce mur. Cela n'a point varié
depuis. Ainsi, pour les cadets, nulle préférence sur ceux des autres
maisons nobles de Bretagne. Voici maintenant pour les aînés Alain IX,
vicomte de Rohan, est sans doute celui qui par ses grands biens, ses
hautes alliances et celles de ses enfants, a fait le plus d'honneur à la
maison, dont il était le chef. Sa mère était fille du connétable de
Clisson\,; sa première femme, dont il ne vint point de postérité
masculine, était fille de Jean V, duc de Bretagne, et de Jeanne fille de
Charles le Mauvais, roi de Navarre. La seconde femme du même Alain, qui
était Lorraine-Vaudemont, continua la postérité à laquelle je
reviendrai. Du premier lit il maria sa seconde fille à Jean d'Orléans,
comte d'Angoulême, deuxième fils du duc d'Orléans, frère de Charles VI,
assassiné par ordre du duc de Bourgogne, et cette Rohan fut mère de
Charles d'Orléans, comte d'Angoulême, père du roi François Ier.
Certainement voilà de la grandeur, et qui fut soutenue par les emplois
et la figure que cet Alain IX, vicomte de Rohan, fit toute sa vie.
Néanmoins Pierre, duc de Bretagne, fils de Jean VI, duc de Bretagne,
frère de la femme défunte alors de ce même vicomte de Rohan, ordonna le
25 mai 1451, en pleins états, à Vannes, que ledit Alain IX, vicomte de
Rohan, aurait séance le premier jour, à la première place au côté
gauche, après les seigneurs de son sang\,; que le second jour cette
place serait occupée par Guy, comte de Laval, et ainsi à l'alternative
jusqu'à ce que ce dernier ou ses successeurs fussent propriétaires du
lieu de Vitré.

Cela fut exécuté de la sorte, et c'est-à-dire que la possession levait
l'alternative, et que le vicomte de Rohan n'en pouvait pas prétendre
avec le baron de Vitré qui le devait toujours précéder. Il faut
remarquer que ce comte de Laval dont il s'agit ici était de la maison de
Montfort en Bretagne depuis longtemps éteinte, et fondue par une
héritière dans celle de La Trémoille qui en a eu Vitré et Laval, que ces
Montfort avaient eu de même par une héritière de la branche aînée de
Laval-Montmorency.

Voilà donc l'aîné de la maison de Rohan et vicomte de Rohan, et au plus
haut point de toute sorte de splendeur, en alternative décidée et subie
avec le comte de Laval, lequel, devenant propriétaire du lieu de Vitré,
le devait toujours précéder, et les juveigneurs ou cadets de la maison
de Rohan semblables en tout et par tout aux juveigneurs de toutes les
autres maisons nobles de Bretagne, et cela par les deux décisions que je
viens de rapporter, qui ont toujours depuis été exécutées.

Jean II, fils d'Alain IX que je viens d'expliquer et de Marie de
Lorraine-Vaudemont, sa seconde femme, épousa, en 1461, Marie, fille de
François Ier, duc de Bretagne, et d'Isabelle Stuart, fille de Jacques
Ier, roi d'Écosse. Cette vicomtesse de Rohan n'eut point de frère, mais
une sœur qui fut première femme sans enfants de François II, dernier duc
de Bretagne, qui, d'une Grailly-Foix, dont la mère était Éléonore de
Navarre, eut Anne, duchesse héritière de Bretagne, deux fois reine de
France, et par qui la Bretagne a été réunie à la couronne, c'est-à-dire
depuis sa mort. Ce vicomte de Rohan n'eut point de mâles qui aient eu
postérité\,: et deux filles qui se marièrent dans leur maison, l'aînée
au fils du maréchal de Gié, la cadette au seigneur de Guéméné.

Ainsi nuls mâles sortis des filles de Bretagne, et jusqu'ici rien qui
sente les princes. Retournons sur nos pas.

Jean Ier, vicomte de Rohan, grand-père d'Alain IX, vicomte de Rohan,
duquel j'ai parlé d'abord, était fils d'une Rostrenan, et figura fort
dans le parti de Charles de Blois, c'est-à-dire de Châtillon, contre
celui de Montfort, c'est-à-dire des cadets de la maison de Bretagne
compétiteurs pour ce duché que le dernier emporta. Ce vicomte de Rohan
épousa l'héritière de Léon dont il eut Alain VIII, père d'Alain IX,
vicomtes de Rohan, puis en secondes noces, en 1377, Jeanne la jeune,
dernière fille de Philippe III, comte d'Évreux, fils d'un fils puîné du
roi Philippe III le Hardi et devenu roi de Navarre par son mariage avec
l'héritière de Navarre, fille du roi Louis X le Hutin. Ainsi cette
vicomtesse de Rohan était sœur de Charles le Mauvais, roi de Navarre, de
Blanche, seconde femme du roi Philippe de Valois, de Marie première
femme de Pierre IV, roi d'Aragon, et d'Agnès, femme de Gaston-Phoebus,
comte de Foix, si célèbre dans Froissart. Aussi faut-il remarquer que
Philippe III, roi de Navarre, était mort en 1343, Jeanne de France, sa
femme, en 1349.

Charles le Mauvais ne mourut qu'en 1385, mais en quel état et depuis
combien d'années\,! Philippe de Valois était mort en 1350\,; Blanche de
Navarre, sa seconde femme, ne mourut qu'en 1398, et n'eut qu'une fille
qui ne fut point mariée. La reine d'Aragon mourut en 1346, et la
comtesse de Foix ne laissa point d'enfants. Il se voit donc par ces
dates que le père, la mère, les sœurs, hors une veuve sans enfants et
retirée, les beaux-frères, hors le comte de Foix, tout était mort avant
le mariage de cette vicomtesse de Rohan\,; et si on y regarde, il ne se
trouvera point de postérité, si ce n'est de Charles le Mauvais qui
survécut de ce mariage, qui toutefois fut extrêmement grand. Il n'en
vint qu'un fils, dont le fils fut père de Louis de Rohan, seigneur de
Guéméné, qui épousa la fille aînée du dernier vicomte de Rohan, et le
maréchal de Gié dont le second fils épousa l'autre fille du dernier
vicomte de Rohan, comme je l'ai dit.

Quoique la branche de Guéméné soit l'aînée par l'extinction de celle des
vicomtes directs, et par le mariage de la fille aînée du dernier
vicomte, parlons d'abord de celle de Gié quoique cadette, parce qu'il
s'y trouvera plutôt matière que dans l'autre, et parce qu'elle est
éteinte.

Le maréchal de Gié a trop figuré pour avoir rien à en dire\,; mais parmi
tous ses emplois et ses alliances, et de son fils aîné à deux filles
d'Armagnac qui leur apportèrent le comté de Guise, il y eut si peu de
princerie en son fait que le parlement, ayant eu ordre d'assister aux
obsèques de l'archevêque de Lyon son fils, mort à Paris en 1536, pendant
une assemblée que François Ier y avait convoquée, le parlement répondit
que la cour, en considération des mérites du feu maréchal de Gié et de
son fils, lui rendrait volontiers l'honneur qu'elle avait coutume de
rendre aux princes et aux grands du royaume. Or, si ce prélat avait été
rang à recevoir cet honneur, le parlement le lui aurait rendu tout de
suite sans répondre\,; et on voit qu'il ne répondit que pour montrer que
c'était, non par rang, mais en considération des mérites du père et du
fils qu'il irait à ses obsèques.

Outre cet archevêque qui fit fort parler de lui dans le clergé, le
maréchal de Gié eut deux autres fils\,; la branche de l'aîné finit à son
petit-fils, sur tous lesquels il n'y a rien à remarquer. Les sœurs de ce
dernier épousèrent, l'aînée un Beauvilliers, dont le duc de Beauvilliers
est descendu\,; la cadette, le marquis de Rothelin, frère et oncle des
ducs de Longueville\,; et de ce mariage vint Léonor, duc de Longueville,
d'où sont sortis tous les autres depuis, et que sa mère et sa femme
firent tant valoir. C'est de ce marquis de Rothelin que les Rothelin
d'aujourd'hui sont bâtards.

Le second fils du maréchal de Gié, gendre cadet du dernier vicomte de
Rohan, n'eut qu'un fils qui fit un grand mariage. Il épousa Isabelle,
fille de Jean, sire d'Albret, et de Catherine de Grailly, dite de Foix,
reine de Navarre. C'est ce qu'il faut expliquer.

Elle était fille de Gaston, prince de Viane, et de Madeleine de France,
sœur put née de Louis XI\,; et le prince de Viane était Grailly, dit de
Foix, dont l'héritière était tombée dans sa maison avec les comtés de
Foix, de Bigorre et de Béarn qu'ils possédaient\,; {[}il{]} était fils
de Gaston IV, comte de Foix, etc., que Charles VII fit comte-pair de
France en 1458, et d'Éléonore, fille de Jean H, roi d'Aragon, et de sa
seconde femme Blanche, reine héritière de Navarre. Éléonore en hérita,
survécut Gaston, son fils, et mourut quarante-deux jours après son
couronnement à Pampelune.

Gaston, son fils, prince de Viane, n'avait laissé qu'un fils et une
fille\,: le fils fut couronné à Pampelune, et mourut trois mois après à
Pau en 1482, empoisonné tout jeune, et sans alliance. Catherine, sa sœur
unique, lui succéda, et fut aussi couronnée à Pampelune avec Jean
d'Albret, son mari, en 1494. Ils se brouillèrent et furent chassés de
leur royaume en 1512, par Ferdinand le Catholique, roi d'Aragon, qui
s'en empara, et depuis la Navarre est demeurée à l'Espagne. Ils en
moururent tous deux de douleur, lui en 1516, elle en 1517, et ne
laissèrent d'enfants qui parurent que Henri d'Albret, roi de Navarre,
une comtesse d'Astarac-Grailly-Foix, morte sans enfants, et Isabelle,
mariée en 1534 à René Ier de Rohan, fils du second fils du maréchal de
Gié, tellement que, par l'événement, René Ier de Rohan épousa la sœur du
père de Jeanne d'Albret, mère de notre roi Henri IV.

Avec ce détail, je pense au moins qu'on ne m'accusera pas d'avoir
dissimulé rien des grandeurs de la maison de Rohan.

De ce mariage de René Ier de Rohan et d'isabelle d'Albret,
{[}sortirent{]} des fils qui ne parurent point, et une fille qui ne
parut que trop. Mais par cela même fatal en bonheur suivi de branche en
branche et de génération en génération à la maison de Rohan, {[}ce
mariage{]} eut la première distinction qui ait été accordée à cette
maison.

M. de Nemours, dont l'esprit, la gentillesse et la galanterie ont été si
célébrés, fit un enfant à cette fille de Rohan, qu'on appelait
M\textsuperscript{lle} de La Garnache, sous promesse de mariage\,; en
même temps il était bien avec M\textsuperscript{me} de Guise. Toutes ces
aventures-là me mèneraient trop loin\,; c'était Anne d'Este, dont la
mère était seconde fille de Louis XII. M. de Guise fut tué par
Poltrot\,; M\textsuperscript{me} de Guise, après avoir gardé les
bienséances, voulut épouser M. de Nemours. Lui ne demandait pas mieux,
et cependant amusait M\textsuperscript{lle} de La Garnache. Enfin,
l'amusement fut si long qu'elle s'en impatienta, et qu'elle en découvrit
la cause. La voilà aux hauts cris, et M\textsuperscript{me} de Guise sur
le haut ton que lui faisaient prendre la splendeur de sa mère et la
puissance de la maison de Guise dont elle disposait, et qui, pour ses
grandes vues, trouvait son compte dans ce second mariage. Il n'en
fallait pas tant pour émouvoir la reine de Navarre, Jeanne d'Albret, et
tout ce qui tenait à son parti et aux princes de Bourbon contre les
Guise.

La reine de Navarre protesta avec tout cet appui qu'elle ne souffrirait
pas que M. de Nemours fit cet affront à une fille qui avait l'honneur
d'être sa nièce\,; et le pauvre M. de Nemours était bien embarrassé.
Personne des intéressés ne faisait là un beau personnage.
M\textsuperscript{me} de Guise voulait enlever M. de Nemours à sa parole
de haute lutte. M. de Nemours convenait de l'avoir donnée\,; il n'osait
y manquer, et pourtant ne la voulait point tenir. La bonne La Garnache
demeurait abusée, et en attendant ce qui arriverait de son mariage,
faisait de sa turpitude la principale pièce de son sac, et toute la
force des cris de ceux qui la protégeaient.

La fin de tout cela fut qu'elle en fut pour sa honte, et ses protecteurs
pour leurs cris, et que M. de Nemours épousa M\textsuperscript{me} de
Guise en 1566. M\textsuperscript{lle} de La Garnache disparut et alla
élever son poupon dans l'obscurité, où il vécut et mourut. Après
plusieurs années, comme la suite infatigable et le talent de savoir se
retourner est encore un apanage spécial de la maison de Rohan,
M\textsuperscript{lle} de La Garnache se remontra à demi, essaya de
faire pitié à M\textsuperscript{me} de Nemours, et d'obtenir quelque
dédommagement par elle. Elle la toucha enfin, et M\textsuperscript{me}
de Nemours obtint personnellement pour elle, et sans aucune suite après
elle, l'érection de la seigneurie de Loudun en duché sans pairie, en
1576.

M\textsuperscript{lle} de La Garnache alors reparut tout à fait sous le
nom de duchesse de Loudun, et jouit du rang de duchesse, qui fut éteint
avec ce duché par sa mort. Telle est la première époque d'un rang dans
la maison, et non à la maison de Rohan, qui, avec toutes ces alliances
si grandes et si immédiates, n'en avait jamais eu, et n'y avait jamais
prétendu.

René II de Rohan, frère de la duchesse de Loudun, et fils comme elle
d'Isabelle de Navarre, ne figura point, non plus que ses autres frères,
mais ils n'eurent point de postérité, et il en eut. Ses deux fils et ses
deux filles firent tous parler d'eux\,; et comme leur père qui s'était
fait huguenot, et encore plus comme leur mère qui fut une héroïne de ce
parti, ils l'embrassèrent. Elle était veuve de Charles de Quellenec,
baron du Pont, qui périt à la Saint-Barthélemy, et fille et héritière de
Jean Larchevêque, seigneur de Soubise, et d'une Bouchard-Aubeterre. Elle
et ses deux filles se sont rendues fameuses dans la Rochelle, où elles
soutinrent les dernières extrémités, jusqu'à manger les cuirs de leurs
carrosses, pendant le siège que Louis XIII y mit.

Sa fille aînée, plus opiniâtre s'il se pouvait qu'elle, mourut de regret
de sa prise fort peu après au château du Parc en Poitou, où elles
avaient été reléguées en sortant de la Rochelle. La mère y mourut deux
ans après en 1631, et l'autre fille les survécut jusqu'en 1646, toutes
les deux point mariées. Elles avaient une autre sœur entre elles deux,
qui fut la première femme d'un prince palatin des Deux-Ponts, en 1604,
et qui mourut en 1607.

Leurs deux frères furent Henri et Benjamin de Rohan, seigneur de
Soubise. Henri fut le dernier chef des huguenots en France. Le duc de
Sully, surintendant des finances, et si bien avec Henri IV, et huguenot
aussi, le favorisa fort auprès d'Henri IV dans la haine du maréchal de
Bouillon. Henri IV le fit duc et pair en 1603, et moins de deux ans
après, M. de Sully lui donna sa fille en mariage.

C'est ce grand homme qui se signala tant à la tête d'un parti abattu, et
qui, réconcilié avec la cour, s'illustra encore davantage par les
négociations dont il fut chargé en Suisse, et par ses belles actions à
la tête de l'armée du roi en Valteline, où il mourut de ses blessures en
1638, avec la réputation d'un grand capitaine et d'un grand homme de
cabinet. Il ne laissa qu'une fille, unique héritière, qui porta tous ses
grands biens en mariage en 1645, malgré sa mère, à Henri Chabot, à
condition de porter lui et leur postérité le nom et les armes de
Rohan\,; et {[}Chabot{]} fut fait duc et pair, comme on sait, par
lettres nouvelles et avec rang du jour de sa réception au parlement. La
mère, qui était huguenote, ne voulait point ce mariage, et la fille, qui
était catholique, soutenue de M. Gaston et de M. le Prince, se moqua
d'elle.

De ce mariage vint le duc de Rohan, la seconde femme de M. de Soubise,
la seconde femme du prince d'Espinoy, et M\textsuperscript{me} de
Coetquen que nous avons tous vus.

M. de Soubise, frère de ce grand-duc de Rohan, ne fit parler de lui que
par l'audace et l'opiniâtreté de ses continuelles défections, quoiqu'à
la paix que le roi donna en 1626 aux huguenots il l'eût fait duc à
brevet. On n'en ouït plus parler en France, depuis la prise de la
Rochelle, et il mourut en Angleterre, sans considération, où il s'était
retiré, sans avoir été marié, vers 1641.

Voilà donc une duchesse à vie, un duc et pair et un duc à brevet dans la
maison de Rohan. Mais cette génération commence à montrer autre chose.
M. de Sully, en faisant le mariage de sa fille, représenta si bien à
Henri IV l'honneur que cette branche de Rohan avait de lui appartenir de
fort près, et d'être même l'héritière de la Navarre, s'il n'avait point
d'enfants, par Isabelle de Navarre, sa grand'tante et leur grand'mère,
qu'il obtint un tabouret de grâce aux deux sœurs de son gendre, l'autre
étant déjà mariée, mais en leur déclarant bien que ce n'était que par
cette unique considération de la proche parenté de Navarre\,; que cette
distinction ne regardait point la maison de Rohan, et ne passerait pas
même au delà de ces deux filles. C'est la première époque de rang, ou
plutôt d'honneurs sans dignité dans la maison de Rohan, et non à cette
maison.

Aux fiançailles et mariage de M. Gaston avec M\textsuperscript{lle} de
Montpensier, princes ni grands n'eurent point de rang, marchèrent entre
eux en confusion, et se placèrent comme ils purent. Les dames ne furent
pas d'avis de faire de même, et voulurent marcher en rang. C'était à
Nantes, et le cardinal de Richelieu faisait la cérémonie. La duchesse de
Rohan qui suivait la duchesse d'Halluyn, qu'on a aussi quelquefois
appelée la maréchale de Schomberg, voulut la précéder. L'autre s'en
défendit\,; la contestation s'échauffa\,; des paroles, elles en vinrent
aux poussades et aux égratignures. Le scandale ne fut pas long, et
sur-le-champ, la dispute fut jugée et décidée en faveur de
M\textsuperscript{me} d'Halluyn, comme l'ancienne de
M\textsuperscript{me} de Rohan, qui subit le jugement.

Voilà la première époque de prétention, et la prétention fut
malheureuse\,: encore n'est-il rien moins que clair qu'elle roulât sur
la maison de Rohan, qui jusqu'alors et bien longtemps depuis n'en avait
aucune\,; mais bien sur l'ancienneté entre duchesses, car on voit que
M\textsuperscript{me} de Rohan ne disputa pas à une autre.
M\textsuperscript{me} d'Halluyn était fille de M. de Pienne, tué du
vivant de son père par ordre du duc de Mayenne, dans la Fère, dont il
était gouverneur en 1592, et son père avait été fait duc et pair au
commencement de 1588. Il avait marié M. de Pienne, son fils, à une fille
du maréchal de Retz, qui n'en eut qu'un fils, mort tout jeune, en 1598,
et M\textsuperscript{me} d'Halluyn, dont il s'agit ici. Elle épousa le
fils aîné du duc d'Épernon, et en faveur de ce mariage, le duché-pairie
d'Halluyn fut de nouveau érigé pour eux, mais avec l'ancien rang du
grand-père de la mariée. Ils se brouillèrent, se démarièrent, et
n'eurent point d'enfants.

En 1620, M\textsuperscript{me} d'Halluyn épousa M. de Schomberg avec des
lettres en continuation de pairie, tellement que M\textsuperscript{me}
de Rohan, dont l'érection était antérieure aux deux continuations de
pairie qu'avait obtenues M\textsuperscript{me} d'Halluyn à tous ses deux
mariages, pouvait bien n'avoir pas grand tort d'être fâchée de la voir
remonter à la première érection de son grand-père, antérieure à celle de
Rohan. On ne voit pas d'ailleurs que cette duchesse de Rohan, qui était
la fille de M. de Sully, et mère de l'héritière qui épousa le Chabot,
ait jamais rien prétendu ni disputé, excepté cette ridicule aventure que
j'ai voulu expliquer afin de ne rien omettre. Passons maintenant à la
branche de Guéméné.

Louis de Rohan, seigneur de Guéméné, frère aîné du maréchal de Gié, dont
je viens d'épuiser la branche, ne fournit rien à remarquer, non plus que
ses trois générations suivantes. La quatrième fut Louis VI de Rohan, qui
fit ériger la seigneurie de Montbazon en comté et celle de Guéméné en
principauté en 1547 et 1549.

Il y a nombre de ces principautés d'érection en France, dont pas une n'a
jamais donné et ne donne encore aucune espèce de distinction à la terre,
que le nom, ni à celui qui en a obtenu l'érection non plus, ni à ses
successeurs. Aussi, ce Louis de Rohan a-t-il vécu, et est-il mort sans
aucun rang ni honneurs, non plus que ses pères, et sans la moindre
prétention. Mais désormais il faut prendre garde à tout ce qui sortira
de lui. Il épousa la fille aînée du dernier Rohan de la branche de Gié,
qui était frère de M\textsuperscript{me}s de Beauvilliers et de
Rothelin-Longueville, comme je l'ai dit ci-dessus, et qui fut
ambassadeur à Rome en 1548, lequel en secondes noces épousa la sœur de
son gendre, dont il n'eut point d'enfants.

De ce Louis VI de Rohan et de Léonore Rohan-Gié, quatre fils et cinq
filles, qui comme les précédentes de leur maison, et même plusieurs
veuves de seigneurs de Rohan, épousèrent des seigneurs, et même des
gentilshommes particuliers.

Les quatre fils furent\,: Louis, en faveur de qui Henri III érigea le
duché de Montbazon en duché-pairie, en 1588\,; il mourut un an après
sans enfants\,; Pierre, prince de Guéméné, qui d'une
Lenoncourt\footnote{La mère d'Anne de Rohan fut Madeleine de
  Rieux-Châteauneuf. Ce fut Hercule de Rohan qui épousa en premières
  noces Marie de Lenoncourt.}, ne laissa qu'une fille, Anne de Rohan,
qui épousa le fils de son frère qui va suivre, et d'elle son cousin
germain\,; Hercule, duc et pair de Montbazon par une érection nouvelle
par Henri IV, en 1595, avec rang de cette nouvelle érection\,; et
Alexandre, marquis de Marigny, et qui mourut sans enfants, ayant pris le
nom de comte de Rochefort, duquel il se faut souvenir.

Hercule susdit, duc de Montbazon, fut grand veneur, gouverneur de Paris
et de l'Ile-de-France, chevalier du Saint-Esprit en son rang de duc, en
1597, homme de tête et d'esprit, qui figura fort, et sa femme et leurs
enfants encore davantage. Lassé de leurs intrigues, qui suivant l'étoile
de la maison de Rohan, étaient utiles à cette maison, mais qui lui
faisaient peu d'honneur, il les laissa faire et se retira en Touraine,
où il demeura longues années, et y mourut à quatre-vingt-six ans, en
1654, sans s'être démis de son duché. Il avait épousé en 1628, en
secondes noces, Marie, fille de Claude d'Avaugour et de Catherine
Fouquet de La Varenne. Le trisaïeul de père en fils de ce Claude
d'Avaugour était le bâtard du dernier duc de Bretagne, et le grand-père
paternel du comte de Vertus d'aujourd'hui.

Du premier lit\,: M. de Montbazon (Louis, prince de Guéméné, depuis duc
de Montbazon), et la connétable de Luynes, mère du duc de Luynes, qui
épousa en secondes noces, en 1622, le duc de Chevreuse, dernier fils du
duc de Guise, tué à Blois, laquelle devint si fameuse sous ce dernier
nom.

Du second lit\,: M. de Soubise, que nous avons tous vu, père du prince
et du cardinal de Rohan, et Anne de Rohan, qui a été la seconde femme du
duc de Luynes, fils de sa sœur, dont elle a eu le comte d'Albert, le
chevalier de Luynes et plusieurs filles, toutes mariées.

Le prince de Guéméné était un homme de beaucoup d'esprit, et encore plus
Anne de Rohan sa femme, fille de Pierre, prince de Guéméné, frère aîné
de son père. Lui, elle et M\textsuperscript{me} de Chevreuse, toute leur
vie, ne furent qu'un, et avec eux, en quatrième, leur belle-mère,
seconde femme de leur père, qui avait autant d'esprit et d'intrigue
qu'eux\,; et, ce qui peut passer pour un miracle, toutes trois
parfaitement belles et fort galantes, sans que leur beauté ni leur
galanterie ait jamais formé le moindre nuage de galanterie ni de
brouillerie entre elles.

Le prince de Guéméné, non seulement voyait trop clair pour ignorer ce
qui se passait dans sa maison, mais il y trouvait son compte, et dès là,
non seulement il le trouvait fort bon, mais il était des confidences
sans en faire semblant au dehors. Leçon utile à la grandeur d'une
maison, quand il y a des beautés qui savent faire usage de leurs
charmes, heureusement fatale à la maison de Rohan, pour le répéter
encore, et que M. de Soubise a si exactement et si utilement suivie.

\hypertarget{chapitre-x.}{%
\chapter{CHAPITRE X.}\label{chapitre-x.}}

1698

~

{\textsc{M. de Luynes fait asseoir pour une fois seulement
M\textsuperscript{lle} de Montbazon, depuis duchesse de Chevreuse, la
veille de leurs noces\,; obtient dispense d'âge et la première place
après les ducs pour le prince de Guéméné, son beau-frère, en la
promotion de 1619.}} {\textsc{- M. de Marigny, frère du duc de
Montbazon, le cinquante-cinquième parmi les gentilshommes en la
promotion de 1619.}} {\textsc{- Art et degré qui procurent le tabouret à
la princesse de Guéméné.}} {\textsc{- Autre tabouret de grâce en même
temps.}} {\textsc{- Tous ôtés, puis rendus. ---}}

~

M. de Soubise et ses deux femmes, la première debout\,; la seconde
assise, belle, le fait prince, etc. --- M\textsuperscript{me}s de
Guéméné assises (1678 et 1679), puis M\textsuperscript{me} de Montauban
(1682). --- MM. de Soubise et comte d'Auvergne s'excluent de l'ordre à
la promotion de 1688\,; colère du roi\,; fausseté insigne sur les
registres de l'ordre. --- Distinction de ceux qui ont rang de princes
étrangers étant en licence. --- Abbé de Bouillon devenu cardinal par le
hasard des coadjutoreries de Langres puis de Reims tombées sur l'abbé Le
Tellier, est le premier qui ait eu ces distinctions en Sorbonne. ---
Abbé de Soubise, depuis cardinal de Rohan, obtient, par ordre du roi,
les mêmes distinctions en Sorbonne. --- Fiançailles du prince de
Montbazon et de la fille du duc de Bouillon dans le cabinet du roi.

Le connétable de Luynes, qui ne l'était pas encore, mais que je nomme
ainsi pour le distinguer de son fils, voulut entrer agréablement dans la
maison du duc de Montbazon, en épousant sa fille, en septembre 1617, et
faire en même temps éclater sa faveur par une distinction
extraordinaire. Il obtint que M\textsuperscript{lle} de Montbazon serait
assise dès la veille de son mariage avec lui, grâce qui se terminait là,
sans influer sur le reste de la famille. C'était en un mot un tabouret
de grâce, et pour une seule fois, pour faire briller la faveur du
favori, et témoigner à M. de Montbazon combien ce mariage était agréable
au roi.

À la promotion de 1619, M. de Luynes obtint une dispense d'âge pour le
frère de sa femme, qui n'avait que vingt et un ans, et qu'on appelait
alors le comte de Rochefort, qui prit après le nom de prince de Guéméné,
jusqu'à la mort de son père. M. de Luynes obtint encore pour lui qu'il
marcherait le premier des gentilshommes de cette promotion, c'est-à-dire
immédiatement après lui-même, qui en était le dernier duc. Mais cela
n'empêcha pas que, jusqu'à ce qu'il le fût devenu lui-même par la mort
de son père, trente-cinq ans après, il n'ait toujours été précédé par
tous les gentilshommes, plus anciens chevaliers de l'ordre que lui, sans
difficulté aucune, ni réclamation de sa part.

Le marquis de Marigny, frère de son père, qui pour combler la famille de
la femme du favori fut aussi de cette même promotion, y marcha le
cinquante-cinquième parmi les gentilshommes, et n'en eut que quatre de
toute la promotion après lui. Ainsi, avec toute la faveur de M. de
Luynes, qui se déploya toujours tout entière sur la maison de sa femme,
avec un emploi aussi important qu'était pour lors le gouvernement de
Paris et de l'Ile-de-France, avec une charge aussi favorable que celle
de grand veneur auprès d'un jeune roi passionné pour la chasse, au point
qu'était Louis XIII, et dont le fils avait la survivance du père, avec
l'exemple des tabourets de grâce des deux sœurs du célèbre duc de Rohan,
et l'avancement d'un jour de celui de M\textsuperscript{me} de Luynes,
on voit que MM. de Rohan n'avaient pas encore imaginé devoir avoir des
honneurs et dès distinctions au-dessus des gens de qualité non titrés,
beaucoup moins à être princes.

Mais voici le commencement. M\textsuperscript{me} de Chevreuse avait été
de tout temps dans la plus intime confidence de la reine. Elle en fut
chassée plus d'une fois par Louis XIII, et de l'aventure du
Val-de-Grâce, où la reine fut fouillée et visitée jusque dans son sein
par le chancelier Séguier, qui pourtant en acquit ses bonnes grâces pour
le reste de sa vie par la manière dont il s'y conduisit,
M\textsuperscript{me}s de Chevreuse et Beringhen se sauvèrent hors du
royaume, et plusieurs autres furent chassés et perdus, que la reine fit
revenir et récompensa tout aussitôt qu'elle fut régente.

M\textsuperscript{me} de Chevreuse était encore alors en Flandre, et
quoiqu'elle se trouvât trompée à son retour dans l'opinion qu'elle avait
conçue d'être absolument maîtresse de l'esprit de la reine et du
gouvernement, elle ne laissa pas de conserver toute sa faveur, malgré le
cardinal Mazarin. Les histoires et les Mémoires de ces temps-là sont
pleins de tout ce que fit la Fronde, qui domina la cour et l'État, et à
qui M. le Prince dut sa prison, puis sa délivrance, et le cardinal
Mazarin, son apparente ruine, par deux fois.

Ces mêmes histoires dépeignent l'hôtel de Chevreuse et l'hôtel de
Guéméné comme le centre de tous les conseils de la Fronde, où M. de
Beaufort et le coadjuteur, depuis cardinal de Retz, étaient en
adoration, et disposaient du parlement de Paris et de tout le parti.
M\textsuperscript{me} de Chevreuse, qui dans des intérêts souvent si
opposés à ceux du cardinal Mazarin conservait toujours sa place dans le
cœur de la reine, et son ascendant sur son esprit timide, défiant,
incertain et variable, avait introduit dans son amitié la princesse de
Guéméné sa belle-sœur, et la duchesse de Montbazon leur belle-mère, mais
surtout M\textsuperscript{me} de Guéméné avait plu infiniment à la reine
par le liant, les grâces et l'artifice caché de son esprit. Cela fut
cultivé d'une part et protégé de l'autre avec tant de soin par
M\textsuperscript{me} de Chevreuse, que M\textsuperscript{me} de Guéméné
fut de tous les particuliers, et la reine l'approchait d'autant plus
d'elle, qu'elle en apprenait tout ce que l'autre voulait bien à la
vérité lui dire d'une cabale où elle était de tout, et dont elle ne
disait à la reine que ce qui était utile à leurs desseins.

Dans ces conversations, ou seule avec la reine, ou en tiers avec elle et
M\textsuperscript{me} de Chevreuse, la reine la faisait asseoir. C'était
une commodité pour causer plus longtemps, qu'elle accordait bien seule
ainsi à d'autres, et même au commandeur de Jars, qui sans façon se
mettait dans un fauteuil. La reine allait souvent au Val-de-Grâce,
M\textsuperscript{me} de Guéméné l'y allait voir. D'autres dames y
étaient quelquefois reçues, et y trouvaient M\textsuperscript{me} de
Guéméné assise, qui se levait, puis se rasseyait sans façon, tellement
que, le Val-de-Grâce devenant peu à peu plus étendu, elle accoutuma
imperceptiblement et poliment ces demi-particuliers à son tabouret. À la
fin, la belle-mère et les belles-sœurs crurent qu'il était temps d'aller
plus loin. M\textsuperscript{me} de Guéméné n'alla plus au Palais-Royal
que de loin à loin, et à mesure que la reine se plaignait de son
absence\,; puis elle la laissa demander pourquoi elle ne la voyait
point, et cessa enfin tout à fait d'y aller. La reine en parlait souvent
à M\textsuperscript{me} de Montbazon et à M\textsuperscript{me} de
Chevreuse. Les excuses s'épuisèrent, et M\textsuperscript{me} de
Chevreuse prit son temps de dire franchement à la reine à quoi il
tenait. La reine surprise voulut se défendre d'accorder ce qui n'avait
eu lieu que comme bonté et familiarité ignorée et sans conséquence.
M\textsuperscript{me} de Chevreuse répondit que tout le monde savait
qu'elle était assise, et l'y voyait au Val-de-Grâce, qui ne pouvait plus
s'appeler un particulier, au nombre de gens où ce particulier s'était
étendu\,; qu'elle ne voyait point de différence entre le Val-de-Grâce et
le Palais-Royal, ni pourquoi sa belle-sœur, assise devant toute la cour
de la reine en un lieu, elle serait debout en un autre\,; et moitié
figue, moitié raisin, avec la fraude en croupe, elle arracha le tabouret
en plein partout, pour la princesse de Guéméné. Ce tabouret ne passa pas
plus avant pour lors dans la maison de Rohan, et n'y produisit point
d'autres distinctions.

En même temps la reine fit la même grâce à la marquise de Sencey, sa
dame d'honneur, chassée pour elle, et à qui, en arrivant à la régence,
elle avait rendu sa charge qui lui avait été ôtée, et récompensé de la
survivance à la comtesse de Fleix, sa fille, toutes deux veuves, qui ont
aussi le tabouret. Elles en jouirent quelques années, jusqu'à ce que
plusieurs personnes de qualité, excitées par M. Gaston et M. le prince,
s'assemblèrent en grand nombre, invitèrent les ducs de se joindre à eux,
et sous le nom de la noblesse, demandèrent la suppression de ces
tabourets, et des honneurs accordés à MM, de Bouillon pour l'échange de
Sedan, que le parlement n'avait pas voulu enregistrer avec ces articles
et quelques autres qui ne le sont pas encore aujourd'hui.

Ces assemblées, dont les princes voulaient effrayer la cour pour
d'autres vues, durèrent assez de semaines pour l'inquiéter par des
demandes plus embarrassantes, qui l'engagèrent à s'accommoder avec
Monsieur et M. le Prince. Les tabourets furent supprimés, et quelques
autres légères demandes accordées, avec quoi ces assemblées finirent
absolument.

Assez longtemps après, la cour prit tout à fait le dessus pour toujours,
et blessée alors des suppressions extorquées, elle rendit les tabourets.

M. de Soubise né, comme il le disait lui-même, mais bien bas à ses amis
particuliers, en riant et s'applaudissant de sa bonne fortune et de sa
sage politique, né gentilhomme avec quatre mille livres de rente, et
devenu prince à la fin avec quatre cent mille livres de rente, avait
épousé une riche veuve qui n'était rien d'elle ni de son premier mari
dont elle n'avait point d'enfants, qui lui donna tout son bien par le
contrat de mariage. Elle ne fut point assise, et M. de Soubise ni pas un
des siens n'imagina de le prétendre. Cette femme mourut en 1660. Avec ce
bien demeuré à M. de Soubise, on songea dans sa famille à le remarier et
à en tirer parti.

M\textsuperscript{me} de Chevreuse, toujours la mieux avec la reine, et
d'autant plus que les troubles étaient bien disparus, et que le cardinal
Mazarin était mort en 1661, qui eût été obstacle aux vues élevées de
M\textsuperscript{me} de Chevreuse, imagina d'unir son crédit à celui de
M\textsuperscript{me} de Rohan, qui sans faveur comme elle, était fort
considérée, pour faire le mariage de sa fille aînée en lui faisant
donner le tabouret. C'était en 1663. M. de Louvois était encore trop
petit garçon, et son père trop fin et trop politique pour oser branler
devant M. de Turenne, comme il s'y éleva longtemps depuis, et ce grand
capitaine était dans l'apogée de sa faveur et de sa considération
personnelle, avec un crédit que rien ne balançait\,; il était lors fort
huguenot, M\textsuperscript{me} de Rohan encore davantage, Cet intérêt
et la figure qu'ils faisaient dans leur religion, les avait intimement
unis, il ne bougeait de chez elle\,; et quand ses filles allaient au bal
ou en quelque autre partie où la bienséance de ce temps-là voulait que
des hommes de nom les accompagnassent, M\textsuperscript{me} de Rohan, à
cause de M. de Turenne, ne les confiait jamais qu'à MM. de Duras ou de
Lorges ses neveux, qui étaient chez elle comme les enfants de la maison,
et j'ai vu cette intimité de mon beau-père avec ces trois dames exister
la même depuis un si grand nombre d'années.

M. de Turenne entra donc dans cette affaire comme dans la sienne
propre\,; M\textsuperscript{me} de Rohan la poursuivit comme une grâce
qu'elle demandait instamment\,; M\textsuperscript{me} de Chevreuse y mit
tout son crédit et toutes ses anciennes liaisons avec la reine, et ils
l'emportèrent. Ils obtinrent presque en même temps de faire
M\textsuperscript{me} de Soubise dame du palais, et une fois à la cour,
sa beauté fit le reste. Le roi ne fut pas longtemps sans en être épris.
Tout s'use\,: l'humeur de M\textsuperscript{me} de Montespan le
fatiguait\,; au plus fort même de sa faveur il avait eu des passades
ailleurs, et lui avait même donné des rivales. Celle-ci sut bien se
conduire\,: Bontems porta les paroles\,; le secret extrême fut exigé, et
la frayeur de M. de Soubise fort exagérée. La maréchale de Rochefort,
accoutumée au métier, fut choisie pour confidente. Elle donnait les
rendez-vous chez elle, où Bontems les venait avertir, et toutes deux,
bien seules et bien affublées, se rendaient par des derrières chez le
roi.

La maréchale de Rochefort m'a conté qu'elle avait pensé mourir une fois
d'embarras\,: il y eut du mécompte. Bontems arriva mal à propos\,; il
fallut, sous divers prétextes, se défaire de la compagnie qu'on avait
laissée entrer, parce qu'on ne comptait sur rien ce jour-là, et
toutefois garder M\textsuperscript{me} de Soubise pour la conduire après
où elle était attendue, et ne pas faire perdre du temps à un amant dont
toutes les heures étoient compassées. Au bout d'un temps assez
considérable, le pénétrant courtisan s'aperçut\,; mais ne se le dit qu'à
l'oreille, et d'oreille en oreille personne n'en douta plus.

M. de Soubise, instruit à l'école de son père et d'un frère aîné,
infiniment plus âgé que lui, ne prit pas le parti le plus honnête, mais
le plus utile. Il se tint toute sa vie rarement à la cour, se renferma
dans le gouvernement de ses affaires domestiques, ne fit jamais semblant
de se douter de rien, et sa femme évita avec grand soin tout ce qui
pouvait trop marquer\,; mais assidue à la cour, imposant à tout ce qui
la composait, dominant les ministres, et ayant tant qu'elle voulait des
audiences du roi dans son cabinet, quand il s'agissait de grâces ou de
choses qui devaient avoir des suites. Afin qu'il ne parût pas qu'elle
les eût obtenues dans des moments plus secrets, elle se mettait tout
habillée aux heures publiques de cour, à la porte du cabinet. Dès que le
roi l'y voyait, il allait toujours à elle avec un air plus qu'ouvert,
mais en quelque sorte respectueux. Si ce qu'elle voulait dire était
court, l'audience se passait ainsi à l'oreille devant tout le monde\,;
s'il y en avait pour plus longtemps, elle demandait d'entrer. Le roi la
menait dans le fond du premier cabinet, joignant la pièce off était tout
le monde, les battants de la porte du cabinet demeuraient ouverts
jusqu'à ce qu'elle sortit de ce même côté, et de celui des autres
cabinets, et cela s'est toujours passé de la sorte.

Mais le plaisant, c'est que ces portes ne demeuraient ouvertes que pour
elle, et se fermaient toujours quand le roi donnait audience à d'autres
dames. Depuis qu'il n'y eut plus rien entre eux, l'amitié et la même
considération subsistèrent, et les mêmes précautions de bienséance. Elle
écrivait très souvent au roi et de Versailles à Versailles. Le roi lui
répondait toujours de sa main, et c'étaient Bontems ou Bloin qui les
rendaient au roi et faisaient passer ses réponses. C'est de la sorte
qu'elle fit M. de Soubise prince par degrés et par occasions, et que peu
à peu elle en obtint tout le rang.

Ce fut pourtant Monsieur dont ils se servirent pour faire asseoir la
Bautru, veuve de M. de Rannes, que le prince de Montauban, frère du
prince de Guéméné, épousa en 1681. Elle jouait fort chez Monsieur. M. de
Montauban n'avait point de rang, quoique sa belle-sœur fût assise, leur
père vivant et point démis par le crédit de M\textsuperscript{me} de
Soubise, sur le même exemple de la belle-sœur de M\textsuperscript{me}
de Chevreuse. Le roi disait toujours que Monsieur lui avait escroqué ce
tabouret\,; et du tabouret, les deux frères devinrent princes comme M.
de Soubise. M. de Guéméné se maria la première fois en 1678, puis en
1679\,; M. de Montauban, son frère, en 1682, et n'a point laissé
d'enfants.

Il est néanmoins vrai que M\textsuperscript{me} de Soubise, qui jamais
ne fut refusée de rien, ne put pourtant venir à bout d'une seule chose.
À la promotion de 1688, le duc de Montbazon, fou depuis longtemps, était
enfermé à Liège. Il était fils du frère aîné de M. de Soubise, ce prince
de Guéméné, chevalier de l'ordre en 1619, dont j'ai parlé ci-dessus. Il
ne s'était point démis de son duché\,; il était interdit, et par
conséquent hors d'état de s'en démettre comme de faire tout autre acte.
M. de Bouillon était exilé.

Cette promotion fut la première où le roi fit passer les ducs à brevet,
les maréchaux de France et les grands officiers de sa maison avant les
gentilshommes de cette même promotion, mais les gentilshommes dès lors
chevaliers de l'ordre continuant à les précéder. Le comte d'Auvergne et
M. de Soubise furent mis sur la liste du roi. Ils demandèrent de
précéder les ducs à brevet et les maréchaux de France de cette même
promotion, et M\textsuperscript{me} de Soubise, que le prince de Guéméné
qui n'en était pas en fût mis, et y tint le rang de duc de Montbazon.

Elle en savait trop pour l'espérer, mais elle compta d'obtenir l'autre
demande en compensation du refus de celle-ci. Pour cette fois elle se
trompa\,; non seulement le roi tint ferme, mais il se fâcha jusqu'à
ordonner à Châteauneuf, greffier de l'ordre, en plein chapitre, de
mettre sur son registre que MM. de Soubise et comte d'Auvergne
s'excusaient d'être de la promotion pour ne vouloir pas prendre l'ordre
dans le rang où leurs pères et leurs prédécesseurs s'étaient tenus
honorés de le recevoir.

Cela fit grand bruit, et la mortification fut cuisante. Mais
M\textsuperscript{me} de Soubise y sut pourvoir\,; elle amadoua et
intimida si bien Châteauneuf, qui était de ses amis et un fort pauvre
homme, qu'elle lui fit écrire sur ses registres que ces messieurs
n'avaient pas pris l'ordre pour n'avoir pas voulu céder à des cadets de
la maison de Lorraine.

Ils n'avaient jamais osé parler d'eux ni des ducs\,; il ne s'était agi
que de passer après le dernier duc et devant le premier des ducs à
brevet, et des maréchaux de France qui n'avaient jamais été mis les
premiers des gentilshommes d'une promotion, et qui encore alors et
depuis n'avaient aucun rang dans l'ordre que de leur ancienneté de
promotion, et dans celle où ils étaient reçus, celui où avec les
gentilshommes ils se trouvaient écrits dans la liste du roi, qui dans
celle-ci les avait écrits les premiers.

Ainsi l'adresse, pour ne rien dire de plus, substitua à une vérité
fâcheuse à leurs idées une fausseté à devenir preuve d'une prétention de
préséance qu'ils n'avaient jamais imaginée. Cela ne fut su que bien des
années après, et le rare est qu'il n'en a été autre chose. Pour MM. de
Bouillon, aucun d'eux jusque-là n'avait été à portée de l'ordre dans
aucune promotion jusqu'à celle-ci.

Le père du maréchal de Bouillon était, comme tous les siens, sans
prétention, et il mourut de ses blessures en 1557, qu'il avait reçues
auprès de Saint-Quentin, longtemps avant l'institution de l'ordre. Le
maréchal de Bouillon vécut et mourut huguenot en 1623. Le duc de
Bouillon ne se fit catholique qu'en 1637, et mourut en 1652 sans
promotion entre-deux, et M. de Turenne son frère ne se convertit qu'en
1668 et fut tué en 1675, aussi sans promotion entredeux. Ce n'était que
depuis la possession des biens de l'héritière de Sedan que le maréchal
de Bouillon, et ses enfants encore plus que lui, avaient commencé à
hasarder quelques prétentions, et le frère puîné du père de cette
héritière, et qui, par transaction toujours exécutée, précéda toute sa
vie en tous actes, lieux et occasions le maréchal de Bouillon, fut le
vingt-quatrième parmi les gentilshommes dans la première promotion, et
son fils le cinquante et unième dans celle de 1619, la même où le
marquis de Marigny dont j'ai parlé fut le cinquante-cinquième.

Voilà une longue parenthèse avant de venir au fait qui l'a engagée, mais
dont la curiosité pourra dédommager. Il faut pourtant en essuyer une
autre dont on ne peut passer le récit pour bien entendre le fait dont il
s'agit cette année.

Je ne sais où s'est prise l'origine du traitement si distingué que
reçoivent en Sorbonne les princes ou ceux qui en ont le rang pendant
leur licence. Ce ne peut être de la maison de Lorraine. Sa puissance a
bien pu dominer cette célèbre école au point de lui faire commettre
l'attentat de dégrader Henri III, et de le déclarer, sans droit ni
autorité quelconque, déchu de la couronne après l'exécution de Blois de
la fin de 1588, et après sa mort, Henri IV exclu de la couronne. Cette
même puissance aurait donc bien pu imaginer ces honneurs et se les faire
rendre\,; mais elle-même ne les prétendait pas alors, au moins le
principal, et qui emporte les autres distinctions qui ne sont que
locales. Elles consistent en celles-ci

Le prince, ou celui qui en a le rang, qui soutient une thèse, a des
gants dans ses mains, et son bonnet sur la tête pendant toute l'action,
et il est traité de \emph{sérénissime prince} tant par ceux qui
argumentent contre lui, que par celui qui préside à la thèse. Il l'est
aussi d'\emph{altesse sérénissime}, et le proviseur de Sorbonne la lui
donne dans ses lettres de doctorat. Quelque grands et puissants qu'aient
été ceux de la maison de Lorraine en France, depuis qu'ils s'y vinrent
établir sous François Ier jusqu'à la destruction de la Ligue sous Henri
IV, aucun d'eux n'a été traité d'altesse que le duc de Lorraine
souverain, et l'aîné ou le chef de leur maison.

De ceux qui ont pour ainsi dire régné en France parmi les troubles
qu'ils y formèrent, nul n'a été plus respecté ni plus grandement traité
par le duc de Mayenne, qui, pendant sa gestion de lieutenant général de
l'État et de couronne de France, n'omit aucune de celles qui sont
réservées à la personne et à l'autorité de nos rois. Il fit, en son
propre nom, des déclarations et des délits qui furent enregistrés au
parlement\,; il fit des maréchaux de France qui en exercèrent les
offices, et dont quelques-uns les conservèrent en faisant leur traité
avec Henri IV\,; il punit de mort et d'exil, et donna grâce de la vie\,;
il disposa en roi des charges, des emplois, des bénéfices de toutes les
sortes, et grand nombre de ses pourvus gardèrent leurs places à la paix.

On ne peut donc pas croire qu'au temps de l'exercice de l'autorité et de
la puissance royale qu'il exerça en plein dans son parti qui était
presque toute la France et Paris surtout, personne de ce parti lui eût
osé ni voulu même refuser aucun des honneurs et des distinctions, même
nouvelles, qu'il eût voulu s'arroger. Les histoires et les Mémoires de
ces malheureux temps rapportent une infinité d'actes de M. de Mayenne,
et de lettres de toutes sortes de personnes à lui écrites. Dans pas une
de ces pièces il ne se trouve d'\emph{altesse}. C'est \emph{vous}
partout, et jusqu'à son propre secrétaire ne lui écrit jamais autrement.
Il est donc vrai qu'il n'imaginait pas de prétendre ce traitement, comme
alors ni le duc de Lorraine ni aucun autre souverain qui se faisaient
donner l'\emph{altesse}, n'imaginaient pas le \emph{sérénissime}.

Ce superlatif ne leur est venu dans la tête et dans l'usage que
longtemps depuis, lorsque leurs cadets se sont fait traiter
d'\emph{altesse}, pour se distinguer d'eux, et cette distinction a été
de courte durée. Les mêmes qui s'étaient fait donner l'\emph{altesse}
comme les souverains, ont pris aussi le \emph{sérénissime} presque
aussitôt qu'ils l'ont vu inventer, et de là est venue de nos jours
l'\emph{altesse royale} qui n'était que pour les enfants de nos rois,
descendue au leur, et à cause de cela Monsieur et Madame de le quitter,
et M. de Savoie, M. le grand-duc de Toscane, et longtemps après M. de
Lorraine, la prendre sous prétexte d'avoir épousé des petites-filles de
France qui en étaient traitées, tandis que les ducs de Lorraine et de
Savoie, gendres de nos rois et leurs beaux-frères, s'étaient contentés
de la simple altesse.

Depuis M. de Mayenne, aucun de sa maison n'a été sur les bancs de
Sorbonne jusqu'à un fils de M. le Grand, et longtemps après, un autre,
tous deux de nos jours, et qui trouvèrent ces distinctions établies en
Sorbonne pour beaucoup moins qu'eux, et qui les ont eues.

Le cardinal de Guise, archevêque de Reims, mort à la suite de Louis
XIII, pendant le siège de Saint-Jean-d'Angély, n'a jamais été que
sous-diacre, et n'avait jamais songé à entrer en licence, beaucoup moins
M. de Guise de Naples, archevêque de Reims, aussi dans son enfance, et
qui ne l'a jamais été que commendataire\footnote{Le commendataire
  percevait les revenus et avait les droits honorifiques attachés à une
  dignité ecclésiastique, sans être chargé de la discipline et souvent
  même sans être engagé dans les ordres.}. D'autres maisons souveraines,
aucun n'a été prélat en France, ni été en Sorbonne, et toutes ces choses
sont des faits certains.

Il faut donc dire que le cardinal de Bouillon est celui en faveur duquel
ils ont été inventés. Il était né en août 1643, et fut cardinal en août
1669. Il avait donc vingt-six ans quand il le fut, et c'est dans cet
intervalle qu'il obtint ces honneurs en Sorbonne. La façon dont il fut
cardinal montrera toute seule comment ces distinctions lui furent
déférées en Sorbonne. M. de Turenne fut fait maréchal général des camps
et armées de France, le 7 avril 1660, la cour étant à Montpellier. Son
neveu avait alors dix-sept ans. Cette époque marque donc bien en quelle
situation était M. de Turenne. Elle ne déchut pas depuis, et personne
n'ignore le degré de faveur, de crédit, d'autorité, où a toujours été ce
grand homme, depuis qu'après tant d'écarts il se fut sincèrement attaché
au roi et au gouvernement la dernière fois. Il serait aussi difficile de
ne savoir pas l'attachement extrême qu'il eut pour la grandeur et les
distinctions de sa maison, qui toute sa vie le conduisit et fut sa
passion dominante, et tous les avantages qu'il sut lui procurer par
toutes sortes d'occasions et de moyens.

Il regarda son neveu comme y pouvant beaucoup contribuer en le poussant
dans l'Église, et M. de Péréfixe, archevêque de Paris, dans la confiance
et le crédit où il était à la cour, comme un instrument très propre à
l'avancer. Il était son ami, et ce prélat s'en faisait un grand honneur.
Il lui recommanda fort son neveu, qui eut l'esprit de lui faire une cour
assidue, et de le gagner aussi personnellement. Il arriva que M. de
Louvois, déjà considérable par soi, aussi bien que par son père, et qui
n'avait ni sa modestie ni sa retenue, imagina de capter si bien l'évêque
de Langres, qu'il fit l'abbé Le Tellier, son frère, son coadjuteur. Ce
prélat était ce fameux abbé de La Rivière, qui avait si Longtemps
gouverné M. Gaston, qui par là avait tant figuré pendant les troubles de
la minorité du roi, qui était devenu ministre, qui avait tant fait
compter tous les partis avec lui, qui avait eu la nomination au
cardinalat, et qui tout homme de rien qu'il était, et enfin perdu, eut
en dédommagement de ce qu'il avait été et prétendu cet évêché,
duché-pairie et force bénéfices. Il savait par expérience active et
passive ce que peuvent les ministres. Il fut ravi de s'acquérir M. de
Louvois et son père, et alla avec les deux frères dire sa résolution à
M. Le Tellier. Celui-ci fut épouvanté d'un siège de cette dignité\,;
mais l'affaire était faite, il ne put s'empêcher de se joindre à eux
pour la faire agréer au roi.

Le bruit qu'elle fit réveilla le cardinal Antoine Barberin,
archevêque-duc de Reims\,; sa puissance et sa chute à Rome, la
protection que le cardinal Mazarin lui avait accordée et à sa famille
fugitive en France, ne lui avait pas donné moins d'expérience et
d'instruction qu'à La Rivière, touchant les ministres. Il accourut dès
le lendemain chez Le Tellier, où il envoya chercher ses fils, leur fit
de grands reproches de s'être adressés à M. de Langres plutôt qu'à lui,
et de ce pas alla demander au roi la coadjutorerie de Reims pour l'abbé
Le Tellier, et l'obtint sur-le-champ.

Une si prodigieuse fortune pour un homme de l'état et de l'âge de l'abbé
Le Tellier, qui n'avait pas encore vingt-sept ans entièrement accomplis,
fit un grand bruit dans le monde, et surprit jusqu'à sa famille et
jusqu'à lui-même. M. de Turenne qui n'aimait pas M. de Louvois, ni guère
mieux M. Le Tellier, en fut piqué au dernier point. C'était de plus un
morceau unique qu'il convoitait pour son neveu, qui déjà plein
d'ambition fut enragé de se le voir ôter, et par l'abbé Le Tellier. Ils
imaginèrent la coadjutorerie de Paris, et avec les avances d'amitié
intime qu'ils avaient avec M. de Péréfixe, ils le lui persuadèrent si
bien et sitôt, qu'il ne le désira pas moins passionnément qu'eux. Il la
demanda au roi, et fut bien étonné d'y trouver de la résistance. Il ne
se rebuta point\,; M. de Turenne vint au secours, qui s'y mit tout
entier comme pour un coup de partie. Le roi dans l'embarras du refus à
M. de Péréfixe qu'il aimait et qu'il considérait fort, et encore plus à
M. de Turenne dans la posture où il était, et qui était pourtant résolu
de ne hasarder pas de faire un second coadjuteur de Retz, en sortit par
proposer à M. de Turenne sa nomination au cardinalat au lieu de la
coadjutorerie, et se trouva heureux et obligé à M. de Turenne de ce
qu'il voulut bien l'accepter. La promotion des couronnes était instante,
ainsi ils n'attendirent pas, et se dépiquèrent ainsi de la coadjutorerie
de l'abbé Le Tellier.

M. de Péréfixe était proviseur de Sorbonne, et en était d'autant plus le
maître, qu'il s'était plus que prêté à toutes les volontés de la cour,
contre M. Arnauld et ses amis, et qu'il avait fait main basse sur la
Sorbonne, et répandu grand nombre de lettres de cachet. D'autre part, le
jeune abbé s'était dévoué aux jésuites, auxquels il a été toute sa vie
abandonné, et dont il a tiré de grands services. Avec ces secours, M. de
Turenne put prétendre pour lui toutes les nouveautés qu'il voulut\,;
elles s'exécutèrent plus tôt que personne ne s'en fut avisé, et une fois
faites et sans disputes ni plaintes, la cour n'en dit rien aussi, et ne
voulut pas courre après, et donner ce dégoût amer à M. de
Turenne\footnote{Il y a ici quelques difficultés de dates que l'on
  trouvera éclaircies dans les notes à la fin du volume.}. N'est-ce
point là voler un peu sur les grands chemins\,? Si on examine bien tout
ce rang de prince étranger, même dans ceux qui le sont par naissance, on
le trouvera tout composé de pareils brigandages.

Sur cet exemple, l'abbé de Soubise prétendit les mêmes distinctions. Il
y trouva de la résistance. M\textsuperscript{me} de Soubise n'eut pas
peine à la vaincre. Le roi a toujours regardé celui-ci avec d'autres
yeux que les autres enfants de M\textsuperscript{me} de Soubise, lui et
un plus jeune qu'on appelait le prince Maximilien\,; car depuis elle,
tout fut et se nomma prince dans cette maison. Mais ce prince Maximilien
fut tué de fort bonne heure, et n'eut pas le temps comme l'abbé de
profiter de l'affection particulière du roi. Il commanda au proviseur et
à la Sorbonne, et l'abbé de Soubise fut traité comme l'avait été le
cardinal de Bouillon. La suite naturelle était que tout finît de même.
Il avait été prieur de Sorbonne pour briller et capter cette école
irritée des ordres du roi à son égard. Il en fallut venir à ses lettres
de doctorat, et c'est le point qui a causé toute cette digression pour
l'entendre. M. de Reims n'y voulut point mettre d'\emph{altesse
sérénissime}. Il était proviseur de Sorbonne, et alléguait que M. de
Péréfixe qui les avait données avec ce traitement à M. de Bouillon,
depuis cardinal, n'était pas duc et pair. M\textsuperscript{me} de
Soubise en vint à bout aussi aisément que du reste. Le roi l'ordonna à
l'archevêque de Reims, et lui dit pour toute raison qu'il ne donnait pas
ces lettres comme archevêque de Reims, mais comme proviseur de Sorbonne,
et qu'il le voulait ainsi\,; on peut juger qu'il fut bientôt obéi.

Presque aussitôt après, le prince de Montbazon, second fils du prince de
Guéméné (car l'aîné était enfermé dans une abbaye), épousa une fille du
duc de Bouillon. M\textsuperscript{me} de Soubise obtint que les
fiançailles se feraient dans le cabinet du roi.

Avant de quitter cette maison, il faut dire que le prince de Guéméné,
mort duc de Montbazon, en février 1667, et frère de M. de Soubise, avait
trente-un an plus que lui, et Aime de Chevreuse, morte en 1679, leur
sœur, suivait le siècle. La princesse de Guéméné, morte duchesse de
Montbazon en 1657, mère de M. de Soubise, était cette belle
M\textsuperscript{me} de Montbazon dont on a fait ce conte qui a trouvé
croyance\,: que l'abbé de Rancé, depuis ce célèbre abbé de la Trappe, en
était fort amoureux et bien traité\,; qu'il la quitta à Paris, se
portant fort bien, pour aller faire un tour à la campagne\,; que bientôt
après, y ayant appris qu'elle était tombée malade, il était accouru, et
qu'étant entré brusquement dans son appartement, le premier objet qui y
était tombé sous ses yeux avait été sa tête, que les chirurgiens, en
l'ouvrant, avait séparée\,; qu'il n'avait appris sa mort que par là, et
que la surprise et l'horreur de ce spectacle joint à la douleur d'un
homme passionné et heureux, l'avait converti, jeté dans la retraite, et
de là dans l'ordre de Saint-Bernard et dans sa réforme. Il n'y a rien de
vrai en cela, mais seulement des choses qui ont donné cours à cette
fiction. Je l'ai demandé franchement à M. de la Trappe, non pas
grossièrement l'amour et beaucoup moins le bonheur, mais le fait, et
voici ce que j'en ai appris était intimement de ses amis, ne bougeait de
l'hôtel de Montbazon, et ami de tous les personnages de la Fronde, de M.
de Châteauneuf, de M\textsuperscript{me} de Chevreuse, de M. de
Montrésor et de ce qui s'appelait alors les importants, mais plus
particulièrement de M. de Beaufort avec qui il faisait très souvent des
parties de chasse, et dans la dernière intimité avec le cardinal de Retz
et qui a duré jusqu'à sa mort. M\textsuperscript{me} de Montbazon mourut
de la rougeole en fort peu de jours. M. de Rancé était auprès d'elle, ne
la quitta point, lui vit recevoir les sacrements, et fut présent à sa
mort. La vérité est que, déjà touché et tiraillé entre Dieu et le monde,
méditant déjà depuis quelque temps une retraite, les réflexions que
cette mort si prompte fit faire à son cœur et à son esprit achevèrent de
le déterminer, et peu après il s'en alla en sa maison de Véretz en
Touraine, qui fut le commencement de sa séparation du monde.

La princesse de Guéméné, si initiée auprès de la reine mère par
M\textsuperscript{me} de Chevreuse, sœur de son mari et de M. de
Soubise, et qui attrapa le tabouret par les bricoles des particuliers et
du Val-de-Grâce, mourut duchesse de Montbazon, en 1685, à
quatre-vingt-un ans. Elle était mère du duc de Montbazon, mort fou en
1699, enfermé depuis longues années à Liège, et du chevalier de Rohan,
décapité pour crime de lèse-majesté, 17 novembre 1674, quelque temps
après avoir vendu à M. de Soyecourt sa charge de grand veneur, qu'il
avait eue en survivance de son père et que M. de La Rochefoucauld eut à
la mort de Soyecourt.

\hypertarget{chapitre-xi.}{%
\chapter{CHAPITRE XI.}\label{chapitre-xi.}}

1698

~

{\textsc{Mariages du fils du duc de La Force et de
M\textsuperscript{lle} de Bosmelet\,; de La Vallière et d'une fille du
duc de Noailles\,; de La Carte et d'une fille du duc de La Ferté\,; de
Sassenage avec une fille du duc de Chevreuse, veuve de Morstein.}}
{\textsc{- Cent vingt mille livres à M. le Grand, et soixante mille
livres au chevalier de Lorraine.}} {\textsc{- Charnacé arrêté pour
fausse monnaie, etc.\,; il déplace plaisamment une maison de paysan qui
l'offusquait.}} {\textsc{- Carrosse de la duchesse de Verneuil exclu des
entrées des ambassadeurs.}} {\textsc{- Querelle de M. le prince de Conti
et du grand prieur qui est mis à la Bastille et n'en sort qu'en allant
demander pardon en propres termes à M. le prince de Conti.}} {\textsc{-
L'électeur de Saxe reconnu roi de Pologne par le roi.}} {\textsc{-
Naissance de mon fils aîné.}} {\textsc{- Éclat entre le duc de Bouillon
et le duc d'Albret, son fils.}} {\textsc{- Curé de Seurre, ami de
M\textsuperscript{me} Guyon, brûlé à Dijon.}} {\textsc{- Réponse de M.
de Cambrai à M. de Meaux.}} {\textsc{- Mort de la duchesse de
Richelieu\,; de la princesse d'Espinoy, douairière\,; ses enfants\,; ses
progrès.}} {\textsc{- Entreprise de M\textsuperscript{lle} de Melun qui
frise de près l'affront.}} {\textsc{- Mort du duc d'Estrées et sa
dépouille.}} {\textsc{- Mort du duc de Chaulnes.}} {\textsc{- Mort de la
duchesse de Choiseul.}}

~

Plusieurs mariages suivirent de près celui de M. de Montbazon. M. de La
Force maria son fils aîné à M\textsuperscript{lle} de Bosmelet, fille
unique, extrêmement riche, d'un président à mortier du parlement de
Rouen, et d'une fille de Chavigny, secrétaire d'État, sœur de la
maréchale de Clérembault, de l'ancien évêque de Troyes, etc.

La Vallière épousa une fille du duc de Noailles. M\textsuperscript{me}
la princesse de Conti, cousine germaine de La Vallière et qui l'aimait
fort, parla libéralement dans le contrat, et lit la noce en sa belle
maison dans l'avenue de Versailles. Ce fut une espèce de fête où
Monseigneur se trouva.

Il s'en fit un autre assez bizarre\,: La Carte, gentilhomme de Poitou,
fort mince et fort pauvre, s'attacha à Monsieur qui prit pour lui un
goût, que sa figure des plus communes ne méritait pas de celui de ce
prince qui s'en entêta extraordinairement, et qui de charge en charge
chez lui, le lit rapidement monter à celle de premier gentilhomme de sa
chambre, et lui fit beaucoup de grâces pécuniaires. À la fin il le
voulut marier. La duchesse de La Ferté avait encore une fille qui avait
un peu rôti le balai, et qui commençait à monter en graine. Elle était
fort bien avec Monsieur qui lui proposa ce mariage\,: elle se fit prier,
et elle voulut que La Carte prit les livrées et les armes de sa fille et
le nom de marquis de La Ferté. Cela l'honorait trop pour n'y pas
consentir avec joie. Mais le duc de La Ferté, de tout temps brouillé
avec sa femme, et non sans cause, séparé d'elle et qui ne la voyait
point, se fit tenir à quatre, et les Saint-Nectaire encore plus, qui
s'opposèrent en forme à la prostitution de leur nom et de leurs armes.
Après bien du vacarme et des propos fâcheux, Monsieur apaisa tout avec
de l'argent. Tous consentirent, et la duchesse de La Ferté donna une
fête à Monsieur en faisant la noce.

Sassenage, autre premier gentilhomme de la chambre de Monsieur, épousa
la veuve de Morstein, fille du duc de Chevreuse. C'était une âme d'élite
du petit troupeau de M\textsuperscript{me} Guyon et de M. de Cambrai,
qui avec toute la profondeur de cette dévotion voulut se remarier. Ce
gendre n'était pas plus fait pour cette famille que M. de Lévi, et
beaucoup moins pour une femme si mystique\,; il a pourtant très bien
vécu avec eux tous.

Le roi, qui venait de payer les dettes de M. de La Rochefoucauld, et qui
aimait fort aussi M. le Grand, ne voulut apparemment pas faire de
jalousie entre ces deux émules en cas que son présent fût éventé\,; il
en fit un de quarante mille écus à M. le Grand, et un autre de vingt
mille écus au chevalier de Lorraine.

Il fit arrêter Charnacé en province où, déjà fort mécontent de sa
conduite en Anjou où il était retiré chez lui, il l'avait relégué
ailleurs, et de là conduire à Montauban, fort accusé de beaucoup de
méchantes choses, surtout de fausse monnaie. C'était un garçon d'esprit
qui avait été page du roi et officier dans ses gardes du corps, fort du
monde, et puis retiré chez lui où il avait souvent fait bien des
fredaines, mais il avait toujours trouvé bonté et protection dans le
roi. Il en fit une entre autres pleine d'esprit et dont on ne put que
rire.

Il avait une très longue et parfaitement belle avenue devant sa maison
en Anjou, dans laquelle était placée une maison de paysan et son petit
jardin qui s'y était apparemment trouvée lorsqu'elle fut plantée, et que
jamais Charnacé ni son père n'avaient pu réduire ce paysan à la leur
vendre, quelque avantage qu'ils lui en eussent offert, et c'est une
opiniâtreté dont quantité de petits propriétaires se piquent, pour faire
enrager des gens à la convenance et quelquefois à la nécessité desquels
ils sont. Charnacé ne sachant plus qu'y faire avait laissé cela là
depuis fort longtemps sans en plus parler. Enfin, fatigué de cette
chaumine qui lui bouchait tout l'agrément de son avenue, il imagina un
tour de passe-passe. Le paysan qui y demeurait, et à qui elle
appartenait, était tailleur de son métier quand il trouvait à l'exercer,
et il était chez lui tout seul, sans femme ni enfants. Charnacé l'envoie
chercher, lui dit qu'il est mandé à la cour pour un emploi de
conséquence, qu'il est pressé de s'y rendre, mais qu'il lui faut une
livrée. Ils font marché comptant\,; mais Charnacé stipule qu'il ne veut
point se fier à ses délais, et que, moyennant quelque chose de plus, il
ne veut point qu'il sorte de chez lui que sa livrée ne soit faite, et
qu'il le couchera, le nourrira et le payera avant de le renvoyer. Le
tailleur s'y accorde et se met à travailler. Pendant qu'il y est occupé,
Charnacé fait prendre avec la dernière exactitude le plan et les
dimensions de sa maison et de son jardin, des pièces de l'intérieur,
jusque de la position des ustensiles et du petit meuble, fait démonter
la maison et emporter tout ce qui y était, remonte la maison telle
qu'elle était au juste dedans et dehors, à quatre portées de mousquet, à
côté de son avenue, replace tous les meubles et ustensiles dans la même
position en laquelle on les avait trouvés, et rétablit le petit jardin
de même, en même temps fait aplanir et nettoyer l'endroit de l'avenue où
elle était, en sorte qu'il n'y parut pas.

Tout cela fut exécuté encore plus tôt que la livrée faite, et cependant
le tailleur doucement gardé à vue de peur de quelque indiscrétion. Enfin
la besogne achevée de part et d'autre, Charnacé amuse son homme jusqu'à
la nuit bien noire, le paye et le renvoie content. Le voilà qui enfile
l'avenue. Bientôt il la trouve longue, après il va aux arbres et n'en
trouve plus. Il s'aperçait qu'il a passé le bout et revient à tâtons
chercher les arbres. Il les suit à l'estime, puis croise et ne trouve
point sa maison. Il ne comprend point cette aventure. La nuit se passe
dans cet exercice, le jour arrive et devient bientôt assez clair pour
aviser sa maison. Il ne voit rien, il se frotte les yeux, il cherche
d'autres objets pour découvrir si c'est la faute de sa vue. Enfin il
croit que le diable s'en mêle, et qu'il a emporté sa maison. À force
d'aller, de venir, et de porter sa vue de tous côtés y il aperçait, à
une assez grande distance de l'avenue, une maison qui ressemble à la
sienne comme deux gouttes d'eau. Il ne peut croire que cela soit\,; mais
la curiosité le fait aller où elle est, et où il n'a jamais vu de
maison. Plus il approche, plus il reconnaît que c'est la sienne. Pour
s'assurer mieux de ce qui lui tourne la tête, il présente sa clef, elle
ouvre, il entre, il retrouve tout ce qu'il y avait laissé, et
précisément dans la même place. Il est prêt à en pâmer, et il demeure
convaincu que c'est un tour de sorcier. La journée ne fut pas bien
avancée que la risée du château et du village l'instruisit de la vérité
du sortilège, et le mit en furie. Il veut plaider\,; il veut demander
justice à l'intendant\,; et partout on s'en moque. Le roi le sut qui en
rit aussi, et Charnacé eut son avenue libre. S'il n'avait jamais fait
pis il aurait conservé sa réputation et sa liberté.

Comme presque tout ce que j'ai écrit depuis que j'ai parlé de la
brillante ambassade de milord Portland s'est passé pendant qu'il était
ici, je ne ferai point de difficulté d'ajouter en cet endroit un oubli
que j'ai fait sur son entrée, dont je n'ai rien dit, parce qu'à la
magnificence près, elle se passa comme toutes les autres\,; mais il y
eut une difficulté. Depuis que M\textsuperscript{me} de Verneuil fut, à
sa grande surprise à elle-même, devenue princesse du sang, elle avait
envoyé son carrosse aux entrées des ambassadeurs qui n'y avaient pas
pris garde. Portland attentif à tout en fut averti, et déclara qu'il ne
souffrirait pas que ce carrosse passât devant le sien\,; que si d'autres
ambassadeurs l'avaient souffert, c'était leur affaire\,; mais qu'il ne
ferait point d'entrée bien résolument, plutôt qu'endurer une nouveauté
sans exemple avec des ambassadeurs d'Angleterre, ou qu'il en écrirait si
on voulait, et en attendait les ordres là-dessus, qui était tout ce
qu'il pouvait faire. Il se fit des allées et des venues qui
n'ébranlèrent point la fermeté de Portland\,; sur quoi on aima mieux que
le carrosse de M\textsuperscript{me} de Verneuil ne se présentât point,
que d'insister davantage ou de se commettre à la réponse d'un pays où
les bâtards des rois sont ce qu'ils ont été partout, et ce qu'ils
devraient toujours être, c'est-à-dire des néants sans état et sans nom,
si ce n'est par les charges et par les dignités qui les en tirent, et
qui les mettent au rang exact de celles dont ils sont revêtus.
Heemskerke, réveillé pour son entrée par cette aventure, forma la même
difficulté que Portland, et eut le même succès.

Il arriva à Meudon une scène fort étrange\,: on jouait après souper, et
Monseigneur s'alla coucher\,; assez de courtisans demeurèrent à jouer ou
à voir jouer. M. le prince de Conti et le grand prieur étaient des
acteurs. Il y eut un coup qui fit une dispute. On a déjà vu en plus d'un
endroit que ce prince et MM. de Vendôme ne s'aimaient pas, et d'une
manière même assez déclarée. La faveur de M. de Vendôme qui ne l'était
pas moins, sa préférence sur les princes du sang pour le commandement
des armées, ses rangs et ses distinctions, crûs à pas de géant, touchant
presque le niveau des princes du sang, avaient tellement augmenté
l'audace du grand prieur qu'il lui échappa dans la dispute une aigreur
et des propos qui eussent été trop forts dans un égal, et qui lui
attirèrent une cruelle repartie, où le prince de Conti tançait à bout
portant et sa fidélité au jeu, et son courage à la guerre, l'un et
l'autre à la vérité fort peu nets. Là-dessus le grand prieur s'emporte,
jette les cartes, et lui demande satisfaction, l'épée à la main, de
cette insulte. Le prince de Conti, d'un sourire de mépris, l'avertit
qu'il lui manquait de respect, mais qu'en même temps il était facile à
rencontrer, parce qu'il allait partout et tout seul. L'arrivée de
Monseigneur tout nu en robe de chambre, que quelqu'un alla avertir,
imposa à tous deux. Il ordonna au marquis de Gesvres qui s'y trouva
d'aller rendre compte au roi de ce qu'il venait d'arriver, et chacun
s'en alla se coucher. Le marquis de Gesvres, au réveil du roi,
s'acquitta de sa commission, sur quoi le roi manda à Monseigneur
d'envoyer, par l'exempt des gardes servant auprès de lui, le grand
prieur à la Bastille. Celui-ci était déjà venu de Meudon pour parler au
roi de son affaire, et fit demander audience par Lavienne. Le roi lui
manda qu'il lui défendait de se présenter devant lui, et lui ordonna de
s'en aller sur-le-champ à la Bastille, où il trouverait ordre de le
recevoir. Il fallut obéir. Un moment après arriva M. le prince de Conti
qui entretint le roi en particulier dans son cabinet.

Le lendemain 30 juillet, M. de Vendôme arriva d'Anet, eut audience du
roi, et de là alla chez M. le prince de Conti. Ce fut un grand émoi à la
cour. Les princes du sang prirent l'affaire fort haut, et les bâtards
{[}furent{]} si embarrassés, que le 2 août, M. du Maine et M. le comte
de Toulouse allèrent voir M, le prince de Conti. Enfin l'affaire
s'accommoda à Marly, le 6 août le matin\,; Monseigneur pria le roi de
vouloir bien pardonner au grand prieur et le faire sortir de la
Bastille, et l'assura que M. le prince de Conti lui pardonnait aussi.
Là-dessus le roi envoya chercher M. de Vendôme. Il lui dit qu'il allait
faire expédier l'ordre pour faire sortir son frère de la Bastille\,;
qu'il pourrait le lendemain l'amener à Marly, où d'abord il voulait
qu'il allât demander pardon à M. le prince de Conti, après à
Monseigneur\,; qu'il le verrait ensuite, et que de là il s'en
retournerait à Paris. Il ajouta qu'au retour à Versailles, le grand
prieur pourrait y venir. La chose fut exécutée de point en point de la
sorte le lendemain jeudi 7 août, les deux pardons demandés et en propres
termes, et M. de Vendôme présent avec son frère. Ce ne fut pas sans que
nature pâtit cruellement en tous les deux\,; mais il fallut avaler le
calice, et calmer les princes du sang, qui étoient extrêmement animés.

Pendant les jours de cette querelle, un envoyé de l'électeur de Saxe qui
venait d'arriver à Paris eut audience du roi, et son maître fut
publiquement reconnu ici roi de Pologne.

Presque en même temps, c'est-à-dire le 29 mai dans la matinée,
M\textsuperscript{me} de Saint-Simon accoucha fort heureusement, et Dieu
nous fit la grâce de nous donner un fils. Il porta, comme j'avais fait,
le nom de vidame de Chartres. Je ne sais pourquoi on a la fantaisie des
noms singuliers\,; mais ils séduisent en toutes nations, et ceux même
qui en sentent le faible les imitent. Il est vrai que les titres de
comte et de marquis sont tombés dans la poussière par la quantité de
gens de rien et même sans terre qui les usurpent, et par là tombés dans
le néant, si bien même que les gens de qualité qui sont marquis ou
comtes (qu'ils me permettent de le dire), ont le ridicule d'être blessés
qu'on leur donne ces titres en parlant à eux. Il reste pourtant vrai que
ces titres émanent d'une érection de terre et d'une grâce du roi, et
quoique cela n'ait plus de distinction, ces titres dans leur origine, et
bien longtemps depuis, ont eu des fonctions, et que leurs distinctions
ont duré bien au delà de ces fonctions. Les vidames, au contraire, ne
sont que les premiers officiers de la maison de certains évêques par un
fief inféodé d'eux, et à titre de leurs premiers vassaux conduisaient
tous leurs autres vassaux à la guerre, du temps qu'elle se faisait ainsi
entre les seigneurs les uns contre les autres, ou dans les armées que
nos rois assemblaient contre leurs ennemis, avant qu'ils eussent établi
leur milice sur le pied que peu à peu elle a été mise, et que peu à peu
ils ont anéanti le service avec le besoin des vassaux, et toute la
puissance et l'autorité des seigneurs. Il n'y eut donc jamais de
comparaison entre le titre de vidame, qui ne marque que le vassal et
l'officier d'un évêque, et les titres qui par fief émanent des rois.
Mais comme on n'a guère connu de vidames que ceux de Laon, d'Amiens, du
Mans et de Chartres, et que entre ceux-là un Moratoire, dont la maison
avait pris le nom de Vendôme pour en avoir épousé l'héritière dont les
Montoire relevaient\,; parce que, dis-je, ce Vendôme s'illustra par sa
gentillesse, ses galanteries, ses grands biens, sa magnificence et la
splendeur du tournoi qu'il donna, et par les intrigues et les grandes
affaires où il n'eut que trop de part, puisqu'elles le firent périr dans
la Bastille, ce nom de vidame de Chartres a paru beau, et ce fief ayant
toujours appartenu aux mêmes qui avaient la terre de la Ferté-Arnaud,
qui de ce Vendôme tomba par sa sœur aux Ferrières, et de ceux-ci encore
par une sœur aux La Fin, Louis XIII l'ayant fait acheter à mon père,
parce qu'il n'y a que vingt lieues de là à Versailles, il acheta en même
temps ce fief dans Chartres qui en est le vidame, et m'en fit porter le
nom, que j'ai fait après porter à mon fils.

Un peu devant le voyage de Compiègne, M. de Bouillon et le duc d'Albret,
son fils aîné, se brouillèrent avec éclat\,; il y avait quelque temps
que, de l'agrément du père, le fils avait fait un voyage à Turenne, pour
en rapporter le présent qui se faisait aux fils aînés du seigneur de
Turenne, la première fois qu'il y allait. Le duc d'Albret y avait mené
des gens d'affaires qui y trouvèrent un testament du maréchal de
Bouillon, portant, à ce qu'ils prétendirent, une substitution dûment
faite et insinuée partout où il appartenait, qui assurait tout d'aîné en
aîné, et qui, par conséquent, liait les mains à M. de Bouillon sur tout
avantage à ses cadets, et le mettait de plus hors d'état de payer ses
créanciers personnels, que sur les revenus pendant sa vie. Au retour de
M. d'Albret, ce feu couva sous la cendre. On tourna M. de Bouillon, on
n'osait tout dire\,; à la fin on vint au fait, et M. d'Albret porta le
testament au lieutenant civil. À quelques semaines de là, M. de Bouillon
étant allé à Évreux, son fils y envoya lui signifier un exploit par un
huissier à la chaîne, qui sont ceux qui peuvent exploiter indifféremment
partout et que chacun qui veut emploie, quand on veut faire une
signification délicate et forte, parce que ceux-là sont toujours fort
respectés et instrumentent avec une grosse chaîne d'or au cou, d'où pend
une médaille du roi. Ils sont en même temps huissiers du conseil, et y
servent avec cette chaîne. Cette démarche causa un grand vacarme\,: M.
de Bouillon jeta les hauts cris, lit ses plaintes au roi, et lui en dit,
dans sa colère, tout ce qu'il put de pis, et il exigea de sa plus proche
famille et de ses amis de ne point voir le duc d'Albret. Le roi
s'expliqua assez partialement en faveur de M. de Bouillon pour mettre
toute la cour de son côté, et ce procédé du fils y mit presque tout le
monde, indépendamment de l'esprit courtisan. M. d'Albret, assez gauche
et assez empêtré de son naturel, n'osa presque plus se montrer, quoique
fort soutenu de M. de La Trémoille, son beau-père, et cette affaire le
renferma fort dans l'obscurité et dans la mauvaise compagnie, quoiqu'il
eût beaucoup d'esprit, et même fort orné, mais avec cela peu agréable.

Un arrêt du parlement de Dijon fit en même temps un grand bruit. Il fit
brûler le curé de Seurre, convaincu de beaucoup d'abominations, en suite
des erreurs de Molinos et fort des amis de M\textsuperscript{me} Guyon.
Cela vint fort mal à propos en cadence avec la réponse de M. de Cambrai
aux \emph{États d'oraison} de M. de Meaux, qui n'eut rien moins que le
succès et l'applaudissement qu'avait eus ce livre, et qu'il conserva
toujours. M. de Paris avait, quelque temps auparavant, fait une visite
aux ducs de Chevreuse et de Beauvilliers. Ils avaient su la belle action
qu'il avait faite à l'égard du dernier, et qui portait sur tous les
deux. Ils se séparèrent donc fort contents de part et d'autre, et ils
firent depuis, dans toutes les suites de cette affaire, une grande
différence de lui aux deux autres prélats.

La duchesse de Richelieu mourut d'une longue, cruelle et bien étrange
maladie. On lui trouva tous les os de la tête cariés jusqu'au cou, et
tout le reste parfaitement sain. Elle était Acigné, de très bonne maison
de Bretagne, et fort proche parente de ma mère, qui était issue de
germaine de sa mère, et fort de ses amies. C'est la seule dont M. de
Richelieu ait eu des enfants.

La princesse d'Espinoy la mère mourut la veille ou le même jour plus
tristement encore. Elle était du voyage de Compiègne, et voulait être de
celui de Marly qui le précédait immédiatement. Allant à Versailles pour
se présenter le soir même pour Marly, elle vint à six chevaux chez
M\textsuperscript{me} de Saint-Simon dont la porte était encore fermée
de sa couche, mais qui lui fut ouverte par l'amitié intime d'elle et de
ses sœurs avec MM. de Duras et de Lorges dont j'ai parlé. Quoiqu'elle
mit beaucoup de rouge, elle la parut tant partout où on n'en met point,
et les veines si grosses, que M\textsuperscript{me} de Saint-Simon ne
put s'empêcher de lui dire qu'elle ferait mieux de se faire saigner que
d'aller à Versailles. M\textsuperscript{me} d'Espinoy répondit qu'elle
en avait été fort tentée par le grand besoin qu'elle s'en sentait, mais
qu'elle n'en avait pas eu le temps à tout ce qu'elle avait eu à faire
avant Compiègne, qu'il fallait qu'elle allât à Marly, et que là elle se
ferait saigner. Du logis elle alla débarquer tout droit chez M. de
Barbezieux, à Versailles\,; elle entra chez lui en bonne santé\,;
l'instant d'après elle se trouva mal\,; on ne fit que la jeter sur le
lit de Barbezieux\,; elle était morte. On lui trouva la tête noyée de
sang. Ce fut une vraie perte pour sa famille et pour ses amis, et elle
en avait beaucoup. C'était une femme d'esprit et de grand sens, bonne et
aussi vraie et sûre que sa sœur de Soubise était fausse, noble,
généreuse, bonne et utile amie, accorte, qui aimait passionnément ses
enfants et qui, excepté ses amis, ne faisait guère de choses sans vues.
Le prince d'Espinoy, qui l'avait épousée en secondes noces, avait obtenu
un tabouret de grâce par son premier mariage avec une fille du vieux
Charost, dont une seule fille, première femme du petit-fils de ce
bonhomme.

M\textsuperscript{me} d'Espinoy était demeurée veuve avec deux fils et
deux filles. M. d'Espinoy avait été chevalier de l'ordre de la promotion
de 1661, et y avait marché le vingt-neuvième, c'est-à-dire le
dix-huitième des gentilshommes, entre le comte de Tonnerre et le
maréchal d'Albret, et n'imaginait pas être prince quoique de grande,
ancienne et illustre maison. Il était mort en 1679, et n'avait jamais
fait aucune figure. M\textsuperscript{me} d'Espinoy, fort laide, était
sœur du duc de Rohan-Chabot et de deux beautés, M\textsuperscript{me} de
Soubise de qui j'ai parlé il n'y a pas longtemps, et assez pour n'en
plus rien dire, et M\textsuperscript{me} de Coetquen, célèbre par le
secret du siège de Gand, que M. de Turenne amoureux d'elle ne lui put
cacher, et qui transpira par elle, en sorte que le roi, qui ne l'avait
dit qu'à M. de Louvois et à lui, leur en parla à tous deux, et que M. de
Turenne eut la bonne foi d'avouer sa faute. Entre une déesse et une
nymphe, cette troisième sœur n'était qu'une mortelle qui vivait avec
M\textsuperscript{me} de Soubise dans l'accortise et la subordination de
sa beauté et de sa faveur, et dans l'amertume de lui avoir vu faire
pièce à pièce MM. de Rohan princes, tandis qu'elle ne savait pas même si
elle obtiendrait la continuation du tabouret de grâce pour son fils.
Tous les biens de ses enfants étaient en Flandre, cela l'avait engagée à
y faire de longs séjours. M. Pelletier de Sousy y était intendant\,; lui
et son frère, le contrôleur général, étaient créatures de M. de Louvois,
par conséquent il était le maître en Flandre. Le besoin que
M\textsuperscript{me} d'Espinoy en eut, et les services qu'il lui
rendit, les lièrent d'une amitié si intime qu'elle dura toute leur vie,
et passa réciproquement à leurs enfants, quoiqu'ils eussent fait tout ce
qu'il fallait pour l'éteindre\,; car M. Pelletier ayant perdu sa femme,
M\textsuperscript{me} d'Espinoy l'épousa, et quoique ce mariage n'ait
jamais été déclaré, il ne fut pourtant ignoré de personne.

C'est cette première liaison avec Pelletier qui forma la sienne avec M.
de Louvois qui devint son intime ami. Il la trouva propre au monde et à
la cour. Il lui conseilla de s'y mettre, elle le crut, elle s'y
introduisit par le gros jeu et par Monsieur, et soutenue par Louvois,
elle fut bientôt de tout\,; ce fut par lui qu'elle obtint le tabouret de
grâce pour son fils qui n'était pas encore dans le monde\,; l'autre fils
mourut en y entrant. Le désir de rendre ce tabouret plus solide lui fit
briguer le mariage de M\textsuperscript{lle} de Commercy, dès lors dans
toute la confiance déclarée de Monseigneur, ainsi que
M\textsuperscript{me} et M\textsuperscript{lle} de Lislebonne, sa mère
et sa sœur aînée. Cette raison, et dans une fille de la maison de
Lorraine, fort belle et fort bien faite, la fit passer sur plusieurs
années plus que n'avait son fils, et sur la médiocrité du bien qui était
nul, et qui alors ne paraissait pas pouvoir augmenter. Le mariage se
fit, et la belle-mère et la belle-fille vécurent toujours dans la plus
étroite amitié. Avec ce surcroît de princes vrais et faux dont son fils
était environné de si près, bien leur fâchait de ne l'être pas aussi.
Elle était intrigante, et le fut assez pour introduire ses filles à la
cour, et en même temps faire en sorte qu'elles ne se trouvassent presque
jamais dans les temps où on s'asseyait, quoiqu'il n'y en eût guère
d'autres de faire sa cour à Versailles, où pourtant j'ai été au souper
du roi derrière toutes les deux (mais cela était extrêmement rare) et
bientôt après qu'elles eurent gagné Marly, où le salon et le manger avec
le roi mettait à l'aise sur les tabourets, elles ne s'y trouvèrent plus,
mais avec un entregent, une politesse à tout le monde qu'on voyait toute
tendue à obtenir tolérance et silence, L'aînée paraissait peu, la
cadette était de tout\,; elle se fourra chez M\textsuperscript{me} la
princesse de Conti, encore plus chez M\textsuperscript{me} la Duchesse,
et tant qu'elle pouvait par elle et par le jeu, dans les parties de
Monseigneur. Sa mère qui savait se conduire la tenait souple et mesurée,
et fort en arrière avec tout le monde. Quand elle l'eut perdue elle
hasarda. À une musique où le roi était, à Versailles,
M\textsuperscript{lle} de Melun, qui s'accoutumait à n'être plus si
polie, se trouva la première après la dernière duchesse. Bientôt après
il en arriva une autre qui alla pour se placer, et à qui tout fit place,
en se baissant comme cela se faisait toujours. M\textsuperscript{lle} de
Melun ne branla pas, et ne fit que se lever et se rasseoir. C'était la
première fois que femme ou fille non titrée, même maréchale de France,
n'eût pas donné sa place en ces lieux-là aux duchesses et aux princesses
étrangères ou en ayant rang. Le roi qui le vit rougit, le montra à
Monsieur, et comme il se tournait de l'autre côté où était
M\textsuperscript{lle} de Melun en levant la voix, Monsieur
l'interrompit, et le prenant par le genou, se leva et lui demanda tout
effrayé ce qu'il allait faire. «\,La faire ôter de là,\,» dit le roi en
colère. Monsieur redoubla d'instances pour éviter l'affront, et se donna
pour caution que cela n'arriverait jamais. Le roi eut peine à se
contenir le reste de la musique. Tout ce qui y était voyait bien de quoi
il était question, et la fille entre deux duchesses se pâmait de honte
et de frayeur jusqu'à perdre toute contenance. Au sortir de là Monsieur
lui lava bien la tête et la rendit sage pour l'avenir. C'était l'hiver
devant la mort de Monsieur\,; mais j'ai voulu l'ajouter ici tout de
suite.

J'anticiperai aussi Compiègne pour parler de deux morts arrivées pendant
que le roi y était, de M. de Chaulnes et du duc d'Estrées. Ce dernier
périt avant cinquante ans, de l'opération de la taille. Il avait refusé
l'ambassade de Rome que son père exerçait quand il mourut, et qui y
avait tellement gâté ses affaires, que son fils ne voulut pas continuer
la même ruine, dont le roi fut un peu fâché\,; il laissa un fils fort
mal à son aise, de sa première femme, fille du fameux Lyonne, ministre
et secrétaire d'État, et n'eut point d'enfants de sa seconde femme qui
était Bautru, sœur de l'abbé de Vaubrun. Le cardinal d'Estrées obtint du
roi le gouvernement de l'Île-de-France, etc., pour son petit-neveu, et
de Monsieur, qui s'en fit honneur, la capitainerie de Villers-Cotterêts
que MM. d'Estrées avaient toujours eue par la bienséance de leur petit
gouvernement de Soissons.

M. de Chaulnes mourut enfin de douleur de l'échange forcé de son
gouvernement de Bretagne, où il était adoré, et qui lui donna jusqu'au
bout, et corps et particuliers, les marques les plus continuelles de sa
vénération, de son attachement et de ses regrets. On eut grande peine à
obtenir de lui la démission du gouvernement de Guyenne, dont on lui
avait d'abord expédié les provisions pour l'échange. Cette démission
était nécessaire pour expédier les mêmes provisions au duc de Chevreuse,
et en même temps la survivance au duc de Chaulnes, mais avec le
commandement et les appointements privativement au duc de Chevreuse.
C'est ainsi que depuis que le roi s'était fait une règle de ne plus
accorder de survivances, il les donnait, en effet, mais sous une autre
forme, et comme à l'envers, mais fort rarement. Ce ne fut qu'environ
deux mois avant la mort de M. de Chaulnes qu'il y consentit enfin, mais
sans un vrai retour, ni de lui ni de la duchesse de Chaulnes, pour M. ni
M\textsuperscript{me} de Chevreuse, et sans avoir jamais voulu ouïr
parler de Guyenne, ni de quoi que ce fût qui eût rapport à ce
gouvernement.

J'ai assez parlé de ce seigneur pour n'avoir rien à y ajouter, si ce
n'est que ce fut une grande perte pour ses amis, et il en avait
beaucoup. Il fut regretté de tout le monde, et, en Bretagne, ce fut un
deuil général. Il ne laissa point d'enfants, mais force dettes. Tous
deux étaient fort magnifiques, et ne s'étaient jamais souciés de laisser
grand'chose au duc de Chevreuse, leur héritier substitué, ou plutôt son
second fils par son mariage. Les profits immenses du droit d'amirauté de
Bretagne, attachés au gouvernement de cette province, et qui pendant les
guerres avaient été fort hauts, avaient fait croire qu'il laisserait
beaucoup de richesses. Il se trouva qu'il avait tout dépensé, et qu'il
avait disposé par un testament en legs pieux et de domestiques, et en
quarante mille livres à son ami intime le chancelier, de tout ce qui lui
restait à donner. M. de Chevreuse en eut cent dix mille livres de rente
du gouvernement, et son second fils, beaucoup de meubles précieux et
d'argenterie avec Chaulnes et Picquigny en payant les dettes.

La duchesse du Choiseul, sœur de La Vallière, mourut aussi en même
temps, pulmonique, belle et faite au tour, avec un esprit charmant, et à
la plus belle fleur de son âge, mais d'une conduite si déplorable,
qu'elle en était tombée jusque dans le mépris de ses amants. J'en ai
suffisamment parlé ailleurs. Son mari, amoureux et crédule, jusqu'à en
avoir quitté le bâton de maréchal de France, comme je l'ai raconté,
brouillé et séparé après coup, ne voulut pas même la voir à sa mort.

\hypertarget{chapitre-xii.}{%
\chapter{CHAPITRE XII.}\label{chapitre-xii.}}

1698

~

{\textsc{Camp de Compiègne superbe\,; magnificence inouïe du maréchal de
Boufflers.}} {\textsc{- Dames s'entassent pour Compiègne.}} {\textsc{-
Ducs couplés à Compiègne.}} {\textsc{- Ambassadeurs prétendent le
pour.}} {\textsc{- Distinction du pour.}} {\textsc{- Logements à la
suite du roi.}} {\textsc{- Voyage et camp de Compiègne.}} {\textsc{-
Plaisante malice du duc de Lauzun au comte de Tessé.}} {\textsc{-
Spectacle singulier.}} {\textsc{- Retour de Compiègne.}}

~

Il n'était question que de Compiègne, où soixante mille hommes venaient
former un camp. Il en fut en ce genre comme du mariage de Mgr le duc de
Bourgogne au sien. Le roi témoigna qu'il comptait que les troupes
seraient belles, et que chacun s'y piquerait d'émulation\,; c'en fut
assez pour exciter une telle émulation qu'on eut après tout lieu de s'en
repentir. Non seulement il n'y eut rien de si parfaitement beau que
toutes les troupes, et toutes à tel point, qu'on ne sut à quels corps en
donner le prix, mais leurs commandants ajoutèrent, à la beauté
majestueuse et guerrière des hommes, des armes, des chevaux, les parures
et la magnificence de la cour, et les officiers s'épuisèrent encore par
des uniformes qui auraient pu orner des fêtes.

Les colonels et jusqu'à beaucoup de simples capitaines eurent des tables
abondantes et délicates, six lieutenants généraux et quatorze maréchaux
de camp employés s'y distinguèrent par une grande dépense, mais le
maréchal de Boufflers étonna par sa dépense et par l'ordre surprenant
d'une abondance et d'une recherche de goût, de magnificence et de
politesse, qui dans l'ordinaire de la durée de tout le camp, et à toutes
les heures de la nuit et du jour, put apprendre au roi même ce que
c'était que donner une fête vraiment magnifique et superbe, et à M. le
Prince, dont l'art et le goût y surpassait tout le monde, ce que c'était
que l'élégance, le nouveau et l'exquis. Jamais spectacle si éclatant, si
éblouissant, il le faut dire, si effrayant, et en même temps rien de si
tranquille que lui et toute sa maison dans ce traitement universel, de
si sourd que tous les préparatifs, de si coulant de source que le
prodige de l'exécution, de si simple, de si modeste, de si dégagé de
tout soin, que ce général qui néanmoins avait tout ordonné et ordonnait
sans cesse, tandis qu'il ne paraissait occupé que des soins du
commandement de cette armée. Les tables sans nombre, et toujours neuves,
et à tous les moments servies à mesure qu'il se présentait ou officiers,
ou courtisans, ou spectateurs\,; jusqu'aux bayeurs les plus inconnus,
tout était retenu, invité et comme forcé par l'attention, la civilité et
la promptitude du nombre infini de ses officiers, et pareillement toutes
sortes de liqueurs chaudes et froides, et tout ce qui peut être le plus
vastement et le plus splendidement compris dans le genre de
rafraîchissements\,; les vins français, étrangers, ceux de liqueur les
plus rares, y étaient comme abandonnés à profusion, et les mesures y
étaient si bien prises que l'abondance de gibier et de venaison arrivait
de tous côtés, et que les mers de Normandie, de Hollande, d'Angleterre,
de Bretagne, et jusqu'à la Méditerranée, fournissaient tout ce qu'elles
avaient de plus monstrueux et de plus exquis à jour et point nommés,
avec un ordre inimitable, et un nombre de courriers et de petites
voitures de poste prodigieux. Enfin jusqu'à l'eau, qui fut soupçonnée de
se troubler ou de s'épuiser par le grand nombre de bouches, arrivait de
Sainte-Reine, de la Seine et des sources les plus estimées, et il n'est
possible d'imaginer rien en aucun genre qui ne fût là sous la main, et
pour le dernier survenant de paille comme pour l'homme le plus principal
et le plus attendu. Des maisons de bois meublées comme les maisons de
Paris les plus superbes, et tout en neuf et fait exprès, avec un goût et
une galanterie singulière, et des tentes immenses, magnifiques, et dont
le nombre pouvait seul former un camp. Les cuisines, les divers lieux,
et les divers officiers pour cette suite sans interruption de tables et
pour tous leurs différents services, les sommelleries, les offices, tout
cela formait un spectacle dont l'ordre, le silence, l'exactitude, la
diligence et la parfaite propreté ravissait de surprise et d'admiration.

Ce voyage fut le premier où les dames traitèrent d'ancienne délicatesse
ce qu'on n'eût osé leur proposer\,; il y en eut tant qui s'empressèrent
à être du voyage, que le roi lâcha la main, et permit à celles qui
voudraient de venir à Compiègne. Mais ce n'était pas où elles
tendaient\,: elles voulaient toutes être nommées, et la nécessité, non
la liberté du voyage, et c'est ce qui leur fit sauter le bâton de
s'entasser dans les carrosses des princesses. Jusqu'alors, tous les
voyages que le roi avait faits, il avait nommé des dames pour suivre la
reine ou M\textsuperscript{me} la Dauphine dans les carrosses de ces
premières princesses. Ce qu'on appela les princesses, qui étaient les
bâtardes du roi, avaient leurs amies et leur compagnie pour elles,
qu'elles faisaient agréer au roi, et qui allaient dans leurs carrosses à
chacune, mais qui le trouvaient bon et qui marchaient sur ce pied-là. En
ce voyage-ci tout fut bon pourvu qu'on allât. Il n'y en eut aucune dans
le carrosse du roi que la duchesse du Lude avec les princesses. Monsieur
et Madame demeurèrent à Saint-Cloud et à Paris.

La cour en hommes fut extrêmement nombreuse, et tellement que pour la
première fois, à Compiègne, les ducs furent couplés. J'échus avec le duc
de Rohan dans une belle et grande maison du sieur Chambaudon, où nous
fûmes nous et nos gens fort à notre aise. J'allai avec M. de La
Trémoille et le duc d'Albret, qui me reprochèrent un peu que j'en avais
fait une honnêteté à M, de Bouillon, qui en fut fort touché. Mais je
crus la devoir à ce qu'il était, et plus encore à l'amitié intime qui
était entre lui et M. le maréchal de Lorges, et qui en outre étaient
cousins germains.

Les ambassadeurs furent conviés d'aller à Compiègne. Le vieux Ferreiro,
qui l'était de Savoie, leur mit dans la tête de prétendre \emph{le
pour}. Il assura qu'il l'avait eu autrefois à sa première ambassade en
France. Celui de Portugal allégua que Monsieur, le menant à Montargis,
le lui avait fait donner par ses maréchaux des logis, ce qui, disait-il,
ne s'était fait que sur l'exemple de ceux du roi\,; et le nonce maintint
que le nonce Cavallerini l'avait eu avant d'être cardinal. Pomponne,
Torcy, les introducteurs des ambassadeurs, Cavoye protestèrent tous que
cela ne pouvait être, et que jamais ambassadeur ne l'avait prétendu, et
il n'y en avait pas un mot sur les registres\,; mais on a vu quelle foi
les registres peuvent porter. Le fait était que les ambassadeurs
sentirent l'envie que le roi avait de leur étaler la magnificence de ce
camp, et qu'ils crurent en pouvoir profiter pour obtenir une chose
nouvelle. Le roi tint ferme\,; les allées et venues se poussèrent jusque
dans les commencements du voyage, et ils finirent par n'y point aller.
Le roi en fut si piqué que lui, si modéré et si silencieux, je lui
entendis dire à son souper, à Compiègne, que s'il faisait bien il les
réduirait à ne venir à la cour que par audience, comme il se pratiquait
partout ailleurs.

\emph{Le pour} est une distinction dont j'ignore l'origine, mais qui en
effet n'est qu'une sottise\,: elle consiste à écrire en craie sur les
logis \emph{pour} M. un tel, ou simplement écrire M. un tel. Les
maréchaux des logis qui marquent ainsi tous les logements dans les
voyages mettent ce \emph{pour} aux princes du sang, aux cardinaux et aux
princes étrangers. M. de La Trémoille l'a aussi obtenu, et la duchesse
de Bracciano, depuis princesse des Ursins. Ce qui me fait appeler cette
distinction une sottise, c'est qu'elle n'emporte ni primauté ni
préférence de logement\,: les cardinaux, les princes étrangers et les
ducs sont logés également entre eux sans distinction quelconque qui est
toute renfermée dans ce mot \emph{pour}, et n'opère d'ailleurs quoi que
ce soit. Ainsi ducs, princes, étrangers, cardinaux sont logés sans autre
différence entre eux après les charges du service nécessaire, après eux
les maréchaux de France, ensuite les charges considérables, et puis le
reste des courtisans. Cela est de même dans les places\,; mais quand le
roi est à l'armée, son quartier est partagé, et la cour est d'un côté et
le militaire de l'autre, sans avoir rien de commun\,; et s'il se trouve
à la suite du roi des maréchaux de France sans commandement dans
l'armée, ils ne laissent pas d'être logés du côté militaire et d'y avoir
les premiers logements.

Le jeudi 28 août, la cour partit pour Compiègne, le roi passa à
Saint-Cloud, coucha à Chantilly, y demeura un jour et arriva le samedi à
Compiègne. Le quartier général était au village de Condun, où le
maréchal de Boufflers avait des maisons outre ses tentes. Le roi y mena
Mgr le duc de Bourgogne et M\textsuperscript{me} la duchesse de
Bourgogne, etc., qui y firent une collation magnifique, et qui y virent
les ordonnances, dont j'ai parlé ci-dessus, avec tant de surprise, qu'au
retour de Compiègne, le roi dit à Livry, qui par son ordre avait préparé
des tables au camp pour Mgr le duc de Bourgogne, qu'il ne fallait point
que ce prince en tînt, que, quoi qu'il pût faire, ce ne serait rien en
comparaison de ce qu'il venait de voir, et que, quand son petit-fils
irait à l'avenir au camp, il dînerait chez le maréchal de Boufflers. Le
roi s'amusa fort à voir et à faire voir les troupes aux dames, leur
arrivée, leur campement, leurs distributions, en un mot, tous les
détails d'un camp, des détachements, des marches, des fourrages, des
exercices, de petits combats, des convois. M\textsuperscript{me} la
duchesse de Bourgogne, les princesses, Monseigneur, firent souvent
collation chez le maréchal, où la maréchale de Boufflers leur faisait
les honneurs. Monseigneur y dîna quelquefois, et le roi y mena dîner le
roi d'Angleterre, qui vint passer trois ou quatre jours au camp. Il y
avait longues années que le roi n'avait fait cet honneur à personne, et
la singularité de traiter deux rois ensemble fut grande. Monseigneur et
les trois princesses enfants y dînèrent aussi, et dix ou douze hommes
des principaux de la cour et de l'armée. Le roi pressa fort le maréchal
de se mettre à table, il ne voulut jamais, il servit le roi et le roi
d'Angleterre, et le duc de Grammont, son beau-père, servit Monseigneur.
Ils avaient vu, en y allant, les troupes à pied, à la tête de leurs
camps\,; et en revenant, ils virent faire l'exercice à toute
l'infanterie, les deux lignes face à face l'une de l'autre. La -veille,
le roi avait mené le roi d'Angleterre à la revue de l'armée.
M\textsuperscript{me} la duchesse de Bourgogne la vit dans son carrosse.
Elle y avait M\textsuperscript{me} la Duchesse, M\textsuperscript{me} la
princesse de Conti et toutes les dames titrées. Deux autres de ses
carrosses la suivirent, remplis de toutes les autres dames.

Il arriva sur cette revue une plaisante aventure au comte de Tessé. Il
était colonel général des dragons. M. de Lauzun lui demanda deux jours
auparavant, avec cet air de bonté, de douceur et de simplicité qu'il
prenait presque toujours, s'il avait songé à ce qu'il lui fallait pour
saluer le roi à la tête des dragons, et là-dessus, entrèrent en récit du
cheval, de l'habit et de l'équipage. Après les louanges, «\,mais le
chapeau, lui dit bonnement Lauzun, je ne vous en entends point parler\,?
--- Mais non, répondit l'autre, je compte d'avoir un bonnet. --- Un
bonnet\,! reprit Lauzun, mais y pensez-vous\,! un bonnet\,! cela est bon
pour tous les autres, mais le colonel général avoir un bonnet\,!
monsieur le comte, vous n'y pensez pas. --- Comment donc\,? lui dit
Tessé, qu'aurais-je donc\,?» Lauzun le fit douter, et se fit prier
longtemps, et lui faisant accroire qu'il savait mieux qu'il ne disait\,;
enfin, vaincu par ses prières, il lui dit qu'il ne lui voulait pas
laisser commettre une si lourde faute, que cette charge ayant été créée
pour lui, il en savait bien toutes les distinctions dont une des
principales était, lorsque le roi voyait les dragons, d'avoir un chapeau
gris. Tessé surpris avoue son ignorance, et, dans l'effroi de la sottise
où il serait tombé sans cet avis si à propos, se répand en actions de
grâces, et s'en va vite chez lui dépêcher un de ses gens à Paris pour
lui rapporter un chapeau gris. Le duc de Lauzun avait bien pris garde à
tirer adroitement Tessé à part pour lui donner cette instruction, et
qu'elle ne fût entendue de personne\,; il se doutait bien que Tessé dans
la honte de son ignorance ne s'en vanterait à personne, et lui aussi se
garda bleu d'en parler.

Le matin de la revue, j'allai au lever du roi, et contre sa coutume, j'y
vis M. de Lauzun y demeurer, qui avec ses grandes entrées s'en allait
toujours quand les courtisans entraient. J'y vis aussi Tessé avec un
chapeau gris, une plume noire et une grosse cocarde, qui piaffait et se
pavanait de son chapeau. Cela qui me parut extraordinaire et la couleur
du chapeau que le roi avait en aversion, et dont personne ne portait
plus depuis bien des années, me frappa et me le fit regarder, car il
était presque vis-à-vis de moi, et M. de Lauzun assez près de lui, un
peu en arrière. Le roi, après s'être chaussé et {[}avoir{]} parlé à
quelques-uns, avise enfin ce chapeau. Dans la surprise où il en fut, il
demanda à Tessé où il l'avait pris. L'autre, s'applaudissant, répondit
qu'il lui était arrivé de Paris. «\,Et pourquoi faire\,? dit le roi. ---
Sire, répondit l'autre, c'est que Votre Majesté nous fait l'honneur de
nous voir aujourd'hui. --- Eh bien\,! reprit le roi de plus en plus
surpris, que fait cela pour un chapeau gris\,? --- Sire, dit Tessé que
cette réponse commençait à embarrasser, c'est que le privilège du
colonel général est d'avoir ce jour-là un chapeau gris. --- Un chapeau
gris\,! reprit le roi, où diable avez-vous pris cela\,? --- {[}C'est{]}
M. de Lauzun, sire, pour qui vous avez créé la charge, qui me l'a
dit\,;» et à l'instant, le bon duc à pouffer de rire et s'éclipser.
«\,Lauzun s'est moqué de vous, répondit le roi un peu vivement, et
croyez-moi, envoyez tout à l'heure ce chapeau au général des Prémontrés.
» Jamais je ne vis homme plus confondu que Tessé. Il demeura les yeux
baissés et regardant ce chapeau avec une tristesse et une honte qui
rendit la scène parfaite. Aucun des spectateurs ne se contraignit de
rire, ni des plus familiers avec le roi d'en dire son mot. Enfin Tessé
reprit assez ses sens pour s'en aller, mais toute la cour lui en dit sa
pensée et lui demanda s'il ne connaissait point encore M. de Lauzun, qui
en riait sous cape, quand on lui en parlait. Avec tout cela, Tessé n'osa
s'en fâcher, et la chose, quoique un peu forte, demeura en plaisanterie,
dont Tessé fut longtemps tourmenté et bien honteux.

Presque tous les jours, les enfants de France dînaient chez le maréchal
de Boufflers\,; quelquefois M\textsuperscript{me} la duchesse de
Bourgogne, les princesses et les dames, mais très souvent des
collations. La beauté et la profusion de la vaisselle pour fournir à
tout, et toute marquée aux armes du maréchal, fut immense et
incroyable\,; ce qui ne le fut pas moins, ce fut l'exactitude des heures
et des moments de tout service partout. Rien d'attendu, rien de
languissant, pas plus pour les bayeurs du peuple, et jusqu'à des
laquais, que pour les premiers seigneurs, à toutes heures et à tous
venants. À quatre lieues autour de Compiègne, les villages et les fermes
étaient remplis de monde, et François et étrangers, à ne pouvoir plus
contenir personne, et cependant tout se passa sans désordre. Ce qu'il y
avait de gentilshommes et de valets de chambre chez le maréchal était un
monde, tous plus polis et plus attentifs les uns que les autres à leurs
fonctions de retenir tout ce qui paraissait, et les faire servir depuis
cinq heures du matin jusqu'à dix et onze heures du soir, sans cesse et à
mesure, et à faire les honneurs, et une livrée prodigieuse avec grand
nombre de pages. J'y reviens malgré moi, parce que quiconque l'a vu ne
le peut oublier ni cesser d'en être dans l'admiration et l'étonnement et
de l'abondance et de la somptuosité, et de l'ordre qui ne se démentit
jamais d'un seul moment ni d'un seul point.

Le roi voulut montrer des images de tout ce qui se fait à la guerre\,;
on fit donc le siège de Compiègne dans les formes, mais fort abrégées\,:
lignes, tranchées, batteries, sapes, etc. Crenan défendait la place. Un
ancien rempart tournait du côté de la campagne autour du château\,; il
était de plain-pied à l'appartement du roi, et par conséquent élevé, et
dominait toute la campagne. Il y avait au pied une vieille muraille et
un moulin à vent, un peu au delà de l'appartement du roi, sur le rempart
qui n'avait ni banquette ni mur d'appui. Le samedi 13 septembre fut
destiné à l'assaut\,; le roi, suivi de toutes les dames, et par le plus
beau temps du monde, alla sur ce rempart, force courtisans, et tout ce
qu'il y avait d'étrangers considérables. De là, on découvrait toute la
plaine et la disposition de toutes les troupes. J'étais dans le
demi-cercle, fort près du roi, à trois pas au plus, et personne devant
moi. C'était le plus beau coup d'œill qu'on pût imaginer que toute cette
armée, et ce nombre prodigieux de curieux de toutes conditions, à cheval
et à pied, à distance des troupes pour ne les point embarrasser, et ce
jeu des attaquants et des défendants à découvert, parce que, n'y ayant
rien de sérieux que la montre, il n'y avait de précautions à prendre
pour les uns et les autres que la justesse des mouvements. Mais un
spectacle d'une autre sorte, et que je peindrais dans quarante ans comme
aujourd'hui, tant il me frappa, fut celui que, du haut de ce rempart, le
roi donna à toute son armée, et à cette innombrable foule d'assistants
de tous états, tant dans la plaine que dessus le rempart même.

M\textsuperscript{me} de Maintenon y était en face de la plaine et des
troupes, dans sa chaise à porteurs, entre ses trois glaces, et ses
porteurs retirés. Sur le bâton de devant, à gauche, était assise
M\textsuperscript{me} la duchesse de Bourgogne, du même côté, en arrière
et en demi-cercle, debout, M\textsuperscript{me} la Duchesse,
M\textsuperscript{me} la princesse de Conti, et toutes les dames, et
derrière elles des hommes. À la glace droite de la chaise, le roi,
debout, et un peu en arrière un demi-cercle de ce qu'il y avait en
hommes de plus distingué. Le roi était presque toujours découvert, et à
tous moments se baissait dans la glace pour parler à
M\textsuperscript{me} de Maintenon, pour lui expliquer tout ce qu'elle
voyait et les raisons de chaque chose. À chaque fois, elle avait
l'honnêteté d'ouvrir sa glace de quatre ou cinq doigts, jamais de la
moitié, car j'y pris garde, et j'avoue que je fus plus attentif à ce
spectacle qu'à celui des troupes. Quelquefois elle ouvrait pour quelques
questions au roi, mais presque toujours c'était lui qui, sans attendre
qu'elle lui parlât, se baissait tout à fait pour l'instruire, et
quelquefois qu'elle n'y prenait pas garde, il frappait contre la glace
pour la faire ouvrir. Jamais il ne parla qu'à elle, hors pour donner des
ordres en peu de mots et rarement, et quelques réponses à
M\textsuperscript{me} la duchesse de Bourgogne qui tâchait de se faire
parler, et à qui M\textsuperscript{me} de Maintenon montrait et parlait
par signes de temps en temps, sans ouvrir la glace de devant, à travers
laquelle la jeune princesse lui criait quelques mots. J'examinais fort
les contenances\,: toutes marquaient une surprise honteuse, timide,
dérobée\,; et tout ce qui était derrière la chaise et les demi-cercles
avait plus les yeux sur elle que sur l'armée, et tout, dans un respect
de crainte et d'embarras. Le roi mit souvent son chapeau sur le haut de
la chaise, pour parler dedans, et cet exercice si continuel lui devait
fort lasser les reins. Monseigneur était à cheval dans la plaine, avec
les princes ses cadets\,; et Mgr le duc de Bourgogne, comme à tous les
autres mouvements de l'armée, avec le maréchal de Boufflers, en
fonctions de général. C'était sur les cinq heures de l'après-dînée, par
le plus beau temps du monde, et le plus à souhait.

Il y avait, vis-à-vis la chaise à porteurs, un sentier taillé en marches
roides, qu'on ne voyait point d'en haut, et une ouverture au bout, qu'on
avait faite dans cette vieille muraille pour pouvoir aller prendre les
ordres du roi d'en bas, s'il en était besoin. Le cas arriva\,: Crenan
envoya Canillac, colonel de Rouergue, qui était un des régiments qui
défendaient, pour prendre l'ordre du roi sur je ne sais quoi. Canillac
se met à monter, et dépasse jusqu'un peu plus que les épaules. Je le
vois d'ici aussi distinctement qu'alors. À mesure que la tête dépassait,
il avisait cette chaise, le roi et toute cette assistance qu'il n'avait
point vue ni imaginée, parce que son poste était en bas, au pied du
rempart, d'où on ne pouvait découvrir ce qui était dessus. Ce spectacle
le frappa d'un tel étonnement qu'il demeura court à regarder la bouche
ouverte, les yeux fixes et le visage sur lequel le plus grand étonnement
était peint. Il n'y eut personne qui ne le remarquât, et le roi le vit
si bien, qu'il lui dit avec émotion\,: «\,Eh bien\,! Canillac, montez
donc. » Canillac demeurait, le roi reprit\,: «\,Montez donc\,; qu'est-ce
qu'il y a\,?» Il acheva donc de monter\,; et vint au roi, à pas lents,
tremblants et passant les yeux à droite et à gauche, avec un air éperdu.
Je l'ai déjà dit\,: j'étais à trois pas du roi, Canillac passa devant
moi, et balbutia fort bas quelque chose. «\,Comment dites-vous\,? dit le
roi\,; mais parlez donc.\,» Jamais il ne put se remettre\,; il tira de
soi ce qu'il put. Le roi, qui n'y comprit pas grand'chose, vit bien
qu'il n'en tirerait rien de mieux, répondit aussi ce qu'il put, et
ajouta d'un air chagrin\,: «\,Allez, monsieur.\,» Canillac ne se le fit
pas dire deux fois, et regagna son escalier et disparut. À peine
était-il dedans, que le roi, regardant autour de lui\,: «\,Je ne sais
pas ce qu'a Canillac, dit-il\,; mais il a perdu la tramontane, et n'a
plus su ce qu'il me voulait dire.\,» Personne ne répondit.

Vers le moment de la capitulation, M\textsuperscript{me} de Maintenon
apparemment demanda permission de s'en aller, le roi cria\,: «\,Les
porteurs de madame\,!» Ils vinrent et l'emportèrent\,; moins d'un quart
d'heure après, le roi se retira, suivi de M\textsuperscript{me} la
duchesse de Bourgogne et de presque tout ce qui était là. Plusieurs se
parlèrent des yeux et du coude en se retirant, et puis à l'oreille bien
bas. On ne pouvait revenir de ce qu'on venait de voir. Ce fut le même
effet parmi tout ce qui était dans la plaine. Jusqu'aux soldats
demandaient ce que c'était que cette chaise à porteurs, et le roi à tout
moment baissé dedans\,; il fallut doucement faire taire les officiers et
les questions des troupes. On peut juger de ce qu'en dirent les
étrangers, et de l'effet que fit sur eux un tel spectacle. Il fit du
bruit par toute l'Europe, et y fut aussi répandu que le camp même de
Compiègne avec toute sa pompe et sa prodigieuse splendeur. Du reste
M\textsuperscript{me} de Maintenon se produisit fort peu au camp, et
toujours dans son carrosse avec trois ou quatre familières, et alla voir
une fois ou deux le maréchal de Boufflers et les merveilles du prodige
de sa magnificence.

Le dernier grand acte de cette scène fut l'image d'une bataille entre la
première et la seconde ligne entières, l'une contre l'autre. M. Rosen,
le premier des lieutenants généraux du camp, la commanda ce jour-là
contre le maréchal de Boufflers, auprès duquel était Mgr le duc de
Bourgogne comme le général. Le roi, M\textsuperscript{me} la duchesse de
Bourgogne, les princes, les dames, toute la cour et un monde de curieux
assistèrent à ce spectacle, le roi et tous les hommes à cheval, les
dames en carrosse. L'exécution en fut parfaite en toutes ses parties et
dura longtemps. Mais quand ce fut à la seconde ligne à ployer et à faire
retraite, Rosen ne s'y pouvait résoudre, et c'est ce qui allongea fort
l'action. M. de Boufflers lui manda plusieurs fois de la part de Mgr le
duc de Bourgogne qu'il était temps. Rosen en entrait en colère et
n'obéissait point. Le roi en rit fort qui avait tout réglé, et qui
voyait aller et venir les aides de camp et la longueur de tout ce
manège, et dit\,: «\,Rosen n'aime point à faire le personnage de
battu.\,» À la fin il lui manda lui-même de finir et de se retirer.
Rosen obéit, mais fort mal volontiers, et brusqua un peu le porteur
d'ordre. Ce fut la conversation du retour et de tout le soir.

Enfin après des attaques de retranchements et toutes sortes d'images de
ce qui se fait à la guerre et des revues infinies, le roi partit de
Compiègne le lundi 22 septembre, et s'en alla avec sa même carrossée à
Chantilly, y demeura le mardi, et arriva le mercredi à Versailles, avec
autant de joie de toutes les dames qu'elles avaient eu d'empressement à
être du voyage. Elles ne mangèrent point avec le roi à Compiègne, et y
virent M\textsuperscript{me} la duchesse de Bourgogne aussi peu qu'à
Versailles. Il fallait aller au camp tous les jours et la fatigue leur
parut plus grande que le plaisir, et encore plus que la distinction
qu'elles s'en étaient proposée. Le roi extrêmement content de la beauté
des troupes, qui toutes avaient habillé, et avec tous les ornements que
leurs chefs avaient pu imaginer, fit donner en partant six cents livres
de gratification à chaque capitaine de cavalerie et de dragons, et trois
cents livres à chaque capitaine d'infanterie. Il en fit donner autant
aux majors de tous les régiments, et distribua quelques grâces dans sa
maison. Il fit au maréchal de Boufflers un présent de cent mille
francs\footnote{Le manuscrit porte ici \emph{francs} et non
  \emph{livres}. Saint-Simon se sert des deux mots indifféremment.}.
Tout cela ensemble coûta beaucoup\,; mais pour chacun ce fut une goutte
d'eau. Il n'y eut point de régiment qui n'en fût ruiné pour bien des
années, corps et officiers, et pour le maréchal de Boufflers, je laisse
à penser ce que ce fut que cent mille livres à la magnificence
incroyable, à qui l'a vue, dont il épouvanta toute l'Europe par les
relations des étrangers qui en furent témoins, et qui tous les jours
n'en pouvaient croire leurs yeux.

\hypertarget{chapitre-xiii.}{%
\chapter{CHAPITRE XIII.}\label{chapitre-xiii.}}

1698

~

{\textsc{La belle-fille de Pontchartrain et son intime liaison avec
M\textsuperscript{me} de Saint-Simon.}} {\textsc{- Amitié intime entre
Pontchartrain et moi.}} {\textsc{- Amitié intime entre l'évêque de
Chartres et moi.}} {\textsc{- Le Charmel\,; ma liaison avec lui.}}
{\textsc{- Méprise de M. de la Trappe au choix d'un abbé, et son insigne
vertu.}} {\textsc{- Changement d'abbé à la Trappe.}}

~

L'intervalle est si court entre le retour du roi le 24 septembre de
Compiègne, et son départ le 2 octobre pour Fontainebleau, que je
placerai ici une chose qui fut entamée avant le premier de ces deux
voyages, et qui ne fut consommée qu'au retour du second. Elle semblera
peu intéressante parmi tout ce qui l'a précédée et la suivra, mais j'y
pris trop de part pour l'omettre, et je ne la puis bien expliquer sans
rappeler ma situation avec quelques personnes. La première me fait trop
d'honneur pour n'être pas embarrassé à la rapporter\,; mais, outre que
la vérité doit l'emporter sur toute autre considération, c'est qu'elle a
influé depuis sur tant de choses importantes qu'il n'est pas possible de
l'omettre.

On a vu en son temps le mariage du fils unique de M. de Pontchartrain
avec une sœur du comte de Roucy, cousine germaine de
M\textsuperscript{me} de Saint-Simon. Ils ne l'avaient désirée que pour
l'alliance, et par la façon dont ils en usèrent pour tous ses proches,
toutefois en trayant, ils firent tout ce qu'il fallait pour en profiter.
Il n'y en eut point qu'ils recherchassent autant que
M\textsuperscript{me} de Saint-Simon, et qu'ils désirassent tant lier
avec leur belle-fille. Elle se trouva très heureusement née, avec
beaucoup de vertu, de douceur et d'esprit, toute Roucy qu'elle était,
beaucoup de sens et de crainte de se méprendre et de mal faire, ce qui
lui donnait une timidité bienséante à son âge. Avec cela, pour peu
qu'elle fût en quelque liberté, toutes les grâces, tout le sel, et tout
ce qui peut rendre une femme aimable et charmante, et avec le temps une
conduite, une connaissance des gens et des choses, un discernement fort
au-dessus d'une personne nourrie dans une abbaye à Soissons, et tombée
dans une maison où dans les commencements elle fut gardée à vue, ce
qu'elle eut le bon esprit d'aimer, et de s'attacher de cœur à tout ce à
quoi elle le devait être. La sympathie de vertus, de goûts, d'esprits,
forma bientôt entre elle et M\textsuperscript{me} de Saint-Simon une
amitié qui devint enfin la plus intime, et la confiance la plus sans
réserve qui pût être entre deux sœurs. M. et M\textsuperscript{me} de
Pontchartrain en étaient ravis. Je ne sais si cette raison détermina M.
de Pontchartrain\,; mais sur la fin de l'hiver de cette année, l'étant
allé voir dans son cabinet, comme depuis ce mariage j'y allais
quelquefois mais pas fort souvent à ces heures-là de solitude, après un
entretien fort court et fort ordinaire, il me dit qu'il avait une grâce
à me demander, mais qui lui tenait au cœur de façon à n'en vouloir pas
être refusé. Je répondis comme je devais à un ministre alors dans le
premier crédit et dans les premières places de son état. Il redoubla,
avec cette vivacité et cette grâce pleine d'esprit et de feu qu'il
mettait à tout quand il voulait, que tout ce que je lui répondais était
des compliments, que ce n'était point cela qu'il lui fallait, c'était
parler franchement, et nettement lui accorder ce qu'il désirait
passionnément et qu'il me demandait instamment\,; et tout de suite il
ajouta\,: «\,l'honneur de votre amitié, et que j'y puisse compter comme
je vous prie de compter sur la mienne, car vous êtes très vrai, et si
vous me l'accordez, je sais que j'en puis être assuré.\,» Ma surprise
fut extrême à mon âge, et je me rabattis sur l'honneur et la
disproportion d'âge et d'emplois. Il m'interrompit, et me serrant de
plus en plus près, il me dit que je voyais avec quelle franchise il me
parlait, que c'était tout de bon et de tout son cœur qu'il désirait et
me demandait mon amitié, et qu'il me demandait réponse précise. Je
supprime les choses honnêtes dont cela fut accompagné. Je sentis en
effet qu'il me parlait fort sérieusement, et que c'était un engagement
que nous allions prendre ensemble\,; je pris mon parti, et après un mot
de reconnaissance, d'honneur, de désir, je lui dis que pour lui répondre
nettement il fallait lui avouer que j'avais une amitié qui passerait
toujours devant toute autre, que c'était celle qui me liait intimement à
M. de Beauvilliers, dont je savais qu'il n'était pas ami\,; mais que
s'il voulait encore de mon amitié à cette condition, je serais ravi de
la lui donner, et comblé d'avoir la sienne. Dans l'instant il
m'embrassa, me dit que c'était là parler de bonne foi, qu'il m'en
estimait davantage, qu'il n'en désirait que plus ardemment mon amitié,
et nous nous la promîmes l'un à l'autre. Nous nous sommes réciproquement
tenu parole plénièrement. Elle a réciproquement duré jusqu'à sa mort
dans la plus grande intimité et dans la confiance la plus entière. Au
sortir de chez lui, ému encore d'une chose qui m'avait autant surpris,
j'allai le dire à M. de Beauvilliers qui m'embrassa tendrement, et qui
m'assura qu'il n'était pas surpris du désir de M. de Pontchartrain, et
beaucoup moins de ma conduite sur lui-même. Le rare est que
Pontchartrain n'en dit rien à son fils ni à sa belle-fille, ni moi non
plus, et personne à la cour ne se douta d'une chose si singulière qu'à
la longue, c'est-à-dire de l'amitié intime entre deux hommes si inégaux
en tout.

J'avais encore un autre ami fort singulier à mon âge. C'était l'évêque
de Chartres. Il était mon diocésain à la Ferté. Cela avait fait qu'il
était venu chez moi, d'abord avec un vieil ami de mon père qui
s'appelait l'abbé Bailly. Peu à peu l'amitié se mit entre nous, et la
confiance. Dans la situation où il était avec M\textsuperscript{me} de
Maintenon, jamais je ne l'employai à rien qu'une seule fois, et bien
légère, qui se trouvera en son temps. Je le voyais souvent chez lui et
chez moi à Paris, et j'étais avec lui à portée de tout.

Un autre encore avec qui je liai amitié fut du Charmel que j'avais vu
plusieurs fois à la Trappe. C'était un gentilhomme tout simple de
Champagne, qui s'était introduit à la cour par le jeu, qui y gagna
beaucoup et longtemps, sans jamais avoir été soupçonné le plus
légèrement du monde. Il prêtait volontiers, mais avec choix, et il se
fit beaucoup d'amis considérables. M. de Créqui le prit tout à fait sous
sa protection. Il lui fit acheter du maréchal d'Humières une des deux
compagnies des cent gentilshommes de la maison du roi au bec de corbin.
Cela n'avait plus que le nom. M. de Créqui, fort bien avec le roi alors,
et avec un air d'autorité à la cour, était premier gentilhomme de la
chambre\,; lui fit avoir des entrées sous ce prétexte de sa charge\,; le
roi le traitait bien et lui parlait souvent\,; il était de tous ses
voyages, et au milieu de la meilleure compagnie de la cour. Tout lui
riait\,: l'âge, la santé, le bien, la fortune, la cour\,; les amis, même
les dames, et des plus importantes, qui l'avaient trouvé à leur gré.
Dieu le toucha par la lecture d'Abbadie\,: \emph{De la vérité de la
religion, chrétienne\,;} il ne balança ni ne disputa, et se retira dans
une maison joignant l'institution de l'Oratoire. Le roi eut peine à le
laisser aller. «\,Quoi, lui dit-il, Charmel, vous ne me verrez jamais\,?
--- Non, sire, répondit-il, je n'y pourrais résister, je retournerais en
arrière, il faut faire le sacrifice entier et s'enfuir.\,» Il passait sa
vie dans toutes sortes de bonnes œuvres, dans une pénitence dure jusqu'à
l'indiscrétion, et allait le carnaval tous les ans à la Trappe\,; il y
demeurait jusqu'à Pâques, où, excepté le travail des mains, il menait en
tout la même vie que les religieux.

C'était un homme d'une grande dureté pour soi, d'un esprit au-dessous du
médiocre, qui s'entêtait aisément et qui ne revenait pas de même, de
beaucoup de zèle qui n'était pas toujours réglé, mais d'une grande
fidélité à sa pénitence, à ses œuvres, et qui se jetait la tête la
première dans tout ce qu'il croyait de meilleur. Avant sa retraite fort
honnête homme et fort sûr, très capable d'amitié, doux et bon homme. On
le connaîtra encore mieux en ajoutant qu'il avait une sœur mariée en
Lorraine à un Beauvau, avec qui il était fort uni, et que son neveu,
fils de ce mariage, épousa une nièce de Couronges, que nous allons voir
venir conclure le mariage de M. de Lorraine avec la dernière fille de
Monsieur. {[}C'est{]} cette nièce, qui, sous le nom de
M\textsuperscript{me} de Craon que portait son mari, fut dame d'honneur
de M\textsuperscript{me} la duchesse de Lorraine, et fit, par le crédit
qu'elle prit auprès de M. de Lorraine, une si riche maison et son mari
grand d'Espagne, puis prince de l'empire, qui a eu depuis
l'administration de la Toscane et la Toison de l'empereur, que j'ai fort
connu par rapport à son oncle et qui est demeuré depuis toujours de mes
amis.

Tout cela dit, venons à ce qui m'a engagé à l'écrire. On a vu en son
temps que M. de la Trappe avait obtenu du roi un abbé régulier de sa
maison et de son choix, auquel il s'était démis pour ne plus penser qu'à
son propre salut après avoir si longtemps contribué à celui de tant
d'autres. On a vu aussi que cet abbé mourut fort promptement après, et
que le roi agréa celui qui lui fut proposé par M. de la Trappe pour en
remplir la place. Mais pour saints, pour éclairés et pour sages que
soient les hommes, ils ne sont pas infaillibles. Un carme déchaussé
s'était jeté à la Trappe depuis peu d'années. Il avait de l'esprit, de
la science, de l'éloquence. Il avait prêché avec réputation. Il savait
fort le monde, et il paraissait exceller en régularité dans tous les
pénibles exercices de la vie de la Trappe. Il s'appelait D.
François-Gervaise, et il avait un frère trésorier de Saint-Martin de
Tours, qui était homme de mérite, et qui se consacra depuis aux
missions, et fut tué en Afrique évêque \emph{in partibus}. Ce carme
était connu de M. de Meaux, dans le diocèse duquel il avait prêché. M.
de la Trappe, son ami, le consulta\,; M. de Meaux l'assura qu'il ne
pouvait faire un meilleur choix.

C'était un homme de quarante ans et d'une santé à faire espérer une
longue vie et un long exemple\,; ses talents, sa piété, sa modestie, son
amour de la pénitence séduisirent M. de la Trappe, et le témoignage de
M. de Meaux acheva de le déterminer. Ce fut donc lui qui, à la prière de
M. de la Trappe, fut nommé par le roi pour succéder à celui qu'il venait
de perdre. Ce nouvel abbé ne tarda pas à se faire mieux connaître après
qu'il eut eu ses bulles\,; il se crut un personnage, chercha à se faire
un nom, à paraître et à n'être pas inférieur au grand homme à qui il
devait sa place et à qui il succédait. Au lieu de le consulter il en
devint jaloux, chercha à lui ôter la confiance des religieux, et n'en
pouvant venir à bout, à l'en tenir séparé. Il fit l'abbé avec lui plus
qu'avec nul autre\,; il le tint dans la dépendance, et peu à peu se mit
à le traiter avec une hauteur et une dureté extraordinaires, et à
maltraiter ouvertement ceux de la maison qu'il lui crut les plus
attachés. Il changea autant qu'il le put tout ce que M. de la Trappe
avait établi, et sans réflexion que les choses ne subsistent que par le
même esprit qui les a établies, surtout celles de ce genre si
particulier et si sublime. Il allait à la sape avec application, et il
suffisait qu'une chose eût été introduite par M. de la Trappe pour y en
substituer une tout opposée. Prélat plus que religieux, ne se prêtant
qu'à ce qui pouvait paraître\,; et devant les amis de M. de la Trappe
(quand ils étaient gens à être ménagés), dans les adorations pour lui,
dont tout aussitôt après il savait se dédommager par les procédés avec
lui les plus étranges.

Outre ce qu'il en coûtait au cœur et à l'esprit de M. de la Trappe,
cette conduite n'allait pas à moins qu'à un prompt renversement de toute
régularité, et à la chute d'un si saint et si merveilleux édifice. M. de
la Trappe le voyait et le sentait mieux que personne et par sa lumière
et par son expérience, lui qui l'avait construit et soutenu de fond en
comble. Il en répandait une abondance de larmes devant son crucifix. Il
savait que d'un mot il renverserait cet insensé, il était peiné pour sa
maison de ne le pas faire, et déchiré de la voir périr\,; mais il était
lui-même si indignement traité tous les jours et à tous les moments de
sa vie, que la crainte extrême de trouver, même involontairement,
quelque satisfaction personnelle à se défaire de cet ennemi et de ce
persécuteur le retenait tellement là-dessus, qu'à moi-même il me
dissimulait ses peines et me persuadait tant qu'il pouvait que cet abbé
faisait très bien en tout, et qu'il en était parfaitement content. Il ne
mentait pas assurément, il se plaisait trop dans cette nouvelle épreuve,
qui se peut dire la plus forte de toutes celles par lesquelles il a été
épuré, et il ne craignait rien tant que de sortir de cette fournaise. Il
excusait donc tout ce qu'il ne pouvait nier, et avalait à longs traits
l'amertume de ce calice. Si M. Maisne et un ou deux anciens religieux le
pressaient sur la ruine de sa maison, à qui il ne pouvait dissimuler ce
qu'ils voyaient et sentaient eux-mêmes, il répondait que c'était l'œuvre
de Dieu, non des hommes, et qu'il avait ses desseins et qu'il fallait le
laisser faire.

M. Maisne était un séculier qui avait beaucoup de lettres, infiniment
d'esprit, de douceur, de candeur, et de l'esprit le plus gai et le plus
aimable, qui depuis plus de trente ans vivait là comme un religieux, et
qui avait écrit, sous M. de la Trappe, la plupart de ses lettres et de
ses ouvrages qu'il lui dictait. Je savais donc par lui et par ces autres
religieux tous les détails de ce qui se passait dans cet intérieur. J'en
savais encore par M. de Saint-Louis\,: c'était un gentilhomme qui avait
passé une grande partie de sa vie à la guerre, jusqu'à être brigadier de
cavalerie, avec un beau et bon régiment. Il était fort connu et fort
estimé du roi, sous qui il avait servi plusieurs campagnes avec beaucoup
de distinction. Les généraux en faisaient tous beaucoup de cas, et M. de
Turenne l'aimait plus qu'aucun autre. La trêve de vingt ans lui fit peur
en 1684\,; il n'était pas loin de la Trappe\,; il y avait vu M. de la
Trappe au commencement qu'il s'y retira\,; il vint s'y retirer auprès de
lui dans la maison qu'il avait bâtie au dehors pour les abbés
commendataires, afin qu'ils ne troublassent point la régularité du
dedans\,; et il y a vécu dans une éminente piété. C'était un de ces
pieux militaires, pleins d'honneur et de courage et de droiture, qui la
mettent à tout sans s'en écarter jamais, avec une fidélité jamais
démentie, et à qui le cœur et le bon sens servent d'esprit et de
lumière, avec plus de succès que l'esprit et la lumière n'en donnent à
beaucoup de gens.

Le temps s'écoulait de la sorte sans qu'il fût possible de persuader M.
de la Trappe contre l'amour de ses propres souffrances, ni d'espérer
rien que de pis en pis de celui qui était en sa place. Enfin il arriva
ce qu'on n'aurait jamais pu imaginer. D. Gervaise tomba dans la punition
de ces philosophes superbes dont parle l'Écriture\,; par une autre
merveille, ses précautions furent mal prises, et par une autre plus
grande encore, le pur hasard, ou pour mieux dire la Providence, le fit
prendre sur le fait. On alla avertir M. de la Trappe, et, pour qu'il ne
pût pas en douter, celui dont il s'agissait lui fut mené. M. de la
Trappe épouvanté, tant qu'on peut l'être, fut tout aussitôt occupé de ce
que pourrait être devenu D. Gervaise. Il le fit chercher partout, et il
fut longtemps dans la crainte qu'il ne se fût allé jeter dans les étangs
dont la Trappe est environnée. À la fin on le trouva caché sur les
voûtes de l'église, prosterné et baigné de larmes. Il se laissa amener
devant M. de la Trappe, à qui il avoua ce qu'il ne pouvait lui cacher.
M. de la Trappe, qui vit sa douleur et sa honte, ne songea qu'à le
consoler avec une charité infinie, en lui laissant pourtant sentir
combien il avait besoin de pénitence et de séparation. Gervaise entendit
à demi-mot, et dans l'état où il se trouvait, il offrit sa démission.
Elle fut acceptée. On manda un notaire à Mortagne, qui vint le
lendemain, et l'affaire fut consommée. M. du Charmel, qui était fort
bien avec M. de Paris, reçut par un exprès cette démission, avec une
lettre de D. Gervaise à ce prélat, qu'il priait de présenter sa
démission au roi.

Il était arrivé deux choses depuis fort peu qui causèrent un étrange
contretemps\,: l'une, que la conduite de D. Gervaise à l'égard de M. de
la Trappe et de sa maison, qui commençait à percer, lui avait attiré une
lettre forte du P. de La Chaise de la part du roi\,; l'autre, qu'il
avait étourdiment accepté le prieuré de l'Estrée auprès de Dreux, pour y
mettre des religieux de la Trappe sans la participation du roi, ce qui
d'ailleurs ne pouvait qu'être nuisible par beaucoup de raisons\,; mais
la vanité veut toujours s'étendre et faire parler de soi. Le roi l'avait
trouvé très mauvais, et lui avait fait mander par le P. de La Chaise de
retirer ses religieux, qui y avait ajouté la mercuriale que ce trait
méritait. À la première, il répondit par une lettre, qu'il tira de
l'amour de M. de la Trappe pour la continuation de ses souffrances,
telle que D. Gervaise la voulut dicter\,; à la seconde, par une
soumission prompte et par beaucoup de pardons. Ce fut donc en cadence de
ces deux lettres, et fort promptement après, qu'arriva la démission que
le roi remit au P. de La Chaise. Lui qui était bon homme ne douta point
qu'elle ne fût le fruit des deux lettres que coup sur coup il lui avait
écrites, tellement que, séduit par la lettre dictée par D. Gervaise
qu'il avait reçue de M. de la Trappe, il persuada aisément au roi de ne
recevoir point la démission, et il le manda à D. Gervaise.

Pendant tout cela nous allâmes à Compiègne. Je crus à propos de suivre
la démission de près. J'allai au P. de La Chaise, qui me conta ce que je
viens d'écrire. Je lui dis que pensant bien faire il avait très mal
fait, et j'entrai avec lui fort au long en matière. Le P. de La Chaise
demeura fort surpris et encore plus indigné de la conduite de D.
Gervaise à l'égard de M. de la Trappe, et tout de suite il me proposa
d'écrire à M. de la Trappe pour savoir au vrai son sentiment à l'égard
de la démission. Il m'envoya la lettre pour la faire remettre sûrement,
dans un lieu où D. Gervaise les ouvrait toutes. Je l'envoyai donc à mon
concierge de la Ferté pour la porter lui-même à M. de Saint-Louis, qui
la remit en main propre, et ce fut ainsi qu'il en fallut user tant que
cette affaire dura. La lettre du P. de La Chaise était telle, que M. de
la Trappe ne put éluder. Il lui manda qu'il croyait que D. Gervaise
devait quitter, et que pour obéir à l'autre partie de sa lettre, qui
était de proposer un sujet au cas qu'il fût d'avis de changer d'abbé, il
lui en nommait un. C'était un ancien et excellent religieux qu'on
appelait D. Malachie, et fort éprouvé dans les emplois de la maison. Je
portai cette réponse au P. de La Chaise à notre retour à Versailles. Il
la reçut très bien. Il m'apprit qu'il lui était venu une requête signée
de tous les religieux de la Trappe qui demandaient D. Gervaise, et il
m'assura en même temps qu'il n'y aurait nul égard, parce qu'il savait
bien qu'il n'y avait point de religieux qui osât refuser sa signature à
ces sortes de pièces. Là-dessus nous voilà allés à Fontainebleau.

D. Gervaise avait mis un prieur à la Trappe de meilleures mœurs que lui,
mais d'ailleurs de sa même humeur, et tout à lui. Ce prieur était à
l'Estrée à retirer les religieux de la Trappe lors de l'aventure de la
démission. Il comprit que celle de l'abbé serait la sienne, et il se
trouvait bien d'être prieur sous lui. Il lui remit donc le courage.
C'est ce qui produisit la requête et toute l'adresse qui suivit. Un soir
à Fontainebleau, que nous attendions le coucher du roi, M. de Troyes
m'apprit avec grande surprise que D. Gervaise y était\,; qu'il avait vu
le matin même le P. de La Chaise, et dit la messe à la chapelle, et que
ce voyage lui paraissait fort extraordinaire et fort suspect. En effet,
il avait su tirer de M. de la Trappe un certificat tel qu'il l'avait
voulu, et accompagné d'un religieux qui lui servait de secrétaire, était
venu le présenter au P. de La Chaise, et plaider lui-même contre sa
démission, repartit aussitôt après, et changea le P. de La Chaise du
blanc au noir. Je ne trouvai plus le même homme\,: plus de franchise,
plus de liberté à parler, en garde sur tout. Je ne pouvais en deviner la
cause. Enfin, j'appris par une lettre de du Charmel, et lui par la
vanterie de D. Gervaise, qu'il avait persuadé, que l'esprit de M. de la
Trappe était tout à fait affaibli\,: qu'on en abusait d'autant plus
hardiment, qu'ayant la main droite tout ulcérée, il ne pouvait écrire ni
signer, qu'il avait auprès de lui un séculier, son secrétaire,
extrêmement janséniste, qui, de concert avec le Charmel, voulait faire
de la Trappe un petit Port-Royal\,; et que pour y parvenir il fallait le
chasser, parce qu'il était entièrement opposé à ce parti\,; et que de là
venaient toutes les intrigues de sa démission. Quelque grossier que fût
un tel panneau, qui ne pouvait couvrir une démission signée et envoyée
par lui-même, le P. de La Chaise y donna en plein, et devint tellement
contraire, qu'il fut impossible de le ramener, ni même de se servir
utilement de M. de Paris qu'il avait rendu suspect au roi dans cette
affaire. Mais la Providence y sut encore pourvoir\,: il s'était passé
depuis dix-huit mois quelque chose d'intime et d'entièrement secret
entre M. de la Trappe et moi, et cette chose était telle, que j'étais
certain de faire tomber tout l'artifice et la calomnie de D. Gervaise,
en la disant à M. de Chartres.

Je passai le reste du voyage de Fontainebleau dans l'angoisse de laisser
périr la Trappe et consumer M. de la Trappe dans cette fournaise ardente
où D. Gervaise le tenait, ou de manquer au secret. Je ne pouvais m'en
consulter à qui que ce fût, et je souffris infiniment avant que de
pouvoir me déterminer. Enfin, la pensée me vint que ce secret n'était
peut-être que pour le salut de la Trappe, et je pris mon parti. J'étais
sûr de celui de M. de Chartres, et le roi était en ce genre l'homme de
son royaume le plus fidèle. M\textsuperscript{me} de Maintenon et M. de
Cambrai ne laissaient pas M. de Chartres longtemps de suite à
Chartres\,; il vint à Saint-Cyr au retour de la cour à Versailles. À
Saint-Cyr, personne ne le voyait\,; je lui envoyai demander à
l'entretenir, il me donna le lendemain. Je lui racontai toute l'histoire
de la Trappe, mais sans parler du motif véritable qui avait fait donner
la démission, qu'en cette extrémité même nous n'avions pas voulu dire au
P. de La Chaise\,; ensuite je lui dis le secret. Il m'embrassa à
plusieurs reprises, il écrivit sur-le-champ à M\textsuperscript{me} de
Maintenon, et dès qu'il eut sa réponse, une heure après, il s'en alla
chez elle trouver le roi à qui il parla\,: c'était un jeudi. Le fruit de
cette conversation fut que le lendemain, qui était le jour d'audience du
P. de La Chaise, où je savais qu'il s'était proposé de se faire ordonner
de renvoyer la démission, il eut là-dessus une dispute si forte avec le
roi, qu'on entendit leur voix de la pièce voisine. Le résultat fut que
le P. de La Chaise eut ordre d'écrire à M. de la Trappe, comme il avait
déjà fait avant la course de D. Gervaise à Fontainebleau, que le roi
voulait savoir son véritable sentiment par lui-même, si la démission
devait avoir lieu ou être renvoyée, et au premier cas, de proposer un
sujet pour être abbé\,; et, pour être certain de l'état et de l'avis de
M. de la Trappe, le valet de chambre du P. de La Chaise en fut le
porteur.

Un donné\footnote{On appelait donnés des séculiers qui se consacraient
  au service d'un monastère. Ils portaient quelquefois le nom d'oblats.}
de la Trappe, d'un esprit fort supérieur à son état, qu'on appelait
frère Chanvier, conduisit ce valet de chambre. Ils arrivèrent exprès
fort tard, pour trouver tout fermé. Ils couchèrent chez M. de
Saint-Louis, et le lendemain, à quatre heures du matin, le valet de
chambre fut introduit avec sa lettre. Il demeura quelque temps auprès de
M. de la Trappe à l'entretenir, pour s'assurer par lui-même de l'état de
son esprit\,; il le trouva dans tout son entier, et il n'est pas étrange
que ce domestique en sortit charmé. Une heure après, il fut rappelé, et
comme M. de la Trappe était instruit des soupçons qui avaient surpris le
P. de La Chaise, et que ce domestique était un homme de sa confiance, il
lui lut lui-même sa réponse, et la fit après cacheter en sa présence
tout de suite, et la lui donna, tellement que ce valet de chambre partit
sans que personne à la Trappe se fût douté qu'il y fût venu. La réponse
était la même que la précédente\,: M. de la Trappe était d'avis que la
démission subsistât, et que le même D. Malachie fût nommé abbé en sa
place. Il n'en fallut pas davantage, et D. Gervaise demeura exclu. Mais
il avait si bien su rendre suspect ce D. Malachie, que le P. de La
Chaise, quoique revenu de très bonne foi de son erreur, ne voulut
jamais, sous prétexte qu'il était Savoyard, et qu'il ne convenait pas à
l'honneur de la France qu'un étranger fût abbé de la Trappe. M. de la
Trappe eut donc ordre de proposer trois sujets. Au lieu de trois il en
mit quatre, et toujours ce D. Malachie le premier. On choisit
{[}celui{]} qui se trouva le premier après lui sur la liste. C'était un
D. Jacques La Court, qui avait été longtemps maître des novices, et en
d'autres emplois dans la maison. On tint cette nomination secrète,
jusqu'à ce que ce même donné de la Trappe dont j'ai parlé eût fait
expédier les bulles. Il fut à Rome avec une lettre de crédit la plus
indéfinie pour tous les lieux où il avait à passer, que lui donna M. de
Pontchartrain en son nom. Il aimait fort la Trappe, et particulièrement
ce frère, à qui il trouvait beaucoup de sens et d'esprit. Le cardinal de
Bouillon, qui se piquait d'amitié pour M. de la Trappe, logea ce frère,
le mena au pape, qui l'entretint plusieurs fois, et qui le renvoya avec
les bulles, entièrement gratis, et la lettre du monde la plus pleine
d'estime et d'amitié pour M. de la Trappe, en considération duquel il
s'expliqua qu'il accordait le gratis encore plus qu'en celle du roi. Au
retour le grand-duc voulut voir ce frère, et le renvoya avec des lettres
et des présents pour M. de la Trappe, de sa fonderie, qui étaient des
remèdes précieux.

Dirais-je un prodige qui ne peut que confondre\,? Tandis qu'on attendait
les bulles, D. Gervaise demeura abbé en plein et incertain de son sort.
Ce même donné, avant de partir pour Rome, trouva par hasard un homme
chargé d'un paquet et d'une boîte à une adresse singulière et venant de
la Trappe. Il crut que rencontrant ce donné à l'abbaye il saurait mieux
trouver celui à qui cela s'adressait, et le frère Chanvier s'en chargea
fort volontiers et l'apporta chez M. du Charmel. La boîte était pleine
de misères en petits présents\,; la lettre, nous l'ouvrîmes, et je puis
dire que c'est la seule que j'aie jamais ouverte. Comme cet imprudent
avait dit au frère Chanvier que l'une et l'autre étaient de D. Gervaise,
nous avions espéré de trouver là toutes ses intrigues qui duraient
encore pour le maintenir, et nous fûmes fort attrapés à la boîte. La
lettre nous consola\,; elle était toute en chiffres, et de près de
quatre grandes pages toutes remplies. Nous ne doutâmes pas alors de
trouver là tout ce que nous cherchions. Je portai la lettre à M. de
Pontchartrain qui les fit déchiffrer. Le lendemain quand je retournai
chez lui, il se mit à rire\,: «\,Vous avez, me dit-il, trouvé la pie au
nid\,; tenez, vous en allez voir des plus belles\,;» puis il ajouta d'un
air sérieux\,: «\,En vérité, au lieu de rire, il faudrait pleurer de
voir de quoi les hommes sont capables, et dans de si saintes
professions\,!»

Cette lettre entière, qui était de D. Gervaise à une religieuse avec qui
il avait été en commerce, et qu'il aimait toujours et dont aussi il
était toujours passionnément aimé, était un tissu de tout ce qui se peut
imaginer d'ordures, et les plus grossières, par leur nom, avec de basses
mignardises de moine raffolé, et débordé à faire trembler les plus
abandonnés. Leurs plaisirs, leurs regrets, leurs désirs, leurs
espérances, tout y était au naturel et au plus effréné. Je ne crois pas
qu'il se dise tant d'abominations en plusieurs jours dans les plus
mauvais lieux. Cela et l'aventure qui causa la démission auraient suffi,
ensemble et séparément, pour faire jeter ce malheureux Gervaise dans un
cachot pour le reste de ses jours, à qui l'aurait voulu abandonner à la
justice intérieure de son ordre. Nous nous en promîmes tous le secret
les quatre qui le savions, et ceux à qui il fallut le dire\,; mais M. de
Pontchartrain crut comme nous qu'il fallait déposer le chiffre et le
déchiffrement à M. de Paris, pour s'en pouvoir servir si l'aveuglement
de cet abandonné et ses intrigues étaient toute autre ressource. Je
portai donc l'un et l'autre chez M. du Charmel, à qui j'eus la malice de
la faire dicter pour en garder un double pour nous. Ce tut une assez
plaisante chose à voir que son effroi, ses signes de croix, ses
imprécations contre l'auteur à chaque infamie qu'il lisait, et il y en
avait autant que de mots. Il se chargea de déposer les deux pièces à M.
de Paris, et je gardai l'autre copie. Heureusement nous n'en eûmes pas
besoin. Cela nous mit à la piste de plusieurs choses, par lesquelles
nous découvrîmes quelle était la religieuse et d'une maison que
M\textsuperscript{me} de Saint-Simon connaissait entièrement et elle
beaucoup aussi. Cet amour était ancien et heureux. Il fut découvert et
prouvé, et D. Gervaise sur le point d'être juridiquement mis \emph{in
pace} par les carmes déchaussés, comme il sortait de prêcher dans le
diocèse de Meaux, et en même temps la religieuse tomba malade à la mort,
et ne voulut jamais ouïr parler des sacrements qu'elle n'eût vu D.
Gervaise. Elle ne les reçut ni ne le vit, et ne mourut point. Dans ce
péril, il se vit perdu sans ressource, et n'en trouva que de se jeter à
la Trappe. À ce prix, ses moines délivrés de lui étouffèrent l'affaire,
et en venant à la Trappe y prendre l'habit il passa chez la religieuse,
entra dans la maison et la transporta de joie. Depuis qu'il fut abbé il
continua son commerce de lettres, ne pouvant mieux, et ce fut une de
celles-là que nous attrapâmes\,; il en fut fort en peine n'ayant point
de nouvelles de son paquet\,; il fit du bruit, il menaça. Pour le faire
taire on lui en apprit le sort tout entier. Cela le contint si bien
qu'il n'osa plus en parler, ni guère plus continuer ses intrigues\,;
mais de honte ni d'embarras il en montra peu, mais beaucoup de chagrin.

Les bulles arrivées, j'allai à la Trappe et je ne demandai point à le
voir. Cela le fâcha, il en fit ses plaintes à M. de la Trappe qui, par
bonté pour un homme qui en méritait si peu, exigea que je le visse. Je
pris un temps qui ne pouvait être que court. En vérité, j'étais plus
honteux et plus embarrassé que lui, qui pourtant savait que j'étais
pleinement instruit de ces deux abominations, et qui n'ignorait pas la
part que j'avais eue au maintien de sa démission. Il ne laissa pas
d'être empêtré, et toujours hypocrite, fort affecté\,; il soutint
presque toujours seul la conversation, me voulut persuader de sa joie
d'être déchargé du fardeau d'abbé, et m'assura qu'il s'allait occuper
dans sa solitude à travailler sur l'Écriture sainte. Avec ces beaux
propos, ce n'était pas plus son compte que celui de la Trappe d'y
demeurer. Il en sortit bientôt après. Il porta la combustion cinq ou six
ans durant dans toutes les maisons où on le mit successivement, et enfin
les supérieurs trouvèrent plus court de le laisser dans un bénéfice de
son frère vivre comme il lui plairait. Il ne cessa de vouloir retourner
à la Trappe, essayer d'y troubler et d'y redevenir abbé, ce qui
m'engagea à la fin à obtenir une lettre de cachet qui lui défendit d'en
approcher plus près de trente lieues, et de Paris, plus de vingt.

Si ce scandale dans un homme de cette profession est extrême, le saint
et prodigieux usage que M. de la Trappe fit de tout ce qu'il en
souffrit, est encore plus surprenant, et qu'à la Trappe la surface même
n'en fut pas agitée et pendant un si long temps. Tout, hors quatre ou
cinq personnes, y fut dans l'entière ignorance, et y est demeuré depuis,
et la paix n'y fut non plus altérée que le silence et toute la
régularité. Ce contraste si effrayant et si complet m'a paru quelque
chose de si rare, que j'ai succombé à l'écrire. Après tant de solitude,
rentrons maintenant dans le monde.

\hypertarget{chapitre-xiv.}{%
\chapter{CHAPITRE XIV.}\label{chapitre-xiv.}}

1698

~

{\textsc{Dot de Mademoiselle pour épouser le duc de Lorraine.}}
{\textsc{- Voyage de Fontainebleau.}} {\textsc{- Douleur et deuil du roi
d'un enfant de M. du Maine, qui cause un dégoût aux princesses.}}
{\textsc{- Tentatives de préséance de M. de Lorraine sur M. le duc de
Chartres.}} {\textsc{- Mariage de Mademoiselle.}} {\textsc{- Division de
préséance entre les Lorraines.}} {\textsc{- Départ de la duchesse de
Lorraine et son voyage.}} {\textsc{- Tracasseries de rang à Bar.}}
{\textsc{- Couronne bizarrement fermée et altesse royale usurpée par le
duc de Lorraine.}} {\textsc{- Venise obtient du roi le traitement entier
de tête couronnée pour ses ambassadeurs.}} {\textsc{- Grande opération
au maréchal de Villeroy.}} {\textsc{- Mort de Boisselot.}} {\textsc{-
Mort de la comtesse d'Auvergne.}} {\textsc{- Mort de l'abbé d'Effiat.}}
{\textsc{- Mort de la duchesse Lanti.}} {\textsc{- Mort de la
chancelière Le Tellier.}} {\textsc{- Mort de l'abbé Arnauld.}}
{\textsc{- Le roi refuse de porter le deuil d'un fils du prince royal de
Danemark.}} {\textsc{- Baron de Breteuil est fait introducteur des
ambassadeurs\,; sa rare ignorance et du marquis de Gesvres.}} {\textsc{-
Abbé Fleury\,; ses commencements\,; ses premiers progrès\,; comment fait
évêque de Fréjus, prince de Conti gagne définitivement son procès contre
M\textsuperscript{me} de Nemours.}} {\textsc{- M\textsuperscript{me} de
Blansac rappelée.}} {\textsc{- Éclat et séparation de Barbezieux et de
sa femme.}}

~

Aussitôt après la paix et la restitution convenue de M. de Lorraine dans
ses États, son mariage fut résolu avec Mademoiselle. Sa dot fut réglée à
neuf cent mille livres, du roi comptant en six mois\,; et quatre cent
mille livres moitié de Monsieur, moitié de Madame, payables après leur
mort\,; et trois cent mille livres de pierreries, moyennant quoi pleine
renonciation à tout, de quelque côté que ce fût, en faveur de M. le duc
de Chartres et de ses enfants mâles. Couronges vint tout régler pour M.
de Lorraine, puis fit la demande au roi, ensuite à Monsieur et à Madame,
et dans la suite présenta à Mademoiselle, de la part de son maître, pour
quatre cent mille livres de pierreries. Je ne sais si elle avait su
qu'elle aurait épousé le fils aîné de l'empereur sans l'impératrice, qui
avait un grand crédit sur son esprit, qui haïssait extrêmement la
France, et qui déclara qu'elle ne souffrirait point que son fils, déjà
couronné et de plus destiné à l'empire, devint beau-frère d'une double
bâtarde. Elle ne fut pas si difficile sur le second degré\,; car ce même
prince, en épousant la princesse d'Hanovre, devint cousin germain de
M\textsuperscript{me} la Duchesse. Quoi qu'il en soit, Mademoiselle,
accoutumée aux Lorrains par Monsieur et même par Madame, car il faut du
singulier partout, fut fort aise de ce mariage, et très peu sensible à
sa disproportion de ses sœurs du premier lit. Ce n'est pas que, mettant
l'Espagne à part, je prétende que M. de Savoie soit de meilleure maison
que M. de Lorraine\,; mais un État à part, indépendant, sans sujétion,
séparé par les Alpes, et toujours en état d'être puissamment soutenu par
des voisins contigus, avec le traitement par toute l'Europe de tête
couronnée, est bien différent d'un pays isolé, enclavé, et toutes les
fois que la France le veut envahi sans autre peine que d'y porter des
troupes, un pays ouvert, sans places, sans liberté d'en avoir, sujet à
tous les passages des troupes françaises, un pays croisé par des grands
chemins marqués, dont la souveraineté est cédée, un pays enfin qui ne
peut subsister que sous le bon plaisir de la France, et même des
officiers de guerre ou de plume qu'elle commet dans ses provinces qui
l'environnent. Mademoiselle n'alla point jusque-là\,: elle fut ravie de
se voir délivrée de la dure férule de Madame, mariée à un prince dont
toute sa vie elle avait ouï vanter la maison, et établie à soixante-dix
lieues de Paris, au milieu de la domination française. Les derniers
jours avant son départ, elle pleura de la séparation de tout ce qu'elle
connaissait\,; mais on sut après qu'elle s'était parfaitement consolée
dès la première couchée, et que du reste du voyage il ne fut plus
question de tristesse.

La cour partit pour Fontainebleau, et, six jours après, le roi et la
reine d'Angleterre y arrivèrent, et on ne songea plus qu'au mariage de
Mademoiselle. Quatre jours avant le départ pour Fontainebleau, M. du
Maine avait perdu son fils unique. Le roi l'était allé voir à Clagny, où
il se retira d'abord, et y pleura fort avec lui. Monseigneur et
Monsieur, l'un et l'autre fort peu touchés, y trouvèrent le roi, et
attendirent longtemps pour voir M. du Maine que le roi sortit d'avec
lui. Quoique fort au-dessous de sept ans, le roi voulut qu'on en prit le
deuil\,; Monsieur désira qu'on le quittât pour le mariage, et le roi y
consentit. M\textsuperscript{me} la Duchesse et M\textsuperscript{me} la
princesse de Conti crurent apparemment au-dessous d'elles de rendre ce
respect à Monsieur, et prétendirent hautement ne le point faire.
Monsieur se fâcha\,; le roi leur dit de le quitter\,; elles poussèrent
l'affaire jusqu'à dire qu'elles n'avaient point apporté d'autres habits.
Le roi se fâcha aussi, et leur ordonna d'en envoyer chercher
sur-le-champ. Il fallut obéir et se montrer vaincues, ce ne fut pas sans
un grand dépit.

M. d'Elbœuf avait tant fait qu'il s'était raccommodé avec M. de
Lorraine. Il était après lui et MM. ses frères l'aîné de la maison de
Lorraine, et comme tel il fut chargé de la procuration pour épouser
Mademoiselle. Cette cérémonie enfanta un étrange prodige qui fut d'abord
su de peu de personnes, mais qui perça à la fin. Il entra dans la tête
des Lorrains de rendre équivoque la supériorité de rang de M. le duc de
Chartres sur M. le duc de Lorraine, et ces obliquités leur ont si
souvent réussi, et frayé le chemin aux plus étranges entreprises, qu'il
leur est tourné en maxime de les hasarder toujours. L'occasion était
faite exprès pour leur donner beau jeu\,: il ne s'agissait que d'exclure
M. et M\textsuperscript{me} de Chartres de la cérémonie. Mademoiselle,
fille ou mariée, conservait son même rang de petite-fille de France, et
sans aucune difficulté précédait, après son mariage comme devant, les
filles de Gaston de même rang qu'elle, et les princesses du sang toutes
d'un rang inférieur au sien. Le chevalier de Lorraine, accoutumé à
dominer Monsieur, osa le lui proposer, et Monsieur, le plus glorieux
prince du monde, et qui savait le mieux et avec le plus de jalousie tout
ce qui concernait les rangs et les cérémonies, partialité à part pour
les Lorrains, Monsieur y consentit. Il en parla à M. son fils, qui lui
témoigna sa surprise, et qui fort respectueusement lui déclara qu'il ne
s'abstiendrait point de la cérémonie et qu'il y garderait son rang
au-dessus de M\textsuperscript{me} sa sœur. Monsieur, qui eut peur du
roi si l'affaire se tournait en aigreur, fila doux et tâcha d'obtenir de
l'amitié et de la complaisance ce qu'il n'osait imposer par voie
d'autorité. Tout fut inutile, encore que Madame favorisât la proposition
de Monsieur, parce qu'elle était en faveur d'un prince qu'elle regardait
comme Allemand, et ils se tournèrent sourdement à la ruse. Pendant
toutes ces menées domestiques, M. de Couronges se désolait de la fermeté
qu'il rencontrait sur beaucoup de points qui tenaient M. de Lorraine
fort en brassière dans son État, principalement celui de l'exacte
démolition des fondements mêmes des fortifications de Nancy. Dans le
désespoir de rien obtenir par lui-même, il s'adressa à Mademoiselle, qui
lui promit qu'elle y ferait de son mieux. Elle tint parole, mais elle ne
fut pas plus écoutée que l'avait été Couronges. Elle en conçut un tel
dépit contre le roi, qu'avec la même légèreté qui lui avait fait
embrasser cette affaire, elle s'emporta avec Couronges jusqu'à le prier
de se hâter de la tirer d'une cour où on ne se souciait que des bâtards,
sans réflexion aucune que toutes vérités, quoique exactes, ne sont pas
bonnes à dire. D'autre part il se trouva des gens bons et officieux qui
lui dirent toutes sortes de sottises de M. de Lorraine, et lui en firent
une peur épouvantable qui lui coûta plus de larmes que les regrets de
son départ, mais qui, grâce à sa légèreté, se séchèrent, comme je l'ai
déjà dit, dès la première journée.

Enfin, le dimanche 12 octobre, sur les six heures du soir, les
fiançailles se firent dans le cabinet du roi, en présence de toute la
cour, et du roi et de la reine d'Angleterre, par le cardinal de Coislin,
premier aumônier, le cardinal de Bouillon, grand aumônier, étant à Rome.
M\textsuperscript{me} la grande-duchesse porta la queue de Mademoiselle.
M. d'Elbœuf en pourpoint et en manteau lui donnait la main, et signa le
dernier de tous le contrat de mariage. Le roi et M\textsuperscript{me}
la duchesse de Bourgogne séparément avaient été voir Mademoiselle avant
les fiançailles, et il y eut beaucoup de larmes répandues. Les rois et
toute la cour entendirent le soir une musique, le souper ne fut qu'à
l'ordinaire de tous les jours. Mademoiselle ne parut plus de tout le
reste du jour après la cérémonie, et le passa à pleurer chez elle, au
grand scandale des Lorrains. Le lendemain sur le midi toute la cour
s'assembla chez la reine d'Angleterre, dans l'appartement de la reine
mère, comme cela se faisait tous les jours, tant qu'elle était à
Fontainebleau tous les voyages. Les princesses n'y osaient manquer,
Monseigneur et toute la famille royale pareillement, et
M\textsuperscript{me} de Maintenon elle-même et tout habillée en grand
habit. On y attendait le roi qui y venait tous les jours prendre la
reine d'Angleterre pour la messe, et qui lui donnait la main tout le
chemin allant et revenant, et faisant toujours passer le roi
d'Angleterre devant lui. Ce ne fut donc ce jour-là que le train de vie
ordinaire, si ce n'est que Mademoiselle y fut amenée par le duc
d'Elbœuf, vêtu comme la veille. Un moment après qu'elle y fut arrivée,
on alla à la chapelle en bas, où M. le duc de Chartres alla et
demeura\,; mais ce fut inutilement pour son rang. Mademoiselle n'y
pouvait être dans le sien. Elle était entre le prie-Dieu du roi et
l'autel, sur un fort gros carreau, à la droite duquel il y en avait un
fort petit pour M. d'Elbœuf, représentant M. de Lorraine. Le cardinal de
Coislin dit la messe et les maria, aussitôt après on se mit en marche,
dans laquelle les princes allaient, comme tous les jours, devant le roi
et les princesses derrière. À la porte de la chapelle, le roi, le roi et
la reine d'Angleterre et les princesses du sang embrassèrent
M\textsuperscript{me} de Lorraine et l'y laissèrent. M. d'Elbœuf la
ramena chez elle se déshabiller, et tout fut fini en ce moment.
M\textsuperscript{me} la duchesse de Chartres demeura à la tribune
quoique tout habillée. C'était elle dont le rang eût été marqué, en
revenant le long de la chapelle, au-dessus de M\textsuperscript{me} de
Lorraine, ce qui fut évité par là. Toute la cour en parla fort haut\,;
mais à ce qu'était M\textsuperscript{me} de Chartres, et à la façon dont
elle avait été mariée, que pouvait-elle faire contre la volonté de
Monsieur et de Madame\,? C'était à M. le duc de Chartres à soutenir cet
assaut et à la faire venir en bas. La fin répondit mal au commencement
que j'ai raconté, et le roi toujours embarrassé, avec Monsieur et
Madame, sur sa fille, n'osa user de son autorité. Mais ce qui fut évité
en public ne le fut pas en particulier. J'appelle ainsi un lieu publie,
mais où la cour n'était pas. M\textsuperscript{me} de Lorraine dîna chez
Monsieur avec Madame, et M. et M\textsuperscript{me} la duchesse de
Chartres, qui tous deux prirent toujours partout le pas et la place à
table sur elle\,; et Monsieur apparemment embarrassé du grand murmure
qui s'était fait de M\textsuperscript{me} de Chartres à la tribune, et
qui avait duré toute la cérémonie, s'expliqua tout haut à son dîner\,;
qu'il ne savait pas ce qu'on avait voulu imaginer, que M. de Lorraine
n'avait jamais prétendu disputer rien à M. de Chartres, et que lui-même
ne l'aurait pas souffert. Après dîner, Monsieur monta dans un carrosse
du roi avec sa fille, M\textsuperscript{me} de Lislebonne et les siens
et M\textsuperscript{me} de Maré gouvernante de Mademoiselle, Madame
dans son carrosse avec ses dames, et M. le duc de Chartres dans le sien
avec des dames de la cour de Monsieur, et s'en allèrent à Paris.
M\textsuperscript{me} la duchesse de Chartres, sous prétexte
d'incommodité, demeura à Fontainebleau.

Cette cérémonie fit un schisme parmi les Lorraines.
M\textsuperscript{me} de Lislebonne prétendit les précéder toutes, comme
fille du duc Charles IV de Lorraine\,; M\textsuperscript{me} d'Elbœuf,
la douairière, et cela soit dit une fois pour toutes, parce que la femme
du duc d'Elbœuf ne paraissait jamais, M\textsuperscript{me} d'Elbœuf,
dis-je, se moqua d'elle\,; et, comme veuve de l'aîné de la maison en
France, et du frère aîné de M. de Lislebonne, se rit de sa belle-sœur et
l'emporta, malgré les pousseries et les colères dont
M\textsuperscript{me} de Lislebonne, quoique fort inutilement, ne se
contraignit pas. Il y avait eu sur cela force pourparlers où la duchesse
du Lude s'était assez mal à propos mêlée, qui n'aboutirent qu'à aigrir
et renouveler les propos de la bâtardise de M\textsuperscript{me} de
Lislebonne, qui se voulait toujours porter pour légitime et qui en fut
mortellement offensée. Je ne sais ce qui arriva à M\textsuperscript{me}
d'Armagnac sur tout cela, mais elle demeura à la tribune avec ses filles
et sa belle-fille.

La ville, mais sans le gouverneur, alla saluer M\textsuperscript{me} de
Lorraine au Palais-Royal. Elle en partit le jeudi 16 octobre, dans un
carrosse du roi, dans lequel montèrent avec elle M\textsuperscript{me}
de Lislebonne, chargée de la conduire, ses deux filles,
M\textsuperscript{me}s de Maré, de Couronges et de Rotzenhausen, une
Allemande favorite de Madame, et mère d'une de ses filles d'honneur.
Desgranges, maître des cérémonies, l'accompagna jusqu'à la frontière, et
elle fut servie par les officiers du roi. À Vitry, où elle coucha, M. de
Lorraine vint, inconnu, voir souper M\textsuperscript{me} la duchesse de
Lorraine\,; puis alla chez M\textsuperscript{me} de Lislebonne qui le
présenta à M\textsuperscript{me} son épouse. Ils furent quelque temps
tous trois ensemble, puis il s'en retourna.

En arrivant à Bar ils furent remariés par des abbés déguisés en évêques,
au refus du diocésain qui voulut un fauteuil chez M. de Lorraine. M. le
Grand, le prince Camille, un de ses fils, le chevalier de Lorraine et M.
de Marsan y étaient déjà. L'évêque d'Osnabrück, frère de M. de Lorraine,
s'y trouva aussi, et mangea seul avec eux. Ce fut une autre
difficulté\,: comme souverain par son évêché, M. de Lorraine voulait
bien lui donner un fauteuil, mais comme à son cadet, il ne lui donnait
pas la main. Comme frère, nos Lorrains lui auraient déféré bien des
choses, mais cette distinction du fauteuil les blessa extrêmement. Cela
fit bien de la tracasserie, et finit enfin par les mettre à l'unisson.
M. d'Osnabrück se contenta d'un siège à dos, et les quatre autres en
eurent de pareils, moyennant quoi, ils mangèrent avec M. et
M\textsuperscript{me} de Lorraine. Ce siège à dos fut étrange devant une
petite-fille de France\,; les princes du sang n'en ont pas d'autres
devant elle\,; mais il passa, et de là vint que les ducs en
prétendirent, lorsqu'ils passèrent depuis par cette petite cour, ce qui
fut rare\,; et que M. de Lorraine en laissa prendre et en prit devant
M\textsuperscript{me} sa femme, d'autant plus volontiers, et manger sa
noblesse avec elle, que cette confusion était l'égalité marquée avec
lui, sans laquelle aucun duc n'eût pu le voir. Je dis égalité, parce
qu'il était raisonnable que ceux de sa maison lui déférassent la main et
ce qu'il voulait, ce qui ne pouvait pas régler les autres. Aucun duc de
Guise, jusqu'au gendre de Gaston inclus, n'a jamais fait difficulté de
toute égalité avec les ducs\,; et en même temps n'a jamais donné la main
chez lui à aucun de la maison de Lorraine. C'est un fait singulier que
je tiens et de ducs et de gens de qualité qui l'ont vu. Ces tracasseries
firent que M. le Grand et les trois autres qui avaient compté
accompagner M. et M\textsuperscript{me} de Lorraine jusqu'à Nancy
prirent congé d'eux à leur départ de Bar, et s'en revinrent.
M\textsuperscript{me} de Lislebonne et ses filles allèrent avec eux, et
y passèrent l'hiver. Le roi ne laissa pas de trouver ce dossier fort
mauvais devant sa nièce, et M. d'Elbœuf, qui alla à Nancy quelque temps
après que M. et M\textsuperscript{me} de Lorraine y furent établis, en
sut bien faire sa cour et dire au roi qu'il se garderait bien, devant
M\textsuperscript{me} de Lorraine, de prendre un autre siège qu'un
ployant, qui est ce que les petites-filles de France donnent ici aux
ducs et aux princes étrangers. M. le Grand en fut fort piqué.

Le jour du mariage, Couronges présenta, de la part de M. de Lorraine,
son portrait enrichi de diamants à Torcy, qui avait dressé le contrat de
mariage. On fut surpris de la couronne qui surmontait ce portrait\,;
elle était ducale, mais fermée par quatre bars, ce qui, aux fleurs de
lis près, ne ressemblait pas mal à celle que le roi avait fait prendre à
Monseigneur. Ce fut une invention toute nouvelle que ses pères n'avaient
pas imaginée, et qu'il mit partout sur ses armes. Il se fit donner en
même temps l'\emph{altesse royale} par ses sujets, que nul autre ne lui
voulut accorder, qui fut une autre nouvelle entreprise, et Meuse qu'il
envoya remercier le roi de sa part, après son mariage, n'osa jamais lui
en donner ici. Je ne sais s'il voulut chercher à s'égaler à M. de
Savoie, et sa chimère de Jérusalem à celle de Chypre, mais M. de Savoie
en avait au moins quelque réalité par le traitement d'ambassadeur de
tête couronnée déféré aux siens à Rome, à Vienne, en France, en Espagne,
et partout où jamais on n'avait ouï parler de simples ambassadeurs de
Lorraine. Cette clôture de couronne, pour être ingénieuse et de forme
agréable pour un orfèvre, était mal imaginée. M. de Lorraine, comme duc
de Lorraine, était un très médiocre souverain, mais souverain pourtant
sans dépendance\,; comme duc de Bar, il l'était aussi, mais mouvant et
dépendant de la couronne, et toutes ses justices à lui (à plus forte
raison celle de tous les Barrois) soumises au parlement de Paris, et ce
fut des armes de Bar qu'il fit la fermeture de sa couronne. Ce ridicule
sauta aux yeux. Ses pères ont eu l'honneur d'être gendres de rois et
d'empereurs\,: un, de roi du Danemark\,; un autre, de notre Henri II\,;
et le père de M. de Lorraine était gendre et beau-frère d'empereurs, et
mari d'une reine douairière de Pologne. C'était, de plus, un des
premiers capitaines de son siècle, un des plus capables du conseil de
l'empereur son beau-frère, et qui avait le plus sa confiance, et
d'autorité et de crédit à sa cour, et dans tout l'empire, duquel, ainsi
que de l'empereur, il était feld-maréchal ou généralissime, avec une
réputation bien acquise en tout genre et singulièrement grande. Jamais
il ne s'était avisé, non plus que ses pères, ni de couronne autre que la
ducale, ni de l'\emph{altesse royale}. Moi et un million d'autres hommes
avons vu sur les portes de Nancy les armes des ducs de Lorraine, en
pierre, avec la couronne purement ducale et le manteau ducal,
apparemment comme ducs de Bar, car en Allemagne, dont la Lorraine tient
fort sans en être, les manteaux de duc ne sont pas usités autour des
armes. Ce duc-ci le quitta aux siennes. Je ne sais ce que sont devenues
ces armes sur les portes de Nancy, où je n'ai pas été depuis ce mariage.
Ces entreprises furent trouvées ridicules, on s'en moqua, mais elles
subsistèrent et tournèrent en droit. C'est ainsi que s'est formé et
accru en France le rang des princes étrangers, par entreprises, par
conjonctures, pièce à pièce, ainsi que je l'ai déjà fait remarquer.
Cette couronne était surmontée d'une couronne d'épines, d'où sortait une
croix de Jérusalem. C'était, pour ne rien oublier, enter le faux sur le
trop faible.

Ce faible, qui était les bars, fut tôt ressenti par ce duc. Sa justice
principale à Bar s'avisa, dans l'ivresse de ses grandeurs nouvellement
imaginées, de nommer le roi dans quelques sentences \emph{le roi très
chrétien}. L'avocat général d'Aguesseau représenta au parlement la
nécessité de réprimer cette audace, ce furent ses propres termes, et
d'apprendre aux Barrois que leur plus grand honneur consistait en leur
mouvance de la couronne. Sur quoi, arrêt du parlement qui enjoint à ce
tribunal de Bar diverses choses, entre autres de ne jamais nommer le roi
que \emph{le roi} seulement, et ce à peine de suspension, interdiction
et même privation d'offices, à quoi il fallut obéir. M. de Lorraine en
fit excuse et cassa celui qui l'avait fait.

Avant de quitter les étrangers il faut dire que la jalousie de Venise
contre Savoie sur le traitement de leurs ambassadeurs, par la prétention
réciproque de la couronne de Chypre, ne cessa de faire instance d'avoir
les mêmes avantages sur le traitement entier de tête couronnée qu'on
venait d'accorder à l'ambassadeur de Savoie depuis le mariage de
M\textsuperscript{me} la duchesse de Bourgogne, et ils l'obtinrent en ce
temps-ci.

Le maréchal de Villeroy, si galant encore à son âge, si paré, d'un si
grand air, si adroit aux exercices et qui se piquait tant d'être bien à
cheval et d'y fatiguer plus que personne, courut si bien le cerf à
Fontainebleau, sans nécessité, qu'il manifesta au monde deux grosses
descentes, une de chaque côté, dont personne ne s'était jamais douté,
tant il les avait soigneusement cachées. Un accident terrible le surprit
à la chasse. On eut peine à le rapporter à bras. Il voulut dérober à la
cour le spectacle de cette sorte de honte pour un homme si bien fait
encore, et si fort homme à bonnes fortunes. Il se fit emporter dès le
lendemain sur un brancard à Villeroy, puis gagner la Seine et à Paris en
bateau. Maréchal, fameux chirurgien, lui fit la double opération avec un
succès qui surprit les connaisseurs en cet art, et le rappela à la vie
qu'il fut sur le point de perdre plus d'une fois. Le roi parut s'y
intéresser beaucoup. Il y gagna la guérison radicale de ses deux
descentes.

Pendant qu'on était à Fontainebleau on apprit la mort de Boisselot, dans
une terre où il s'était retiré lieutenant général. Il avait été
capitaine aux gardes, et s'était acquis une grande réputation en Irlande
par l'admirable et longue défense de Limerick, assiégé par le prince
d'Orange en personne, par laquelle il retarda longtemps la conquête de
toute cette île.

La femme du comte d'Auvergne mourut aussi chez elle à Berg-op-Zoom\,:
elle était fille unique et héritière d un prince de Hohenzollern et de
l'héritière de Berg-op-Zoom. C'était une femme de bonne mine, qui
imposait, d'un esprit doux et poli, au-dessous du médiocre, mais d'une
vertu, d'un mérite et d'une conduite rare dont elle ne se démentit
jamais, et dont elle eut bon besoin toute sa vie.

L'abbé d'Effiat mourut en même temps dans un beau logement à l'Arsenal,
que lui avait donné le maréchal de La Meilleraye, grand maître de
l'artillerie, son beau-frère. Il était fils du maréchal d'Effiat et
d'une Fourcy, frère de Cinq-Mars, grand écuyer de France, décapité à
Lyon avec M. de Thou, 12 septembre 1642, sans avoir été marié, et du
père du marquis d'Effiat, premier écuyer de Monsieur et chevalier de
l'ordre, qui à quelques legs près eut tout ce riche héritage. L'abbé
d'Effiat avait soixante-dix ans, et toute sa vie avait été fort galant
et fort du grand monde. Tout vieux et tout aveugle qu'il était devenu,
il en était encore tant qu'il pouvait, et avait la manie, quoique depuis
plus de vingt ans aveugle, de ne le vouloir pas paraître. Il était
averti, et retenait fort bien les gens et les meubles qui étaient dans
une chambre, les plats qu'on devait servir chez lui et leur arrangement,
et se gouvernait en conséquence comme s'il eût vu clair. On avait pitié
de cette faiblesse et on ne faisait pas semblant de s'en apercevoir. Il
avait de l'esprit, la conversation agréable, savait mille choses et
était un fort bon homme.

La duchesse Lanti mourut aussi à Paris, d'un cancer qu'elle y avait
apporté de Rome, dans l'espérance d'y trouver sa guérison. On a vu
ailleurs qui était son mari, et qu'elle était sœur de la duchesse de
Bracciano qui fit son mariage. Elle n'avait rien, et Lanti se trouva
fort honoré d'épouser une La Trémoille, sœur d'une femme qui à tous
égards tenait le premier rang dans Rome, et qui lui procura l'ordre du
Saint-Esprit. Elle laissa des enfants, et le roi fit donner à sa fille
qu'elle avait amenée de quoi s'en retourner à Rome.

La chancelière Le Tellier mourut enfin à plus de quatre-vingt-dix ans,
ayant conservé sa tête et sa santé jusqu'à la fin, et grande autorité
dans sa famille, à qui elle laissa trois millions de biens.

M. de Pomponne perdit l'abbé Arnauld son frère. C'était un homme fort
retiré et grand homme de bien, qui n'avait jamais fait parler de lui
dans les affaires du fameux Arnauld son oncle. Il vivait dans un
bénéfice qu'il avait.

Le prince royal de Danemark perdit son fils. Cette cour fit tout ce
qu'elle put pour engager la nôtre d'en porter le deuil, mais le roi ne
voulut point avoir cette complaisance. Il ne portait le deuil que des
têtes couronnées ou des princes qui étaient ses parents, et il n'avait
point de parenté avec la maison d'Oldenbourg qui est celle des rois de
Danemark.

Bonnœill, introducteur des ambassadeurs, était mort il y avait cinq ou
six mois. C'était un fort honnête homme, différent de Sainctot à qui son
père, seul introducteur, avait vendu la moitié de sa charge. Le père et
le fils entendaient fort bien leur métier. Breteuil, qui, pour être né à
Montpellier pendant l'intendance de son père, se faisait appeler le
baron de Breteuil, eut cette charge d'introducteur au retour de
Fontainebleau. C'était un homme qui ne manquait pas d'esprit mais qui
avait la rage de la cour, des ministres, des gens en place ou à la mode,
et surtout de gagner de l'argent dans les partis en promettant sa
protection. On le souffrait et on s'en moquait. Il avait été lecteur du
roi, et il était frère de Breteuil, conseiller d'État et intendant des
finances. Il se fourrait fort chez M. de Pontchartrain, où Caumartin,
son ami et son parent, l'avait introduit. Il faisait volontiers le
capable quoique respectueux, et on se plaisait à le tourmenter. Un jour,
à dîner chez M. de Pontchartrain, où il y avait toujours grand monde, il
se mit à parler et à décider fort hasardeusement. M\textsuperscript{me}
de Pontchartrain le disputa, et pour fin lui dit qu'avec tout son savoir
elle pariait qu'il ne savait pas qui avait fait le \emph{Pater}. Voilà
Breteuil à rire et à plaisanter, M\textsuperscript{me} de Pontchartrain
à pousser sa pointe, et toujours à le défier et à le ramener au fait. Il
se défendit toujours comme il put, et gagna ainsi la sortie de table.

Caumartin, qui vit son embarras, le suit en rentrant dans la chambre, et
avec bonté lui souffle «\,Moïse.\,» Le baron, qui ne savait plus où il
en était, se trouva bien fort, et au café remet le \emph{Pater} sur le
tapis, et triomphe. M\textsuperscript{me} de Pontchartrain alors n'eut
plus de peine à le pousser à bout, et Breteuil, après beaucoup de
reproches du doute qu'elle affectait, et de la honte qu'il avait d'être
obligé à dire une chose si triviale, prononça magistralement que c'était
Moïse qui avait fait le \emph{Pater}. L'éclat de rire fut universel. Le
pauvre baron confondu ne trouvait plus la porte pour sortir. Chacun lui
dit son mot sur sa rare suffisance. Il en fut brouillé longtemps avec
Caumartin, et ce \emph{Pater} lui fut longtemps reproché. Son ami le
marquis de Gesvres, qui quelquefois faisait le lecteur et retenait
quelques mots qu'il plaçait comme il pouvait, causant un jour dans les
cabinets du roi, et admirant en connaisseur les excellents tableaux qui
y étaient, entre autres plusieurs crucifiements de Notre Seigneur, de
plusieurs grands martres, trouva que le même en avait fait beaucoup, et
tous ceux qui étaient là. On se moqua de lui, et on lui nomma les
peintres différents qui se reconnaissent à leur manière. «\,Point du
tout, s'écria le marquis, ce peintre s'appelait INRI, voyez-vous pas son
nom sur tous ces tableaux\,?» On peut imaginer ce qui suivit une si
lourde bêtise, et ce que put devenir un si profond ignorant.

On a vu en son temps la disgrâce, puis la mort de Daquin, premier
médecin du roi. Il avait un frère évêque de Fréjus qui était un homme
fort extraordinaire. Il demanda à se défaire de son évêché en faveur de
son neveu. Tout fut bon au roi pourvu qu'il se démît, et l'abbé Daquin
d'ailleurs avait plu au roi dans l'exercice de son agence du clergé.
L'oncle ne fut pas longtemps d'accord avec lui-même, et il vexa
tellement et si mal à propos son neveu qu'il abdiqua Fréjus pour n'avoir
point à lutter contre son oncle. Le roi approuva fort ce procédé, et
trouva celui du vieil évêque extrêmement mauvais. Séez vint à vaquer
tout à propos, et fut donné au neveu, et en même temps l'oncle eut ordre
de désemparer de Fréjus et de laisser les lieux libres. Voilà donc
Fréjus tout à fait vacant.

L'abbé Fleury languissait après un évêché depuis longues années, le roi
s'était buté à ne lui en point donner. Il n'estimait pas sa conduite, et
disait qu'il était trop dissipé, trop dans les bonnes compagnies, et que
trop de gens lui parlaient pour lui. Il l'avait souvent refusé. Le P. de
La Chaise y avait échoué, et le roi s'était expliqué qu'il ne voulait
plus que personne lui en parlât davantage. Il y avait quatre ou cinq ans
qu'après une longue espérance le pauvre abbé était tombé dans cette
espèce d'excommunication, et il la comptait d'autant plus sans ressource
qu'il avait essayé la faveur naissante de M. de Paris qui n'avait pas
mieux réussi que les autres, en sorte que le pauvre garçon ne savait que
devenir. Il était sans bien et presque sans bénéfices, il était trop
petit compagnon pour quitter sa charge par dépit\,; et la garder aussi
sans espérance, c'était le dernier mépris. Son père était receveur des
décimes du diocèse de Lodève. Il s'était fourré parmi les valets du
cardinal Bonzi, dont il avait obtenu la protection du temps de sa faveur
à la cour et qu'il pouvait tout en Languedoc. L'abbé Fleury était fort
beau et fort bien fait dans sa première jeunesse, et en a conservé les
restes toute sa vie. Il plut fort au bon cardinal\,; il voulut en
prendre soin, il le fit chanoine de l'église de Montpellier, où il fut
ordonné prêtre en 1674 après avoir fait à Paris des études telles
quelles dans un grenier de ces petits collèges à bon marché. Le cardinal
Bonzi, qui était grand aumônier de la reine, se fit une affaire de lui
en faire avoir une charge d'aumônier, ce qui parut assez étrange. Sa
figure adoucit les esprits\,; il se trouva discret, doux, liant\,; il se
fit des amies et des amis, et se fourra dans le monde sous la protection
du cardinal Bonzi. La reine mourut et le cardinal obtint pour lui une
charge d'aumônier du roi. On en cria beaucoup, mais on s'accoutume à
tout. Fleury respectueux et d'un esprit et d'une humeur qui avait su
plaire, d'une figure qui plaisait peut-être encore plus, d'une modestie,
d'une circonspection, d'une profession qui rassurait, gagna toujours du
terrain, et il eut la fortune et l'entregent d'être d'abord souffert,
puis admis, dans les meilleures compagnies de la cour, et de se faire
des protecteurs ou des amis illustres des personnages principaux en
hommes et en femmes dans le ministère et dans les premières places ou
dans le premier crédit. Il était reçu chez M. de Seignelay\,; il ne
bougeait de chez M. de Croissy, puis de chez M. de Pomponne et M. de
Torcy, où, à la vérité, il était comme ailleurs sans conséquence, et
suppléait souvent aux sonnettes avant qu'on en eût l'invention. Le
maréchal et la maréchale de Villeroy l'avaient très souvent, les
Noailles extrêmement, et il eut le bon esprit de se lier étroitement
avec ce qu'il y avait de meilleur et de plus distingué parmi les
aumôniers du roi, comme les abbés de Beuvron et de Saint-Luc, et avec
d'autres de son métier qui lui faisaient honneur. Le maréchal de
Bellefonds, le vieux Villars, M\textsuperscript{me} de Saint-Géran, M.
et M\textsuperscript{me} de Castries, il ne sortait point de chez eux,
et passait ainsi une vie très agréable et très honorable pour lui\,;
mais le roi n'avait pas tort de n'y trouver rien d'ecclésiastique, et
quoiqu'il se conduisit fort sagement, il était difficile que tout en fût
ignoré. Il en était donc là et sans moyen quelconque d'avancer ni de
reculer, fort plaint du gros du monde, mais sans secours pour sortir de
ce bourbier, lorsque Fréjus vaqua.

M. de Paris, qui l'en vit touché jusqu'aux larmes, en prit si
généreusement pitié que, malgré les défenses du roi, il se hasarda de
faire encore une tentative. Elle fut mal reçue, et de façon à fermer la
bouche à tout autre\,; mais le prélat fit effort de crédit et de bien
dire pour représenter au roi que c'était déshonorer et désespérer un
homme et sans une cause éclatante à quoi on s'en pût prendre, et insista
si fortement et si longtemps, que le roi d'impatience lui mit la main
sur l'épaule, et le serrant et le remuant\,: «\,Ho bien\,! monsieur, lui
dit-il, vous le voulez donc que je fasse l'abbé Fleury évêque de Fréjus,
et malgré toutes les raisons que je vous ai dites et redites, vous
insistez sur ce que c'est un diocèse au bout du royaume et en pays
perdu\,; il faut donc vous céder pour n'en être plus importuné, mais je
le fais à regret, et souvenez-vous bien, et je vous le prédis, que vous
vous en repentirez.\,» Ce fut de la sorte qu'il eut Fréjus, arraché par
M. de Paris à la sœur de son front et de toute la force de ses bras.
L'abbé Fleury fut comblé de joie et de reconnaissance pour un service si
peu attendu, et qui le tirait de l'état du monde le plus cruel et le
plus violent, auquel il ne voyait point d'issue\,; mais le roi fut
prophète, et bien plus qu'il ne pensait, mais d'une tout autre sorte. Le
nouvel évêque se pressa le moins qu'il put de se confiner à Fréjus. Il
fallut pourtant bien y aller. Ce qu'il y fit pendant quinze ou seize ans
n'est pas de mon sujet\,; ce qu'il a fait depuis, cardinal et plutôt roi
absolu que premier ministre, c'est ce que tous les historiens ne
laisseront pas ignorer à la postérité.

M. le prince de Conti, plus heureux et peut-être plus actif au parlement
qu'en Pologne, gagna enfin définitivement son grand procès contre
M\textsuperscript{me} de Nemours, pour les biens de Longueville, dans le
milieu de décembre, et de vingt-trois juges eut vingt voix. Outre treize
ou quatorze cent mille livres qui lui furent adjugées, ses prétentions
sur Neuchâtel devinrent bien plus considérables.

Le roi dans cette fin d'année résolut d'entreprendre trois grands
ouvrages qui auraient dû même être faits depuis longtemps\,: la chapelle
de Versailles, l'église des Invalides et l'autel de Notre-Dame de Paris.
Ce dernier était un vœu de Louis XIII, fait lorsqu'il n'avait plus le
temps de l'accomplir, et dont il avait chargé son successeur qui avait
été plus de cinquante ans sans y songer.

Il permit aussi à M\textsuperscript{me} de Blansac de reparaître à la
cour, et de voir M\textsuperscript{me} la duchesse de Chartres qui en
eut une grande joie. Celle de la maréchale de Rochefort fut tôt après
troublée par l'apoplexie de son fils dont il eut attaques sur attaques.
C'était fort peu de chose à la valeur près, et un jeune homme
excessivement débauché.

M. de Barbezieux finit l'année par un éclat dont il se serait pu passer.
Il avait, comme on l'a vu, épousé M\textsuperscript{lle} d'Alègre. Il la
traitait comme un enfant, et ne se contraignait pas de ses galanteries
et de sa vie accoutumée. M. d'Elbœuf, comme on l'a vu encore, en fit
l'amoureux à grand bruit pour insulter Barbezieux. La jeune femme,
piquée de la conduite de son mari à son égard, crut de mauvais conseils
et rendit son mari jaloux. Il s'abandonna à cette passion, tout lui
grossit, il crut voir ce qu'il ne voyait point et il lui arriva ce qui
n'est jamais arrivé à personne, de se déclarer publiquement cocu, d'en
vouloir donner les preuves, de ne le pouvoir, et de n'en être cru de qui
que ce soit. On n'a jamais vu homme si enragé que celui-là, de ne
pouvoir passer pour cocu. Tout ce qui se trouva ne fut qu'imprudences,
étourderies et folies d'une jeune innocente sottement conseillée, qui
veut ramener par où on les égare, et ce fut tout. Mais Barbezieux
furieux ne fut plus capable de raison. Il pria d'Alègre par un courrier
qu'il lui dépêcha en Auvergne de revenir sur-le-champ, et la lettre fut
si bien tournée, qu'Alègre, qui n'était pas un habile homme, ne douta
pas que ce ne fût pour quelque grand avancement que son gendre lui
procurait. Il fut donc étrangement surpris en arrivant, quand il apprit
de quoi il s'agissait. Les séparer, il le fallait bien dans la crise où
l'affaire était tombée. M\textsuperscript{me} de Barbezieux était
prisonnière chez son mari et malade. Le mari prétendait qu'elle la
faisait, et voulait la mettre dans un couvent\,; le père et la mère la
voulaient garder chez eux. Enfin, après un grand vacarme, et pour fort
peu de chose, le roi fort importuné du beau-père et du gendre, décida
que M\textsuperscript{me} de Barbezieux irait chez son père et sa mère
jusqu'à entière guérison, après laquelle ils la mèneraient dans un
couvent en Auvergne. Pour le bien, Barbezieux le remit tout entier, et
s'en rapporta à d'Alègre, de ce qu'il conviendrait pour l'éducation et
l'entretien de ses deux filles. On plaignit fort d'Alègre, et sa fille
encore plus, et on tomba rudement sur Barbezieux. Ce qu'il fit encore de
plus mal, ce furent les niches de toutes les sortes qu'il s'appliqua
depuis à faire à d'Alègre, et d'y employer l'autorité et le crédit de sa
charge.

\hypertarget{chapitre-xv.}{%
\chapter{CHAPITRE XV.}\label{chapitre-xv.}}

1699

~

{\textsc{Année 1699. M. le duc de Berry chevalier de l'ordre.}}
{\textsc{- Mort du duc de Brissac.}} {\textsc{- Difficultés à succéder à
la dignité de duc et pair de Brissac.}} {\textsc{- Entreprises
lorraines.}} {\textsc{- Étrange hardiesse de la princesse d'Harcourt, le
jour de la première audience de milord Jersey chez M\textsuperscript{me}
la duchesse de Bourgogne.}} {\textsc{- Noir artifice des Lorrains que je
mis au net avec le roi le soir même.}} {\textsc{- Plainte du duc de
Rohan au roi qui ordonne à la princesse d'Harcourt de demander pardon à
la duchesse de Rohan, et qui l'exécute en public chez
M\textsuperscript{me} de Pontchartrain.}} {\textsc{- Places des
princesses du sang au cercle et lieux arrangés.}}

~

M. le duc de Berry fut nommé chevalier de l'ordre le premier jour de
cette année, et fut reçu à la Chandeleur.

Le duc de Brissac mourut à Brissac le premier ou le second jour de cette
année. Il était frère unique de la maréchale de Villeroy, et mon
beau-frère, sans enfants de ma sœur avec qui il avait très mal vécu,
comme je l'ai dit au commencement de ces Mémoires. Il n'en eut point non
plus de la sœur de Vertamont, premier président du grand conseil, qu'il
épousa pour son grand bien, qu'il mangea si parfaitement que, n'ayant
pas même de douaire ni de reprises pour elle, elle continua à vivre
comme elle faisait depuis longtemps chez son frère, qui lui donnait
jusqu'à des souliers et des chemises. Elle était bossue avec un visage
assez agréable, et beaucoup d'esprit et fort orné qui l'était encore
plus, et beaucoup de douceur et de vertu. M. de Brissac savait beaucoup,
et avait infiniment d'esprit et du plus agréable, avec une figure de
plat apothicaire, grosset, basset, et fort enluminé. C'était de ces
hommes nés pour faire mépriser l'esprit, et pour être le fléau de leur
maison. Une vie obscure, honteuse, de la dernière et de la plus vilaine
débauche, à quoi il se ruina radicalement à n'avoir pas de pain
longtemps avant de mourir, sans table, sans équipage, sans rien jamais
qui eût paru, sans cour, sans guerre, et sans avoir jamais vu homme ni
femme qu'on pût nommer. Cossé était fils du frère cadet de son père,
mort chevalier de l'ordre. Il avait épousé depuis plusieurs années une
fille de Bechameil, qui était surintendant de Monsieur, sœur de la femme
de Desmarets, neveu de M. Colbert, chassé et longtemps exilé à sa mort,
et de Nointel que Monsieur fit faire intendant en Bretagne, puis
conseiller d'État.

J'appris cette mort à Versailles, où j'étais presque toujours. Je
compris aussitôt que Cossé trouverait des difficultés à être duc de
Brissac par le fond de la chose même, et par la sottise de bien des
ducs. Je sentis en même temps combien il importait à la durée des duchés
qu'il le fût, et je me hâtai dès le lendemain matin d'en parler à M. de
La Trémoille, à M. de La Rochefoucauld, et à quelques autres, que je
persuadai si bien qu'ils me promirent d'appuyer Cossé tant qu'ils
pourraient, et de prendre même fait et cause pour lui si cela devenait
nécessaire. Je ne m'étais pas trompé à ne pas perdre de temps. Le soir
même, comme on attendait le coucher du roi, le duc de Rohan parut dans
le salon, devenu depuis la chambre du roi, il vint à moi, et me dit que
plusieurs ducs l'avaient vu à Paris, sur la prétention de Cossé, dont on
ne doutait pas\,; qu'ils étaient fort résolus à s'y opposer, et
l'avaient prié de m'en parler de leur part. Le duc de Grammont s'était
aussi chargé de m'en parler, et à plusieurs autres. Il s'étendit sur
l'avantage de gagner un rang d'ancienneté, et de diminuer le nombre des
ducs. Je lui répondis que j'étais surpris que lui, qui était plus
instruit que ceux dont il me parlait, eût pu se laisser prendre à leurs
raisons\,; qu'il était à la vérité fort à désirer que les rangs
d'ancienneté parmi les ducs ne fussent pas troublés par des chimères et
des prétentions qui n'avaient que du crédit, comme celle de M. de
Luxembourg et plusieurs autres\,; et qu'il plût au roi de ne plus
prodiguer si facilement cette dignité\,; mais que de chercher à
l'éteindre sur un issu de mâle en mâle d'un duc, c'était l'éteindre un
jour sur nous-mêmes, puisqu'il n'y avait aucun de nous à qui cela ne pût
arriver dans sa maison en plusieurs façons\,; que je croyais, au
contraire, qu'il était d'un intérêt très principal de conserver le plus
longuement qu'il était possible les duchés dans les maisons où elles
étaient, et pour l'honneur de la dignité et pour l'intérêt des maisons,
quand c'était une succession de mâle en mâle, et non pas des extensions
chimériques, par des femelles ou par des parentés masculines qui ne
sortaient point de celui en faveur duquel le duché était érigé\,; que le
cas de Cossé était simple, que son père était fils puîné et frère puîné
des ducs de Brissac, et lui cousin germain de celui qui venait de
mourir, par conséquent en tout droit et raison de l'être, et nous en
tout intérêt de l'y aider\,; qu'à l'égard du rang, je ne pouvais
m'empêcher de lui dire que c'était une raison misérable, et que, autant
qu'il était insupportable de céder à des chimères, ou à des entreprises,
ou à des nouveautés, autant était-il agréable de suivre une règle
honorable entre nous de précéder ses cadets, et de n'avoir aucune peine
à avoir des anciens et à leur céder partout. M. de Rohan n'était pas à
un mot, ni aisé à persuader. Après avoir écouté ses répliques, et qu'il
eut vu que je ne nie rendais point, il me dit d'un ton plus haut que lui
et ces messieurs auraient beaucoup de déplaisir si je ne voulais pas
être des leurs\,; mais que leur résolution était prise d'intervenir
contre Cossé, et de demander que le duché-pairie de Brissac fût déclaré
éteint. À ce mot je le pris par le bras, et lui répondis que, si lui et
ces messieurs tenaient bon, nous verrions donc un schisme, parce que
j'avais parole de MM. de La Trémoille, de Chevreuse, de La
Rochefoucauld, de Beauvilliers et de plusieurs autres, de prendre, si le
cas y échéait, fait et cause pour Cossé\,; et qu'on verrait alors qui
aurait plus de raison et meilleure grâce, de ceux qui soutiendraient la
conservation de la dignité au descendant si proche et de mâle en mâle de
celui pour qui elle avait été érigée, ou de ceux qui en voudraient
porter l'éteignoir sur lui, et en donner l'exemple pour leur postérité à
eux-mêmes. M. de Rohan fut bien étonné à ce propos\,; j'en profitai, et
lui proposai d'en parler à ceux que je lui venais de nommer, qui étoient
à Versailles, et qu'il trouverait si aisément sous sa main. Le roi vint
se déshabiller et finit notre conversation. Elle fut efficace.

Je la rendis le lendemain à ceux que j'avais gagnés, qui me promirent de
nouveau de prendre fait et cause. Ils s'en expliquèrent à d'autres
fortement, tellement que les ducs de Rohan, de Grammont, et les autres
qui avaient pris la résolution de s'opposer à Cossé, n'osèrent pousser
leur pointe, ni même en parler davantage. Je pouvais, quoique fort
jeune, avoir quelque poids dans cette affaire, après ce qui s'était
passé en celle de M. de Luxembourg. Le duc de Brissac est plus ancien
que moi, et je n'avais aucune habitude avec Cossé, qui était un bavard
fort borné et fort peu compté, qui avalait du vin avec force mauvaise
compagnie, et n'en voyait pas fort ordinairement de bonne. Son cousin
avait trop étrangement vécu avec ma sœur et avec mon père, pour que je
pusse m'intéresser à sa maison par rapport à elle, et j'étais depuis
plusieurs années en procès avec M. de Brissac et ses créanciers pour la
restitution de la dot de ma sœur. C'étaient là des raisons de meilleur
aloi que celles que le duc de Rohan m'avait alléguées, et qui ne
pouvaient être contre-balancées par la maréchale de Villeroy, dont je
fus depuis ami intime, mais avec qui alors je n'étais guère encore qu'en
connaissance, et en aucune avec son mari. Mais l'intérêt général me
détermina et me toucha assez pour hasarder ma dette. Cossé, qui sut
l'obligation qu'il m'avait, accourut me remercier et m'offrir de me
mettre hors d'intérêt sur ce procès, que j'avais déjà gagné une fois, et
qu'on avait renouvelé par des chicanes. Il m'en pressa même, mais je ne
le voulus pas, parce que tous les créanciers de son cousin lui auraient
pu faire la même loi sur cet exemple, comme beaucoup même firent sans
cela\,; il n'aurait pu y suffire, ni atteindre à la propriété de Brissac
essentielle pour en recueillir la dignité. Je sentais bien ce que je
hasardais avec une succession ruinée, ventilée, en proie aux frais et
aux chicanes, et à Cossé lui-même à qui il resterait peu ou point de
bien, après s'être épuisé pour une acquisition si essentielle, où chaque
intéressé le rançonnerait\,; mais la même considération générale de la
conservation des duchés dans les maisons me fit aussi courir
volontairement le hasard de ce qui pourrait arriver de ce procès.

Cossé avait bien des difficultés à surmonter\,: il fallait être
propriétaire du duché de Brissac par succession, non par acquisition, et
pour cela avoir la renonciation de la maréchale de Villeroy et de ses
enfants, qu'ils donnèrent aussitôt\,; et ce qui fut le plus long et le
plus difficile, s'accommoder avec un monde de créanciers du feu duc de
Brissac, et à leur perte, parce que les biens ne suffisaient pas. Outre
ces embarras domestiques, la chose en soi en portait avec elle. Il
n'était point le vrai héritier, et il ne le devenait que par la
renonciation de la maréchale de Villeroy et de ses enfants. Il était
donc par cette raison très équivoque que le duché ne fût pas éteint,
parce que la règle des grands fiefs est que la mort saisit le vif sans
intervalles, et ce vif n'était point lui, mais la maréchale de Villeroy
et ses enfants après elle, incapable comme femelle de recueillir ni
transmettre une dignité purement masculine, ce qui en opérait
l'extinction\,; par conséquent la renonciation de cette femelle pouvait
très bien n'avoir pas plus d'effet en faveur de Cossé, que la succession
qu'elle abandonnait en avait sur elle\,: c'est-à-dire la tradition de la
terre sans la dignité, puisqu'elle ne pouvait pas donner ou abandonner
autre chose que ce qu'en acceptant la succession elle recevrait, qui
était la terre, non la dignité, dont son sexe la rendait incapable et
conséquemment l'éteignait en sa personne, la succession passant
nécessairement par elle, soit qu'elle l'acceptât ou qu'elle y renonçât.
À ces raisons on pouvait encore ajouter que ces successions de dignité
en collatéral étaient de droit étroit, et qu'il ne pouvait dépendre
d'une volonté de particulière de faire un homme duc ou de l'empêcher de
l'être, ce qui arrivait pourtant en ce cas par l'acceptation ou par la
renonciation de la maréchale de Villeroy. On ne peut nier la force de
ces arguments\,; mais la réponse se trouvait écrite dans les lettres
d'érection de Brissac, qui étaient pour le maréchal de Brissac et pour
tous ses hoirs sortis de son corps, et de degré en degré, en légitime
mariage et successeurs mâles. Ainsi, son second fils, père de Cossé, et
sa postérité masculine, étaient appelés au défaut de la postérité
masculine aînée. Le cas arrivait, et il était clair que l'intention du
roi concesseur était\,: que tout mâle sorti par mâle du maréchal de
Brissac recueillit à son rang d'aînesse la dignité de duc et pair. Il
est vrai que par successeur la nécessité était imposée d'avoir la
terre\,; mais puisqu'on ne pouvait nier la volonté du roi concesseur
être telle qu'elle vient d'être expliquée, la conséquence suit
évidemment en faveur de la renonciation. Mais ce n'était pas là tout\,:
l'érection appelait bien les collatéraux, mais l'enregistrement du
parlement les avait exclus, et c'était au parlement à qui l'on avait
affaire, non pas contentieusement avec des parties, mais pour recevoir
Cossé en qualité de duc de Brissac et de pair de France, après que les
affaires liquidées avec les créanciers l'auraient mis en état de s'y
présenter.

Je n'avais eu garde de laisser sentir au duc de Rohan aucune de ces
difficultés. Celle des créanciers, qui était publique, l'avait occupé
lui et les ducs qui s'étaient voulu opposer, et ils n'avaient envisagé
qu'en gros, et à travers un brouillard, celle de la nécessité de la
renonciation de la maréchale de Villeroy. Je fus le conseil de Cossé,
non sur les discussions des créanciers, mais sur ce qui regardait
intrinsèquement la succession à la dignité. Il venait presque tous les
jours chez moi, ou y envoyait tant que l'affaire dura, qui ne fut pas
sans épines fréquentes et fortes, et qui passa la révolution de cette
année.

À la suite de ce récit de pairie, j'en ferai un autre, à peu près de la
même matière, sur ce qui arriva le 6 janvier chez M\textsuperscript{me}
la duchesse de Bourgogne, à l'audience de M. le comte de Jersey,
ambassadeur d'Angleterre. Je serais trop long, et sortirais du dessein
de ces Mémoires, si j'entreprenais d'expliquer l'origine, les
entreprises et les progrès du rang et des prétentions de la maison de
Lorraine en France, et à son exemple de celui des princes étrangers.
Pour me rabattre au fait dont il s'agit, il suffira de savoir qu'aux
cérémonies de la cour, entrées, mariages des rois, baptêmes, obsèques,
il y a eu souvent des disputes entre les duchesses et les princesses
étrangères pour la préséance, que les rois ont cru de leur intérêt de
laisser subsister sans les décider, pour entretenir une division qu'ils
se sont crue utile, à quoi ce n'est pas ici le lieu de répondre.

Dans l'ordinaire de la vie, comme cercles, audiences, comédies, en un
mot, tous les lieux journaliers et de cour, et de commerce du monde,
jamais il n'y en avait, et entre elles, elles se plaçaient
indifféremment comme elles se rencontraient. La reine avait des dames du
palais, duchesses et princesses\,: M\textsuperscript{me}s de Chevreuse,
de Beauvilliers, de Noailles, et plusieurs autres duchesses\,; la
princesse de Bade, sœur du comte de Soissons, tante paternelle de
l'autre et du prince Eugène devenu depuis si fameux, fille de
M\textsuperscript{me} de Carignan, princesse du sang, vivant et à Paris
et à la cour, mère du prince Louis de Bade qui s'est illustré à la tête
des armées de l'empereur et de l'empire, et dont la fille a épousé M. le
duc d'Orléans petit-fils de Monsieur, longues années depuis\,;
M\textsuperscript{me} d'Armagnac, la même dont il va être ici question,
M\textsuperscript{lle} d'Elbœuf et d'autres encore. Jamais de dispute,
et jamais entre elles elles n'ont pris garde à rien, et cela avait
toujours duré ainsi jusque vers la fin de la vie de
M\textsuperscript{me} la Dauphine-Bavière, que la princesse d'Harcourt
commença la première à devenir hargneuse, et M\textsuperscript{me}
d'Armagnac aussi. La première avait peu à peu gagné toute la protection
de M\textsuperscript{me} de Maintenon, avec qui Brancas son père avait
été longtemps plus que bien, et il fallait à M\textsuperscript{me} de
Maintenon une raison aussi forte pour pouvoir prendre en faveur une
personne qui en était aussi peu digne. Comme toutes celles de peu qui ne
savent rien que ce que le hasard leur a appris, et qui ont longtemps
langui dans l'obscurité avant que d'être parvenues,
M\textsuperscript{me} de Maintenon était éblouie de la principauté, même
fausse, et ne croyait pas que rien le pût disputer à la véritable.

La maison de Lorraine n'ignorait pas cette disposition. M. le Grand
balançait qui que ce fût dans l'esprit du roi\,; et le chevalier de
Lorraine qui avait infiniment d'esprit, et tout celui des Guise, avait
Monsieur en croupe, à qui le roi, qui ignorait beaucoup de choses, se
rapportait fort ordinairement sur tout ce qui fait partie du cérémonial.
Ce fut donc par l'avis du chevalier de Lorraine, que sa belle-sœur et la
princesse d'Harcourt commencèrent à entreprendre. Il compta avec raison
avoir affaire aux personnes du monde les moins unies, les moins
concertées, les moins attentives, qui ne s'apercevraient de rien
qu'après coup, qui ne sauraient par ces défauts comment se défendre, sur
quoi le passé lui répondait de l'avenir\,; car c'est de la sorte que de
conjonctures, d'entreprises, et pièce à pièce, que leur rang s'est formé
et maintenu, et qu'il prétendait l'étendre et l'agrandir. Il comptait
encore, ou que de guerre lasse on les laisserait faire, ou qu'à force de
disputes, avec les appuis que je viens d'expliquer et la prédilection
constante de la reine mère pour les princes, dont il était resté quelque
chose au roi, ils tireraient toujours avantage de ces disputes en
partie, et peut-être en tout par l'importunité. Ils avaient la princesse
de Conti auprès de Monseigneur à leur disposition, par
M\textsuperscript{me} de Lislebonne et ses filles, et par les mêmes
immédiatement qui eurent enfin toute sa confiance. Avec tout cela, ils
ne firent, pour ainsi dire, que ballotter dans ces commencements. L'état
de M\textsuperscript{me} la Dauphine, toujours mourante, retranchait
beaucoup d'occasions, et il y en eut encore beaucoup moins depuis sa
mort, jusqu'à ce que M\textsuperscript{me} la duchesse de Bourgogne
commençât à tenir une cour. Ils ne voulaient pas se hasarder sous les
yeux du roi, qu'ils n'eussent essayé ailleurs, et qu'ils ne l'eussent
accoutumé à leurs entreprises, mais elles se produisirent hautement dès
le lendemain de son mariage, au premier cercle qu'elle tint. Les
princesses ne se mirent plus au-dessous des duchesses\,: après, elles
prétendirent la droite, et l'eurent souvent par leur concert et leur
diligence. Ils avaient affaire à une dame d'honneur qui craignait tout,
qui voulait être l'amie de tout le monde, qui n'ignorait pas la
prédilection de Mine de Maintenon, qui tremblait devant elle, et qui,
basse et de fort peu d'esprit, se trouvait toujours embarrassée, et n'y
savait d'issue qu'en souffrant tout et laissant tout entreprendre, et
l'âge de Mine la duchesse de Bourgogne ne lui permettait ni de savoir ce
qui devait être, ni d'imposer.

Tel était l'état des choses à cet égard, quand les Lorrains, lassés de
leurs faibles avantages de diligence et de ruse où ils se trouvaient
quelquefois prévenus, résolurent d'en usurper de plus réels et se
crurent en état de les emporter. Soit hasard ou dessein prémédité, le
leur éclata à la première audience que milord Jersey eut de
M\textsuperscript{me} la duchesse de Bourgogne, le mardi 6 janvier de
cette année. De part et d'autre, les dames arrivèrent avant qu'on pût
entrer. Les duchesses, qui s'étaient trouvées les plus diligentes, se
trouvèrent les premières à la porte et entrèrent les premières. La
princesse d'Harcourt et d'autres Lorraines suivirent. La duchesse de
Rohan se mit la première à droite. Un moment après, avant qu'on fût
assis, et comme les dernières arrivaient encore, titrées et non titrées
(et il y avait grand nombre de dames), la princesse d'Harcourt se glisse
derrière la duchesse de Rohan, et lui dit de passer à gauche. La
duchesse de Rohan répond qu'elle se trouvait bien là, avec grande
surprise de la proposition, sur quoi la princesse d'Harcourt n'en fait
pas à deux fois, et grande et puissante comme elle était, avec ses deux
bras lui fait faire la pirouette, et se met en sa place.
M\textsuperscript{me} de Rohan ne sait ce qui lui arrive, si c'est un
songe ou vérité, et, voyant qu'il s'agissait de faire tout de bon le
coup de poing, fait la révérence à M\textsuperscript{me} la duchesse de
Bourgogne et passe de l'autre côté, ne sachant pas trop encore ce
qu'elle faisait ni ce qui lui arrivait, dont toutes les dames furent
étrangement étonnées et scandalisées. La duchesse du Lude n'osa dire
mot\,; et M\textsuperscript{me} la duchesse de Bourgogne à son âge
encore moins, mais sentit l'insolence et le manque de respect.
M\textsuperscript{me} d'Armagnac, et ses fille et belle-fille, qui
voulait aussi la droite à l'audience de l'ambassadeur, qui se donnait
dans la pièce qui précédait celle du lit où on était, contente de
l'expédition qu'elle venait de voir, se tint vers la porte de ces deux
pièces, qui était le côté gauche de celle du lit, y fit asseoir ses
fille et belle-fille, quoique après les duchesses, dit qu'il y avait
trop de monde et s'en alla dans la pièce de l'audience garder la droite,
et se mit dans le cercle qui était arrangé tout prêt vers le bas bout de
la droite. La toilette finie, on passa dans la pièce de l'audience.
M\textsuperscript{me} de Saint-Simon était grosse de six semaines ou
deux mois. Elle était venue tard et des dernières du côté gauche,
tellement que lorsqu'on se leva elle n'eut qu'un pas à faire pour gagner
la pièce de l'audience. Ce brouhaha d'y passer était toujours assez
long\,; elle se trouvait mal et ne pouvait se tenir debout. Elle alla
donc s'asseoir, en attendant qu'on vint, sur le premier tabouret qu'elle
y trouva du cercle même tout arrangé\,; et comme le côté droit de ce
cercle était le plus près de la porte des deux pièces, elle se trouva à
deux sièges au-dessus de M\textsuperscript{me} d'Armagnac, mais celle-ci
tournée en cercle et en dedans, et M\textsuperscript{me} de Saint-Simon
en dehors tournée le visage à la muraille, de manière qu'elles étaient
toutes deux comme adossées. M\textsuperscript{me} d'Armagnac, qui vit
qu'elle se trouvait un peu mal, lui offrit de l'eau de la reine de
Hongrie. Comme on se mit à passer un peu après, elle lui dit qu'étant la
première arrivée, elle ne croyait pas qu'elle voulût se mettre au-dessus
d'elle. M\textsuperscript{me} de Saint-Simon, qui ne s était mise là
qu'en attendant, ne répondit point, et dans le même moment s'alla mettre
de l'autre côté, où elle s'assit même avant qu'on fût rangé\,; et fit
mettre une duchesse devant elle pour la cacher jusqu'à ce qu'on fût
placé.

J'appris ce qui s'était passé à la toilette, et je sus par des dames du
palais que M\textsuperscript{me} la duchesse de Bourgogne était fort
bien disposée, et qu'elle comptait d'en parler au roi et à
M\textsuperscript{me} de Maintenon. Je crus qu'il était important de ne
pas souffrir un affront, et à propos d'en tirer parti. Nous conférâmes
quelques-uns ensemble. Le maréchal de Boufflers alla parler à M. de
Noailles, et moi à M. de La Rochefoucauld, au retour du roi, qui était
allé tirer. L'avis fut que M. de Rohan devait le lendemain matin
demander justice au roi, sans être accompagné, parce que le roi
craignait et haïssait tout ce qui sentait un corps. J'allai aussi voir
M. de La Trémoille qui allait souper chez le duc de Rohan, à la ville,
qui n'avait point de logement\,; M. de La Trémoille me promit de le
disposer à ce que nous désirions.

Comme j'étais au souper du roi, M\textsuperscript{me} de Saint-Simon
m'envoya dire de venir sur-le-champ lui parler dans la grande cour, où
elle m'attendait dans son carrosse\,; j'y allai. Elle me dit qu'elle
venait d'être avertie par M\textsuperscript{me} de Noailles, sortant de
chez la duchesse du Lude, qui l'avait trouvée sortant de chez M. de
Duras, qui était l'appartement joignant, que les trois frères lorrains
avaient été au tirer du roi\,; qu'ils s'y étaient toujours tenus tous
trois tout seuls, séparés de tout ce qui y était, et peu de gens avaient
la liberté de suivre le roi et aucun de l'approcher, excepté le
capitaine des gardes en quartier, qui était le duc de Noailles, qu'ils
avaient paru disputer entre eux\,; et M. de Marsan le plus agité\,;
qu'enfin après un long débat, M. le Grand les avait quittés, s'était
avancé au roi, lui avait parlé assez longtemps\,; que M. de Noailles
avait entendu que c'était une plainte qu'il faisait de ce qu'à
l'audience du matin M\textsuperscript{me} de Saint-Simon avait pris la
place de M\textsuperscript{me} d'Armagnac, et s'était mise au-dessus
d'elle, à quoi le roi n'avait pas distinctement répondu, et fort en un
mot\,; après quoi M. le Grand était allé rejoindre ses frères, et était
toujours demeuré en particulier avec eux. M\textsuperscript{me} de
Saint-Simon, bien étonnée de l'étrange usage qu'ils faisaient de la
chose du monde la plus simple et la plus innocente, et du mensonge
qu'ils y ajoutaient, conta ce qui lui était arrivé à
M\textsuperscript{me} de Noailles, qui fut d'avis que j'en fisse parler
au roi le soir même. Ces messieurs, fort embarrassés de soutenir ce que
la princesse d'Harcourt avait fait à la duchesse de Rohan, en quelque
disgrâce qu'eussent toujours vécu le duc de Rohan et elle, et qui
craignaient des plaintes au roi, saisirent ce qui était arrivé à
M\textsuperscript{me} de Saint-Simon pour se plaindre les premiers et
tâcher de compenser l'un par l'autre. Voilà un échantillon de l'artifice
de ces messieurs, et d'un mensonge public et dont toute l'audience était
témoin. Cet artifice, tout mal inventé qu'il fût, me mit en colère.
J'allai trouver M. de La Rochefoucauld, qui voulut absolument que je
parlasse au roi à son coucher. «\,Je le connais bien, me dit-il,
parlez-lui hardiment, mais respectueusement\,; ne touchez que votre
affaire\,; n'entamez point celle des ducs, et laissez faire M. de Rohan
demain, c'est la sienne. Croyez-moi, ajouta-t-il, des gens comme vous
doivent parler eux-mêmes\,; votre liberté et votre modestie plairont au
roi, il l'aimera cent fois mieux.\,» J'insistai\,; lui aussi. Je voulus
voir si le conseil partait du cœur ou de l'esprit, et je lui proposai de
monter vite chez M. le maréchal de Lorges, et que je l'engagerais à
parler. «\,Non, encore un coup, non, reprit le duc, cela ne vaut rien,
parlez vous-même. Si au petit coucher j'en puis trouver le moyen, je
parlerai à mon tour.\,» Cela me détermina.

Je remonte chez le roi, et voulus m'avancer au duc de Noailles, qui
sortait de prendre l'ordre. Il ne jugea pas devoir paraître avec moi, et
me dit en passant de parler au coucher. Boufflers, à qui Noailles avait
conté l'affaire, m'en dit autant, et qu'il ne s'avancerait point pour
prendre l'ordre que je n'eusse parlé. Je m'approchai de la cheminée du
salon, et quand le roi vint, je me contentai de le voir aller se
déshabiller. Comme il eut donné le bonsoir, et qu'à son ordinaire il se
fut retiré le dos au coin de la cheminée pour donner l'ordre, tandis que
tout ce qui n'avait pas les entrées sortait, je m'avançai à lui, et lui
aussitôt se baissa pour m'écouter en me regardant fixement. Je lui dis
que je venais d'apprendre tout à l'heure la plainte que M. le Grand lui
avait faite de M\textsuperscript{me} de Saint-Simon, que rien au monde
ne me touchait tant que l'honneur de son estime et de son approbation,
et que je le suppliais de me permettre de lui conter le fait, et tout de
suite j'enfile ma narration telle que je l'ai faite ci-dessus, et sans
en oublier une seule circonstance. Je m'en tins là, suivant le conseil
de M. de La Rochefoucauld. Je n'ajoutai aucune plainte ni des Lorrains
ni de M. le Grand, et je me contentai de lui avoir donné, par le simple
et véritable exposé du fait, un parfait démenti. Le roi ne m'interrompit
jamais d'un seul mot depuis que j'eus ouvert la bouche. Quand j'eus
fini, il me répondit : «\,Cela est bien, monsieur, d'un air très
gracieux et content, il n'y a rien à cela,\,» en souriant avec un signe
de tête comme je me retirais. Après quelques pas faits, je me rapprochai
du roi avec vivacité, je l'assurai de nouveau que tout ce que je lui
avais avancé était vrai de point en point, et je reçus la même réponse.

L'heure de parler au roi était tellement indue, les spectateurs avaient
trouvé le discours si long et si actif de ma part, et si bien reçu à
l'air du roi, que leur curiosité était extrême de savoir ce qui m'avait
pu engager à une démarche si peu usitée, quoique la plupart se
doutassent bien en gros qu'il s'agissait de l'affaire du matin. Beaucoup
de courtisans attendaient dans les antichambres. Le maréchal de
Boufflers prit l'ordre, et me trouva avec le duc d'Humières. Je leur
rendis ma conversation, je fis ensuite quelques tours par rapport à
M\textsuperscript{me} la duchesse de Bourgogne, et je m'en allai après
chez le duc de Rohan, comme je l'avais promis. Ma conversation avec le
roi avait déjà couru partout, à cause de l'heure indue où je Pavais eue.
Ils ne m'attendaient plus, et avaient envoyé chez moi le fils du duc de
Rohan pour tâcher d'en apprendre quelque chose. Ils me pressèrent
là-dessus. La présence du duc d'Albret me retint, et celle encore de la
comtesse d'Egmont. Enfin, après bien des assurances et des instances, il
fallut les satisfaire, et je m'y portai pour donner courage au duc de
Rohan. Ce qu'il fallut essuyer de disparates de sa part rie se peut
imaginer, avec une déraison surnageante à désoler. À la fin pourtant il
promit de parler au roi le lendemain, comme nous le voulions, et je les
quittai là-dessus à trois heures après minuit.

Le lendemain de bonne heure je retournai voir le maréchal de Boufflers
pour qu'il instruisît M. de Noailles, et je fus rendre compte de ma
soirée à M. le maréchal de Lorges qui n'en savait pas un mot, et à qui
jusque-là je n'avais pas eu le temps d'en parler. Il alla aussi dire au
roi ce dont je venais de le prier, et cependant je me montrai fort chez
le roi, où je vis le maréchal de Villeroy très animé, tout ami intime
qu'il fût des trois frères et beau-frère de l'aîné. J'envoyai cependant
messager sur messager au duc de Rohan pour l'avertir des moments et le
presser de venir. Enfin il arriva, comme le roi allait sortir de la
messe. Il se mit à la porte du cabinet et quelques ducs avec lui. Comme
le roi approcha, il s'avança. Le roi le fit entrer et le mena à la
fenêtre de son cabinet, et la porte se ferma aussitôt, en sorte qu'il
demeura seul avec le roi. Les maréchaux de Villeroy, Noailles, Boufflers
et quelques autres ducs se tinrent à la porte. Je crus en avoir assez
fait, et je regardais de la cheminée du salon toute cette pièce entre
eux et moi, mais dans la même. Cela dura près d'un petit quart d'heure.
Le duc de Rohan sortit fort animé, le duc de Noailles ne fit qu'entrer
et sortir pour prendre l'ordre, et tous vinrent à moi à la cheminée,
puis nous sortîmes dans la chambre du roi où nous nous mîmes en tas à la
cheminée. Là le duc de Rohan nous rendit sa conversation, où rien ne fut
oublié. Il demanda justice sur sa femme de la princesse d'Harcourt,
s'étendit sur les entreprises des Lorrains et l'impossibilité d'éviter
des querelles continuelles\,; il fit valoir le respect violé à
M\textsuperscript{me} la duchesse de Bourgogne par la princesse
d'Harcourt, et gardé par la duchesse de Rohan, expliqua bien le fait de
M\textsuperscript{me} de Saint-Simon et de M\textsuperscript{me}
d'Armagnac, et le noir et audacieux artifice des Lorrains pour se tirer
d'affaire par ce faux change\,; en un mot, parla avec beaucoup de force,
d'esprit et de dignité. Le roi lui répondit qu'il l'avait laissé dire
pour en être encore mieux informé par lui\,; qu'il l'était dès la veille
par M\textsuperscript{me} la duchesse de Bourgogne, par la duchesse du
Lude qui lui avaient dit les mêmes choses\,; qu'il l'avait été le soir
par moi, et ce matin encore par M. le maréchal de Lorges, et qu'il nous
en avait parfaitement crus l'un et l'autre\,; qu'il louait fort le
respect et la modération de M\textsuperscript{me} de Rohan, et trouvait
la princesse d'Harcourt fort impertinente. Il s'expliqua en termes durs
sur les Lorrains, et par deux fois l'assura qu'il y mettrait ordre et
qu'il serait content. Je sus ensuite par mes amies du palais que
M\textsuperscript{me} de Saint-Simon avait été servie à souhait par
M\textsuperscript{me} la duchesse de Bourgogne, et qu'il y avait eu une
dispute assez forte entre le roi et M\textsuperscript{me} de Maintenon,
qui obtint à toute peine que la princesse d'Harcourt qui allait toujours
à Marly n'en fût pas exclue le lendemain. M\textsuperscript{me}
d'Armagnac et ses fille et belle-fille qui s'étaient présentées, pas une
n'y fut.

Toute cette journée se passa encore en mesures. Le lendemain, le roi
alla à Marly. M\textsuperscript{me} la duchesse de Bourgogne n'y
couchait pas encore, mais elle y allait tous les jours. Nous demeurâmes
tard à Versailles pour la bien instruire par ce qui l'environnait. Elle
fit merveilles le lendemain. La princesse d'Harcourt essuya du roi une
rude sortie\,; et M\textsuperscript{me} de Maintenon lui lava fort la
tête, en sorte que tout le voyage ce fut autre nature, la douceur et la
politesse même, mais avec la douleur et l'embarras peints sur toute sa
personne. Ce ne fut pas tout. Elle eut ordre de demander pardon en
propres termes à la duchesse de Rohan, et ce fut encore à
M\textsuperscript{me} de Maintenon à qui elle dut que ce ne fût pas chez
la duchesse, et qui fit régler que, n'ayant point de logement, la chose
se passerait en plénière compagnie chez M\textsuperscript{me} de
Pontchartrain. En même temps la duchesse du Lude eut ordre du roi de
déclarer à la maison de Lorraine\,: que le mariage de M. de Lorraine ne
leur donnait rien de plus et ne leur faisait pas d'un fétu, ce fut
l'expression. Elle s'en acquitta, et deux jours après le retour de
Marly, la duchesse de Rohan se rendit à heure prise chez
M\textsuperscript{me} la chancelière, où il y avait beaucoup de dames et
de gens de la cour à dîner. La princesse d'Harcourt y vint qui lui fit
des excuses, l'assura qu'elle l'avait toujours particulièrement honorée,
et qu'en un mot elle lui demandait pardon de ce qui s'était passé.
M\textsuperscript{me} de Rohan reçut tout cela fort gravement, et
répondit fort froidement. La princesse d'Harcourt redoubla de
compliments, lui dit qu'elle savait bien que ce devrait être chez elle
qu'elle aurait dû lui témoigner son déplaisir, qu'elle comptait bien
aussi d'y aller s'acquitter de ce devoir, et lui demander l'honneur de
son amitié, à quoi si elle pouvait réussir elle s'estimerait la plus
heureuse du monde. C'était là tomber d'une grande audace à bien de la
bassesse. Dire poliment ce que le roi avait prescrit aurait suffi. Mais
elle était si battue de l'oiseau qu'elle crut n'en pouvoir trop dire
pour en faire sa cour, et voilà comme sont les personnes qui en sont
enivrées\,! elles se croient tout permis, et quand cela bâte mal, elles
se croient perdues, et se roulent dans les dernières soumissions pour
plaire et pour se raccrocher. Telle fut la fin de cette étrange histoire
qui nous donna enfin repos.

Pendant le voyage de Marly, j'appris que M. le Grand, outré de ce que
leur entreprise leur était retombée à sus en plein, se plaignait de ce
que, parlant au roi et au monde, je lui avais donné un démenti. Dès le
même jour que le roi retourna à Versailles, j'y allai\,; j'affectai de
me montrer partout, et de me donner licence parfaite en propos sur le
grand écuyer et sur sa famille. Je m'attendis à quelque sortie brusque
de sa part ou de la leur, en me rencontrant\,; ma réponse aussi était
toute prête, et ma résolution prise de leur parler si haut que ce fût à
eux à courir\,; mais tout brutal et tout furieux qu'il était, et toute
piquée qu'était sa famille, aucun d'eux ne s'y commit\,; je fus même
surpris que, l'ayant tôt après rencontré, il me salua le premier, mais
de cette époque nous sûmes de part et d'autre à quoi nous en tenir. Sept
ou huit jours après, la comtesse de Jersey eut sa première audience de
M\textsuperscript{me} la duchesse de Bourgogne. Les duchesses y eurent
la droite, et les Lorraines la gauche, et mêlées entre elles. Elles
s'étaient avisées depuis quelque temps de se déplacer par aînesse, comme
font les princesses du sang\,; le roi le leur avait défendu, elles y
étaient encore revenues, et le roi l'avait trouvé très mauvais. Il vint
à cette audience pour saluer l'ambassadrice, comme cela se fait toujours
à pareilles audiences. Après l'avoir saluée, il demeura au milieu du
cercle, auprès d'elle, regarda et considéra le cercle de tous les côtés,
puis il dit tout haut que ce cercle était fort bien arrangé comme cela.
Ce fut une nouvelle mortification aux Lorrains.

En ce même cercle, M\textsuperscript{me} la Princesse était à la tête
des duchesses, en retour comme elles, et coude à coude de la première.
M\textsuperscript{me} la Duchesse était de même, à gauche, à la tête des
Lorrains. Les princesses du sang avaient essayé de se mettre en face du
cercle à lieux arrangés, à distance de M\textsuperscript{me} la duchesse
de Bourgogne, mais sur la même ligne qu'elle\,; le roi l'avait trouvé
fort mauvais et défendu. Il n'y a que les fils et filles de France qui
se placent de la sorte, même le roi et la reine y étant, et les
petits-fils et les petites-filles de France, dans les deux coins, à demi
tournés, ni en face de tout, ni entièrement de côté, et le roi voulut
que cela fût de même pour M\textsuperscript{me} la duchesse de
Bourgogne, et cela avait toujours été ainsi avec M\textsuperscript{me}
la Dauphine-Bavière.

\hypertarget{chapitre-xvi.}{%
\chapter{CHAPITRE XVI.}\label{chapitre-xvi.}}

1699

~

{\textsc{Mort de la duchesse de Chaulnes.}} {\textsc{- Mort de
Chamarande père.}} {\textsc{- Problème brûlé par arrêt du parlement.}}
{\textsc{- Voyage de M\textsuperscript{me} de Nemours, du prince de
Conti et des autres prétendants à Neuchâtel.}} {\textsc{- Paix de
Carlowitz.}} {\textsc{- Prince électoral de Bavière, héritier et nommé
tel de la monarchie d'Espagne, et sa mort.}} {\textsc{- Neuvième
électorat reconnu.}} {\textsc{- Mort du célèbre chevalier Temple.}}
{\textsc{- Trésor inutilement cherché pour le roi chez l'archevêque de
Reims.}} {\textsc{- Mort du chevalier de Coislin.}} {\textsc{- Mort de
La Feuillée.}} {\textsc{- M. de Monaco, ambassadeur à Rome\,; ses
prétentions, son succès.}} {\textsc{- Monseigneur des secrétaires d'État
et aux secrétaires d'État.}} {\textsc{- Fauteuil de l'abbé de Cîteaux
aux états de Bourgogne.}} {\textsc{- M\textsuperscript{me} de
Saint-Géran rappelée.}} {\textsc{- Mariage du comte d'Auvergne avec
M\textsuperscript{lle} de Wassenaer.}} {\textsc{- Ambassade de Maroc.}}
{\textsc{- Torcy ministre\,; bizarrerie de serments.}} {\textsc{-
Reineville, lieutenant des gardes du corps, disparu.}} {\textsc{-
Permillac se tue.}}

~

La duchesse de Chaulnes mourut dans tous les premiers jours de cette
année, n'ayant pu survivre son mari plus de quelques mois. Ils avaient
passé leur vie dans la plus intime union. C'était, pour la figure
extérieure, un soldat aux gardes, et même un peu suisse habillé en
femme\,; elle en avait le ton et la voix, et des mots du bas peuple\,;
beaucoup de dignité, beaucoup d'amis, une politesse choisie, un sens et
un désir d'obliger qui tenaient lieu d'esprit, sans jamais rien de
déplacé, une grande vertu, une libéralité naturelle, et noble avec
beaucoup de magnificence, et tout le maintien, les façons, l'état et la
réalité d'une fort grande dame, en quelque lieu qu'elle se trouvât,
comme M. de Chaulnes l'avait de même d'un fort grand seigneur. Elle
était, comme lui, adorée en Bretagne, et fut pour le moins aussi
sensible que lui à l'échange forcé de ce gouvernement. On a vu ailleurs
qui elle était, et de qui veuve en premières noces et sans enfants de
ses deux maris. Elle ne fit que languir et s'affliger depuis la mort de
M. de Chaulnes, et ne voulut presque voir personne dans le peu qu'elle
vécut depuis.

Le bonhomme Chamarande la suivit de fort près, universellement estimé,
considéré et regretté. J'en ai suffisamment parlé ailleurs pour n'avoir
rien à y ajouter ici\,; il avait une assez bonne abbaye, chose avec
raison devenue dès lors si rare aux laïques.

Villacerf essuya un grand dégoût par le désordre qui se trouva dans les
fonds des bâtiments. Un nommé Mesmin, son principal commis, en qui il se
fiait de tout, abusa longtemps de sa confiance. Les plaintes des
ouvriers et des fournisseurs, longtemps retenues par l'amitié et par la
crainte, éclatèrent enfin\,; il fallut répondre et voir clair.
Villacerf, dont la probité était hors de tout soupçon, et qui s'en
pouvait rendre le témoignage à lui-même, parla fort haut\,; mais quand
ce fut à l'examen, Mesmin s'enfuit, et il se trouva force friponneries.
Villacerf en conçut un si grand déplaisir, qu'il se défit des bâtiments.
Le roi qui l'aimait, mais qui jugeait que sa tête n'était plus la même,
lui donna douze mille livres de pension, outre qu'il en avait déjà, et
accepta sa démission\,; et à peu de jours de là, donna les bâtiments à
Mansart, son premier architecte, qui était neveu du fameux architecte
Mansart, mais d'une autre famille. Il s'appelait Hardouin, et pour
s'illustrer dans son métier, où il n'était pas habile, il prit le nom de
son oncle, et fut meilleur et plus habile et heureux courtisan que le
vieux Mansart n'avait été architecte.

Il parut un livre intitulé \emph{Problème}, sans nom d'auteur, qui fit
un grand vacarme\,: l'auteur consultait, par toutes les plus malignes
raisons pour et contre, savoir lequel on devait croire sur des questions
théologiques de M. de Noailles, évêque de Châlons, ou du même M. de
Noailles, archevêque de Paris. Il prétendait que ce prélat était devenu
contraire à lui-même, et avait dit blanc et noir sur les mêmes
questions, favorablement aux jansénistes, étant à Châlons, et
défavorablement, étant à Paris. Ce fut le premier coup qui lui fut
porté. Il ne douta pas qu'il ne lui vint des jésuites\,; sa doctrine
était fort différente de la leur, et jamais il n'avait été bien avec
eux. Il était devenu archevêque de Paris sans eux\,; toutes ses liaisons
de prélats et d'ecclésiastiques étaient contraires aux leurs. L'affaire
de M. de Cambrai était une nouvelle matière de division entre eux,
d'autant plus sensible aux jésuites qu'ils n'osaient toucher cette
corde-là, qui les avait pensé perdre. C'en était plus qu'il n'en fallait
pour persuader M. de Paris que ce livre si injurieux était sorti de leur
boutique. Ils eurent beau protester d'injure en public et en
particulier, et aller lui témoigner leur désaveu et leur peine qu'il
prit cette opinion d'eux, ils furent froidement écoutés, et comme des
gens qui ne persuadaient pas, mais qu'on voulait bien faire semblant de
croire. Le livre fut condamné et exécuté au feu, par arrêt du parlement,
et les jésuites, contre qui tout se souleva, en burent toute la honte,
et ne le pardonnèrent jamais à M. de Paris.

Au bout d'assez longtemps, le pur hasard lui fit trouver le véritable
auteur du \emph{Problème}, et avec de telles preuves, que l'auteur même
demeura convaincu jusqu'à ne pouvoir le désavouer. Il n'était pas loin,
puisqu'il logeait dans l'archevêché. C'était un docteur de beaucoup
d'esprit, d'une grande érudition, et qui avait toujours vécu en très
homme de bien. Il s'appelait Boileau, différent de l'ami de Bontems qui
a souvent prêché devant le roi, et différent encore du célèbre poète et
de l'auteur des \emph{Flagellants}. M. de Paris, qui cherchait à
s'attacher des gens de bien les plus éclairés pour l'aider dans la
grande place qu'on le força de remplir, avait pris ce M. Boileau chez
lui, le traitait avec tous les égards et toute la confiance qu'il aurait
pu témoigner à son propre frère, et le tenait à ses dépens. Boileau
était un homme sauvage qui se barricadait dans sa chambre, et qui
n'ouvrait qu'à ceux qui avaient le signal de lui de frapper un certain
nombre de coups, et encore à certaines heures. Il ne sortait de ce
repaire que pour aller à l'église ou chez M. l'archevêque, travaillait
obscurément, vivait en pénitent fort solitaire, avait une plume belle,
forte, éloquente, et beaucoup de suite et de justesse. Qui eût cru que
le \emph{Problème} fût sorti de celle-là\,? M. de Paris en fut touché
extrêmement. On peut juger que ce docteur délogea à l'heure même, et
qu'il n'eût pas été difficile à M. de Paris de le faire enfermer pour le
reste de ses jours. Il prit un parti bien contraire, et bien digne d'un
grand évêque. Il vaqua à peu de jours de là un canonicat de
Saint-Honoré, qui sont fort bons. Il le lui donna. Boileau, qui n'avait
pas de quoi vivre, l'accepta, et acheva de se déshonorer. Il n'était pas
content de ce que M. de Paris ne levait pas bouclier pour les
jansénistes, et qu'il ne mit pas tout son crédit à faire tout ce qu'ils
auraient voulu. C'est ce qui lui fit faire ce livre dont les jésuites
surent bien triompher.

M. le prince de Conti, ayant gagné son procès contre
M\textsuperscript{me} de Nemours, songea à en tirer la meilleure pièce,
qui était Neuchâtel. Pour abréger matière, il engagea le roi à envoyer
M. de Torcy de sa part à M\textsuperscript{me} de Nemours lui faire
diverses propositions, qui toutes aboutissaient à ne point plaider
devant MM. de Neuchâtel, à l'en laisser jouir sa vie durant, et à faire
avec sûreté qu'après elle cette principauté revint à M. le prince de
Conti. M\textsuperscript{me} de Nemours qui avait beaucoup d'esprit et
de fermeté, et qui se sentait la plus forte à Neuchâtel, vint dès le
lendemain parler au roi, refusa toutes les propositions, et moyennant
qu'elle promit au roi de n'employer aucune voie de fait, elle lui fit
trouver bon qu'elle allât à Neuchâtel soutenir son droit. M. le prince
de Conti l'y suivit, Matignon y alla aussi, et enfin les ducs de
Lesdiguières et de Villeroy, qui tous y prétendaient droit après
M\textsuperscript{me} de Nemours. Ces trois derniers descendaient des
deux sœurs de M. de Longueville, grand-père de M\textsuperscript{me} de
Nemours\,: les deux ducs de l'aînée, mariée au fils aîné du maréchal de
Retz, et M. de Villeroy n'y prétendait que du même droit et après M. de
Lesdiguières\,; la cadette mariée au fils du maréchal de Matignon. Le
vieux Mailly et d'autres gens se firent ensuite un honneur d'y prétendre
par des généalogies tirées aux cheveux. Il y a eu sur cette grande
affaire des factums curieux de tous ces prétendants. Le public
désintéressé jugea en faveur de M. de Lesdiguières. On les peut voir
avec satisfaction. Je ne m'embarquerai pas dans le détail de cette
célèbre et inutile dispute, où un tiers sans droit mangea l'huître et
donna les écailles aux prétendants.

Je ne m'engagerai pas non plus dans la discussion des affaires des
Impériaux et des Turcs\,; je me contenterai de dire que l'empereur, qui
avait grand besoin de la paix, l'eut avec eux au commencement de cette
année, par le traité de Carlowitz, où la Pologne et la république de
Venise furent comprises, {[}paix{]} assez avantageuse pour l'état
présent des affaires, mais où Venise se plaignit amèrement de
l'empereur, et après quelques mois, ne pouvant mieux, la signa.

Il y avait cinq ou six mois que le roi d'Espagne, hors de toute
espérance d'avoir des enfants, et dans une infirmité de toute sa vie qui
s'augmentait à vue d'œill, avait voulu fixer la succession de sa vaste
monarchie, indigné qu'il était de tous les projets de la partager après
lui qui lui revenaient sans cesse. La reine sa femme avait beaucoup de
crédit sur son esprit, et elle-même était entièrement gouvernée par une
Allemande qu'elle avait amenée avec elle, qu'on appelait la comtesse de
Berlips, et qui amassait pour elle et pour les siens des trésors à
toutes mains. Cette reine était sœur de l'impératrice, mais en même
temps elle l'était comme elle de l'électeur palatin, par conséquent
parente et de même maison de l'électeur de Bavière. Malgré la haine des
deux branches électorales, depuis l'affaire de Bohème, on crut que
l'amour de la maison l'avait emporté sur celui des proches, et que la
reine, menée par la Berlips, avait eu grande part à la disposition du
roi d'Espagne.

Il fit un testament par lequel il appela à la succession entière de
toutes ses couronnes et États le prince électoral de Bavière, qui avait
sept ans. Sa mère, qui était morte, était fille unique du premier lit de
l'empereur Léopold, et de Marguerite-Thérèse, sœur du roi d'Espagne,
tous deux seuls du second lit de Philippe IV et de la fille de
l'empereur Ferdinand III\,; je dis seuls, parce que tous les autres sont
morts sans alliance. La reine épouse de notre roi était par cette raison
seule du premier lit du même Philippe IV, et d'une fille de notre roi
Henri IV et sœur aînée du père du roi d'Espagne et de l'impératrice,
mère de l'électrice de Bavière, dont le fils, en faveur duquel ce
testament se fit, était en effet le véritable héritier de la monarchie
d'Espagne, si on a égard aux renonciations du mariage du roi et de la
paix des Pyrénées. Dès que ce testament fut fait, le cardinal
Portocarrero le dit en grand secret au marquis d'Harcourt, qui dépêcha
d'Igulville au roi avec cette nouvelle. Le roi, ni lors, ni depuis
qu'elle fut devenue publique, n'en parut pas avoir le plus léger
mécontentement. L'empereur n'en dit rien aussi. Il espérait bien cette
vaste succession, et réunir dans sa branche tous les États de sa maison.
Mais son conseil avait ses ressources accoutumées. Il n'y avait pas
longtemps qu'il s'en était servi pour se défaire de la reine d'Espagne,
fille de Monsieur, qui n'avait point d'enfants, et qui prenait à son gré
trop de crédit sur le roi son mari. Le prince électoral de Bavière
mourut fort brusquement les premiers jours de février, et personne ne
douta que ce ne fût par l'influence du conseil de Vienne. Ce coup remit
l'empereur dans ses premières espérances, et plongea l'Europe dans la
douleur et dans le trouble des mesures à prendre sur l'ouverture de
cette prodigieuse succession, que chacun regardait avec raison comme ne
pouvant pas être éloignée.

Presque en même temps le neuvième électorat érigé en faveur du duc
d'Hanovre, qui avait causé tant de mouvements dans l'empire, et qui
était entré dans la guerre et dans la paix, fut reconnu par une partie
de l'Allemagne et de l'Europe.

L'Angleterre presque en même temps perdit, dans un simple particulier,
un de ses principaux ornements, je veux dire le chevalier Temple, qui a
également figuré avec la première réputation dans les lettres et dans
les sciences, et dans celle de la politique et du gouvernement, et qui
s'est fait un grand nom dans les plus grandes ambassades et les
premières méditations de paix générale. C'était avec beaucoup d'esprit,
d'insinuation, de fermeté et d'adresse, un homme simple d'ailleurs, qui
ne cherchait point à paraître, et qui aimait à se réjouir, et à vivre
libre en vrai Anglais, sans aucun souci d'élévation, de biens ni de
fortune. Il avait partout beaucoup d'amis, et des amis illustres qui
s'honoraient de son commerce. Dans un voyage qu'il fit en France pour
son plaisir, le duc de Chevreuse, qui le connaissait par ses ouvrages,
le vit fort. Ils se rencontrèrent un matin dans la galerie de
Versailles, et les voilà à raisonner machines et mécaniques. M. de
Chevreuse, qui ne connaissait point d'heure quand il raisonnait, le tint
si longtemps que deux heures sonnèrent. À ce coup d'horloge, M. Temple
interrompit M. de Chevreuse, et, le prenant par le bras\,: «\,Je vous
assure, monsieur, lui dit-il, que de toutes les sortes de machines, je
n'en connais aucune qui soit si belle, à l'heure qu'il est, qu'un
tournebroche, et je m'en vais tout courant en éprouver l'effet,\,» lui
tourna le dos et le laissa fort étonné qu'il pût songer à dîner.

Des ministres aussi désintéressés que celui-là sont bien rares. Les
nôtres n'en avaient pas le bruit. Il vint des avis au roi et fort
réitérés qu'il y avait huit millions enterrés dans la cour de la maison
du feu chancelier Le Tellier. Le roi, qui n'en voulut rien croire, fut
pourtant bien aise que cela revint à l'archevêque de Reims, à qui était
la maison, et qui y logeait, et se rendit aisément à la prière qu'il lui
fit de faire fouiller partout en présence de Chamillart, intendant des
finances. On bouleversa tous les endroits que la donneuse d'avis
indiqua, on ne trouva rien, on eut la honte de l'avoir crue, et elle eut
la prison pour salaire de ses avis.

Les honnêtes gens de la cour regrettèrent un cynique, qui vécut et
mourut tel au milieu de la cour et du monde, et qui n'en voyait que ce
qui lui en plaisait\,; ce fut le chevalier de Coislin, frère du duc et
du cardinal de ce nom, et frère de mère comme eux de la maréchale de
Rochefort. C'était un très honnête homme de tous points, et brave,
pauvre, mais à qui son frère le cardinal n'avait jamais laissé manquer
de rien, et un homme fort extraordinaire, fort atrabilaire et fort
incommode. Il ne sortait presque jamais de Versailles, sans jamais voir
le roi, et avec tant d'affectation, que je l'ai vu, moi et bien
d'autres, se trouver par hasard sur le passage du roi, gagner au pied
d'un autre côté. Il avait quitté le service maltraité par M. de Louvois,
ainsi que son frère, à cause de M. de Turenne, à qui il s'était attaché,
et qui l'aimait. Il ne l'avait de sa vie pardonné au ministre ni au
maître, qui souffrait cette folie par considération pour ses frères. Il
logeait au château dans l'appartement du cardinal, et mangeait chez lui
où il y avait toujours fort bonne compagnie. Si quelqu'un lui
déplaisait, il se faisait porter un morceau dans sa chambre, et si étant
à table il survenait quelqu'un qu'il n'aimait point, il jetait sa
serviette et s'en allait bouder ou achever de dîner tout seul. On
n'était pas toujours à l'abri de ses sorties, et la maison de son frère
fut bien plus librement fréquentée après sa mort, quoique presque tout
ce qui y allait fût fait à ses manières, qui mettaient souvent ses
frères au désespoir, surtout le cardinal, qu'il tyrannisait.

Un trait de lui le peindra tout d'un coup. Il était embarqué avec ses
frères, et je ne sais plus quel quatrième, à un voyage du roi, car il le
suivait toujours sans le voir, pour être avec ses frères et ses amis. Le
duc de Coislin était d'une politesse outrée, et tellement quelquefois
qu'on en était désolé. Il complimentait donc sans fin les gens chez qui
il se trouvait logé dans le voyage, et le chevalier de Coislin ne
sortait point d'impatience contre lui. Il se trouva une bourgeoise
d'esprit, de bon maintien et jolie, chez qui on les marqua. Grandes
civilités le soir, et le matin encore davantage. M. d'Orléans, qui
n'était pas lors cardinal, pressait son frère de partir, le chevalier
tempêtait, le duc de Coislin complimentait toujours. Le chevalier de
Coislin qui connaissait son frère, et qui comptait que ce ne serait pas
sitôt fait, voulut se dépiquer et se vengea bien. Quand ils eurent fait
trois ou quatre lieues, le voilà à parler de la belle hôtesse et de tous
les compliments, puis, se prenant à rire, il dit à la carrossée que,
malgré toutes les civilités sans fin de son frère, il avait lieu de
croire qu'elle n'aurait pas été longtemps fort contente de lui. Voilà le
duc de Coislin en inquiétude, qui ne peut imaginer pourquoi, et qui
questionne son frère\,: «\,Le voulez-vous savoir\,? lui dit brusquement
le chevalier de Coislin\,; c'est que, poussé à bout de vos compliments,
je suis monté dans la chambre où vous avez couché, j'y ai poussé une
grosse selle tout au beau milieu sur le plancher, et la belle hôtesse ne
doute pas à l'heure qu'il est que ce présent ne lui ait été laissé par
vous avec toutes vos belles politesses.\,» Voilà les deux autres à rire
de bon cœur, et le duc de Coislin en furie qui veut prendre le cheval
d'un de ses gens et retourner à la couchée déceler le vilain, et se
distiller en honte et en excuses. Il pleuvait fort, et ils eurent toutes
les peines du monde à l'en empêcher, et bien plus encore à les
raccommoder. Ils le contèrent le soir à leurs amis, et ce fut une des
bonnes aventures du voyage. À qui les a connus, il n'y a peut-être rien
de si plaisant.

Le bonhomme La Feuillée, lieutenant général, grand-croix de Saint-Louis
et gouverneur de Dôle, etc., qu'on a vu ci-devant le mentor de
Monseigneur en Flandre, mourut bientôt après dans une grande estime de
probité, de valeur et de capacité à la guerre.

M. de Monaco partit dans ces temps-ci pour Rome. Il avait accepté
l'ambassade étant à Monaco, d'où il était venu recevoir ses ordres et
ses instructions. On a vu ci-devant qu'il avait obtenu le rang de prince
étranger au mariage de son fils, en 1688, avec une fille de M. le Grand,
chose à quoi ses pères n'avaient jamais pensé, et qu'il fut le dernier
jour de la même année chevalier de l'ordre en son rang d'ancienneté
parmi les ducs. Il prétendit que M. de Torcy avec qui il allait avoir un
commerce de lettres nécessaire et continuel, lui écrivit
\emph{monseigneur}, comme les secrétaires d'État l'écrivent aux Lorrains
et aux Bouillon, et il l'obtint tout de suite. Quand le roi en parla à
Torcy, il fut bien étonné et se récria fort. Il s'appuya principalement
sur ce que MM. de Rohan, dont le rang de prince étranger est antérieur à
celui de Monaco, n'avaient point ce traitement des secrétaires d'État,
et frappa si bien le roi par cette distinction, qu'il a constamment
refusée à M\textsuperscript{me} de Soubise, qu'il l'emporta. À son tour,
M. de Monaco fut bien surpris lorsque le roi lui dit que M. de Torcy lui
avait allégué des raisons si fortes, qu'il n'avait pu s'empêcher de s'y
rendre. M. de Monaco insista sur le dégoût et de la chose et du
changement, mais le roi tint ferme et le pria de n'y plus songer. M. de
Monaco outré partit brouillé avec Torcy, et l'effet de cette brouillerie
se répandit sur toute son ambassade, au détriment des affaires, qui en
souffrirent beaucoup.

Arrivé à Rome, il se mit à prétendre l'\emph{altesse}, ce qu'aucun de
ses pères n'avait imaginé. On a vu, à propos du cordon bleu donné à
Vaïni, que le cardinal de Bouillon y eut la même prétention, et ne put
jamais la faire réussir. Il traversa celle de M. de Monaco et n'y eut
pas grande peine. Personne ne voulut tâter de cette nouveauté, et lui
qui n'en voulut pas démordre passa le reste de sa vie dans une grande
solitude à Rome, ce qui gâta encore beaucoup les affaires dont il était
chargé, et brouillé de plus avec le cardinal de Bouillon\,; et voilà le
fruit des chimères et de leurs concessions

Pour venir au fond de la prétention sur les secrétaires d'État, il n'est
pas douteux qu'ils écrivaient \emph{monseigneur} à tous les ducs. J'ai
encore, par le plus grand hasard du monde, trois lettres à mon père,
lors à Blaye, de M. Colbert. Par la matière, quoique peu importante, et
mieux encore par les dates, on voit qu'il écrivit la première, n'étant
encore que contrôleur général, mais en chef, après la disgrâce de M.
Fouquet, et que, lorsqu'il écrivit les deux autres, il était contrôleur
général, secrétaire d'État, ayant le département de la marine, et
ministre d'État. Je ne sais comment elles se sont conservées, mais
toutes trois et dedans et dessus traitent mon père de
\emph{monseigneur}. M. de Louvois est celui qui changea ce style, et qui
persuada au roi qu'il y était intéressé, parce que ses secrétaires
d'État parlaient en son nom et donnaient ses ordres. Il parlait sans
contradicteur à un roi jaloux de son autorité, qui n'aimait de grandeur
que la sienne, et qui ne se donnait pas le temps, ni moins encore la
peine de la réflexion sur ce sophisme. M. de Louvois était craint,
chacun avait besoin de lui, les ducs n'ont jamais eu coutume de se
soutenir. Il écrivit \emph{monsieur} à un, puis à un autre, après à un
troisième\,; on le souffrit\,; après, cela fit exemple, et le
\emph{monseigneur} fut perdu. M. Colbert ensuite l'imita. Il n'y avait
pas plus de raison de s'offenser de l'un que de l'autre. On avait aussi
souvent besoin de lui que de M. de Louvois, et cela s'établit. La même
raison combattit pour les deux autres secrétaires d'État qui, bien que
moins accrédités, étaient secrétaires d'État comme les deux premiers, et
soutenus d'eux en ce style, et la chose fut finie. M. de Turenne, alors
en grande splendeur, et brouillé avec M. de Louvois, mit tout son crédit
à se faire conserver le \emph{monseigneur} que les secrétaires d'État
lui avaient donné, et à son frère, depuis leur rang de prince étranger,
obtenu par l'échange de Sedan et par la faveur du cardinal Mazarin qui
se jeta entre leurs bras. Cette continuation du même style à un homme
aussi principal dans l'État devint une grande distinction pour sa
maison, qu'il eut grand soin d'y faire comprendre. Cette planche fit à
plus forte raison le plain-pied de la maison de Lorraine. Celle de Rohan
n'était alors qu'un passage, et n'osa, par conséquent, ni se parangonner
aux deux autres, ni se mettre à dos des ministres aussi accrédités, et
depuis n'a pu les réduire à changer leur style avec elle. La facilité
avec laquelle M. de Louvois fit ce grand pas lui ouvrit une plus vaste
carrière. Bientôt après il exigea tant qu'il put d'être traité de
\emph{monseigneur} par ceux qui lui écrivaient. Le subalterne subit
aisément ce joug nouveau. Quand il y eut accoutumé le commun, il haussa
peu à peu, et à la fin il le prétendit de tout ce qui n'était point
titré. Une entreprise si nouvelle et si étrange causa une grande
rumeur\,; il l'avait prévu, et y avait préparé le roi par la même
adresse qui lui avait réussi à l'égard des ducs. Il se contenta d'abord
de mortifier ceux qui résistèrent, et bientôt après il fit ordonner par
le roi que personne non titré ne lui écrirait plus que
\emph{monseigneur}. Quantité de gens distingués en quittèrent le
service, et ont été poursuivis dans tout ce qu'ils ont pu avoir
d'affaires jusqu'à leur mort. La même chose qui était arrivée sur le
\emph{monseigneur} aux ducs des autres secrétaires d'État leur réussit
de même à tous quatre pour se le faire donner comme M. de Louvois\,; et
le rare est que ni lui ni les trois autres ne l'ont jamais prétendu ni
eu de pas un homme de robe. Us poussèrent après jusqu'à l'inégalité de
la {[}suscription{]} avec tout ce qui n'est point titré, et même avec
les évêques, archevêques, excepté les pères ecclésiastiques, et tout
leur a fait joug.

Une autre dispute fit en ce même temps quelque bruit. M. d'Autun,
président né des états de Bourgogne, disputait depuis quelque temps à
l'abbé de Cîteaux d'avoir un fauteuil dans cette assemblée. Cet honneur,
selon lui, n'était dû dans le clergé qu'aux évêques et non pas à un
moine, quoique chef d'un grand ordre. M. de Cîteaux, à qui cela
s'adressait, alléguait la dignité de son abbaye, dont l'autorité
s'étendait dans tout le monde catholique, et son ancienne possession,
que M. d'Autun traitait de vieil abus. Il y eut sur cela force factums
de part et d'autre. L'abbé de Cîteaux se trouvait lors une fort bonne
tête et fort apparenté dans la robe\,; il s'appelait M. Larcher, et qui
n'oublia pas de faire souvenir le chancelier Boucherat qu'il comptait
deux grands-oncles paternels parmi ses prédécesseurs, chose, bien
qu'élective, qui le flattait d'autant plus que sa famille, toute
nouvelle, n'avait rien de mieux à se vanter. Le roi à la fin voulut
juger l'affaire au conseil de dépêches. M. le Prince, gouverneur de
Bourgogne, et Ferrand, intendant de la province, furent consultés\,;
leur avis fut favorable à M. de Cîteaux, qui gagna son procès.

Le retour de M\textsuperscript{me} de Blansac à la cour, que M. de La
Rochefoucauld avait obtenu tout à la fin de l'année dernière, fut d'un
bon augure à une autre exilée. M\textsuperscript{me} de Saint-Géran, en
femme d'esprit, comme on l'a vu ici en son temps, n'avait point voulu
profiter de la liberté qui lui avait été laissée dans son éloignement de
la cour. Elle s'était retirée à Rouen dans le couvent de Bellefonds,
ainsi nommé des biens que la famille du maréchal de Bellefonds y a
faits, et du nombre de ses sœurs et de ses parentes qui y ont été
supérieures et religieuses. M\textsuperscript{me} de Saint-Géran avait
passé sa jeunesse chez le maréchal de Bellefonds et chez la vieille
Villars sa tante\,; ce fut la retraite qu'elle choisit, et d'où elle ne
sortit pas une seule fois. Elle avait beaucoup d'amis à la cour, qui
firent si bien valoir sa conduite, qu'elle fut rappelée, accueillie
comme en triomphe, et incontinent après logée au château, et de tout
mieux qu'auparavant, mais de sa part avec plus de précaution et de
sagesse.

Le comte d'Auvergne, qui n'était ni d'âge ni de figure à être amoureux,
l'avait été toute sa vie et l'était éperdument de M\textsuperscript{lle}
de Wassenaer, lorsque sa femme mourut. Il vint aussitôt après demander
permission au roi de l'épouser et de l'amener en France. La grâce était
singulière, pour ne rien dire de la bienséance si fort blessée dans
cette précipitation. M\textsuperscript{lle} de Wassenaer était
Hollandaise, d'une maison ancienne, chose rare en ce pays-là, et fort
distinguée parmi le peu de noblesse qui y est demeurée, par conséquent
calviniste. Il était donc contre tous les édits et déclarations du roi,
depuis la révocation de l'édit de Nantes et l'expulsion des huguenots,
d'en épouser une, et contre toutes les règles que le roi s'était
prescrites et qu'il avait exactement tenues, d'en souffrir la demeure en
France. Le roi avait passé sa vie à être amoureux, M\textsuperscript{me}
de Maintenon aussi. Le comte d'Auvergne les toucha par la similitude, et
leur dévotion, par l'espérance de gagner une âme à Dieu en procurant la
conversion de cette fille, ce qui ne se pouvait que par ce mariage. Il
obtint donc tout ce qu'il demanda, et s'en retourna au plus vite
l'épouser et la ramener en France. Elle parut à Paris et à la cour
mériter l'amour d'un plus jeune cavalier, et sa vertu, sa douceur, sa
conduite charmèrent encore plus que sa figure et le public et la famille
même du comte d'Auvergne, jusqu'à ses enfants, avec qui elle accommoda
leurs affaires et mit la paix entre eux. On verra bientôt qu'elle ne
tarda pas à se convertir, mais de la meilleure foi du monde, et après
s'être donné tout le temps et tout le soin d'être bien instruite et
pleinement convaincue.

Une ambassade du roi de Maroc, que Saint-Olon, envoyé du roi en ce
pays-là, en ramena, amusa tout Paris à aller voir ces Africains. C'était
un homme de bonne mine et de beaucoup d'esprit, à ce qu'on dit, que cet
ambassadeur. Le roi fut flatté de cette démarche d'un barbare, et le
reçut comme il est usité pour ces ambassadeurs non européens, turcs ou
moscovites, jusqu'au czar Pierre Torcy et Pontchartrain, qui furent ses
commissaires, crurent en être venus à bout lorsqu'il dédit et Saint-Olon
et l'interprète, et qu'il ne voulut plus de commerce avec eux,
prétendant qu'ils l'avaient engagé sans qu'il leur eût rien dit qui les
y pût conduire. Cela fit un assez étrange contraste, le jour même d'une
conférence à Versailles où il était venu avec eux de Paris, et ne voulut
jamais les remmener. Il déclara qu'il ne ferait point la paix, et on fut
longtemps à le ramener et à finir avec lui un traité.

Torcy entrait dans tout sous Pomponne son beau-père, qui lui facilitait
souvent de porter lui-même les dépêches au conseil. À force d'y entrer
de la sorte pour des moments, le roi, content de sa conduite, lui dit
enfin de s'asseoir et de demeurer. Cet instant le constitua ministre
d'État. Il est impossible que le secrétaire d'État des affaires
étrangères ne le soit, à moins d'être doublé par un père ou un
beau-père. Toute sa fonction consiste aux dépêches étrangères et aux
audiences qu'il donne aux ambassadeurs et autres ministres étrangers. Il
faut donc qu'il rapporte les affaires et les dépêches au conseil, et
dans ce conseil il n'entre que des ministres. Torcy avait entre
trente-quatre et trente-cinq ans alors\,; il avait voyagé et fort
utilement dans toutes les cours de l'Europe. Il était sage, instruit,
extrêmement mesuré\,; tout applaudit à cette grâce. Il est plaisant que
les plus petites charges aient toutes un serment, et que les ministres
d'État n'en prêtent point, qui sur tous autres y devraient être obligés.
C'est une de ces singularités dont on ne voit point de raison, puisque
ceux qui ont le plus de charges sur leur tête, dont ils ont prêté
serment de chacune, en prêtent encore un nouveau s'ils obtiennent une
nouvelle charge. En petit, les intendants des provinces qui eu sont
despotiquement les maîtres n'en prêtent point non plus, tandis que les
plus petits lieutenants de roi de province, inconnus dans leurs
provinces, où souvent ils n'ont jamais mis le pied, souvent encore aussi
peu connus partout ailleurs, et qui en toute leur vie n'ont pas la plus
légère fonction, prêtent tous serment et entre les mains du roi.

On vit en ce temps-ci, à six semaines ou deux mois de distance, deux
cruels effets du jeu. Reineville, lieutenant des gardes du corps,
officier général distingué à la guerre, fort bien traité du roi, et fort
estimé des capitaines des gardes, disparut tout d'un coup sans avoir pu
être trouvé nulle part, quelque soin qu'on prit à le chercher\,; c'était
un homme d'esprit qui avait un maintien de sagesse qui imposait. Il
aimait le jeu, il avait perdu ce qu'il ne pouvait payer\,; il était
homme d'honneur, il ne put soutenir son infortune. Douze ou quinze ans
après, il fut reconnu par hasard dans les troupes de Bavière, où il
était allé se jeter pour avoir du pain et vivre inconnu. Permillac fit
bien pis, car il se tua un matin dans son lit, d'un coup de pistolet
dans la tête, pour avoir perdu tout ce qu'il n'avait pas ni ne pouvait
avoir, ayant été gros et fidèle joueur toute sa vie. C'était un homme de
beaucoup d'esprit et jusque-là de sens, que ses talents et sa
distinction avaient avancé à la guerre\,; bien gentilhomme d'ailleurs,
et fort au gré de tous les généraux, ayant toujours eu la confiance du
général de l'armée, où il faisait supérieurement le détail de la
cavalerie, et toujours avec la meilleure compagnie de l'armée. Il
servait toujours sur le Rhin. Il avait pris de l'amitié pour moi et moi
pour lui. Tout le monde le plaignit, et je le regrettai fort.

\hypertarget{chapitre-xvii.}{%
\chapter{CHAPITRE XVII.}\label{chapitre-xvii.}}

1699

~

{\textsc{Condamnation à Rome du livre de l'archevêque de Cambrai.}}
{\textsc{- Conduite du cardinal de Bouillon.}} {\textsc{- Belle réponse
du duc de Beauvilliers au roi.}} {\textsc{- Soumission illustre de
l'archevêque de Cambrai.}} {\textsc{- Acceptation du jugement du pape
par les assemblées d'évêques par métropoles en jugeant.}} {\textsc{-
Enregistrement au parlement.}} {\textsc{- Procédé de l'archevêque de
Cambrai et de l'évêque de Saint-Omer à l'assemblée provinciale.}}
{\textsc{- Mort du comte de Mailly, de Thury, de Frontenac, de Racine\,;
sa funeste distraction.}} {\textsc{- Mort du duc de La Force.}}
{\textsc{- Valincour mis à l'histoire du roi en la place de Racine.}}
{\textsc{- Mort de l'évêque de Luçon, Barillon.}} {\textsc{- Mariage du
duc de Choiseul avec M\textsuperscript{me} Brûlart.}} {\textsc{- Mariage
du roi des Romains\,; pourquoi la part différée.}} {\textsc{- Style de
s'écrire entre l'empereur et le roi.}} {\textsc{- Traitements
d'ambassadeurs de tête couronnée à l'ambassadeur du grand-duc à Vienne,
nulle part ailleurs.}} {\textsc{- Naissance du prince de Piémont.}}
{\textsc{- Le roi paye les dettes de M\textsuperscript{me} la Duchesse
et de Monseigneur, et lui double ses mois.}} {\textsc{- Augmentation de
quarante-deux mille livres d'appointements à M. de La Rochefoucauld.}}
{\textsc{- Pension secrète de vingt mille livres à l'évêque de
Chartres.}} {\textsc{- M. de Vendôme change l'administration de ses
affaires, et va publiquement suer la vérole.}} {\textsc{- Mort de
Savary, assassiné.}} {\textsc{- Mort de l'abbé de La Châtre.}}
{\textsc{- Le roi fait revenir tous les prétendants de Neuchâtel.}}
{\textsc{- Deux vols au roi fort étranges.}} {\textsc{- Vaïni à la
cour.}} {\textsc{- Fériol ambassadeur à Constantinople.}} {\textsc{-
Situation du comte de Portland.}} {\textsc{- Courte disgrâce de la
comtesse de Grammont.}}

~

L'affaire de M. de Cambrai touchait à son terme et faisait plus de bruit
que jamais. Ce prélat faisait tous les jours quelque nouvel ouvrage pour
éclaircir et soutenir ses \emph{Maximes des saints}, et y mettait tout
l'esprit imaginable. Ses trois antagonistes y répondaient chacun à
part\,; l'amertume à la fin surnagea de part et d'autre, et, à
l'exception de M. de Paris qui se contint toujours dans une grande
modération, M. de Cambrai et MM. de Meaux et de Chartres se traitèrent
fort mal. Le roi pressait le jugement à Rome, où, fort mécontent de la
conduite du cardinal de Bouillon à cet égard, il crut hâter l'affaire en
donnant à M\textsuperscript{me} de Lévi le logement de M. de Cambrai à
Versailles, et défendant à ce prélat de plus prendre la qualité de
précepteur des enfants de France dont il lui avait déjà ôté les
appointements, et le fit dire au pape et à la congrégation établie pour
juger. En effet, le cardinal de Bouillon, lié comme on l'a vu ci-dessus
avec M. de Cambrai, ses principaux amis, et les jésuites, quoique chargé
des affaires du roi à Rome, et recevant ordres sur ordres de presser le
jugement et la condamnation de M. de Cambrai, mettait tout son crédit à
le différer et à éviter qu'il fût condamné. Il en reçut des reproches du
roi fort durs qui ne lui firent pas changer de conduite au fond, niais
qui lui firent chercher des excuses et des couleurs. Mais quand il vit
enfin qu'il n'y avait plus à reculer, il ne rougit point d'être
solliciteur et juge en même temps et de solliciter contre les ordres du
roi, directement contraires, en faveur de M. de Cambrai, pour qui
l'ambassadeur d'Espagne sollicitait aussi au nom du roi son maître. Ce
ne fut pas tout\,: le jour du jugement il ne se contenta pas d'opiner
pour M. de Cambrai de toute sa force, mais il essaya d'intimider les
consulteurs. Il interrompit les cardinaux de la congrégation, il
s'emporta, il cria, il en vint aux invectives, de manière que le pape,
instruit de cet étrange procédé et scandalisé à l'excès, ne put
s'empêcher de dire de lui\,: «\,\emph{È un porto ferito}, c'est un
sanglier blessé. » Il s'enferma chez lui à jeter feu et flammes, et ne
put même se contenir quand il fut obligé de reparaître. Le pape prononça
la condamnation, qui fut dressée en forme de constitution, et où la cour
de Rome, sûre de l'impatience du roi de la recevoir, inséra des termes
de son style que la France n'admet point. Le nonce qui la reçut par un
courrier la porta aussitôt au roi qui en témoigna publiquement sa joie.
Le nonce parla au roi entre son lever et la messe. C'était un dimanche
22 mars. Le roi revenant de la messe trouva M. de Beauvilliers dans son
cabinet pour le conseil qui allait se tenir. Des qu'il l'aperçut il fut
à lui, et lui dit\,: «\,Eh bien\,! monsieur de Beauvilliers, qu'en
direz-vous présentement\,? voilà M. de Cambrai condamné dans toutes les
formes. --- Sire, répondit le duc d'un ton respectueux mais néanmoins
élevé, j'ai été ami particulier de M. de Cambrai, et je le serai
toujours, mais s'il ne se soumet pas au pape, je n'aurai jamais de
commerce avec lui.\,» Le roi demeura muet, et les spectateurs en
admiration d'une générosité si ferme d'une part et d'une déclaration si
nette de l'autre, mais dont la soumission ne portait que sur l'Église.

Rome, à même de faire pis, montra par la condamnation même qu'elle était
plus donnée au roi qu'appesantie sur M. de Cambrai. Vingt-trois
propositions du livre des \emph{Maximes des saints} y furent qualifiées
téméraires, dangereuses, erronées, mais \emph{in globo}, et le pape
excommunie ceux qui le liront ou le garderont chez eux. Monsieur, qui
était venu de Paris dîner avec le roi, en sut la nouvelle en arrivant.
Le roi lui en parla pendant le dîner avec une satisfaction qui
s'épanchait, et encore à M. de La Rochefoucauld en allant au sermon, qui
répondit fort honnêtement sur M. de Cambrai, comme ne doutant pas qu'il
ne se soumit\,: c'était un personnage bon à faire à l'égard des gens
dans cette situation dont il n'avait jamais été ami.

M. de Cambrai apprit presque en même temps son sort, dans un moment qui
eût accablé un homme qui aurait eu en soi moins de ressources. Il allait
monter en chaire\,; il ne se troubla point\,; il laissa le sermon qu'il
avait préparé, et, sans différer un moment de prêcher, il prit son thème
sur la soumission due à l'Église\,; il traita cette matière d'une
manière forte et touchante, annonça la condamnation de son livre,
rétracta son opinion qu'il y avait exposée, et conclut son sermon par un
acquiescement et une soumission parfaite au jugement que le pape venait
de prononcer. Deux jours après il publia un mandement fort court, par
lequel il se rétracta, condamna son livre, en défendit la lecture,
acquiesça et se soumit de nouveau à sa condamnation, et par les termes
les plus concis, les plus nets, les plus forts s'ôta tous les moyens
d'en pouvoir revenir. Une soumission si prompte, si claire, si publique,
fut généralement admirée. Il ne laissa pas de se trouver des censeurs
qui auraient voulu qu'il eût comme copié la constitution, et qui se
firent moquer d'eux. M. de Meaux qui était à la cour reçut les
compliments de tout le monde qui courut chez lui en foule. M. de
Chartres était à Chartres, où il demeura, et M. de Paris montra une
grande modération. M\textsuperscript{me} de Maintenon parut au comble de
sa joie.

La difficulté fut après sur l'enregistrement au parlement, à cause de la
forme de cette bulle et des termes qui s'y trouvaient contraires aux
libertés de l'Église gallicane, libertés qui ne sont ni des nouveautés
ni des concessions ou des privilèges, mais un usage constant
d'attachement à l'ancienne discipline de l'Église, qui n'a point fléchi
aux usurpations de la cour de Rome, et qui ne l'a point laissée empiéter
comme elle a fait sur les Églises des autres nations. On prit donc un
expédient pour mettre tout à couvert sans trop de retardement\,; ce fut
une lettre du roi à tous les métropolitains de son royaume, par laquelle
il leur mandait d'assembler chacun ses suffragants pour prononcer sur la
condamnation que le pape venait de faire du livre des \emph{Maximes des
saints} de M. de Cambrai, de la constitution duquel il leur envoya en
même temps un exemplaire. L'obéissance fut d'autant plus prompte que
cette sorte d'assemblée par provinces ecclésiastiques sentait fort les
conciles provinciaux, quoique limitée à une matière, et que
l'interruption de ces sortes de conciles, dont les évêques avaient abusé
en y mêlant pour leur autorité force affaires temporelles, était un de
leurs plus grands regrets. Par ce tour nos évêques furent censés
examiner le livre et la censure, et n'adhérer au jugement du pape que
comme juges eux-mêmes de la doctrine, et jugeant avec lui. Ils en firent
des procès-verbaux qu'ils envoyèrent à la cour, et de cette manière il
n'y eut plus de difficulté, et le parlement enregistra la condamnation
de M. de Cambrai, en conséquence de l'adhésion des évêques de France en
forme de jugement.

M. de Cambrai subit ce dernier dégoût avec la même grandeur d'âme qu'il
avait reçu et adhéré à sa condamnation. Il assembla ses suffragants
comme les autres métropolitains, et y trouva de quoi illustrer sa
patience comme il avait illustré sa soumission. Valbelle, évêque de
Saint-Omer, Provençal ardent à la fortune, n'eut pas honte, comptant
plaire, d'ajouter douleur à la douleur. Il proposa dans l'assemblée
qu'il n'y suffisait pas de condamner le livre des \emph{Maximes des
saints}, si on n'y condamnait pas en même temps tous les ouvrages que M.
de Cambrai avait faits pour le soutenir. L'archevêque répondit
modestement qu'il adhérait de tout son cœur à la condamnation de son
livre des \emph{Maximes des saints}, et qu'il n'avait pas attendu, comme
on le savait, cette assemblée pour donner des marques publiques de son
entière soumission au jugement qui avait été rendu, mais qu'il croyait
aussi qu'il ne devait pas l'étendre à ce qui n'était point jugé\,; que
le pape était demeuré dans le silence sur tous les écrits faits pour
soutenir le livre condamné\,; qu'il croyait devoir se conformer
entièrement au jugement du pape en condamnant comme lui le livre qu'il
avait condamné, et demeurant comme lui dans le silence sur tous les
autres écrits à l'égard desquels il y était demeuré. Il n'y avait rien
de si sage, de si modéré, ni de plus conforme à la raison, à la justice
et à la vérité que cette réponse. Elle ne satisfit point M. de
Saint-Omer, qui voulait se distinguer et faire parler de lui. Il prit
feu, et insista par de longs et violents raisonnements que M. de Cambrai
écouta paisiblement sans rien dire. Quand le Provençal fut épuisé, M. de
Cambrai dit qu'il n'avait rien à ajouter à la première réponse qu'il
avait faite à la proposition de M. de Saint-Omer, ainsi que c'était aux
deux autres prélats à décider, à l'avis desquels il déclarait par avance
qu'il s'en rapporterait sans répliquer. MM. d'Arras et de Tournai se
hâtèrent d'opiner pour l'avis de M. de Cambrai, et imposèrent avec
indignation à M. de Saint-Omer, qui ne cessa de murmurer et de menacer
entre ses dents. Il se trouva fort loin de son compte. Le gros du monde
s'éleva contre lui\,; la cour même le blâma, et quand il y reparut, il
n'y trouva que de la froideur parmi ceux même qu'il regardait comme ses
amis, et qui ne l'étaient ni de M. de Cambrai ni des siens.

Il mourut en ce même temps un des hommes de la cour qui avait le nez le
plus tourné à une grande fortune\,; ce fut le comte de Mailly. Il était
fils du vieux Mailly et de la Bécasse, qu'on appelait ainsi à cause de
son long nez, qui était devenue seule héritière de la riche branche de
Montcavrel de la maison de Montchi, dont les Hocquincourt faisaient une
autre branche. Le père et la mère, quoique gens de qualité et de
beaucoup d'esprit tous deux, n'ont guère été connus que par le nombre de
procès qu'ils ont su gagner, la belle maison vis-à-vis le pont Royal
qu'ils ont bâtie, et les grands biens qu'ils ont amassés et acquis,
étant nés l'un et l'autre fort pauvres. Le maréchal de Nesle, leur aîné,
était mort maréchal de camp de ses blessures au siège de Philippsbourg,
en 1688, et n'avait laissé qu'un fils et une fille de la dernière de
l'illustre maison de Coligny, belle comme le jour, qu'il avait épousée
malgré père et mère, et le comte de Mailly, dont il s'agit ici, leur
quatrième fils. On a vu comme M\textsuperscript{me} de Maintenon en fit
le mariage avec M\textsuperscript{lle} de Saint-Hermine, fille d'un de
ses cousins germains, lorsque j'ai parlé du mariage de
M\textsuperscript{me} la duchesse de Chartres dont elle fut dame
d'atours, et ensuite de M\textsuperscript{me} la duchesse de Bourgogne.
Mailly était un homme bien fait, d'un visage agréable mais audacieux,
comme était son esprit et sa conduite. Il avait été élevé auprès de
Monseigneur et c'était celui pour qui ce prince avait témoigné et depuis
conservé la plus constante affection, et la plus marquée. C'était même à
qui l'aurait de son côté de M. le prince de Conti et de M. de Vendôme.
Beaucoup d'esprit, de grâces, un grand air du monde, de la valeur, une
ambition démesurée qui l'aurait mené bien loin, et à laquelle il aurait
tout sacrifié. Il avait trouvé le moyen à son âge de plaire au roi, et
M\textsuperscript{me} de Maintenon le regardait comme son véritable
neveu. Rien moins avec tout cela que bas avec personne\,; les ministres
et les généraux d'armée le comptaient. Mais pour ne s'y pas méprendre il
fallait s'attendre qu'il tournerait toujours à la faveur et à tout ce
qui pouvait le conduire. Il avait été de fort bonne heure menin de
Monseigneur, et mestre de camp général des dragons, qu'il vendit au duc
de Guiche dès qu'il fut maréchal de camp. Il avait neuf mille livres de
pension personnelle, et sa femme douze mille, outre leurs emplois. Il
était frère de l'archevêque d'Arles et de l'évêque de Lavaur. Nous
avions dîné chez M. le maréchal de Lorges à un grand repas qu'il donnait
à milord Jersey, parce que l'intérêt de milord Feversham, son frère, lui
faisait cultiver les ambassadeurs d'Angleterre.

Mailly est extrêmement de mes amis\,; après dîner, nous retournâmes
ensemble à Versailles. Mon carrosse rompit entre Sèvres et Chaville, à
ne pouvoir être raccommodé de longtemps\,; nous primes le parti
d'achever le voyage à pied, mais il lui prit une subite fantaisie de
retourner à Paris, quoi que je pusse faire pour l'en détourner. Il prit
par les bois de Meudon pour n'être point vu, et pour arriver dans le
quartier des Incurables, où logeait une créature qu'il entretenait\,;
moi, je gagnai Versailles par Montreuil, pour n'être pas aussi
rencontré. Je ne sais si cette traite à pied lui aigrit l'humeur de la
goutte qu'il avait quelquefois, mais dans la nuit il fut pris auprès de
sa demoiselle, si vivement et si subitement, par la gorge, qu'elle crut
qu'il allait étouffer. Il ne dura que deux fois vingt-quatre heures sans
avoir pu être transporté\,; sa femme y était accourue.
M\textsuperscript{me} de Maintenon, dès qu'elle la sut veuve, alla
elle-même à Paris la chercher, et la ramena dans son carrosse
extrêmement affligée. Elle eut pour ses enfants les neuf mille livres de
pension qu'avait son mari, et sur l'exemple de M\textsuperscript{me} de
Béthune, dame d'atours de la reine, elle servit au bout de ses six
semaines. Il fut peu regretté de la cour, et même dans le monde, mais la
perte fut grande pour sa maison.

Thury, frère cadet de M. de Beuvron, mourut aussi. On le voyait assez
souvent à la cour\,; c'était un homme fort appliqué à ses affaires\,; ni
lui ni son frère n'avaient guère servi. Il était resté un vieux conte
d'eux, du temps qu'ils étaient à l'armée. Ils se promenaient à la tête
du camp, il tomba une pluie assez douce après une grande sécheresse.
«\,Mon frère, s'écria l'un, que de foin\,! --- Mon frère, que
d'avoine\,!» répondit l'autre. On le leur a souvent reproché.

On eut nouvelles de la mort du comte de Frontenac à Québec, où il était,
pour la seconde fois, gouverneur général, depuis près de dix ans. Il
avait tellement gagné la confiance des sauvages, la première fois qu'il
eut cet emploi, qu'on fut obligé de le prier d'y retourner. Il y fit
toujours parfaitement bien, et ce fut une perte. Le frère de Callières
commandait sous lui, et lui succéda. M. de Frontenac s'appelait Buade\,;
son grand-père avait été gouverneur de Saint-Germain, premier maître
d'hôtel du roi, et chevalier de l'ordre en 1619. Celui-ci était fils
d'une Phélypeaux, nièce et fille de deux secrétaires d'État, et il était
frère de M\textsuperscript{me} de Saint-Luc, dont le mari était
chevalier de l'ordre et lieutenant général de Guyenne, fils du maréchal
de Saint-Luc, et père du dernier Saint-Luc, mari d'une Pompadour, sœur
de M\textsuperscript{me} d'Hautefort. C'était un homme de beaucoup
d'esprit, fort du monde, et parfaitement ruiné. Sa femme, qui n'était
rien, et dont le père s'appelait Lagrange-Trianon, avait été belle et
galante, extrêmement du grand monde, et du plus recherché. Elle et son
amie, M\textsuperscript{lle} d'Outrelaise, qui ont passé leur vie logées
ensemble à l'Arsenal, étaient des personnes dont il fallait avoir
l'approbation\,; on les appelait les Divines. J'en ai dit quelque chose
à propos du nom d'Orondat du vieux Villars. Un si aimable homme et une
femme si merveilleuse ne duraient pas aisément ensemble\,; ainsi, le
mari n'eut pas de peine à se résoudre d'aller vivre et mourir à Québec,
plutôt que mourir de faim ici, en mortel auprès d'une Divine.

Presque en même temps, on perdit le célèbre Racine, si connu par ses
belles pièces de théâtre. Personne n'avait plus de fonds d'esprit, ni
plus agréablement tourné\,; rien du poète dans son commerce, et tout de
l'honnête homme, de l'homme modeste, et sur la fin, de l'homme de bien.
Il avait les amis les plus illustres à la cour, aussi bien que parmi les
gens de lettres\,: c'est à eux à qui je laisse d'en parler mieux que je
ne pourrais faire. Il fit, pour l'amusement du roi et de
M\textsuperscript{me} de Maintenon, et pour exercer les demoiselles de
Saint-Cyr, deux chefs-d'œuvre en pièces de théâtre\,: \emph{Esther} et
\emph{Athalie}, d'autant plus difficiles qu'il n'y a point d'amour, et
que ce sont des tragédies saintes, où la vérité de l'histoire est
d'autant plus conservée que le respect dû à l'Écriture sainte n'y
pourrait souffrir d'altération. La comtesse d'Ayen et
M\textsuperscript{me} de Caylus sur toutes excellèrent à les jouer,
devant le roi et le triage le plus étroit et le plus privilégié chez
M\textsuperscript{me} de Maintenon. À Saint-Cyr, toute la cour y fut
plusieurs fois admise, mais avec choix. Racine fut chargé de l'histoire
du roi, conjointement avec Despréaux, son ami. Cet emploi, ces pièces,
dont je viens de parler, ses amis lui acquirent des privances. Il
arrivait même quelquefois que le roi n'avait point de ministres chez
M\textsuperscript{me} de Maintenon, comme les vendredis, surtout quand
le mauvais temps de l'hiver y rendait les séances fort longues\,; ils
envoyaient chercher Racine pour les amuser. Malheureusement pour lui, il
était sujet à des distractions fort grandes.

Il arriva qu'un soir qu'il était entre le roi et M\textsuperscript{me}
de Maintenon, chez elle, la conversation tomba sur les théâtres de
Paris. Après avoir épuisé l'opéra, on tomba sur la comédie. Le roi
s'informa des pièces et des acteurs, et demanda à Racine pourquoi, à ce
qu'il entendait dire, la comédie était si fort tombée de ce qu'il
l'avait vue autrefois, Racine lui en donna plusieurs raisons, et conclut
par celle qui, à son avis, y avait le plus de part, qui était que, faute
d'auteurs et de bonnes pièces nouvelles, les comédiens en donnaient
d'anciennes, et entre autres ces pièces de Scarron, qui ne valaient rien
et qui rebutaient tout le monde. À ce mot, la pauvre veuve rougit, non
pas de la réputation du cul-de-jatte attaquée, mais d'entendre prononcer
son nom, et devant le successeur. Le roi s'embarrassa, le silence qui se
fit tout d'un coup réveilla le malheureux Racine, qui sentit le puits
dans lequel sa funeste distraction le venait de précipiter. Il demeura
le plus confondu des trois, sans plus oser lever les yeux ni ouvrir la
bouche. Ce silence ne laissa pas de durer plus que quelques moments,
tant la surprise fut dure et profonde. La fin fut que le roi renvoya
Racine, disant qu'il allait travailler. Il sortit éperdu et gagna comme
il put la chambre de Cavoye. C'était son ami, il lui conta sa sottise.
Elle fut telle, qu'il n'y avait point à la pouvoir raccommoder. Oncques
depuis, le roi ni M\textsuperscript{me} de Maintenon ne parlèrent à
Racine, ni même le regardèrent. Il en conçut un si profond chagrin,
qu'il en tomba en langueur, et ne vécut pas deux ans depuis. Il les mit
bien à profit pour son salut. Il se fit enterrer à Port-Royal des
Champs, avec les illustres habitants duquel il avait eu des liaisons dès
sa jeunesse, que sa vie poétique avait même peu interrompues,
quoiqu'elle fût bien éloignée de leur approbation. Le chevalier de
Coislin s'y était fait porter aussi, auprès de son célèbre oncle, M. de
Pontchâteau. On ne saurait croire combien le roi fut piqué de ces deux
sépultures.

Le duc de La Force, qui mourut dans ce même temps, ne fit pas tant de
vide et de regrets, nonobstant sa naissance et sa dignité. C'était un
très bon et honnête homme, et rien de plus, qui à force d'exils, de
prisons, d'enlèvements de ses enfants, et de tous les tourments dont on
s'était pu aviser, s'était fait catholique. Le roi eut soin de le bien
faire assister, pour qu'il mourût tel. Sa femme, enfin, avait eu
permission de se retirer en Angleterre, et d'y jouir de son bien. Elle y
fut en estime et en considération, et y eut le rang de duchesse.

Peu après la mort de Racine, Valincour fut choisi pour travailler à
l'histoire du roi en sa place, avec Despréaux. Je ne sais quelle
connaissance il avait eue auprès de M\textsuperscript{me} de Montespan.
Ce fut par elle qu'il fut mis auprès de M. le comte de Toulouse, dès sa
première jeunesse reconnue, et bientôt après fut secrétaire général de
la marine. C'était un homme d'infiniment d'esprit, et qui savait
extraordinairement\,; d'ailleurs un répertoire d'anecdotes de cour où il
avait passé sa vie dans l'intrinsèque, et parmi la compagnie la plus
illustre et la plus choisie, solidement vertueux et modeste, toujours
dans sa place, et jamais gâté par les confiances les plus importantes et
les plus flatteuses. D'ailleurs très difficile à se montrer, hors avec
ses amis particuliers, et peu à peu très longtemps devenu grand homme de
bien. C'était un homme doux, gai, salé, sans vouloir l'être, et qui
répandait naturellement les grâces dans la conversation, très sûr et
extrêmement aimable, qui avait su conserver la confiance du roi, être
considéré de M\textsuperscript{me} de Maintenon, et ne lui être point
suspect, en demeurant publiquement attaché à M\textsuperscript{me} de
Montespan jusqu'à sa mort, et à tous les siens après elle. M. le comte
de Toulouse avait aussi toute confiance en lui, quoique parfaitement
brouillé avec M. d'O et sans nul commerce ensemble. On ne l'en estimait
pas moins, quoique lui-même estimât fort peu ce gouverneur de la
personne et de la maison de son maître.

Un saint et savant évêque finit aussi ses jours, Barillon, évêque de
Luçon, frère de Barillon, longtemps ambassadeur en Angleterre, et de
Morangis, tous deux conseillers d'État. C'était un homme qui ne sortait
presque jamais de son diocèse, où il menait une vie tout à fait
apostolique. Il était fort estimé et dans la première considération dans
le monde et parmi ses confrères, ami intime de M. de la Trappe, et ami
aussi de mon père, ainsi que ses frères. Il vint trop tard à Paris se
faire tailler, et en mourut de la manière la plus sainte, la plus
édifiante, et qui répondit le mieux à toute sa vie.

Le duc de Choiseul, las de sa misère, épousa une sœur de l'ancien évêque
de Troyes, et de la maréchale de Clérembault, fille de Chavigny,
secrétaire d'État. Elle était veuve de Brûlart, premier président du
parlement de Dijon, et fort riche, dont elle n'avait qu'une fille.
Quoique vieille, elle voulut tâter de la cour et du tabouret\,; elle en
trouva un à acheter, et le prit.

Malgré la paix, l'empereur gardait peu de bienséances\,: il fut plus de
trois mois sans donner part au roi du mariage du roi des
Romains\footnote{On donnait le titre de \emph{roi des Romains} au prince
  désigné par les électeurs pour succéder au trône impérial. Ce prince
  était alors Joseph, né eu 1676, roi des Romains depuis 1690, empereur
  en 1705, mort en 1711.}, son fils, avec la seconde fille de la
duchesse d'Hanovre qui avait été ici longtemps, et que j'ai rapporté
ci-dessus en être sortie de dépit de son aventure avec
M\textsuperscript{me} de Bouillon. Elles étaient à Modène, où l'aînée
avait épousé le duc de Modène qui avait quitté le chapeau qu'il avait
porté longtemps, pour succéder à son frère, mort sans enfants, et se
marier. Le prince de Salm, grand maître de la maison du roi des Romains,
dont il avait été gouverneur, et en grand crédit auprès de lui et dans
la cour de l'empereur, fit ce mariage. Il était veuf de la sœur de
M\textsuperscript{me} la Princesse et de la duchesse d'Hanovre, et
compta avec raison faire un grand coup pour lui que de faire sa nièce
reine des Romains. L'empereur les fit venir de Modène, et fit célébrer
ce mariage dans le mois de janvier. Ce fut par un simple courrier qu'il
en donna enfin part, chargé d'une lettre en italien de sa main pour le
roi. Il n'avait aucun ministre ici. La morgue impériale est telle
qu'elle refuse encore la \emph{Majesté} au roi, dans les lettres qu'on
appelle de chancellerie, c'est-à-dire qui commencent par les titres
\emph{très haut}, etc., et sont contresignées. La morgue française n'en
veut point recevoir sans \emph{Majesté}, de sorte que ces sortes de
lettres sont bannies entre eux, et qu'ils s'écrivent toujours l'un à
l'autre de leur main, avec la \emph{Majesté} réciproque, et une égalité
en tout parfaite. Le roi y avait Villars, avec caractère d'envoyé. La
préséance de la France sur l'Espagne ne permettait pas d'avoir un
ambassadeur à Vienne, que cette cour eût fait précéder tant qu'elle
aurait pu par celui d'Espagne, pour la dignité de la maison d'Autriche.

Villars avait reçu une incivilité très forte dans l'appartement de
l'empereur, du prince de Lichtenstein, sur ce qu'il ne voyait point
l'archiduc, que les ministres du roi ne visitaient point, à cause de
quelque embarras de cérémonial. Villars prétendit une réparation
authentique, se retira de la cour, et dépêcha un courrier. Le nonce de
Vienne et les autres ministres étrangers s'en mêlèrent inutilement\,;
Villars eut ordre de s'en revenir sans prendre congé, si la réparation
ne lui était point faite. On la différa tant, que Villars résolut de
partir\,: sur le point qu'il allait monter en voiture, on le pria de
rester, et on l'assura de la satisfaction, et en effet, deux jours
après, elle fut achevée d'être concertée, et sur-le-champ exécutée par
les excuses que le prince de Lichtenstein lui alla faire chez lui. Ce
fut apparemment ce petit démêlé qui retarda tant la part du mariage du
roi des Romains, car l'un et l'autre se fit tout de suite, et fort peu
après l'embarras du cérémonial chez l'archiduc, à la satisfaction du
roi, qui ordonna à Villars d'aller chez lui. L'empereur donna une grande
distinction au grand-duc\,; ce fut le traitement d'ambassadeurs de tête
couronnée aux siens qui ne l'avaient dans aucune cour. M. de Savoie fut
outré de cette égalité avec lui\,; je ne sais si ce fut pour le
mortifier, ou pour l'argent de Florence. La naissance d'un prince de
Piémont l'en consola bientôt après\,: dans le moment, un lieutenant des
gardes partit pour en porter la nouvelle à Monsieur. Dès qu'il fut
arrivé, l'ambassadeur de Savoie le vint dire au roi, et entra dans son
cabinet, où il était enfermé en attendant le marquis de Rover, que M. de
Savoie {[}avait{]} envoyé exprès au roi qui reçut très bien
l'ambassadeur, et l'envoya à Saint-Cyr trouver M\textsuperscript{me} la
duchesse de Bourgogne qui y était. Elle fut si sensible à cette
nouvelle, qu'elle en pleura de joie.

Le roi, qui venait de payer les dettes de M\textsuperscript{me} la
Duchesse qui étaient fortes du jeu, et aux marchands, paya aussi celles
de Monseigneur, qui allaient à cinquante mille livres, se chargea de
payer ses bâtiments de Meudon, et au lieu de quinze cents pistoles qu'il
avait par mois, le mit à cinquante mille écus. Pontchartrain, en habile
homme, fit sa cour de cette affaire-là à ce prince à qui il en porta la
nouvelle, sans qu'il eût rien demandé ni parlé au contrôleur général,
lequel s'acquit par là Monseigneur pour toujours. Il avait toujours eu
grand soin d'aller au-devant de tout ce qui pouvait lui plaire, et il
combla par ce présent un fils accoutumé à trembler devant son père, et
que le père n'avait pas envie d'en désaccoutumer. M. de La
Rochefoucauld, toujours nécessiteux et piteux, au milieu des richesses
et en proie à ses valets, obtint, sa vie durant seulement, quarante-deux
mille livres de rente d'augmentation d'appointements sur sa charge de
grand veneur, quoiqu'on ait vu, il n'y a pas longtemps, ici, que le roi
lui avait payé ses dettes. Il donna aussi, mais avec un grand secret, et
qui a toujours duré, vingt mille livres de pension à M. de Chartres. Ses
voyages et ses ouvrages lui dépensaient beaucoup\,; il craignit de n'y
pouvoir suffire et de laisser des dettes qui ne se pourraient payer\,;
il demanda une abbaye. Il tenait par la confiance du mariage du roi,
dont le roi avait trouvé bon que M\textsuperscript{me} de Maintenon lui
fit la confidence, et il était sur le pied de leur en parler et de leur
en écrire à l'un et à l'autre. Le roi ne voulut point lui donner
d'abbaye\,; il en avait déjà une, il trouva que cela ferait un contraste
désagréable avec M. de Cambrai qui avait rendu la sienne lorsqu'il fut
archevêque\,; et pour éviter le qu'en-dira-t-on, au lieu d'abbaye, il
lui fit cette pension, qui lui était payée par mois.

M. de Vendôme songea aussi enfin à ses affaires et à sa santé. Il était
extrêmement riche et n'avait jamais un écu pour quoi qu'il voulût faire.
Le grand prieur son frère s'était emparé de sa confiance avec un abbé de
Chaulieu, homme de fort peu, mais de beaucoup d'esprit, de quelques
lettres, et de force audace, qui l'avait introduit dans le monde sous
l'ombre de MM. de Vendôme, des parties desquels il s'ennoblissait. On
avait souvent et inutilement parlé à M. de Vendôme sur le misérable état
où sa confiance le réduisait. Le roi lui en avait dit son avis, et
l'avait pressé de penser à sa santé que ses débauches avaient mise en
fort mauvais état. À la fin il en profita, il pria Chemerault qui lui
était fort attaché de dire au grand prieur de sa part qu'il le priait de
ne se plus mêler de ses affaires, et à l'abbé de Chaulieu de cesser d'en
prendre soin. Ce fut un compliment amer au grand prieur qui faisait
siens les revenus de son frère, et en donnait quelque chose à l'abbé de
Chaulieu. Jamais il ne le pardonna sincèrement à son frère, et ce fut
l'époque, quoique sourde, de la cessation de leur identité, car leur
union se pouvait appeler telle. L'abbé de Chaulieu eut une pension de
six mille livres de M. de Vendôme, et eut la misère de la recevoir.
Crosat, un des plus riches hommes de Paris, à toutes sortes de métiers,
se mit à la tête des affaires de M. de Vendôme, après quoi il prit
publiquement congé du roi, de Monseigneur, des princesses et de tout le
monde, pour s'en aller se mettre entre les mains des chirurgiens qui
l'avaient déjà manqué une fois. C'est le premier et unique exemple d'une
impudence pareille. Ce fut aussi l'époque qui lui fit perdre terre. Le
roi lui dit qu'il était ravi qu'il eût enfin pourvu à ses affaires, et
qu'il eût pris le parti de pourvoir aussi à sa santé, et qu'il
souhaitait que ce fût avec un tel succès qu'on le pût embrasser au
retour en sûreté. Il est vrai qu'une race de bâtards pouvait en ce
genre-là prétendre quelque privilège, mais d'aller en triomphe où jamais
on ne fut qu'en cachant sa honte sous les replis les plus mystérieux,
épouvanta et indigna tout à la fois, et montra tout ce que pouvait une
naissance illégitime sur un roi si dévot, si sérieux et en tout genre si
esclave de toutes les bienséances. Au lieu d'Anet il fut à Clichy chez
Crosat pour être plus à portée de tous les secours de Paris. Il fut près
de trois mois entre les mains des plus habiles, qui y échouèrent. Il
revint à la cour avec la moitié de son nez ordinaire, ses dents tombées
et une physionomie entièrement changée, et qui tirait sur le niais\,; le
roi en fut si frappé qu'il recommanda aux courtisans de n'en pas faire
semblant de peur d'affliger M. de Vendôme. C'était assurément y prendre
un grand intérêt. Comme il était parti pour cette expédition médicale en
triomphe, il en revint aussi triomphant par la réception du roi, dont
l'exemple gagna toute la cour. Cela et le grand remède qui lui avait
affaibli la tête la lui tourna tout à fait, et depuis cette époque ce ne
fut plus le même homme. Le miroir cependant ne le contentait pas, il ne
parut que quelques jours et s'en alla à Anet voir si le nez et les dents
lui reviendraient avec les cheveux.

Deux aventures étranges effrayèrent et firent faire bien des
réflexions\,: Savary fut trouvé assassiné dans sa maison à Paris. Il
n'avait qu'un valet et une servante qui furent trouvés en même temps
assassinés, tous trois tout habillés et en différents endroits de la
maison, sans la moindre chose volée. L'apparence fut que ce crime fut
commis de jour, et que ce fut une vengeance par des écrits qui se sont
trouvés chez lui. C'était un bourgeois de Paris dont le frère venait de
mourir évêque de Séez, à qui Daquin succéda. Il était à son aise, bien
nippé, sans emploi, et vivait en épicurien. Il avait beaucoup d'amis, et
quelques-uns de la plus haute volée. Il recevait chez lui des parties de
toute espèce de plaisirs, mais choisies et resserrées, et la politique
n'en était pas bannie quand on en voulait traiter. On n'a jamais su la
cause de cet assassinat, mais on en trouva assez pour n'oser
approfondir, et l'affaire en demeura là. On ne douta guère qu'un très
vilain petit homme ne l'eût fat faire, mais d'un sang si supérieurement
respecté, que toute formalité tomba dans la frayeur de le trouver au
bout, et qu'après le premier bruit tout le monde cessa d'oser parler de
cette tragique histoire.

L'autre aventure n'imposa aucun silence. On a vu ci-devant celle de
l'abbé de Caudelet sur l'évêché de Poitiers, et que l'abbé de La Châtre
demeura convaincu d'être l'auteur de la calomnie, et de l'avoir fait
réussir. Allant de Saint-Léger à Pontchartrain avec Garsault, qui avait
le haras du roi à Saint-Léger, la calèche légère et découverte dans
laquelle ils étaient tous deux fut emportée par les chevaux. La frayeur
les fit jeter dehors, l'abbé se brisa contre des pierres, et les roues
lui passèrent sur le corps. Il vécut encore vingt-quatre heures, et
mourut sans avoir eu un instant de connaissance. Garsault, extrêmement
blessé, recouvra la sienne pour en faire un bon usage pendant deux mois
qu'il fut en proie aux chirurgiens, au bout desquels il mourut aussi.

L'affaire de M. le prince de Conti allait mal à Neuchâtel, où il était
logé dans la ville sans aucune considération. Les ducs de Lesdiguières
et de Villeroy y logeaient de même. M\textsuperscript{me} de Nemours
était dans le château dans toute la splendeur de souveraine reconnue, et
toute l'autorité dont elle faisait sentir l'éclat et le poids à un
Bourbon avec toute la volupté du dépit et de la vengeance. Le canton de
Berne avait voulu lui prêter main-forte comme allié de Neuchâtel, et
Puysieux, ambassadeur en Suisse, n'avait pu en arrêter de fortes
démonstrations. Le roi sentit toute l'indécence du séjour du prince de
Conti en un lieu si éloigné des moindres égards pour lui. Il lui fit
donc mander de revenir, et il donna le même ordre aux ducs de
Lesdiguières et de Villeroy, à Matignon et à M\textsuperscript{me} de
Nemours elle-même, qui se fit un peu tirer l'oreille pour obéir. Elle en
fit des plaintes amères à MM. de Neuchâtel et aux Suisses, qui ne s'en
unirent que plus fortement à elle, et s'en aliénèrent de plus en plus
des intérêts de M. le prince de Conti. Il arriva à Paris et les autres
prétendants longtemps devant elle. Elle fit allant et revenant tout ce
grand voyage dans sa chaise à porteurs, avec force carrosses et grands
équipages, et un chariot derrière elle rempli de seize porteurs pour en
relayer. Il y avait en cette voiture plus d'air de singularité et de
grandeur que de raison d'âge ou d'incommodité. Elle allait de même de
Paris à Versailles, et ses officiers lui donnaient à dîner à Sèvres. Le
roi qui craignait la force de sa part la reçut honnêtement, et l'assura
toujours qu'il ne prendrait point de parti entre ses sujets, et dans la
vérité il ne fit rien dans tout le cours de cette affaire en faveur de
M. le prince de Conti que ce qu'il ne put éviter par pure bienséance.
L'acquisition de Neuchâtel ne l'éloignait pas de France pour toujours
comme la couronne de Pologne, aussi en eut-il bien plus d'envie, et le
roi infiniment moins.

On lui fit à la grande écurie, à Versailles, un vol bien hardi la nuit
du 3 au 4 juin. Le roi étant à Versailles, toutes les housses et les
caparaçons furent emportés\,; il y en eut pour plus de cinquante mille
écus\,; les mesures furent si bien prises que qui que ce soit ne s'en
aperçut dans une maison si habitée, et que dans une nuit si courte tout
fut emporté sans que jamais on ait pu en avoir de nouvelles. M. le Grand
entra en furie et tous ses subalternes aussi. On dépêcha sur tous les
chemins, on fouilla Paris et Versailles, et le tout inutilement. Cela me
fait souvenir d'un autre vol qui eut quelque chose de bien plus étrange,
et qui arriva fort peu avant la date du commencement de ces Mémoires. Le
grand appartement, c'est-à-dire depuis la galerie jusqu'à la tribune,
était meublé de velours cramoisi avec des crépines et des franges d'or.
Un beau matin elles se trouvèrent toutes coupées. Cela parut un prodige
dans un lieu si passant tout le jour, si fermé la nuit et si gardé à
toutes heures. Bontems, au désespoir, fit et fit faire toutes les
perquisitions qu'il put, et toutes sans aucun succès. Cinq ou six jours
après, j'étais au souper du roi, il n'y avait que Daquin, premier
médecin du roi, entre le roi et moi, et personne entre moi et la table.
Vers l'entremets, j'aperçus je ne sais quoi de fort gros et comme noir
en l'air sur la table, que je n'eus le temps de discerner ni de montrer
par la rapidité dont ce gros tomba sur le bout de la table, devant
l'endroit du couvert de Monsieur et de Madame qui étaient à Paris, et
qui se mettaient toujours au bout de la table à la gauche du roi, le dos
aux fenêtres qui donnent sur la grande cour. Le bruit que cela fit en
tombant, et la pesanteur de la chose pensa l'enfoncer, et fit bondir les
plats, mais sans en renverser aucun, et de hasard cela tomba sur la
nappe et point dans des plats. Le roi, au coup que cela fit, tourna la
tête à demi, et sans s'émouvoir en aucune sorte\,: «\,Je pense, dit-il,
que ce sont mes franges. » C'en était en effet un paquet plus large
qu'un chapeau de prêtre avec ses bords tout plats, et haut en manière de
pyramide mal faite d'environ deux pieds. Cela était parti de loin
derrière moi vers la porte mitoyenne des deux antichambres, et un
frangeon détaché en l'air était tombé sur le haut de la perruque du roi,
que Livry qui était à sa gauche aperçut et ôta. Il s'approcha du bout de
la table, et vit en effet que c'étaient les franges tortillées en
paquet, et tout le monde les vit comme lui. Cela fit un moment de
murmure. Livry, voulant ôter ce paquet, y trouva un billet attaché\,; il
le prit et laissa le paquet. Le roi tendit la main et dit\,:
«\,Voyons.\,» Livry avec raison ne voulut pas\,; et, se retirant en
arrière, le lut tout bas, et par derrière le roi le donna à Daquin avec
qui je le lus entre ses mains. Il y avait dedans, d'une écriture
contrefaite et longue comme de femme, ces propres mots\,: «\,Reprends
tes franges, Bontems, la peine en passe le plaisir, mes baisemains au
roi.\,» Il était roulé et point fermé\,; le roi le voulut encore prendre
des mains de Daquin qui se recula, le sentit, le frotta, tourna et
retourna, puis le montra au roi sans le lui laisser toucher. Le roi lui
dit de le lire tout haut quoique lui-même le lût en même temps.
«\,Voilà, dit le roi, qui est bien insolent\,;» mais d'un ton tout uni
et comme historique. Il dit après qu'on ôtât ce paquet. Livry le trouva
si pesant qu'à peine le put-il lever de dessus la table, et le donna à
un garçon bleu qui vint se présenter. De ce moment le roi n'en parla
plus, et personne n'osa plus en rien dire, au moins tout haut, et le
reste du souper se passa tout comme chose non avenue.

Outre l'excès de l'impudence et de l'insolence, c'est un excès de péril
qui ne se peut comprendre. Comment lancer de si loin un paquet de cette
pesanteur et de ce volume, sans être environné de complices, et au
milieu d'une foule telle qu'elle était toujours au souper du roi, où à
peine pouvait-on passer dans ces derrières\,? Comment, malgré ce cercle
de complices, le grand mouvement des bras pour une vibration aussi forte
put-il échapper à tant d'yeux\,? Le duc de Gesvres était en année. Ni
lui ni personne ne s'avisa de faire fermer les portes, que du temps
après que le roi fut sorti de table. On peut juger si les coupables
étaient demeurés là, ayant eu plus de trois quarts d'heure toutes les
issues libres pour se retirer. Les portes fermées il ne se trouva qu'un
seul homme que personne ne connut et qu'on arrêta. Il se dit gentilhomme
de Saintonge, et connu du duc d'Uzès, gouverneur de la province. Il
était à Versailles, on l'envoya prier de venir. Il allait se coucher. Il
vint aussitôt, reconnut ce gentilhomme, en répondit, et sur ce
témoignage on le laissa avec des excuses. Jamais depuis on n'a pu rien
découvrir de ce vol, ni de la singulière hardiesse de sa restitution.

Vaïni, qui avec la permission du roi s'était paré du cordon bleu à Rome,
vint le recevoir de sa main le jour de la Pentecôte. Il fut fort bien
reçu. Après le lui avoir donné, le roi ne voulut pas avoir l'air du
repentir. Les courtisans, qui aiment la nouveauté, et les amis surtout
du cardinal de Bouillon qui le leur avait fort recommandé, lui firent
beaucoup d'accueil. Il fut toujours en bonne compagnie. Il passa trois
mois à la cour ou à Paris. Le roi lui fit présent lui-même d'une belle
croix de diamants, et l'avertit de prendre garde qu'on ne la lui coupât.
Il s'en retourna en Italie, charmé de tout ce qu'il avait vu, et de la
bonne réception qu'il avait reçue. En ce même temps, Fériol s'en alla
relever Châteauneuf, notre ambassadeur à Constantinople, en la même
qualité.

On sut que Portland n'avait pas eu tort à Paris d'être en peine de la
faveur naissante de Keppel, Hollandais comme lui, mais jeune, hardi et
bien fait, que le roi d'Angleterre avait fait comte d'Albemarle. La
jalousie éclata à son retour, et la froideur se mit entre lui et son
maître. Il remit toutes, ses charges et ses emplois, passa en Hollande,
et dit au roi d'Angleterre que ce serait en ce pays-là où il se
réservait à lui faire sa cour. Peu après le roi d'Angleterre y passa
aussi, comme il faisait toutes les années. Il s'y rapprocha de son
ancien favori, le remmena avec lui en Angleterre, où il continua d'être
chargé comme devant des principales affaires. Mais il ne reprit point
ses charges ni sa faveur première, et Albemarle demeura affermi dans la
sienne.

Le roi, qui passait toujours à Versailles l'octave du Saint-Sacrement, à
cause des deux processions et des saluts, allait aussi toujours à Marly
après le salut de l'octave. Il découvrit cette année que la comtesse de
Grammont avait été passer quelques jours de cette octave à Port-Royal
des Champs, où elle avait été élevée, et pour lequel elle avait conservé
beaucoup d'attachement. C'était un crime qui pour tout autre aurait été
irrémissible\,; mais le roi avait personnellement pour elle une vraie
considération et une amitié qui déplaisait fort à M\textsuperscript{me}
de Maintenon, mais qu'elle n'avait jamais pu rompre, et qu'elle
souffrait parce qu'elle ne pouvait faire autrement. Elle ne laissait pas
de lui montrer souvent sa jalousie par des traits d'humeur quoique
mesurés, et la comtesse qui était fort haute, et en avait tout l'air et
le maintien avec une grande mine, des restes de beauté, et plus d'esprit
et de grâce qu'aucune femme de la cour, ne se donnait pas la peine de
les ramasser, et montrait de son côté à M\textsuperscript{me} de
Maintenon, par son peu d'empressement pour elle, qu'elle ne lui rendait
le peu qu'elle faisait que par respect pour le goût du roi. Ce voyage
donc que M\textsuperscript{me} de Maintenon tâcha de mettre à profit ne
mit la comtesse qu'en pénitence, non en disgrâce. Elle qui était
toujours de tous les voyages de Marly, et partout où le roi allait, n'en
fut point celui-ci. Ce fut une nouvelle. Elle en rit tout bas avec ses
amis. Mais d'ailleurs elle garda le silence et s'en alla à Paris. Deux
jours après elle écrivit au roi par son mari qui avait liberté d'aller à
Marly, mais elle n'écrivit ni ne fit rien dire à M\textsuperscript{me}
de Maintenon. Le roi dit au comte de Grammont qui cherchait à justifier
sa femme, qu'elle n'avait pu ignorer ce qu'il pensait d'une maison toute
janséniste qui est une secte qu'il avait en horreur. Fort peu après le
retour à Versailles, la comtesse de Grammont y arriva, et vit le roi en
particulier chez M\textsuperscript{me} de Maintenon. Il la gronda, elle
promit qu'elle n'irait plus à Port-Royal, sans toutefois l'abjurer le
moins du monde\,; ils se raccommodèrent, et au grand déplaisir de
M\textsuperscript{me} de Maintenon, il n'y parut plus.

\hypertarget{chapitre-xviii.}{%
\chapter{CHAPITRE XVIII.}\label{chapitre-xviii.}}

1699

~

{\textsc{Pensées et desseins des amis de M. de Cambrai.}} {\textsc{- Duc
de Beauvilliers prend à la grande direction la place du chancelier
absent.}} {\textsc{- Naissance de mon second fils.}} {\textsc{- Voyage
très singulier d'un maréchal de Salon en Provence, à la cour.}}
{\textsc{- Le roi partial pour M. de Bouillon contre M. d'Albret.}}
{\textsc{- Mort de Saint-Vallier, du duc de Montbazon, de Mirepoix, de
la duchesse Mazarin, de M\textsuperscript{me} de Nevet, de la reine de
Portugal.}} {\textsc{- Séance distinguée de M. du Maine en la chambre
des comptes.}} {\textsc{- Filles d'honneur de la princesse de Conti
douairière mangent avec M\textsuperscript{me} la duchesse de
Bourgogne.}} {\textsc{- Dédicace de la statue du roi à la place de
Vendôme.}} {\textsc{- Cause du retardement de l'audience de
Zinzendorf.}} {\textsc{- Le roi ne traite le roi de Danemark que de
Sérénité, et en reçoit la Majesté.}} {\textsc{- Mort de la duchesse
douairière de Modène.}} {\textsc{- Fortune et mort du chancelier
Boucherat.}} {\textsc{- Candidats pour les sceaux\,: Harlay, premier
président\,; Courtin, doyen du conseil\,; d'Aguesseau, Pomereu, La
Reynie, Caumartin, Voysin, Pelletier-Sousy.}} {\textsc{- Fortune de
Pontchartrain fait chancelier.}}

~

Les amis de M. de Cambrai s'étaient flattés que le pape, charmé d'une
soumission si prompte et si entière, et qui avait témoigné plus de
déférence pour le roi que tout autre sentiment dans le jugement qu'il
avait rendu, le récompenserait de la pourpre\,; et en effet il y eut des
manèges qui tendaient là. Ils prétendent encore que le pape en avait
envie, mais qu'il n'osa jamais voyant que, depuis cette soumission, sa
disgrâce n'était en rien adoucie. Le duc de Béthune, qui venait toutes
les semaines à Versailles, y dînait assez souvent chez moi, et ne
pouvait avec nous s'empêcher de parler de M. de Cambrai\,: il savait
qu'il y était en sûreté, et outre cela mon intimité avec M. de
Beauvilliers. Cette espérance du cardinalat perdue, il se lâcha un jour
chez moi jusqu'à nous dire qu'il avait toujours cru le pape
infaillible\,; qu'il en avait souvent disputé avec la comtesse de
Grammont, mais qu'il avouait qu'il ne le croyait plus depuis la
condamnation de M. de Cambrai. Il ajouta qu'on savait bien que ç'avait
été une affaire de cabale ici et de politique à Rome, mais que les temps
changeaient, et qu'il espérait bien que ce jugement changerait aussi et
serait rétracté, et qu'il y avait de bons moyens pour cela. Nous nous
mimes à rire, et à lui dire que c'était toujours beaucoup que ce
jugement l'eût fait revenir de l'erreur de l'infaillibilité des papes,
et que l'intérêt qu'il prenait en l'affaire de M. de Cambrai eût été
plus puissant à lui dessiller les yeux, que la créance de tous les
siècles et tant et tant de puissantes raisons qui détruisaient ce nouvel
et dangereux effet de l'orgueil et de l'ambition romaine, et de
l'intérêt de ceux qui le soutenaient jusqu'à en vouloir faire un
pernicieux dogme.

Parlant des amis de M. de Cambrai, cela me fait souvenir de réparer ici,
quoiqu'en matière fort différente, un oubli que j'ai fait d'une chose
qui se passa au dernier voyage de Fontainebleau. La petite direction se
tient toujours chez le chef du conseil des finances qui y préside, et la
grande direction dans la salle du conseil des parties\footnote{Voy. t.
  Ier, fin du volume, note sur les conseils du roi, et entre autres sur
  les conseils de grande et petite direction.}\,: le chancelier y
préside, et lorsque étant absent, et qu'il y a eu un garde des sceaux,
il y a présidé de sa place, et a toujours laissé vide celle du
chancelier. Il faut comprendre quand il n'est pas exilé, au moins à ce
que j e pense, parce qu'alors il fait partout ses fonctions, et prend
même au parlement la place que le chancelier y tient. En ce voyage de
Fontainebleau, où le chancelier malade n'alla point, M. de Beauvilliers
prit sa place à la grande direction\,: il y avait présidé d'autres fois
en l'absence du chancelier, sans prendre sa place, et l'avait laissée
vide. Le roi le sut, et dit qu'étant duc et pair et président à la
grande direction par l'absence du chancelier, il devait prendre sa place
et ne la plus laisser vide. Cela fut ainsi exécuté depuis, et fort
souvent encore après à Versailles, par les infirmités de M. le
chancelier.

Le 12 août, M\textsuperscript{me} de Saint-Simon accoucha fort
heureusement, et Dieu nous fit la grâce de nous donner un second fils,
qui porta le nom de marquis de Ruffec, belle terre en Angoumois que ma
mère avait achetée de la sienne.

Un événement singulier fit beaucoup raisonner tout le monde. Il arriva
en ce temps-ci tout droit à Versailles un maréchal de la petite ville de
Salon en Provence, qui s'adressa à Brissac, major des gardes du corps,
pour être conduit au roi à qui il voulait parler en particulier. Il ne
se rebuta point des rebuffades qu'il en reçut, et fit tant que le roi en
fut informé et lui fit dire qu'il ne parlait pas ainsi à tout le monde.
Le maréchal insista, dit que, s'il voyait le roi, il lui dirait des
choses si secrètes et tellement connues à lui seul, qu'il verrait bien
qu'il avait mission pour lui parler et pour lui dire des choses
importantes\,; qu'en attendant au moins, il demandait à être renvoyé à
un de ses ministres d'État. Là-dessus le roi lui fit dire d'aller
trouver Barbezieux, à qui il avait donné ordre de l'entendre. Ce qui
surprit beaucoup, c'est que ce maréchal qui ne faisait qu'arriver, et
qui n'était jamais sorti de son lieu ni de son métier, ne voulut point
de Barbezieux, et répondit tout de suite qu'il avait demandé à être
renvoyé à un ministre d'État, que Barbezieux ne l'était point, et qu'il
ne parlerait point qu'à un ministre. Sur cela le roi nomma Pomponne, et
le maréchal, sans faire ni difficulté ni réponse, l'alla trouver\,; ce
qu'on sut de son histoire est fort court\,: le voici. Cet homme revenant
tard de dehors se trouva investi d'une grande lumière auprès d'un arbre,
assez près de Salon. Une personne vêtue de blanc, et pardessus à la
royale, belle, blonde et fort éclatante, l'appela par son nom, lui dit
de la bien écouter, lui parla plus d'une demi-heure, lui dit qu'elle
était la reine qui avait été l'épouse du roi, lui ordonna de l'aller
trouver et de lui dire les choses qu'elle lui avait communiquées, que
Dieu l'aiderait dans tout son voyage, et qu'à une chose secrète qu'il
dirait au roi, et que le roi seul au monde savait, et qui ne pouvait
être sue que de lui, il reconnaîtrait la vérité de tout ce qu'il avait à
lui apprendre. Que si d'abord il ne pouvait parler au roi, qu'il
demandât à parler à un de ses ministres d'État, et que surtout il ne
communiquât rien à autres quels qu'ils fussent, et qu'il réservât
certaines choses au roi tout seul. Qu'il partit promptement, et qu'il
exécutât ce qui lui était ordonné hardiment et diligemment, et qu'il
s'assurât qu'il serait puni de mort s'il négligeait de s'acquitter de sa
commission. Le maréchal promit tout, et aussitôt la reine disparut, et
il se trouva dans l'obscurité auprès de son arbre. Il s'y coucha au pied
ne sachant s'il rêvait ou s'il était éveillé, et s'en alla après chez
lui, persuadé que c'était une illusion et une folie dont il ne se vanta
à personne. À deux jours de là, passant au même endroit, la même vision
lui arriva encore, et les mêmes propos lui furent tenus. Il y eut de
plus des reproches de son doute et des menaces réitérées, et pour fin,
ordre d'aller dire à l'intendant de la province ce qu'il avait vu, et
l'ordre qu'il avait reçu d'aller à Versailles, et que sûrement il lui
fournirait de quoi faire le voyage. À cette fois, le maréchal demeura
convaincu. Mais flottant entre la crainte des menaces et les difficultés
de l'exécution, il ne sut à quoi se résoudre, gardant toujours le
silence de ce qui lui était arrivé.

Il demeura huit jours en cette perplexité, et enfin comme résolu à ne
point faire le voyage, lorsque, repassant encore par le même endroit, il
vit et entendit encore la même chose, et des menaces si effrayantes
qu'il ne songea plus qu'à partir. À deux jours de là, il fut trouver à
Aix l'intendant de la province, qui sans balancer l'exhorta à poursuivre
son voyage, et lui donna de quoi le faire dans une voiture publique. On
n'en a jamais su davantage. Il entretint trois fois M. de Pomponne, et
fut chaque fois plus de deux heures avec lui. M. de Pomponne en rendit
compte au roi en particulier, qui voulut que Pomponne en parlât plus
amplement à un conseil d'État où Monseigneur n'était point, et où il n'y
avait que les ministres, qui lors, outre lui, étaient le duc de
Beauvilliers, Pontchartrain et Torcy, et nuls autres. Ce conseil fut
long, peut-être aussi y parla-t-on d'autre chose après. Ce qui arriva
ensuite fut que le roi voulut entretenir le maréchal\,; il ne s'en cacha
point\,; il le vit dans ses cabinets, et le fit monter par le petit
degré qui en descend sur la cour de Marbre par où il passe pour aller à
la chasse, ou se promener. Quelques jours après, il le vit encore de
même, et à chaque fois fut près d'une heure seul avec lui, et prit garde
que personne ne fût à portée d'eux. Le lendemain de la première fois
qu'il l'eut entretenu, comme il descendait par ce même petit escalier
pour aller à la chasse, M. de Duras, qui avait le bâton et qui était sur
le pied d'une considération et d'une liberté de dire au roi tout ce
qu'il lui plaisait, se mit à parler de ce maréchal avec mépris, et à
dire le mauvais proverbe, que cet homme-là était un fou ou que le roi
n'était pas noble. À ce mot, le roi s'arrêta, et se tournant au maréchal
de Duras, ce qu'il ne faisait presque jamais en marchant\,: «\,Si cela
{[}est{]}, lui dit-il, je ne suis pas noble, car je l'ai entretenu
longtemps\,; il m'a parlé de fort bon sens, et je vous assure qu'il est
fort loin d'être fou.\,» Ces derniers mots furent prononcés avec une
gravité appuyée qui surprit fort l'assistance, et qui en grand silence
ouvrit fort les yeux et les oreilles. Après le second entretien, le roi
convint que cet homme lui avait dit une chose qui lui était arrivée il y
avait plus de vingt ans, et que lui seul savait, parce qu'il ne l'avait
jamais dite à personne, et il ajouta que c'était un fantôme qu'il avait
vu dans la forêt de Saint-Germain, et dont il était sûr de n'avoir
jamais parlé. Il s'expliqua encore plusieurs fois très favorablement sur
ce maréchal, qui était défrayé de tout par ses ordres, qui fut renvoyé
aux dépens du roi, qui lui fit donner assez d'argent outre sa dépense,
et qui fit écrire à l'intendant de Provence de le protéger
particulièrement, et d'avoir soin que, sans le tirer de son état et de
son métier, il ne manquât de rien le reste de sa vie. Ce qu'il y a eu de
plus marqué, c'est qu'aucun des ministres d'alors n'a jamais voulu
parler là-dessus. Leurs amis les plus intimes les ont poussés et tournés
là-dessus, et à plusieurs reprises, sans avoir pu en arracher un mot, et
tous, d'un même langage, leur ont donné le chance, se sont mis à rire et
à plaisanter sans jamais sortir de ce cercle, ni enfoncer cette surface
d'une ligne. Cela m'est arrivé avec M. de Beauvilliers et M. de
Pontchartrain, et je sais par leurs plus intimes et leurs plus familiers
qu'ils n'en ont rien tiré davantage, et pareillement de ceux de Pomponne
et de Torcy.

Ce maréchal, qui était un homme d'environ cinquante ans, qui avait
famille, et bien famé dans son pays, montra beaucoup de bon sens dans sa
simplicité, de désintéressement et de modestie. Il trouvait toujours
qu'on lui donnait trop, ne parut {[}avoir{]} aucune curiosité, et dès
qu'il eut achevé de voir le roi et M. de Pomponne, ne voulut rien voir
ni se montrer, parut empressé de s'en retourner, et dit que, content
d'avoir accompli sa mission, il n'avait plus rien à faire que s'en aller
chez lui. Ceux qui en avaient soin firent tout ce qu'ils purent pour en
tirer quelque chose\,; il ne répondait rien, ou disait\,: «\,Il m'est
défendu de parler,\,» et coupait court sans se laisser émouvoir par
rien. Revenu chez lui, il ne parut différent en rien de ce qu'il était
auparavant, ne parlait ni de Paris ni de la cour, répondait en deux mots
à ceux qui l'interrogeaient, et montrait qu'il n'aimait pas à l'être, et
sur ce qu'il avait été faire pas un mot de plus que ce que je viens de
rapporter. Surtout nulle vanterie\,; ne se laissait point entamer sur
les audiences qu'il avait eues, et se contentait de se louer du roi
qu'il avait vu, mais en deux mots et sans laisser entendre s'il l'avait
vu en curieux ou d'une autre manière, et ne voulant jamais s'en
expliquer. Sur M. de Pomponne, quand on lui en parlait, il répondait
qu'il avait vu un ministre, sans expliquer comment ni combien, qu'il ne
le connaissait pas, et puis se taisait sans qu'on pût lui en faire dire
davantage. Il reprit son métier, et a vécu depuis à son ordinaire. C'est
ce que les premiers de la province en ont rapporté, et ce que m'en a dit
l'archevêque d'Arles, qui passait du temps tous les ans à Salon, qui est
la maison de campagne des archevêques d'Arles aussi bien que le lieu de
la naissance et de la sépulture du fameux Nostradamus. Il n'en faut pas
tant pour beaucoup faire raisonner le monde. On raisonna donc beaucoup
sans avoir rien pu trouver, ni qu'aucune suite de ce singulier voyage
ait pu ouvrir les yeux sur rien. Des fureteurs ont voulu se persuader,
et persuader aux autres, que ce ne fut qu'un tissu de hardie friponnerie
dont la simplicité de ce bonhomme fut la première dupe.

Il y avait à Marseille une M\textsuperscript{me} Arnoul, dont la vie est
un roman, et qui, laide comme le péché, et vieille, pauvre, et veuve, a
fait les plus grandes passions, a gouverné les plus considérables des
lieux où elle s'est trouvée, se fit épouser par ce M. Arnoul, intendant
de marine à Marseille, avec les circonstances les plus singulières, et,
à force d'esprit et de manège, se fit aimer et redouter partout où elle
vécut, au point que la plupart la croyaient sorcière. Elle avait été
amie intime de M\textsuperscript{me} de Maintenon, étant
M\textsuperscript{me} Scarron\,; un commerce secret et intime avait
toujours subsisté entre elles jusqu'alors. Ces deux choses sont
vraies\,; la troisième, que je me garderai bien d'assurer, est que la
vision et la commission de venir parler au roi fut un tour de
passe-passe de cette femme, et que ce dont le maréchal de Salon était
chargé par cette triple apparition qu'il avait eue, n'était que pour
obliger le roi à déclarer M\textsuperscript{me} de Maintenon reine. Ce
maréchal ne la nomma jamais, et ne la vit point. De tout cela jamais on
n'en a su davantage.

L'affaire de M. de Bouillon avec son fils faisait grand bruit. Elle
était portée pour des incidents au conseil des parties. Le roi fit en
cette occasion ce qu'il n'avait jamais fait auparavant ni ne fit depuis.
Il prit parti pour M. de Bouillon, fit mander de sa part par
Pontchartrain à Maboul, maître des requêtes, de rapporter sans délai, et
dit lui-même au duc d'Albret qu'il ne voulait que justice entre lui et
son père, mais qu'il voulait couper court aux procédures et aux
procédés, et protéger son père, qui était un de ses plus anciens
domestiques, et qui l'avait toujours bien servi. On peut imaginer si
après ces déclarations M. de Bouillon fut lui-même bien servi par ses
juges, et quel tour prit son affaire dans le monde, où le duc d'Albret
n'osa presque se montrer de fort longtemps.

Le gros Saint-Vallier qui avait été longtemps capitaine de la porte, et
qui après avoir vendu au frère du P. de La Chaise, s'était retiré en son
pays de Dauphiné, mourut à Grenoble. Sa femme, belle, spirituelle et
galante, y régnait sur les cœurs et sur les esprits. Elle avait été fort
du monde, et en était devenue le centre dans cette province, d'où on ne
la revit presque plus à Paris, où elle avait conservé des amis, et à la
cour.

Le duc de Montbazon mourut aussi dans les faubourgs de Liège, où il
était enfermé depuis bien des années dans une abbaye. Le prince de
Guéméné son fils devint par sa mort duc de Montbazon, et se fit recevoir
au parlement. Il fut le premier qui, devenant duc, n'en prit pas le nom
et conserva le sien. Ce fut un raffinement de princerie. On en rit et on
le laissa faire.

Le marquis de Mirepoix mourut en ce même temps. Il était dans les
mousquetaires noirs, médiocre emploi pour un homme de sa naissance, mais
il était fort mal à son aise, et ne laissa point d'enfants de la fille
aînée de la duchesse de La Ferté. Il était de mes amis. C'était un homme
d'honneur et de valeur. J'avais été presque élevé avec son frère,
beaucoup plus jeune que lui. La maréchale de Duras, sœur du duc de
Ventadour, l'avait pris chez elle comme son fils, et l'avait élevé avec
son fils aîné, et nous nous voyions tous les jours\,; je les perdis
depuis de vue\,; le duc de Duras entra dans le monde et me laissa fort
derrière. Il avait bien des années plus que moi. L'autre s'amouracha de
la fille d'un cabaret en Alsace, et s'enterra si bien avec elle qu'on ne
l'a pas vu depuis. Le fils de ce mariage est le marquis de Mirepoix
d'aujourd'hui.

La duchesse Mazarin finit aussi son étrange carrière en Angleterre, où
elle était depuis plus de vingt-cinq ans. Sa vie a fait tant de bruit
dans le monde que je ne m'arrêterai pas à en parler. Malheureusement
pour elle, sa fin y répondit pleinement, et ne laissa de regrets qu'à
Saint-Évremond, dont la vie, la cause de la fuite, et les ouvrages sont
si connus. M\textsuperscript{me} de Bouillon, et ce que
M\textsuperscript{me} Mazarin avait ici de plus proches, partirent pour
l'aller trouver, la trouvèrent morte en arrivant à Douvres et revinrent
tout court. M. Mazarin, depuis si longtemps séparé d'elle et sans aucun
commerce, fit rapporter son corps, et le promena près d'un an avec lui
de terre en terre. Il le déposa un temps à Notre-Dame de Liesse, où les
bonnes gens la priaient comme une sainte et y faisaient toucher leurs
chapelets. À la fin, il l'envoya enterrer avec son fameux oncle, en
l'église du collège des Quatre-Nations à Paris.

MM\hspace{0pt}. de Matignon perdirent en même temps une sœur très
aimable, veuve sans enfants de M. de Nevet, en Bretagne où elle était
allée pour des affaires\,: elle logeait avec eux à Paris. Ils étaient
tous fort des amis de mon père et de ma mère.

La reine de Portugal, sœur de l'impératrice, de la reine d'Espagne et de
l'électeur palatin, mourut aussi, et laissa plusieurs enfants. Elle
était seconde femme du roi don Pedro, qui avait, de concert avec la
reine sa belle-sœur, détrôné son frère comme fou et imbécile, qu'il tint
enfermé aux Terceires en 1669, puis à Cintra, à sept lieues de Lisbonne,
jusqu'à sa mort en 1683. Il épousa en même temps cette meure reine, sœur
de la duchesse douairière de Savoie, grand'mère de M\textsuperscript{me}
la duchesse de Bourgogne, qui prétendit que ce premier mari ne l'avait
jamais été. Elles étaient filles du duc de Nemours, tué en duel à Paris,
pendant les guerres civiles, par le duc de Beaufort, frère de sa
femme\,; et ce duc de Nemours était frère aîné du duc de Nemours mari de
la Longueville, qu'on a vu perdre ce grand procès contre M. le prince de
Conti, et faire ensuite le voyage de Neuchâtel. De ce mariage, don
Pedro, qui ne prit que le titre de régent du vivant du roi son frère,
n'eut qu'une seule fille qui mourut prête à être mariée\,; sa mère
mourut trois mois après son premier mari.

Le roi donna encore des distinctions à ses bâtards, dont il ne perdait
point d'occasions. M. du Maine, grand maître de l'artillerie, comme
ordonnateur en cette partie, avait à être reçu à la chambre des comptes,
et sa place devait être au-dessus du doyen, comme l'avaient eue les
autres grands maîtres de l'artillerie. Le roi voulut qu'il la prit entre
le premier et le second président, et cela fut exécuté ainsi. Il accorda
aussi à M\textsuperscript{me} la princesse de Conti que ses deux filles
d'honneur mangeassent avec M\textsuperscript{me} la duchesse de
Bourgogne. Jamais dame d'honneur de princesse du sang n'avait entré dans
les carrosses, ni mangé. Le roi donna cette distinction à celles de ses
bâtardes, et la refusa toujours à celles des autres princesses du sang.
Pour les filles d'honneur de M\textsuperscript{me} la princesse de Conti
(et M\textsuperscript{me} la Duchesse n'en avait plus depuis longtemps),
elles obtinrent d'abord d'aller à Marly, puis de manger à table quand
Madame n'y était pas, avant le mariage de M\textsuperscript{me} la
duchesse de Bourgogne, à la fin de manger avec elles.

En accordant de nouveaux honneurs, privativement à tous autres, à ce qui
sortait de sa personne, elle-même semblait aussi en mériter de nouveaux.
Mais tout était épuisé en ce genre\,: on ne fit donc que recommencer ce
qui s'était fait à sa statue de la place des Victoires, en découvrant,
le 13 août, après midi, celle qu'on avait placée dans la place de
Vendôme. Le duc de Gesvres, gouverneur de Paris, à cheval, à la tête du
corps de ville, y fit les tours, les révérences, et les autres
cérémonies tirées et imitées de la consécration de celle des empereurs
romains. Il n'y eut à la vérité ni encens ni victimes\,; il fallut bien
donner quelque chose au titre de roi très chrétien. Il y eut un beau feu
le soir sur la rivière, que Monsieur et Madame allèrent voir du Louvre.
Monseigneur en pompe, la seule fois de sa vie, avait été spectateur de
la dédicace de la statue de la place des Victoires, de chez le maréchal
de La Feuillade qui en avait été l'inventeur. Son fils, mal avec le roi,
se lassa en ce temps-ci de la dépense dont il était chargé par le
testament de son père, de faire allumer tous les soirs les falots des
quatre coins de cette place. Le roi voulut bien l'en décharger.

Il refusa presque en même temps audience au comte de Zinzendorf, envoyé
de l'empereur, nouvellement arrivé\,; parce qu'il prétendit n'en point
prendre des fils de France puînés, à cause que les envoyés du roi à
Vienne ne voient pas l'archiduc, et le roi veut qu'il prenne toutes ces
audiences en sortant de la sienne. Villars, comme on a vu, eut ordre de
voir l'archiduc, chez lequel on ajusta le cérémonial qui en empêchait,
après qu'il eut reçu chez lui la satisfaction du prince de
Lichtenstein\,: ainsi la difficulté de Zinzendorf tomba d'elle-même.

Une autre difficulté suivit celle-là de près. Le roi de Danemark mourut.
Le prince royal devenu roi en donna part au roi, et n'en voulut pas
recevoir la réponse sans le traitement de \emph{Majesté}, que jamais
ceux de Danemark n'ont eu des nôtres, et se sont toujours contentés de
la \emph{Sérénité\,;} le roi à son tour refusa de prendre le deuil qu'il
a toujours porté des têtes couronnées, même sans parenté, comme il n'y
en a point avec le roi de Danemark. Cela dura quelques mois de la
sorte\,; à la fin, le roi de Danemark céda, et reçut la lettre du roi en
réponse dans le style accoutumé, et le roi prit le deuil, comme s'il
n'eût pas été passé depuis longtemps.

La vieille duchesse de Modène, de la maison Barberine, mourut aussi,
mère du duc de Modène, et seconde femme de son père, qui de son premier
mariage avec la Martinozzi, sœur de la mère du prince de Conti, et
toutes deux filles de la sœur aînée du cardinal Mazarin, avait eu la
reine d'Angleterre qui est à Saint-Germain.

M. Boucherat, chancelier et garde des sceaux de France, mourut à Paris
le mercredi 2 septembre l'après-dînée, et sur les huit heures du soir.
MM. d'Harlay et de Fourcy, ses gendres, rapportèrent les sceaux au roi,
qui partit le lendemain jeudi et alla coucher à Fontainebleau, où il
emporta les sceaux. Le père et le grand-père de M. Boucherat étaient
auditeurs des comptes à Paris, et son bisaïeul avocat au parlement. Il
ne faut pas aller plus loin. Il avait un frère conseiller au parlement,
fort épais, qui lui ressemblait beaucoup, qu'il fit conseiller
d'honneur. Lui fut d'abord correcteur à la chambre des comptes, puis
conseiller aux requêtes du palais, et en 1643 maître des requêtes. Il
fut en cette charge connu de M. de Turenne qui prit confiance en lui et
le chargea de ses affaires, qui, dans l'éclat et le crédit où il était,
n'étaient pas difficiles à gérer. Cet attachement fit sa fortune. M. de
Turenne lui procura des intendances, des commissions extraordinaires en
plusieurs grandes provinces où il le soutint fort, une place de
conseiller d'État en 1662, et une de conseiller d'honneur au parlement
en 1671. Il n'est pas de l'étendue de ces Mémoires d'expliquer comment
il fut fait chancelier à Fontainebleau, le jour de la Toussaint 1685,
par la mort de M. Le Tellier. À celle de M. de Louvois, il eut le
râpé\footnote{Saint-Simon explique plus loin ce qu'on entendait par
  \emph{râpé des ordres du roi}. «\,Ce nom, dit-il, est pris de l'eau
  qu'on passe sur le marc du raisin après qu'il a été pressé, et tout le
  jus ou le moût tiré qui est le vin. Cette eau fermente sur ce marc et
  y prend une couleur et une impression de petit vin ou piquette, et
  cela s'appelle un \emph{râpé} de vin. On va voir que la comparaison
  est juste et le nom bien appliqué. Pierre, par exemple, a une charge
  de l'ordre depuis quelques années il la vend à Paul et obtient le
  brevet ordinaire. Jean veut se parer de l'ordre sans bourse délier.
  Avec l'agrément du roi, et le marché fait et déclaré avec Paul, Jean
  se met entre Pierre et lui, fait un achat simulé de la charge de
  Pierre, et y est reçu par le roi. Quelques semaines après, il donne sa
  démission, fait une vente simulée à Paul, et obtient le brevet
  accoutumé. » C'était ce brevet de vétéran des ordres du roi que l'on
  appelait le \emph{râpé des ordres}.} de chancelier de l'ordre, dont M.
de Barbezieux eut la charge. Il avait alors soixante-neuf ans, et il
touchait au décanat du conseil. Qui eût voulu faire, exprès un
chancelier de cire l'eût pris sur M. Boucherat. Jamais figure n'a été si
faite exprès\,; la vérité est qu'il n'y fallait pas trop chercher autre
chose, et il est difficile de comprendre comment M. de Turenne s'en
coiffa, et comment ce magistrat soutint les emplois, quoique fort
ordinaires, par lesquels il passa. Il ne fut point ministre, et MM. de
Louvois et Colbert, qui étaient lors les principaux, contribuèrent fort
à son élévation pour n'avoir aucun ombrage à craindre. De sa première
femme Fr. Marchand, il eut M\textsuperscript{me}s de Fourcy et de
Morangis\,; de la seconde qui était une Loménie, veuve d'un
Nesmond\footnote{La seconde femme du chancelier Boucherat,
  Anne-Françoise de Loménie, n'était pas veuve d'un Nesmond, mais de
  Nicolas Bretel, seigneur de Grémonville, ambassadeur de France à
  Venise de 1645 à 1647 et mort à Paris en 1648. La \emph{Biographie
  universelle} et le \emph{Dictionnaire de la noblesse} se sont trompés
  sur la date de la mort de Nicolas Bretel. Cette date est fixée par les
  papiers de la famille de Grémonville.}, avec trois filles, il n'en eut
que M\textsuperscript{me} d'Harlay. Ses trois gendres furent conseillers
d'État, et le dernier, ambassadeur plénipotentiaire à la paix de
Ryswick, comme il a été dit en son temps. Le chancelier avait
quatre-vingt-quatre ans quand il mourut. Il y avait longtemps qu'il
était infirme, et que M. et M\textsuperscript{me} d'Harlay qui logeaient
avec lui, ses secrétaires, et surtout Boucher, qui était le premier et
qui ne s'y est pas oublié, faisaient tout et lui faisaient tout faire.

M. de Pontchartrain, le premier président, MM. Courtin, d'Aguesseau,
Pomereu, La Reynie, conseillers d'État, et les deux premiers au conseil
royal des finances, furent ceux dont on parla le plus. Quelques-uns
parlèrent aussi de MM. de Caumartin et Voysin.

Le premier président, comme on l'a déjà vu, avait eu deux fois parole du
roi d'être chancelier\,; la première, étant procureur général lorsqu'il
donna l'invention du chausse-pied de la légitimation du chevalier de
Longueville, sans nommer la mère, pour faire passer celle des enfants de
M\textsuperscript{me} de Montespan\,; la dernière, étant premier
président, lorsqu'il inventa pour eux ce rang au-dessus des pairs, si
approchant, quoique inférieur, de celui des princes du sang\,; mais
l'affaire de M. de Luxembourg sur la préséance, qui le brouilla sans
ménagement avec les dues, et qui outra M. de La Rochefoucauld contre
lui, les rendit inutiles. M. de La Rochefoucauld, qui n'ignorait ni ces
paroles ni leur cause, se fit une application continuelle de le perdre
là-dessus dans l'esprit du roi, et lui donna tant de coups d'estramaçon,
dont il ne se cachait pas, qu'il vint à bout de ce qu'il désirait. Aucun
de nous ne se cacha de lui nuire en tout ce qu'il put, et tous se
piquèrent de faire éclater leur joie quand ils le virent frustré de
cette grande espérance. Le dépit qu'il en conçut fut public et si
extrême qu'il en devint encore plus absolument intraitable, et qu'il
s'écriait souvent, dans une amertume qu'il ne pouvait contenir, qu'on le
laisserait mourir dans la poussière du palais. Sa faiblesse fut telle
qu'il ne put s'empêcher six semaines après de s'en plaindre au roi à
Fontainebleau, où il fit le bon valet avec sa souplesse et sa fausseté
accoutumées. Le roi le paya de propos et de la commission de travailler
à la diminution du blé dans la ville et banlieue de Paris où il était
devenu cher, et d'ordonner au prévôt des marchands et au lieutenant de
police de n'y rien faire que de concert avec lui. Il fit semblant d'être
content des discours et de cette coriandre, et n'en vécut pas moins
enragé. Sa santé et sa tête à la fin en furent attaquées jusqu'à le
forcer à quitter sa place, d'où il tomba dans le mépris après avoir
aiguisé force haines.

M. Courtin, doyen du conseil, illustre par sa probité et par sa
capacité, par la douceur et l'agrément de son commerce, et par ses
belles et importantes ambassades, s'était expliqué avec le roi,
lorsqu'il refusa celle de Ryswick, et depuis, la place du conseil royal
des finances, que son âge, sa santé et l'état de ses yeux qu'il était
prêt à perdre ne lui permettaient plus de penser qu'à finir.

M. d'Aguesseau avait beaucoup d'esprit mais encore plus réglé et plus
sage. Il avait excellé dans les premières intendances, et il écrivait
d'affaires {[}de façon{]} qu'on n'avait jamais pu faire d'extraits de
ses lettres. Sa capacité était profonde et vaste\,; son amour du bien
ardent, mais prudent\,; sa modestie en tout retraçait les premiers et
les plus anciens magistrats\,; sa douceur extrême\,; ses opinions justes
et concises quand il s'était une fois décidé, à quoi la crainte de
l'injustice et la défiance de soi-même le rendait souvent trop incertain
et trop lent\,; assez capable d'amitié et tout à fait incapable de
haine\,; grand et aisé travailleur\,; exact à tout et ne perdant jamais
un instant\,; d'une piété solide, unie et de toute sa vie\,; éclairé en
tout, et si appliqué à ses devoirs qu'il n'avait jamais connu qu'eux et
ne s'était en aucun temps mêlé avec le monde. Tant de vertus et de
talents lui avaient acquis l'amour et la vénération publique, et une
grande estime du roi\,; mais il avait eu une fille dans celles \emph{de
l'Enfance}, de M\textsuperscript{me} de Mondonville que les jésuites
avaient si étrangement su détruire. Lui, et sa femme, aussi vertueuse
que lui et de plus d'esprit encore, mais dont l'extérieur n'était pas
aimable comme le sien, étaient soupçonnés de jansénisme. Avec cette tare
c'était merveille comme ses vertus et ses talents l'avaient porté sans
autre secours où il était arrivé, mais c'eût été un vrai miracle si
elles l'eussent conduit plus loin.

Pomereu était un aigle qui brillait d'esprit et de capacité, qui avait
été le premier intendant de Bretagne, qui avait eu de grandes et
importantes commissions, et qui avait recueilli partout une grande
réputation, mais il était fantasque, qui avait même quelques temps
courts dans l'année où sa tête n'était pas bien libre et où on ne le
voyait point. D'ailleurs c'était un homme ferme, transcendant, qui avait
et qui méritait des amis. Il l'était fort de mon père et il était
demeuré le mien.

La Reynie, usé d'âge et de travail, est celui qui a mis la place de
lieutenant de police dans la considération et l'importance où on l'a vue
depuis, et où elle serait désirable s'il avait pu l'exercer toujours\,;
mais, noyé dans les détails d'une inquisition naissante et qui a été
portée de plus en plus loin après lui, il n'était plus en âge ni en état
de venir au grand et de travailler d'une manière supérieure. Du reste,
esprit, capacité, sagesse, lumières, probité, tout fit regretter qu'il
eût pour ainsi dire dépassé la première place de son état.

Caumartin, cousin germain et ami confident de Pontchartrain, tel que je
l'ai représenté en parlant de l'affaire de son frère avec M. de Noyon,
avait beaucoup d'amis et du haut parage\,; mais l'insolence de son
extérieur, qui pourtant n'en avait que l'écorce, lui aliénait le gros du
monde. Un amas de blé dont il fut fort accusé dans un temps de cherté,
et diverses autres choses dont il se justifia très bien, avaient laissé
un nuage dans l'esprit du roi dont il ne put jamais revenir pour aucune
place. C'était fort la mode à Fontainebleau, tous les voyages, d'aller
chez lui à Saint-Ange, qui en est à quatre lieues, qu'il avait fort bien
ajusté. Le roi, tout maître qu'il fût toujours de soi-même, ne pouvait
s'empocher de marquer par quelque mot que cela ne lui était point
agréable.

Voysin et sa femme, dans la faveur de M\textsuperscript{me} de
Maintenon, depuis qu'elle avait logé chez eux, aux voyages du roi en
Flandre dont il était intendant, n'était pas encore mûr à beaucoup près.

Pelletier de Sousy, conseiller d'État, et tiercelet de ministre, par un
travail réglé avec le roi une fois par semaine, par Marly où ce même
travail lui procurait de coucher, et par la distinction de paraître
comme eux la canne à la main sans manteau, avait reçu une entorse de la
probité de son frère, quand il quitta la place de contrôleur général et
que le roi, pour l'obliger, lui proposa de la donner à Sousy, {[}ce{]}
qui le fixa pour toujours où il était. Son fils avait eu sa place
d'intendant des finances. Le roi le trouvait bien établi avec raison et
ne songea pas un moment à lui.

D'autres à portée des sceaux, il n'y en avait point. Le premier
président seul, véritable antagoniste, étant exclu, le choix du roi fut
bientôt fait. L'habitude y contribua et M\textsuperscript{me} de
Maintenon acheva d'y déterminer son goût, qui lui fut toujours favorable
dans les {[}temps{]} mêmes de nuages et de brouillard.

M. de Pontchartrain était petit-fils du premier Phélypeaux, qui fut
secrétaire d'État à la place de Forget, sieur de Fresne, trois semaines
avant la mort funeste d'Henri IV, par le crédit de la reine sa femme
dont il était secrétaire des commandements. Il mourut en 1621, pendant
le siège de Montauban. Son fils eut sa charge\,; mais, comme il n'avait
que huit ans, d'Herbault, frère aîné de son père, l'exerça par
commission et se la fit donner après en titre, dépouillant son neveu. La
Vrillière, son fils, l'eut après lui, et de père en fils elle leur est
demeurée. Le neveu dépouillé fut conseiller au parlement, puis président
en la chambre des comptes à Paris, et mourut dans cette charge en 1685.
Il fut un des juges de M. Fouquet, que l'on tira tous des diverses cours
supérieures du royaume. Sa probité fut inflexible aux caresses et aux
menaces de MM. Colbert, Le Tellier et de Louvois, réunis pour la perte
du surintendant. Il ne put trouver matière à sa condamnation, et par
cette grande action se perdit sans ressource. Il était pauvre\,; tout
son désir et celui de son fils, dont il s'agit ici, était de faire
tomber sa charge sur sa tête en s'en démettant. La vengeance des
ministres fut inflexible à son tour, il n'en put jamais avoir
l'agrément\,; tellement que ce fils demeura dix-huit ans conseiller aux
requêtes du palais, sans espérance d'aucune autre fortune. Je le lui ai
ouï dire souvent, et combien il était affligé d'être exclu d'avoir la
charge de son père. Il logeait chez lui avec sa femme, fille de Maupeou,
président aux enquêtes, n'avaient qu'un carrosse pour eux deux, et lui
un cabinet pour travailler, où on entrait du haut du degré sans rien
entre-deux, et couchaient au second étage. Sa mère, qui était morte en
1653, était fille du célèbre Talon, avocat général au parlement, puis
conseiller d'État, qui a laissé des Mémoires si curieux et si rares des
troubles de la minorité, en forme presque de journal.

Mon père était ami des Talon et des Phélypeaux, et lui et ma mère ont vu
cent fois MM. de Pontchartrain, père et fils, vivant comme je le
remarque. Le fils avait un frère et deux sœurs. Le frère fut conseiller
au grand conseil, puis maître des requêtes, bon homme et fort homme
d'honneur, mais qui serait demeuré toute sa vie maître des requêtes,
sans la fortune de son aîné qui le fit conseiller d'État et intendant de
Paris. Les sœurs épousèrent\,: l'aînée, M. Bignon, avocat général au
parlement, après son célèbre père, puis conseiller d'État, celui qui par
amitié et sans parenté voulut bien être mon tuteur lorsqu'à la mort de
ma sœur je fus son légataire universel. L'autre sœur épousa M. Habert de
Montmort, conseiller au parlement, fils de celui qui fut un des premiers
membres de l'Académie française lorsque le cardinal de Richelieu la
forma. Celle-ci mourut sans enfanta, dès 1661. L'autre mourut en 1690,
et ne vit point la fortune de son frère, qui l'aimait si tendrement
qu'il a toujours traité ses enfants comme les siens, et en a fait deux
conseillers d'État, et un autre conseiller d'État d'Église, et vécut
intimement et avec déférence dans sa fortune avec M. Bignon, son
beau-frère, jusqu'à sa mort. Tel était l'état de cette famille si mal
aisée et si reculée, que lorsque le père mourut en 1685 ils n'en furent
guère plus à leur aise.

Quoique simple conseiller aux requêtes du palais, et ne vivant point en
amitié avec La Vrillière ni Châteauneuf, son fils, de qui seuls il
pouvait tirer quelque lustre, parce qu'il ne leur pouvait pardonner la
charge de secrétaire d'État, Pontchartrain, né galant, et avec un feu et
une grâce dans l'esprit que je n'ai point vus dans aucun autre si ce
n'est en M. de la Trappe, se distinguait dans les ruelles et les
sociétés à sa portée, et plus encore par sa capacité, sa grande facilité
et son assiduité au palais. Je lui ai ouï dire bien des fois que son
château en Espagne était d'arriver, avec l'âge, à une place de
conseiller d'honneur au parlement, et d'avoir une maison dans le cloître
Notre-Dame. Il vécut ainsi jusqu'en 1677 que la place de premier
président du parlement de Rennes vaqua, et que les affaires de la
province la rendirent assez longtemps vacante par la difficulté de la
remplir. M. Colbert qui par sa place avait grand désir que celle-ci fût
bien remplie, à cause des états où le premier président de Bretagne est
toujours second commissaire du roi, et pour avoir un homme de qui il pût
tirer conseil sur ce qui se passait dans le commerce de cette province
si maritime, en raisonnait souvent dans son cabinet avec ses plus
familiers. De ce nombre était Hotman, qu'il avait fait intendant des
finances et intendant de Paris, en la capacité duquel il avait beaucoup
de confiance. Hotman avait épousé une Colbert, cousine germaine de
Villacerf et de Saint-Pouange, mais qui, n'étant pas comme eux fille
d'une sœur de M. Le Tellier, était demeurée avec son mari fort attachée
à M. Colbert dont elle était comme eux issue de germaine. Hotman était
un homme qui ne craignait point de dire son avis, et qui, malgré
l'aversion qu'il connaissait en M. Colbert pour Pontchartrain et pour
toute sa famille, lui en proposa le fils comme celui qu'il jugeait le
plus propre à être premier président de Rennes. Il en dit tant de bien
sur ce qu'il en savait qu'il persuada M. Colbert. Ce fut donc ainsi que
l'ennemi de Pontchartrain débourba son fils par une sorte de nécessité.
La surprise qu'ils en eurent fut grande, et augmenta quand ils apprirent
que c'était à Hotman à qui ils devaient cette fortune, avec qui ils
n'avaient aucune liaison. Ils avaient si peu pensé à cet emploi que la
difficulté pécuniaire de le remplir les mit sur le point du refus. Leurs
amis les pouillèrent et les encouragèrent\,; et voilà Pontchartrain en
Bretagne. Hotman, qui mourut sans enfants en 1683, eut le loisir de
s'applaudir du choix qu'il avait proposé\,; Pontchartrain y mit le
parlement et la justice sur un pied tout différent qu'il n'avait été,
lit toutes les fonctions d'intendant dans une province qui n'en
souffrait point encore, mit tout en bon ordre et se fit aimer partout.
Il y eut de grands démêlés d'affaires avec le duc de Chaulnes qui était
adoré en Bretagne, et qui n'était pas accoutumé qu'autre que lui et les
états, dont il était le maître, se mêlassent de rien dans le pays.

On a vu en son lieu que M. Pelletier, contrôleur général des finances,
le tira de là en 1687 pour le faire intendant des finances qu'il fit
toutes sous lui tant qu'il les garda, et comment il les lui fit donner,
en 1689, quand il voulut quitter ce pénible emploi. Pontchartrain eut
toutes les peines du monde à l'accepter, et au lieu de la reconnaissance
qu'il devait à Pelletier de lui avoir fait faire un si grand pas, il lui
en voulut mal, le lui déclara, et ne put jamais le lui pardonner\,: bien
estimable de craindre des fonctions si friandes pour tant d'autres, et
qui portent avec elles les richesses, l'autorité et la faveur\,; fort
blâmable, je ne puis m'empêcher de l'avouer, de n'avoir pas senti plus
que le dégoût des finances de quel accul de fortune il l'avait tiré, et
en quelle place, et en quelle passe son amitié et sa probité le mettait,
et aux dépens de son propre frère. Un an après, la mort de Seignelay
combla ses vœux, quand il se vit revêtu de sa charge de secrétaire
d'État avec le département de la marine et celui de la maison du roi. Il
fit alors instances pour être déchargé des finances. Il ne faisait que
d'y entrer en chef\,; la guerre aussi ne faisait que commencer. En homme
d'esprit il avait bien pris avec M. de Louvois, qui n'en voulait point
d'autre aux finances, et M\textsuperscript{me} de Maintenon, à qui sa
femme et lui avaient également plu, était encore plus éloignée d'un
changement. Le contrôleur général était de tous les ministres celui
qu'elle courtisait le plus. Elle y avait un intérêt principal pour mille
affaires qu'elle protégeait, et pour faire auprès du roi tout ce qui
allait à éloigner ou à approcher à son gré les gens et les choses, parce
que c'était lui d'ordinaire qui y avait la principale influence.
Personne n'était si propre à cette sorte de manège que Pontchartrain.
C'était un très petit homme, maigre, bien pris dans sa petite taille,
avec une physionomie d'où sortaient sans cesse les étincelles de feu et
d'esprit, et qui tenait encore beaucoup plus qu'elle ne promettait.
Jamais tant de promptitude à comprendre, tant de légèreté et d'agrément
dans la conversation, tant de justesse et de promptitude dans les
reparties, tant de facilité et de solidité dans le travail, tant
d'expédition, tant de subite connaissance des hommes ni plus de tour à
les prendre. Avec ces qualités, une simplicité éclairée et une sage
gaieté surnageaient à tout, et le rendaient charmant en riens et en
affaires. Sa propreté était singulière et s'étendait à tout, et à
travers toute sa galanterie, qui subsista dans l'esprit jusqu'à la fin,
beaucoup de piété, de bonté, et j'ajouterai d'équité avant et depuis les
finances, et dans cette gestion même autant qu'elle en pouvait
comporter. Il en avouait lui-même la difficulté, et c'est ce qui les lui
rendait si pénibles, et il s'en expliquait même souvent avec amertume
aux parties qui la lui remontraient. Aussi voulut-il souvent les
quitter, et ce ne fut que par ruses que sa femme les lui fit garder en
lui demandant, tantôt deux, tantôt quatre, tantôt huit jours de délai.

C'était une femme d'un grand sens, sage, solide, d'une conduite
éclairée, égale, suivie, unie, qui n'eut rien de bourgeois que sa
figure\,; libérale, galante en ses présents, et en l'art d'imaginer et
d'exécuter des fêtes\,; noble, magnifique au dernier point, et avec
cela, ménagère et d'un ordre admirable. Personne, et cela est
surprenant, ne connaissait mieux la cour ni les gens qu'elle, et
n'avait, comme son mari, plus de tours et de grâces dans l'esprit. Elle
lui fut d'un grand usage pour le conseil et la conduite, et il eut le
bon esprit de le connaître et d'en profiter\,; leur union fut toujours
intime. Sa piété fut toujours un grand fonds de vertu qui augmenta sans
cesse, qui l'appliqua aux lectures et à la prière, qui lui fit, quand
elle put, embrasser toutes sortes de bonnes œuvres, et qui la rendit la
mère des pauvres\,; avec cela, gaie, et de fort bonne compagnie, où tous
deux mettaient beaucoup dans la conversation, et fort loin de
bavarderie, et tous deux fort capables d'amitié, et lui de servir et de
nuire. Ce qu'ils ont donné aux pauvres est incroyable\,:
M\textsuperscript{me} de Pontchartrain avait toujours les yeux et les
mains ouvertes à leurs besoins, toujours en quête de pauvres honteux, de
gentilshommes et de demoiselles dans le besoin, de filles dans le
danger, pour les tirer de péril et de peine, en mariant ou en plaçant
les unes, donnant des pensions aux autres, et tout cela, dans le dernier
secret. Outre de grandes sommes réglées aux pauvres de leur paroisse, en
tous lieux ils étaient ingénieux à assister\,; et ce tour, et cette
galanterie qu'elle avait dans l'esprit, elle l'employait toute à
secourir des personnes qui cachaient leurs besoins, qu'elle faisait
semblant d'ignorer elle-même. C'était une grosse femme, très laide, et
d'une laideur ignoble et grossière, qui ne laissait pas d'avoir de
l'humeur qu'elle domptait autant qu'il lui était possible. Jamais il n'y
eut de meilleurs parents, ni de meilleurs amis que ce couple, ni de gens
plus polis, on pourrait ajouter quelquefois plus respectueux, et qui se
souvenaient le mieux de ce qu'ils étaient et de ce qu'étaient les
autres, quoiqu'à travers ce levain que mêlent en tout la faveur,
l'autorité et les places.

Ils furent longtemps parfaitement bien avec M\textsuperscript{me} de
Maintenon\,; mais peu à peu, il y eut des froideurs entre elle et
Pontchartrain qu'elle ne maniait pas avec la facilité qu'elle voulait.
Sa femme, qu'elle goûta toujours, et dans tous les temps, tâchait de
rendre Pontchartrain plus complaisant\,; et pour l'amour d'elle,
M\textsuperscript{me} de Maintenon en souffrit des roideurs qu'elle
n'eût jamais passées à un autre\,; mais la pelote grossit tant qu'elle
fut ravie de s'en défaire honnêtement par les sceaux. Il fut ministre
d'État fort peu après avoir été fait secrétaire d'État\,; il avait lu
assez pour être instruit de beaucoup de choses, à travers son
application et son assiduité à ses fonctions et son goût pour le monde
et la bonne compagnie. Il était élevé dans le parlement et dans ses
maximes, duquel il n'était rien moins qu'esclave\,; mais il en avait
pris le bon sur les maximes de France à l'égard de Rome. Ces matières,
qui se présentaient souvent au conseil sous divers aspects, ne lui
échappaient sous aucun. L'extrême facilité de son appréhension, et
l'agilité ferme et forte de son élocution, blessaient souvent le duc de
Beauvilliers là-dessus, dont l'esprit et la conscience ne pouvaient être
d'accord sur ces matières, et qui, en gros, était toujours pour les
maximes de France, mais dans le détail, s'en échappait toujours en
faveur de Rome. Cela les avait aigris l'un contre l'autre, et
quelquefois jusqu'à l'indécence de la part de Pontchartrain qui, ayant
plus de fond que le duc, ne le ménageait pas en ces occasions, et les
rendit ennemis autant que des gens de bien le peuvent être. Le nombre
immense de créations d'offices et d'affaires extraordinaires, auxquelles
la nécessité de la guerre engagea, ne laissa pas de tomber en partie sur
Pontchartrain, et c'était ce qui le pressait sans cesse de quitter les
finances. Il le fut d'établir la capitation et le dixième\footnote{Voy.
  sur la capitation, t. Ier, p.~227, note. L'impôt du dixième consistait
  dans la dîme, ou dixième partie des revenus de toute espèce. Tous les
  Français, nobles et roturiers, y étaient soumis.} inventés l'un et
l'autre par le puissant Bâville, le maître du Languedoc sous le nom
d'intendant, et qui les proposait sans cesse pour en faire sa cour.
Pontchartrain eut horreur de deux impôts que leur facilité à imposer et
à augmenter rendrait continuels et d'une pesanteur extrême. Il rejeta le
dernier, sans souffrir qu'on le mit en délibération, et ne put éviter
l'autre.

Le jour même que Boucherat mourut, l'après-dînée, qui, comme je l'ai
remarqué, était un mercredi, veille du départ du roi pour Fontainebleau,
personne, dès le matin, ne crut qu'il passât la journée. Le roi, au
sortir du conseil, dit à Pontchartrain qui en sortit le dernier\,:
«\,Seriez-vous bien aise d'être chancelier de France\,? --- Sire,
répondit-il, si je vous ai demandé instamment plus d'une fois de me
décharger des finances pour demeurer simple ministre et secrétaire
d'État, vous pouvez imaginer si je les quitterais de bon cœur pour la
première place où je puisse arriver. --- Eh bien\,! dit le roi, n'en
parlez à personne sans exception\,; mais si le chancelier meurt, comme
il est peut-être mort à cette heure, je vous fais chancelier, et votre
fils sera secrétaire d'État en titre, et exercera tout à fait. Vous
continuerez, pour ce voyage, à loger dans votre appartement ordinaire,
parce que j'ai donné les logements de la chancellerie où j'ai bien vu
que le chancelier ne viendrait pas, et que cela m'embarrasserait à
reloger ceux que j'y ai mis.\,» Pontchartrain embrassa les genoux du
roi, saisit l'occasion de demander et d'obtenir de conserver son
logement de Versailles au château, et se retira dans la plus grande joie
qu'il ait jamais sentie, moins d'être chancelier, quoiqu'il en fût
comblé, à ce que je lui ai ouï dire, que d'être délivré du fardeau des
finances, qui lui devenait, malgré la paix, plus insupportable tous les
jours. Cela alla du mercredi au samedi que Pontchartrain devait arriver
à Fontainebleau. Ce soir-là, le roi entrant chez M\textsuperscript{me}
de Maintenon, il dit au maréchal de Villeroy, capitaine des gardes en
quartier, de faire avertir chez Pontchartrain qu'il vint lui parler dès
qu'il serait arrivé. Il y fut d'abord, et il en sortit chancelier de
France. On était à la comédie, un officier des gardes y vint dire au
maréchal de Villeroy que le roi avait fait apporter les sceaux chez
M\textsuperscript{me} de Maintenon, et qu'on avait vu M. de
Pontchartrain les emporter de là chez lui. On s'y attendait plus qu'à
aucun autre. Toute l'attention se tourna à qui serait contrôleur
général\,; on n'attendit pas longtemps.

\hypertarget{chapitre-xix.}{%
\chapter{CHAPITRE XIX.}\label{chapitre-xix.}}

1699

~

{\textsc{Fortune de Chamillart fait contrôleur général des finances.}}
{\textsc{- Mariage de Dreux avec la fille aînée de Chamillart.}}
{\textsc{- Belle action de Chamillart.}} {\textsc{- Logement de
Monseigneur à Fontainebleau.}} {\textsc{- Princesse de Montbéliard à
Fontainebleau.}} {\textsc{- Tabouret de la chancelière.}} {\textsc{-
Femmes des gardes des sceaux.}} {\textsc{- Cour du chancelier.}}
{\textsc{- Trois cent mille livres au maréchal de Villeroy, maître à
Lyon\,; et pension de cent mille livres au duc d'Enghien.}} {\textsc{-
Mort de l'abbé de Charost.}} {\textsc{- Mort de Villacerf\,; sa
familiarité avec le roi.}} {\textsc{- Mort de la comtesse de Fiesque.}}
{\textsc{- Famille, fortune et mort de M. de Pomponne.}} {\textsc{-
Changements d'ambassadeurs.}} {\textsc{- Retour de Fontainebleau.}}

~

Le soir même, au sortir du souper, le roi dit dans son cabinet à
Monseigneur et à Monsieur qu'il avait écrit un billet de sa main à
Chamillart par un des gens de M\textsuperscript{me} de Maintenon, par
lequel il lui mandait qu'il lui donnait la place de contrôleur général.
Cela se répandit au coucher, et de là par toute la cour. Le courrier ne
l'avait pas trouvé à Paris et le fut chercher à Montfermeil qui en est à
quatre lieues, vers Chelles et Livry. Il arriva le lendemain dimanche
après midi.

C'était un grand homme qui marchait en dandinant, et dont la physionomie
ouverte ne disait mot que de la douceur et de la bonté, et tenait
parfaitement parole. Son père, maître des requêtes, mourut en 1675
intendant à Caen, où il avait été près de dix ans. L'année suivante, le
fils fut conseiller au parlement. Il était sage, appliqué, peu éclairé,
et il aima toujours la bonne compagnie. Il était de bon commerce et fort
honnête homme. Il aimait le jeu, mais un jeu de commerce, et jouait bien
tous les jeux. Cela l'initia un peu hors de sa robe\,; mais sa fortune
fut d'exceller au billard. Le roi, qui s'amusait fort de ce jeu, dont le
goût lui dura fort longtemps, y faisait presque tous les soirs d'hiver
des parties avec M. de Vendôme et M. le Grand, et tantôt le maréchal de
Villeroy, tantôt le duc de Grammont. Ils surent que Chamillart y jouait
fort bien, ils voulurent en essayer à Paris. Ils en furent si contents,
qu'ils en parlèrent au roi, et le vantèrent tant, qu'il dit à M. le
Grand de l'amener la première fois qu'il irait à Paris. Il vint donc, et
le roi trouva qu'on ne lui en avait rien dit de trop. M. de Vendôme et
M. le Grand l'avaient pris en amitié et en protection encore plus que
les deux autres, et firent en sorte qu'il fût admis une fois pour toutes
dans la partie du roi, où il était le plus fort de tous. Il s'y comporta
si modestement et si bien, qu'il plut au roi et au courtisan, dont il se
trouva protégé à l'envi au lieu d'en être moqué, comme il arrive à un
nouveau venu inconnu et de la ville. Le roi le goûta de plus en plus, et
il en parla tant à M\textsuperscript{me} de Maintenon qu'elle le voulut
voir. Il s'en tira si bien avec elle, que, peut-être pour flatter le
goût du roi, elle lui dit de la venir voir quelquefois, et à la fin elle
le goûta autant pour le moins que le roi. Malgré ses voyages continuels
à Versailles, où il ne couchait point, il fut assidu les matins au
palais, et continua d'y rapporter. Cela lui acquit l'affection de ses
confrères, qui lui surent gré de faire son métier comme l'un d'eux, et
de vivre avec eux à l'ordinaire, sans donner dans l'impertinence qui
suit souvent les distinctions en beaucoup de gens, et cela lui fit un
mérite à la cour et auprès du roi. Peu à peu il se fit des amis, et le
roi voulut qu'il fût maître des requêtes, pour être plus libre et plus
en état d'être avancé. Alors, il lui donna un logement au château, chose
fort extraordinaire pour un homme comme lui, et même unique. C'était en
1686. Trois ans après il fut nommé intendant de Rouen. Il pria le roi,
avec qui déjà il était très librement, de vouloir bien ne le pas
éloigner de lui\,; mais le roi lui dit que c'était pour cela même qu'il
l'envoyait à Rouen qui est si proche, et il lui permit de venir de temps
en temps passer six semaines à Versailles. Il le mena à Marly et le mit
de son jeu au brelan et à d'autres. Il prit des croupiers parce que le
jeu était gros\,: il y fut heureux.

Au bout de trois ans d'intendance où il ne se méconnut pas plus qu'il
avait fait au parlement, il vaqua une charge d'intendant des finances
que le roi lui donna de son mouvement en 1689, où, comme on voit, il
demeura dix ans, et toujours sur le même pied avec le roi, quoique le
billard ne fit plus à la mode. Il cultiva si bien M\textsuperscript{me}
de Maintenon depuis qu'il fut devenu sédentaire à Paris et à la cour,
qu'elle le choisit pour administrer les revenus et toutes les affaires
temporelles de Saint-Cyr, ce qui lui donna un rapport continuel avec
elle. Il se fit beaucoup d'amis à la cour\,: M. de Chevreuse, dont les
terres venaient presque jusqu'à Versailles par le duché de Chevreuse et
par celui de Montfort, avait fait et refait divers échanges avec la
maison de Saint-Cyr, dans lesquels le roi et M\textsuperscript{me} de
Maintenon étaient entrés, et avait beaucoup de terres limitrophes et
même enclavées avec les leurs. Cela donna lieu à Chamillart de
travailler souvent avec lui, et occasion d'acquérir véritablement son
amitié et celle du duc de Beauvilliers, qui a duré autant que leur vie.
Avec tant de véhicules, celui de Saint-Cyr surtout et la protection de
M\textsuperscript{me} de Maintenon, qui se faisait un si grand intérêt
d'avoir un contrôleur général tout à fait à elle, ce choix ne fut pas un
instant balancé, et le roi s'en applaudit publiquement.

Il vécut dans cet emploi avec une douceur, une patience, une affabilité
qui y était inconnue, et qui lui gagna tout ce qui avait affaire à lui.
Il ne se rebutait point des propositions les plus ineptes ni des
demandes les plus absurdes et les plus réitérées\,; son tempérament y
contribuait par un flegme qui ne se démentait jamais, mais qui n'avait
rien de rebutant\,: sa manière de refuser persuadait du déplaisir qu'il
en ressentait, et celle d'accorder ajoutait à la grâce. Il était en
effet extrêmement porté à obliger et à servir, et fâché et éloigné de
faire la moindre peine. Il se fit aimer passionnément des intendants des
finances, dont ses manières émoussèrent le dépit de voir leur cadet
devenu leur maître, et adorer de ses commis et des financiers. Toute la
cour l'aima de même par la facilité de son accès, par sa politesse et
par une infinité de services, et le roi lui marqua continuellement une
affection qui se peut dire d'ami, et qui augmenta tous les jours. Sa
femme et lui étaient enfants des deux sœurs. Elle était vertueuse et
fort polie\,; mais elle ne savait que jouer, sans l'aimer, mais faute de
savoir faire autre chose ni que dire, après avoir demandé à chacun
comment il se portait\,: la cour ne put la former, et, à dire vrai,
c'était la meilleure et la plus sotte femme du monde, et la plus inutile
à son mari.

Hors son fils alors enfant, Chamillart fut malheureux en famille,
malheur grand pour chacun, mais extrême pour un ministre qui n'a le
temps de rien, et qui a un besoin principal, pour se soutenir et pour
faire, d'avoir autour de soi un groupe qui rassemble et concilie le
monde, qui soit instruit à tout moment des intrigues de ce qui se passe,
et de l'histoire du jour, qui sache raisonner et combiner, et qui soit
capable de le mettre en deux mots au fait de tout tous les jours. Il
avait deux frères plus sots encore que sa femme, et le second y joignait
la suprême impertinence à la sublime bêtise, et tous deux, malgré la
faveur, se faisaient moquer d'eux sans cesse et ouvertement. L'un était
évêque de Dôle, qu'il fit évêque de Senlis, à qui il ne manquait qu'un
béguin et des manches pendantes\,: bon homme et bon prêtre d'ailleurs,
qu'il fallait envoyer à Mende ou à quelque évêché comme cela riche et au
bout du royaume. L'autre qui était dans la marine, il le passa à terre,
et le maria à la fille de Guyet, bien faite, sage et raisonnable, mais
dont le père, qui fut intendant des finances, était un sot et un
impertinent pommé, et sa femme un esprit aigre, qui se croyait une
merveille. Ce gendre, dont la cervelle de plus était mal timbrée, vécut
fort mal avec eux. Rebours, cousin germain de Chamillart et de sa femme,
travailla sous lui d'abord, puis devint intendant des finances. C'était,
je pense, le véritable original du marquis de Mascarille, et fort
impertinent au fond. L'abbé de La Proustière, aussi leur cousin germain,
suppléait pour le ménage, les affaires et l'arrangement domestique à
l'incapacité de M\textsuperscript{me} Chamillart\,: c'était le meilleur
homme et le plus en sa place, et le plus respectueux du monde, mais
grand bavard, et savait fort rarement ce qu'il disait ni même ce qu'il
voulait dire. Avec de tels entours, il fallait toute l'amitié du roi et
de M\textsuperscript{me} de Maintenon pour soutenir Chamillart, dont les
talents ne suppléaient pas aux appuis domestiques. Il éprouva encore un
autre malheur fort singulier.

Dreux et lui étaient conseillers en la même chambre et intimes amis\,;
Dreux fort riche, et Chamillart fort peu accommodé. Leurs femmes
accouchèrent en même temps d'un fils et d'une fille. Dreux, par amitié,
demanda à Chamillart d'en faire le mariage. Chamillart, en âge d'avoir
d'autres enfants, le représenta à son ami, et qu'en attendant que ces
enfants qui venaient de naître fussent en état de se marier, il
trouverait avec ses biens des partis bien plus convenables que sa fille.
Dreux, homme droit, franc, et qui aimait Chamillart, persévéra si bien
qu'ils s'en donnèrent réciproquement parole. Avec les années, la chance
avait tourné. Dreux était demeuré conseiller au parlement, et Chamillart
devenu tout ce que nous venons de voir, mais toujours amis intimes. Sept
ou huit mois avant que Chamillart devînt contrôleur général, il alla
trouver Dreux, et avec amitié lui dit que leurs enfants étaient en âge
de se marier et de les acquitter de leur parole. Dreux, très touché
d'une proposition qui, par la fortune, était si disproportionnée de la
sienne, et qui faisait celle de son fils, fit tout ce qu'un homme
d'honneur peut faire pour le détourner d'une affaire qui n'était plus
dans les termes ordinaires, et qui dans les suites ferait l'embarras de
sa famille, lui rendit sa parole, refusa et dit que c'était lui-même qui
lui en manquait, parce qu'il lui en voulait manquer. Ce combat d'amitié
et de probité dura plusieurs jours de part et d'autre. À la fin
Chamillart bien résolu à partager sa fortune avec son ami l'emporta, et
le mariage se fit. Il obtint pour son gendre l'agrément du régiment
d'infanterie de Bourgogne, et tôt après sa fortune, de la charge de
grand maître des cérémonies que Blainville lui vendit, et le roi prit
prétexte de cette charge pour faire entrer M\textsuperscript{me} Dreux
dans les carrosses, et manger avec M\textsuperscript{me} la duchesse de
Bourgogne. C'est le premier exemple de deux noms de bourgeois se décorer
d'eux-mêmes, et sans prétexte de terre, du nom de marquis et de comte\,;
car tout aussitôt M. Dreux devint M. le marquis de Dreux\footnote{Voy.,
  à la fin du volume une note de MM, de Dreux-Nancré et de Dreux-Brezé
  qui établit que M. de Dreux était de grande et ancienne maison.}, et
Chamillart le frère M. le comte de Chamillart, tant la faveur enchérit
toujours sur les plus folles nouveautés que la bassesse du monde crée et
adopte. Ce nouveau marquis se montra un fort brave homme, mais bête,
obscur, brutal, et avec le temps, audacieux, insolent, et quelque chose
de pis encore, et sans se défaire des bassesses de son état et de son
éducation. Sa femme ne fut heureuse ni par lui ni avec lui, et méritait
infiniment de l'être\,: une grande douceur, beaucoup de vertu et de
sagesse, bien de l'esprit, et avec le temps, de connaissance du monde et
des gens, du manège mais sans rien de mauvais, et si fort en tout temps
en sa place, qu'elle se fit aimer de tout le monde, même des ennemis de
son père, et fit tant de pitié, qu'elle fut toujours et dans tous les
temps accueillie partout, et traitée avec une distinction personnelle
très marquée.

Je ne puis quitter Chamillart sans en rapporter une action qui, pour
n'être pas ici en sa place et avoir dû être racontée plus haut, mérite
de n'être pas oubliée. Ce fut du temps qu'il était conseiller au
parlement, et qu'il jouait au billard avec le roi trois fois la semaine
sans coucher à Versailles. Cela lui rompait fort les jours et les heures
sans le détourner, comme je l'ai dit, de son assiduité au palais. Il y
rapporta dans ces temps-là un procès. Celui qui le perdit lui vint crier
miséricorde. Chamillart le laissa s'exhaler avec ce don de tranquillité
et de patience qu'il avait. Dans le discours du complaignant, il insista
fort sur une pièce qui faisait le gain de son procès, et avec laquelle
il ne comprenait pas encore qu'il l'eut perdu. Il rebattit tant cette
pièce que Chamillart se souvint qu'il ne l'avait pas vue, et lui dit
qu'il ne l'avait pas produite. L'autre à crier plus fort et qu'elle
l'était. Chamillart insistant et l'autre aussi, il prit les sacs qui se
trouvèrent là, parce que l'arrêt ne faisait qu'être signé\,; ils les
visitèrent, et la pièce s'y trouva produite. Voilà l'homme à se désoler,
et cependant Chamillart à lire la pièce et à le prier de lui donner un
peu de patience. Quand il l'eut bien lue et relue\,: «\,Vous avez
raison, lui dit Chamillart, elle m'était inconnue, et je ne comprends
pas comment elle m'a pu échapper\,: elle décide en votre faveur. Vous
demandiez vingt mille livres, vous en êtes débouté par ma faute, c'est à
moi à vous les payer. Revenez après-demain. » Cet homme fut si surpris
qu'il fallut lui répéter ce qu'il venait d'entendre\,; il revint le
surlendemain. Chamillart cependant avait battu monnaie de tout ce qu'il
avait, et emprunté le reste. Il lui compta les vingt mille livres, lui
demanda le secret et le congédia\,; mais il comprit de cette aventure
que les examens et les rapports de procès ne pouvaient compatir avec ce
billard de trois fois la semaine. Il n'en fut pas moins assidu au
palais, ni attentif à bien juger, mais il ne voulut plus être rapporteur
d'aucune affaire, et remit au greffe celles dont il se trouvait chargé,
et pria le président d'y commettre. Cela s'appelle une belle, prompte et
grande action dans un juge, et encore plus dans un juge aussi
étroitement dans ses affaires qu'il y était alors.

Monseigneur logeait à Fontainebleau dans un appartement enclavé qui ne
lui plaisait point. Il eut envie de ceux de MM. du Maine et de Toulouse
contigus, en bas dans la cour en ovale\,; mais le roi ne les voulut
point déloger. Il fit espérer pour l'année suivante un autre logement à
Monseigneur, qui fut obligé de demeurer en attendant dans le sien. Celui
de la reine mère lui aurait mieux convenu qu'aucun, mais il était occupé
tout le milieu de chaque voyage, et celui-ci encore par le roi et la
reine d'Angleterre, et demeurait vide le reste du temps.

Il vint à Fontainebleau du fond de la Silésie une fille de la maison de
Wirtemberg, d'une arrière-branche de Montbéliard-Eltz, et c'est cette
principauté d'Eltz qui est en Silésie. Elle avait perdu son père il y
avait six mois, et sans savoir que M. de Chaulnes avec l'héritière de
Picquigny sa mère avaient tout donné au second fils de M. de Chevreuse,
s'il mourait sans enfants, elle venait recueillir la succession d'Ailly
dont elle avait eu une mère\,; elle était dans un deuil à faire peur, et
ne marchait que dans un carrosse drapé comme en ont les veuves et sans
armes, et ses chevaux caparaçonnés et croisés de blanc jusqu'à terre,
ses gens des manteaux longs et des crêpes traînants\,: on lui demanda de
qui un si grand deuil. «\,Hélas\,! dit-elle en sanglotant ou faisant
semblant, c'est de monseigneur mon papa. » Cela parut si plaisant que
chacun lui fit la même question pour donner lieu à la réponse, et voilà
comme sont les Français. Ce qui leur parut si ridicule, et qui l'était
en effet à nos oreilles, ne l'était en soi qu'à demi. Personne de
quelque distinction, même fort éloignée de celle des maisons souveraines
d'Allemagne, en parlant de ses parents en allemand, ne dit jamais
autrement que monsieur mon père, madame ma mère, mademoiselle ma sœur,
monsieur mon frère, monsieur mon oncle, madame ma tante, monsieur mon
cousin, et supprimer le monsieur ou le madame serait une grossièreté
pareille à tutoyer parmi nous. De monseigneur il n'y en a point en
allemand, de papa voilà le ridicule, surtout entre cinquante et soixante
ans qu'avait cette bonne Allemande\,; mais cela, joint aux sanglots, à
l'équipage d'enterrement, fit le ridicule complet. Elle vit le roi le
matin un moment, puis M\textsuperscript{me} la duchesse de Bourgogne, à
qui le roi avait mandé de la baiser et de la faire asseoir la dernière
de toutes les duchesses\,; et Sainctot, introducteur des ambassadeurs,
la mena partout par ordre du roi. Ce fut la duchesse du Lude qui la
présenta\,; elle demeura deux jours à Fontainebleau et une quinzaine à
Paris, puis s'en retourna comme elle était venue.

M\textsuperscript{me} la chancelière prit son tabouret à la toilette de
M\textsuperscript{me} la duchesse de Bourgogne le samedi 19 septembre,
après laquelle elle suivit dans le cabinet où il y eut audience d'un
abbé Rinini en cercle. La duchesse du Lude, son amie et encore plus des
places et de la faveur, avait arrangé cela tout doucement pour étendre
ce tabouret. Le roi qui le sut lui lava la tête et avertit le chancelier
que sa femme avait fait une sottise, qu'il ne trouverait pas bon qu'elle
recommençât\,; aussi s'en garda-t-elle bien depuis. Cela fit grand bruit
à la cour. Pour entendre ce fait, il faut remonter bien haut, et savoir
qu'aucun office de la couronne ne donne le tabouret à la femme de
l'officier, non pas même celui de connétable.

Le chancelier Séguier avait donné sa fille aînée, très riche, à un parti
très pauvre, et qui d'ailleurs n'y aurait pas prétendu. C'était au père
des duc et cardinal de Coislin, pour faire sa cour au cardinal de
Richelieu, le meilleur parent qui fût au monde, qui était cousin germain
de M. de Coislin qu'il fit chevalier de l'ordre et colonel général des
Suisses, et dont il maria les sœurs au comte d'Harcourt, {[}l'une{]}
étant veuve de Puylaurens, et l'autre au dernier duc d'Épernon, fils et
successeur des charges de ce célèbre duc d'Épernon. Séguier était dans
la plus intime faveur du cardinal\,; il était ambitieux, il trouva sa
belle auprès de lui, il lui demanda le tabouret pour sa femme\,; le
cardinal lui fit beaucoup de difficultés et céda enfin à force de
persévérance. Quand ce fut à attacher le grelot, avec toute sa puissance
et tout son crédit, il demeura court, et n'osa. Il connaissait Louis
XIII, dont le goût ni la politique n'était ni le désordre dans sa cour,
ni la confusion des états. Le chancelier pressait le cardinal\,; il
s'était engagé à lui, et en effet il avait grande envie de lui faire
obtenir cette grâce\,; dans son embarras, il alla chez mon père, ce qui
lui arrivait souvent en ces temps-là, comme je l'ai remarqué en parlant
de mon père, et lui exposa son désir, et l'extrême plaisir qu'il lui
ferait s'il voulait bien tâcher à le faire réussir, en lui avouant
franchement que lui-même n'osait en rompre la glace. Mon père eut la
bonté, il ne m'appartient pas de dire la simplicité, de s'en charger\,;
le roi trouva la proposition fort étrange, et pour abréger ce qui se
passa dans des temps et des mœurs si éloignées des nôtres, il accorda
quoique à regret que la chancelière aurait le tabouret à la toilette
sans le pouvoir prétendre, ni s'y présenter en aucun autre temps, parce
qu'en ce temps-là, comme je l'ai remarqué sur M\textsuperscript{me} de
Guéméné, la toilette n'était point une heure de cour, mais particulière,
à porte fermée, qui n'était ouverte qu'à cinq ou six dames des plus
familières.

Quand après la toilette devint temps et lieu public de cour, la
chancelière y conserva son tabouret\,; mais jamais elle ne s'y est
présentée à aucune audience, cercle, Biner, etc. La duchesse du Lude,
qui était sa petite-fille, aurait bien voulu faire accroire que ce
tabouret s'étendait à toute la matinée jusqu'au dîner exclusivement pour
y comprendre les audiences, et gagner ainsi le terrain pied à pied. Mais
le roi y mit si bon ordre, et la chose tellement au net, que cela
demeura barré pour toujours. Pour le roi, la chancelière ne le voyait
jamais qu'à la porte de son cabinet où elle se tenait debout tout
habillée pour lui faire sa cour lorsqu'il rentrait de la messe, et il
s'arrêtait toujours à elle pour lui dire un mot, et cela arrivait deux
fois l'année, et aux occasions s'il s'en présentait. Chez les filles de
France elle n'était assise non plus qu'à la toilette, mais ce tabouret,
tout informe qu'il fût, soutenu de l'exemple de la même chancelière
Séguier, qui fut enfin assise tout à fait quand le cardinal Mazarin fit
duc à brevet son mari avec tant d'autres (dont il disait qu'il en ferait
tant qu'il serait honteux de l'être et honteux de ne l'être pas) {[}fut
cause{]} que les chancelières sans avoir pu étendre ce tabouret ni oser
prendre les distinctions des duchesses comme la housse, etc., n'ont pas
laissé pourtant d'obtenir insensiblement des princesses du sang le
fauteuil, et je pense aussi la reconduite des duchesses, mais cédant à
toutes partout, même à brevet jusqu'à aujourd'hui, et sans tortillage ni
difficulté. Il n'avait jamais été question des femmes des gardes des
sceaux, et aucune n'a eu le tabouret ni prétendu. Mais M. d'Argenson
étant devenu garde des sceaux, et en même temps le seul vrai maître des
finances pendant la régence de M. le duc d'Orléans, la facilité de ce
prince qui faisait litière d'honneurs, et qui n'en haïssait pas les
mélanges et les désordres, fit asseoir la femme du garde des sceaux à la
toilette de M\textsuperscript{me} sa fille, et de M\textsuperscript{me}
sa mère, les seules filles de France alors, et cet exemple a fait
asseoir M\textsuperscript{me} Chauvelin à la toilette de la reine,
lorsque son mari eut les sceaux avec toute la faveur et toute la
confiance du cardinal Fleury, plus roi que premier ministre.

Avant de quitter la matière du chancelier, il faut dire que, lui et sa
femme n'étant plus nommés que du nom unique de leur office, leur fils
prit le nom de Pontchartrain et se comtifia, son père, ayant extrêmement
augmenté cette terre qu'il avait fait ériger en comté. Il ouvrit la
porte de sa cour aux évêques, aux gens d'une qualité un peu distinguée,
sans être titrés, et pour toute la robe au seul premier président du
parlement de Paris. On le souffrit, et on trouva même qu'il en avait
beaucoup rabattu de son prédécesseur, et il était vrai. Reste à savoir
si Boucherat qui, le premier, avait imaginé d'égaler sa cour à celle du
roi, pouvait avoir raison.

En ce voyage de Fontainebleau, le roi donna trois cent mille livres au
maréchal de Villeroy, à prendre en trois ans sur Lyon, des riches
revenus duquel lui et le prévôt des marchands qu'il nommait, étaient les
seuls dispensateurs sans rendre compte. Peu après, le roi donna cent
mille livres de pension au duc d'Enghien, encore enfant\,; M. le Duc,
son père, n'en avait que quatre-vingt-dix mille.

L'abbé de Charost mourut en ce temps-ci à Paris, chez son père, où il
vivait fort pieusement et fort retiré. Il était fils aîné du duc de
Béthune, et frère aîné du duc de Charost\,; il était fort bossu, avait
renoncé à tout pour une pension médiocre, et s'était fait prêtre. Il
n'avait qu'une abbaye, et jamais il n'avait été question de lui pour
l'épiscopat. J'ai ouï dire qu'il en aurait été fort digne.

Le bonhomme Villacerf ne put survivre plus longtemps au malheur qui lui
était arrivé de l'infidélité de son principal commis des bâtiments, dont
j'ai parlé au commencement de l'année. Il rie porta pas santé depuis, ne
remit pas le pied à la cour depuis s'être démis des bâtiments, et acheva
enfin de mourir. C'était un bon et honnête homme, qui était déjà vieux,
et qui ne put s'accoutumer à avoir été trompé et à n'être plus rien. Il
avait passé une longue vie, toujours extrêmement bien avec le roi, et si
familier avec lui, qu'étant d'une de ses parties de paume autrefois, où
il jouait fort bien, il arriva une dispute sur sa balle\,; il était
contre le roi, qui dit qu'il n'y avait qu'à demander à la reine qui les
voyait jouer de la galerie\,: «\,Par\ldots\,! sire, répondit Villacerf,
cela n'est pas mauvais\,; s'il ne tient qu'à faire juger nos femmes, je
vais envoyer quérir la mienne.\,» Le roi et tout ce qui était là rirent
beaucoup de la saillie. Il était cousin germain, et dans la plus intime
et totale confiance de M. de Louvois, qui, du su du roi, l'avait fait
entrer en beaucoup de choses secrètes, et le roi avait toujours conservé
pour lui beaucoup d'estime, d'amitié et de distinction. C'était un homme
brusque, mais franc, vrai, droit, serviable et très bon ami\,; il en
avait beaucoup, et fut généralement plaint et regretté.

La comtesse de Fiesque, cousine germaine paternelle de la feue duchesse
d'Arpajon, de feu Thury et du marquis de Beuvron, mourut pendant
Fontainebleau, extrêmement âgée. Elle avait passé sa vie dans le plus
frivole, du grand monde\,; deux traits entre deux mille la
caractériseront. Elle n'avait presque rien, parce qu'elle avait tout
fricassé ou laissé piller à ses gens d'affaires\,; tout au commencement
de ces magnifiques glaces, alors fort rares et fort chères, elle en
acheta un parfaitement beau miroir. «\,Hé, comtesse, lui dirent ses
amis, où avez-vous pris cela\,? --- J'avais, dite-elle, une méchante
terre, et qui ne me rapportait que du blé, je l'ai vendue, et j'en ai eu
ce miroir. Est-ce que je n'ai pas fait merveilles\,? du blé ou ce beau
miroir\,!» Une autre fois, elle harangua son fils, qui n'avait presque
rien, pour l'engager à se marier, et à se remplumer par un riche
mariage, et la voilà à moraliser sur l'orgueil qui meurt de faim plutôt
que faire une mésalliance. Son fils, qui n'avait aucune envie de se
marier, la laissa dire, puis, voulant voir où cela irait, fit semblant
de se rendre à ses raisons. La voilà ravie\,! Elle lui étale le parti,
les richesses, l'aisance, une fille unique, les meilleures gens du
monde, et qui seraient ravis, auprès de qui elle avait des amis qui
feraient immanquablement réussir l'affaire, une jolie figure, bien
élevée et d'un âge à souhait. Après une description si détaillée, le
comte de Fiesque la pressa de nommer cette personne en qui tant de
choses réparaient la naissance\,; la comtesse lui dit que c'était la
fille de Jacquier, qui était un homme connu de tout le monde, et qui
s'était acquis l'estime et l'affection de M. de Turenne, les armées
duquel il avait toujours fournies de vivres, et s'était enrichi. Voilà
le comte de Fiesque à rire de tout son cœur, et la comtesse à lui
demander, en colère, de quoi il riait, et s'il trouvait ce parti si
ridicule. Le fait était que Jacquier n'eut jamais d'enfants. La
comtesse, bien surprise, pense un moment, avoue qu'il a raison, et
ajoute en même temps que c'est le plus grand dommage du monde, parce que
rien ne lui eût tant convenu. Elle était pleine de semblables disparates
qu'elle soutenait avec colère, puis en riait la première. On disait
d'elle qu'elle n'avait jamais eu que dix-huit ans. Elle était veuve, dès
1640, de M. de Piennes-Brouilly, tué à Arras, dont elle n'eut qu'une
fille, mère de Guerchy. Les Mémoires de Mademoiselle, avec qui elle
passa toute sa vie, souvent en vraies querelles pour des riens, et sans
toutefois pouvoir se passer l'une de l'autre, la font très bien
connaître. Elle n'eut ni frères ni sœurs, et son père était aîné de
celui de Beuvron.

Une autre mort fit plus de bruit, et laissa un grand vide pour le
conseil et pour les honnêtes gens, ce fut celle de Pomponne, fils du
célèbre Arnauld d'Andilly, et neveu du fameux M. Arnauld. Cette famille
illustre en science, en piété et par beaucoup d'autres endroits, n'a pas
besoin d'être expliquée ici\,; elle l'est par tant de beaux ouvrages que
je m'en tiendrai ici à M. de Pomponne. M. d'Andilly, par ses emplois et
par l'amitié dont la reine mère l'honorait avant et même depuis sa
retraite à Port-Royal des Champs, malgré les tempêtes du jansénisme, fit
employer son fils dès sa première jeunesse en plusieurs affaires
importantes en Italie, où il fit des traités et conclut des ligues avec
plusieurs princes. Son père, extrêmement aimé et estimé, lui donna
beaucoup de protecteurs, dont M. de Turenne fut un des principaux.
Pomponne passa par l'intendance des armées à Naples et en Catalogne, et
partout avec tant de sagesse, de modération et de succès, que sa
capacité, soutenue des amis de son père et de ceux que lui-même s'était
procurés, le fit choisir en 1665 pour l'ambassade de Suède. Il y demeura
trois ans, et passa après à celle de Hollande\,; il réussit si bien en
toutes deux qu'il fut renvoyé en Suède, où, combattu par tout l'art de
la maison d'Autriche, il vint à bout de conclure cette fameuse ligue du
Nord, si utile à la France en 1671. Le roi en fut si content qu'ayant
perdu peu de mois après M. de Lyonne, ministre et secrétaire d'État des
affaires étrangères, il ne crut pouvoir mieux remplacer un si grand
ministre que par Pomponne. Toutefois il en garda le secret, et ne le
manda qu'à lui par un billet de sa main, avec ordre d'achever en Suède,
le plus tôt qu'il pourrait, ce qui demandait nécessairement à l'être de
la même main, et de revenir incontinent après. Il arriva au bout de deux
mois, dans la même année 1671, et fut déclaré aussitôt. Son père, retiré
dès 1644, eut la joie de voir son fils arrivé par son mérite dans une
place si importante, et mourut trois ans après à quatre-vingt-cinq ans.
Pomponne parut encore plus digne de cette charge par la manière dont il
l'exerça, qu'avant d'en avoir été revêtu. C'était un homme qui excellait
surtout par un sens droit, juste, exquis, qui pesait tout et faisait
tout avec maturité, mais sans lenteur\,; d'une modestie, d'une
modération, d'une simplicité de mœurs admirables, et de la plus solide
et de la plus éclairée piété. Ses yeux montraient de la douceur et de
l'esprit\,; toute sa physionomie, de la sagesse et de la candeur\,; un
art, une dextérité, un talent singulier à prendre ses avantages en
traitant\,; une finesse, une souplesse sans ruse qui savait parvenir à
ses fins sans irriter\,; une douceur et une patience qui charmait dans
les affaires\,; et avec cela une fermeté, et, quand il le fallait, une
hauteur à soutenir l'intérêt de l'État et la grandeur de la couronne que
rien ne pouvait entamer. Avec ces qualités il se fit aimer de tous les
ministres étrangers comme il l'avait été dans les divers pays où il
avait négocié. Il en était également estimé et il en avait su gagner la
confiance. Poli, obligeant, et jamais ministre qu'en traitant, il se fit
adorer à la cour, où il mena une vie égale, unie, et toujours éloignée
du luxe et de l'épargne, et ne connaissant de délassement de son grand
travail qu'avec sa famille, ses amis et ses livres. La douceur et le sel
de son commerce étoient charmants, et ses conversations, sans qu'il le
voulût, infiniment instructives. Tout se faisait chez lui et par lui
avec ordre, et rien ne demeurait en arrière sans jamais altérer sa
tranquillité.

Ces qualités étaient en trop grand contraste avec celles de Colbert et
de Louvois pour en pouvoir être souffertes avec patience. Tous deux en
avaient sans doute de très grandes, mais, si elles paraissaient
quelquefois plus brillantes, elles n'étaient pas si aimables\,; et s'ils
avaient des amis, Pomponne avait aussi les siens particuliers, et
quoique moins puissant, peut-être en plus grand nombre, et de plus
qu'eux était généralement aimé. Chacun des deux autres tendait toujours
à embler la besogne d'autrui, et c'est ce qui les avait rendus ennemis
l'un de l'autre\,; tous deux voulaient, sous divers prétextes, manier
les affaires étrangères, et tous deux s'en trouvaient également,
sagement, mais doucement repoussés. Non seulement ils n'y purent jamais
surprendre la moindre prise, mais la grande connaissance qu'avait
Pomponne des affaires générales de l'Europe, et en particulier celle que
son application, ses voyages, ses négociations lui avaient acquise des
maisons, des ministres, des cours étrangères, de leurs intérêts et de
leurs ressorts, lui donnaient un tel avantage sur ces matières, que sans
sortir de sa modération et de sa douceur, ils n'osaient le contredire au
conseil ou devant le roi\,; il les avait souvent mis sans reparties
lorsqu'ils l'avaient hasardé. Hors de toute espérance d'embler rien sur
un homme si instruit et si sage, et qui se contentait de son ministère
sans leur donner jamais prise par vouloir empiéter sur le leur, ils
furent longtemps à chercher comment pouvoir entamer un homme si
difficile à prendre, et si insupportable à leur ambition vis-à-vis
d'eux. Ce désir de s'en délivrer, pour mettre en sa place quelqu'un qui
ne pût pas si bien se défendre, réunit pour un temps ces deux ennemis.
Ils se concertèrent\,; le jansénisme fut leur ressource. C'était en
effet le miracle du mérite de Pomponne que, fils, frère, neveu, cousin
germain et parent le plus proche, ou lié des nœuds les plus intimes avec
tout ce qu'on avait rendu le plus odieux au roi et en gros et
personnellement, il pût conserver ce ministre dans un poste de la
première confiance. Les deux autres allant toujours l'un après l'autre à
la sape, et s'aidant d'ailleurs de tout ce qui pouvait concourir à leur
dessein, s'aperçurent de leurs progrès sur l'esprit du roi. Ils le
poussèrent et vinrent enfin à bout de se faire faire un sacrifice sous
le prétexte de la religion. Ce ne fut pourtant pas sans une extrême
répugnance. Le roi, si parfaitement content de la gestion de Pomponne,
ne voyait en lui que mesure et sagesse sur tout ce qui regardait le
jansénisme. Il avait peine à se défier de lui, même sur ce point, et le
danger et le scandale de se servir du neveu de M. Arnauld dans ses
affaires les plus secrètes et les plus importantes ne lui paraissait
point en comparaison du danger et de la peine de s'en priver. À force
d'attaques continuelles il céda à la fin, et, comme la dernière goutte
d'eau est celle qui fait répandre le vase, un rien perdit M. de Pomponne
après tant d'assidues préparations. Ce fut en 1679.

On traitait le mariage de M\textsuperscript{me} la Dauphine, et on
attendait le courrier qui devait en apporter la conclusion. Dans ces
moments critiques Pomponne supputa, et crut qu'il aurait le temps
d'aller passer quelques jours à Pomponne. M\textsuperscript{me} de
Soubise était bien au fait de tout\,; c'était le temps florissant de sa
beauté et de sa faveur. Elle était amie de Pomponne, mais elle n'osait
s'expliquer\,; elle se contenta de le conjurer de remettre ce petit
voyage, et de l'avertir qu'elle voyait des nuages qui ne devaient pas
lui permettre de s'absenter\,; elle le pressa autant qu'il lui fut
possible. Les gens les plus parfaits ne sont pas sans défauts\,; il ne
put comprendre tout ce que M\textsuperscript{me} de Soubise voulait
qu'il entendît, ni avoir la complaisance de sacrifier ce petit voyage à
son conseil et à son amitié. Pomponne est à six lieues de Paris. Pendant
son absence, arriva le courrier de Bavière, et en même temps une lettre
à M. de Louvois qui avait ses gens partout\,: c'était la conclusion avec
le détail de tous les articles du traité et du mariage. Louvois va tout
aussitôt porter sa lettre au roi, qui s'étonne de n'avoir point de
nouvelles par ailleurs. Les dépêches de Pomponne étaient en chiffres, et
celui qui déchiffrait se trouva à l'Opéra, où il s'était allé divertir
en l'absence de son maître. Tandis que le temps se passe à l'Opéra, puis
à déchiffrer, et cependant à aller et à venir de Pomponne, Colbert et
Louvois ne perdirent pas de temps. Ils mirent le roi en impatience et en
colère, et s'en surent si bien servir que Pomponne en arrivant à Paris
trouva un ordre du roi de lui envoyer les dépêches et sa démission, et
de s'en retourner à Pomponne.

Ce grand coup frappé, Louvois, dont Colbert qui avait ses raisons avait
exigé de ne pas dire un mot de toute cette menée à son père, se hâta de
lui aller conter la menée et le succès\,: «\,Mais, lui répondit
froidement l'habile Le Tellier, avez-vous un homme tout prêt pour mettre
en cette place\,? --- Non, lui répondit son fils, on n'a songé qu'à se
défaire de celui qui y était, et maintenant la place vide ne manquera
pas, et il faut voir de qui la remplir. --- Vous n'êtes qu'un sot, mon
fils, avec tout votre esprit et vos vues, lui répliqua Le Tellier. M.
Colbert en sait plus que vous, et vous verrez qu'à l'heure qu'il est, il
sait le successeur, et il l'a proposé\,; vous serez pis qu'avec l'homme
que vous avez chassé, qui avec toutes ses bonnes parties n'était pas au
moins plus à M. Colbert qu'à vous. Je vous le répète, vous vous en
repentirez.\,» En effet, Colbert s'était assuré de la place pour son
frère Croissy, lors à Aix-la-Chapelle, comme je l'ai dit en rapportant
sa mort, et ce fut un coup de foudre pour Le Tellier et pour Louvois qui
les brouilla plus que jamais avec Colbert, et par une suite nécessaire
avec ce frère. Pomponne sentit sa chute et son vide, mais il le supporta
en homme de bien et de courage, avec tranquillité. Il eut peu après
liberté de venir et de demeurer à Paris. Aucun de ses amis ne le
délaissa, tout le monde prit part à sa disgrâce. Les étrangers en
regrettant sa personne qu'ils aimaient, et lui continuant toujours des
marques de considération dans les occasions qui s'en pouvaient
présenter, furent bien aises d'être soulagés de sa capacité.

Le roi après quelque temps voulut voir Pomponne par derrière dans ses
cabinets. Il le traita en prince qui le regrettait, et lui parla même de
ses affaires. De temps en temps, mais rarement cela se répétait, et
toujours sur le même pied de la part du roi. À la fin en une de ces
audiences, le roi lui témoigna la peine qu'il avait ressentie en
l'éloignant, et qu'il ressentait encore, et Pomponne y ayant répondu
avec le respect et l'affection qu'il devait, le roi continua à lui
parler avec beaucoup d'estime et d'amitié. Il lui dit qu'il avait
toujours envie de le rapprocher de lui, qu'il ne le pouvait encore, mais
qu'il lui demandait sa parole de ne point s'excuser, et de revenir dans
son conseil dès qu'il le manderait, et en attendant de lui garder le
secret de ce qu'il lui disait. Pomponne le lui promit, et le roi
l'embrassa. L'événement a fait voir ce que le roi pensait alors. C'était
de se défaire de M. de Louvois en l'envoyant à la Bastille. La
parenthèse, en serait déplacée ici, je pourrai avoir lieu ailleurs de
raconter un fait si curieux. Dans le moment que ce ministre fut mort, le
roi écrivit de sa propre main à Pomponne de revenir sur-le-champ prendre
sa place dans ses conseils. Un gentilhomme ordinaire du roi fut chargé
en secret de ce message par le roi même. Il trouva cet illustre
disgracié à Pomponne qui s'allait mettre au lit. Le lendemain matin il
vint à Versailles, et débarquer chez Bontems qui le mena par les
derrières chez le roi. On peut juger des grâces de cette audience. Le
roi ne dédaigna pas de lui faire des excuses de l'avoir éloigné et de
l'avoir rapproché si tard\,; il ajouta qu'il craignait qu'il n'eût peine
à voir Croissy faire les fonctions qu'il avait si dignement remplies.
Pomponne, toujours modeste, doux, homme de bien, répondit au roi, que
puisqu'il le voulait rattacher à son service, et qu'il s'était engagé à
lui d'y rentrer, il ne songerait qu'à le bien servir, et que pour bien
commencer, et ôter, en tant qu'en lui était, toutes les occasions de
jalousie, il s'en allait de ce pas chez Croissy lui apprendre les bontés
du roi, et lui demander son amitié. Le roi, touché au dernier point
d'une action si peu attendue, l'embrassa et le congédia. La surprise de
Croissy fut sans pareille quand il s'entendit annoncer M. de Pomponne.
On peut juger qu'elle ne diminua pas quand il apprit ce qui l'amenait.
Celle de la cour, qui n'avait pas songé à un retour après douze années
de disgrâce et qui n'en avait pas eu le moindre vent, fut grande aussi,
mais mêlée de beaucoup de joie\,; il entra au premier conseil qui se
tint, et M. de Beauvilliers en même temps.

Pomponne, dès le même jour, eut un logement au château assez grand, et
vécut avec toutes sortes de mesures et de prévenances avec Croissy, qui
y répondit de son côté, et qui avait bien compris qu'il fallait le
faire. Leur alliance que le roi voulut, je l'ai racontée en son temps.
Pomponne et son gendre vécurent ensemble en vrai père et en véritable
fils. Il y trouva tout ce qu'il pouvait désirer pour devenir un bon et
sage ministre. Il y ajouta du sien toutes les lumières et toute
l'instruction qu'il put, dont Torcy sut bien profiter. M. de Pomponne
lia une amitié étroite avec M. de Beauvilliers. La confiance était
intime entre eux et avec le duc de Chevreuse. Il fut aussi fort uni avec
Pelletier, et honnêtement avec les autres ministres ou secrétaires
d'État. Il mourut le 26 septembre de cette année, à Fontainebleau, à
quatre-vingt-un ans, dans le désir depuis longtemps de la retraite, que
l'état de sa famille ne lui avait pas encore permise. Sa tête et sa
santé étaient entières. Il n'avait jamais été malade\,; il mangea un
soir du veau froid et force pêches\,; il en eut une indigestion qui
l'emporta en quatre jours. Il reçut ses sacrements avec une grande
piété, et fit une fin aussi édifiante que sa vie. Torcy, son gendre, eut
les postes, et sa veuve douze mille livres de pension. C'était une femme
avare et obscure qu'on ne voyait guère. Elle avait une sœur charmante
par son esprit, par ses grâces, par sa beauté, par sa vertu, femme de M.
de Vins, qui était lieutenant général et qui eut les mousquetaires
noirs. Ils avaient un fils unique, beau, aimable, spirituel comme la
mère, avec qui j'avais été élevé. M. de Pomponne était ami particulier
de mon père, et ils logeaient chez lui. Ce jeune homme fut tué à
Steinkerque, à sa première campagne. Le père et surtout la mère ne s'en
sont jamais consolés\,; elle n'a presque plus voulu voir personne
depuis, absorbée dans la douleur et dans la piété tout le reste de sa
longue vie. Je regrettai extrêmement son fils. M. de Pomponne ne fut pas
heureux dans ceux qui se destinèrent au monde. Le cadet, qui promettait,
fut tué de bonne heure à la tête d'un régiment de dragons. L'aîné,
épais, extraordinaire, avare, obscur, quitta le service, devint
apoplectique, et fut toute sa vie compté pour rien jusque dans sa
famille. L'abbé de Pomponne fut aumônier du roi. Il se retrouvera
occasion d'en parler.

Le roi revint de Fontainebleau et nomma Briord ambassadeur à la Haye en
la place de Bonrepos, qui demanda à revenir\,; et Phélypeaux, lieutenant
général, qui était à Cologne, ambassadeur à Turin\,; Bonac, neveu de
Bonrepos, alla à Cologne.

\hypertarget{chapitre-xx.}{%
\chapter{CHAPITRE XX.}\label{chapitre-xx.}}

1699

~

{\textsc{Mgr et M\textsuperscript{me} la duchesse de Bourgogne mis
ensemble.}} {\textsc{- Menins de Mgr le duc de Bourgogne\,: Gamaches,
d'O, Cheverny, Saumery.}} {\textsc{- M\textsuperscript{me} de Saumery.}}
{\textsc{- Emplois de Cheverny et son aventure à Vienne.}} {\textsc{-
Mort de M\textsuperscript{me} de Montchevreuil.}} {\textsc{- Mgr le duc
de Bourgogne entre au conseil des dépêches.}} {\textsc{- Castel dos
Rios, ambassadeur d'Espagne en France.}} {\textsc{- Mort d'Arrouy dans
la Bastille.}} {\textsc{- Voyage à Paris du duc et de
M\textsuperscript{me} la duchesse de Lorraine pour l'hommage lige de
Bar.}} {\textsc{- Ducs de Lorraine, l'un connétable, l'autre grand
chambellan.}} {\textsc{- Princes du sang précédent les souverains non
rois partout.}} {\textsc{- M. de Lorraine étrangement incognito.}}
{\textsc{- M\textsuperscript{me} et M. de Lorraine à Paris, qui va
saluer le roi.}} {\textsc{- Adresse continuelle à l'égard de M. et
M\textsuperscript{me} la duchesse de Chartres.}} {\textsc{-
M\textsuperscript{me} de Lorraine malade de la petite vérole.}}
{\textsc{- Hommage lige au roi par le duc de Lorraine pour le duché de
Bar.}} {\textsc{- M. de Lorraine à Meudon et à Marly, où il prend
congé.}} {\textsc{- M. de Lorraine prend congé de Monseigneur à l'Opéra,
et de Mgr et de M\textsuperscript{me} la duchesse de Bourgogne sans les
avoir vus auparavant, et part en poste payée par le roi.}} {\textsc{-
M\textsuperscript{me} de Lorraine à Versailles, puis à Marly, prend
congé et part.}}

~

En arrivant de Fontainebleau le jour même, Mgr et M\textsuperscript{me}
la duchesse de Bourgogne furent mis ensemble. Le roi les voulut aller
surprendre comme ils se mettraient au lit\,; il s'y prit un peu trop
tard, il trouva les portes fermées, et il ne voulut pas les faire
ouvrir. Peu de jours après, il nomma quatre hommes qui étaient souvent à
la cour pour se tenir assidus auprès de Mgr le duc de Bourgogne, qui
dans la vérité ne pouvaient guère être plus mal choisis, Cheverny,
Saumery, Gamaches et d'O. Des deux derniers j'en ai parlé assez pour
n'avoir rien à y ajouter. Le bon Gamaches était un bavard qui n'avait
jamais su ce qu'il disait ni ce qu'il faisait, et dont M. de Chartres et
ses amis de plaisir s'étaient moqués tant que le roi l'avait tenu auprès
de lui. Il ne savait rien, pas même la cour ni le monde où il avait fort
peu été, ni la guerre non plus, quoiqu'il eût toujours servi et avec
beaucoup d'honneur et de valeur\,; du reste un fort honnête homme. D'O
était ce mentor de M. le comte de Toulouse, qui de son appartement de
Versailles devint lieutenant général des armées navales. Son assiduité
chez son premier maître était difficile à accorder avec cet emploi, mais
il savait accorder toutes choses, témoin sa dévotion importante et le
galant métier de sa femme pour faire fortune par l'un des deux, et
peut-être par tous les deux ensemble.

Cheverny était Clermont-Gallerande\,; son père avait été maître de la
garde-robe et chevalier de l'ordre en 1661, dont on a d'excellents
Mémoires en forme d'annales, sous le nom de Monglat qu'il portait. Sa
femme, fille du fils du chancelier de Cheverny, était une femme
extrêmement du grand monde, qui avait été gouvernante des filles de
Gaston, et sur le pied de laquelle il ne faisait pas bon marcher. L'un
et l'autre fort riches s'étaient parfaitement ruinés, et avaient marié
leur fils à la sœur de Saumery. C'était un homme qui présentait plus
d'esprit, de morale, de sens et de sentiments qu'il n'en avait en
effet\,; beaucoup de lecture, peu ou point de service, une conversation
agréable et fournie, beaucoup de politique, d'envie de plaire et de
crainte de déplaire, un extérieur vilain et même dégoûtant, toute
l'encolure d'un maître à écrire, et toujours mis comme s'il l'eût été,
en tout un air souffreteux, et une soif de cour et des agréments de cour
qui allait à la bassesse\,; avec tout cela ce tuf se cachait sous
d'autres apparences, et j'en ai été la dupe fort longtemps\,; d'ailleurs
un honnête homme.

Saumery était petit-fils d'un valet d'Henri IV, qui l'avait suivi du
Béarn, et qui, comme beaucoup de ce peuple, s'appelait Johanne. Il fut
jardinier de Chambord, et sur la fin de sa vie concierge, non pas de ces
concierges gouverneurs et capitaines comme il y en a toujours eu à
Fontainebleau et à Compiègne, mais concierge effectif, comme nous en
avons tous dans nos maisons. Il gagna du bien\,; il mit son fils dans
les troupes, qui était fort bien fait, et trouva à le marier à une
bourgeoise de Blois à sa portée. M. Colbert, encore \emph{in minoribus},
épousa l'autre sœur. Sa fortune avança ses beaux-frères. L'un
s'enrichit, acheta Menars, devint intendant de Paris, et est mort
président à mortier\,; il était frère de M\textsuperscript{me} Colbert.
Saumery devint gouverneur de Chambord, en eut la capitainerie des
chasses et celle de Blois\,: c'était un fort honnête homme et qui ne
s'en faisait point du tout accroire\,; il se tenait à Chambord, où il
est mort fort vieux, et paraissait rarement à la cour, où on en faisait
cas pour sa valeur et sa probité. Je l'ai vu, il était fort grand, avec
ses cheveux blancs et l'air tout à fait vénérable. Son fils aîné, qui
est celui dont il s'agit, servit quelque temps subalterne, et se retira
de bonne heure avec un coup de mousquet dans le genou, et se fit maître
des eaux et forêts d'Orléans, etc. Il était dans cet obscur emploi et
inconnu à tout le monde, lorsque M. de Beauvilliers l'en tira pour le
faire un des sous-gouverneurs des enfants de France. Jamais homme si
intrigant, si valet, si bas, si orgueilleux, si ambitieux, si dévoué à
la fortune, et tout cela sans fond aucun, sans voile, sans pudeur\,; on
en verra d'étranges traits. Jamais homme aussi ne tira tant parti d'une
blessure. Je disais de lui qu'il boitait audacieusement, et il était
vrai. Il parlait des personnages les plus distingués, dont à peine il
avait jamais vu les antichambres, comme de ses égaux et de ses amis
particuliers. Il racontait des traits qu'il avait ouï dire, et n'avait
pas honte de dire devant des gens qui avaient au moins le sens commun\,:
«\,Le pauvre mons Turenne me disait,\,» qui, à son âge et à son petit
emploi, n'a peut-être jamais su qu'il fût au monde, et le monsieur tout
du long, il n'en honorait personne. C'était mons de Beauvilliers, mons
de Chevreuse, et ainsi de ceux dont il ne disait pas le nom tout court,
et il le disait de presque tout le monde jusqu'aux princes du sang. Je
lui ai ouï dire bien des fois «\,la princesse de Conti\,» en parlant de
la fille du roi, et «\,le prince de Conti\,» en parlant de M. son
beau-frère. Pour les premiers seigneurs de la cour, il était rare quand
il leur donnait le \emph{monsieur} ou le \emph{mons}. C'était\,: «\,le
maréchal d'Humières,\,» et ainsi des autres, et des gens de la première
qualité, très ordinairement par leur nom, sans qualité devant. La
fatuité et l'insolence étaient complètes\,; et si, à force de monter
cent escaliers par jour, de dire des riens à l'oreille, de faire
l'important et le gros dos, il imposait à une partie de la cour, et, par
ses valetages et ses blâmes de complaisance bien bas en confidence, il
s'était acquis je ne sais combien de gens\footnote{Voy., à la fin du
  volume, la note rectificative de M. le marquis de Saumery.}.

Sa femme, fille de Besmaux, gouverneur de la Bastille, était une grande
créature aussi impertinente que lui, qui portait les chausses, et devant
qui il n'osait pas souffler. Son effronterie ne rougissait de rien, et
après force galanteries, elle s'était accrochée à M. de Duras qu'elle
gouvernait, et chez qui elle était absolument et publiquement la
maîtresse, et vivait à ses dépens. Elle en acquit le nom de
M\textsuperscript{me} la connétable, parce que M. de Duras était doyen
des maréchaux de France. On ne l'appelait pas autrement\,: elle-même
était la première à en rire. Enfants, complaisants, domestiques, tout
était en respect et en dépendance devant elle, et M\textsuperscript{me}
de Duras aussi dans le peu et le rare qu'elle venait de sa campagne\,;
l'âge du maréchal faisait qu'on s'en scandalisait moins. Voilà les gens
que le roi mit autour de Mgr le duc de Bourgogne, qui chas soit fort
souvent\,; et de ces quatre il n'y avait que Gamaches qui pût monter à
cheval, ou qui en voulût prendre la peine. Le rare fut qu'ils n'eurent
ni nom d'emploi, ni brevet, ni appointements, mais de beaux propos en
les y mettant, et l'agrément d'être, sans demander, de tous les voyages
de Marly, et cela seul tournait les têtes.

Cheverny était menin de Monseigneur\,: il avait été envoyé à Vienne, et
ambassadeur après en Danemark, où lui et sa femme avaient gagné le
scorbut et laissé leur santé et leurs dents. La femme, avec plus
d'esprit et de mesure, ne tenait pas mal de son frère. À Vienne il
arriva à Cheverny une aventure singulière. Il devait avoir un soir
d'hiver sa première audience de l'empereur. Il alla au palais\,; un
chambellan l'y reçut, le conduisit deux ou trois pièces, ouvrit la
dernière, l'y fit entrer, se retira de la porte même et la ferma. Entré
là, il se trouve dans une pièce plus longue que large, mal meublée, avec
une table tout au bout\,; sur laquelle, pour toute lumière dans la
chambre, il y avait deux bougies jaunes, et un homme vêtu de noir, le
dos appuyé contre la table. Cheverny assez mal édifié du lieu, se croit
dans une pièce destinée à attendre d'être introduit plus loin, et se met
à regarder à droite et à gauche, et à se promener d'un bout à l'autre.
Ce passe-temps dura près d'une demi-heure. À la fin, comme un des tours
de sa promenade l'approchait assez de cette table, et de cet homme noir
qui y était appuyé, et qu'à son air et à son habit il prit pour un valet
de chambre qui était là de garde, cet homme qui jusqu'alors l'avait
laissé en toute liberté sans remuer ni dire un mot, se prit à lui
demander civilement ce qu'il faisait là. Cheverny lui répondit qu'il
devait avoir audience de l'empereur, qu'on l'avait fait entrer, et qu'il
attendait là d'être introduit pour avoir l'honneur de lui faire sa
révérence. «\,C'est moi, lui répliqua cet homme, qui suis l'empereur.\,»
Cheverny à ce mot pensa tomber à la renverse, et fut plusieurs moments à
se remettre, à ce que je lui ai ouï conter. Il se jeta aux pardons, à
l'obscurité, et à tout ce qu'il put trouver d'excuses. Je pense après
que son compliment fut mal arrangé. Un autre que l'empereur en eût ri,
mais Léopold, incapable de perdre sa gravité, demeura dans le même
sang-froid, qui acheva de démonter le pauvre Cheverny. Il contait bien,
et cette histoire était excellente à entendre de lui.

M\textsuperscript{me} de Montchevreuil revenant de Fontainebleau le même
jour que le roi, 22 octobre, avec M\textsuperscript{me} de Maintenon,
dans le carrosse et en compagnie de laquelle elle allait toujours, se
trouva si mal au Plessis qu'il y fallut arrêter longtemps. On eut toutes
les peines du monde à l'amener à Versailles, où elle mourut le quatrième
jour. M\textsuperscript{me} de Maintenon en fut fort affligée, beaucoup
de gens tâchèrent de persuader qu'ils l'étaient, mais dans le fond
chacun s'en trouva soulagé comme d'une délivrance. J'ai suffisamment
parlé de M. et de M\textsuperscript{me} de Montchevreuil à propos du
mariage de M. du Maine, pour n'avoir rien à y ajouter. Quelques jours
après, le roi vit le bonhomme Montchevreuil dans son cabinet par les
derrières, par où comme gouverneur autrefois de M. du Maine il
continuait d'entrer. Le roi le traita comme un ami intime aurait fait
son ami. À la situation où il était avec lui, cela n'était pas
surprenant.

Ce même jour de la mort de M\textsuperscript{me} de Montchevreuil, 25
octobre, le roi dit le soir à Mgr le duc de Bourgogne qu'il le ferait
entrer au premier conseil de dépêches\,; et ajouta que, pour les
premiers, il voulait qu'il ne fît qu'écouter pour apprendre et se
former, pour se mettre en état de bien opiner ensuite. Ce fut une grande
joie pour ce prince\,; Monseigneur n'y était pas entré si jeune.
Monsieur en était, mais il en était resté là.

Castel dos Rios, gentilhomme catalan, fort pauvre, était arrivé à Paris
au commencement du voyage de Fontainebleau, avec caractère d'ambassadeur
d'Espagne\,; il avait été nommé pour aller en la même qualité en
Portugal, mais il arriva que celui qui devait venir en France étant plus
distingué et beaucoup plus accrédité à la cour d'Espagne, il fit changer
la destination, et alla en Portugal comme à une ambassade de faveur, et
fit envoyer l'autre à celle d'exil. C'est ainsi qu'elle était regardée.
Il voulut venir à Fontainebleau trouver la cour et en fut refusé\,; il
s'en plaignit fort, on lui répondit qu'on avait bien fait attendre M.
d'Harcourt trois mois à Madrid avant de lui permettre de voir le roi
d'Espagne\,; qu'ainsi il pouvait bien avoir patience six semaines avant
de voir le roi. Au retour il eut audience. Ce qu'il avait à y traiter
était en effet d'une importance à ne pas souffrir volontiers des
délais\,; il pressa le roi de deux choses de la part du roi son
maître\,: l'une, d'employer son autorité pour faire révoquer à la
Sorbonne la condamnation qu'elle avait faite des livres d'une béate
espagnole qui s'appelle Marie d'Agreda. Le temps était mal pris\,; ces
livres étaient tout à fait dans les sentiments de M. de Cambrai, que le
roi venait de faire condamner à Rome. L'autre chose était de faire
établir en dogme, par tout son royaume, l'immaculée conception de la
sainte Vierge, et par conséquent faire plus que l'Église, qui a été plus
retenue là-dessus\,; aussi se moqua-t-on de l'ambassadeur et de son
maître avec les plus belles paroles du monde. Ce fut là toute la matière
de son audience. Qui aurait cru que cette ambassade eût tourné quatorze
mois après comme elle fit, et que cette espèce d'exil eût fait à
l'ambassadeur la fortune la plus complète.

Le pauvre d'Arrouy\footnote{On écrit ordinairement d'Harouis.
  M\textsuperscript{me} de Sévigné, qui parle souvent de ce trésorier
  des états de Bretagne, annonce sa ruine à sa fille dans une lettre du
  19 février 1690\,: «\,La déroute de notre pauvre d'Harouis est bien
  plus aisée à comprendre\,; passionné de faire plaisir à tout le monde,
  sans mesure, sans raison\,; cette passion offusquant toutes les
  autres, et même la justice\,: voilà un autre prodige.\,»} mourut en ce
temps-ci à la Bastille, où il était depuis dix ou douze ans\,; il avait
été longtemps trésorier des états de Bretagne. C'était le meilleur homme
du monde et le plus obligeant\,; il ne savait que prêter de l'argent et
point presser pour se faire payer\,; avec cette conduite il s'obéra si
bien, que, quand il fallut compter, il ne put jamais se tirer d'affaire.
La confiance de la province et de tout le monde était si grande en lui,
qu'on l'avait laissé plusieurs années sans compter\,: ce fut sa ruine.
Beaucoup de gens y perdirent gros. La Bretagne y demeura pour beaucoup,
et il demeura entièrement ruiné. C'est, je crois, l'unique exemple d'un
comptable de deniers publics avec qui ses maîtres et tout le public
perdent, sans que sa probité en ait reçu le plus léger soupçon. Les
perdants mêmes le plaignirent, tout le monde s'affligea de son
malheur\,; c'est ce qui fit que le roi se contenta d'une prison
perpétuelle. Il la souffrit sans se plaindre et la passa dans une grande
piété, fort visité de beaucoup d'amis et secouru de plusieurs. Cela
n'empêcha pas son fils de devenir maître des requêtes et intendant de
province, avec réputation d'esprit et de probité. Il se fit aimer et
estimer, et il aurait été plus loin, si la piété tant de lui que de sa
femme dont il n'avait point d'enfants, ne les avait engagés à tout
quitter pour ne penser qu'à leur salut. J'ai fort vu cette
M\textsuperscript{me} d'Arrouy à Pontchartrain, qui avait beaucoup
d'esprit, et un esprit très aimable et orné, extrêmement dans les
meilleures œuvres, et extrêmement janséniste. Je me suis souvent fort
diverti à disputer avec elle. J'étais ravi quand je l'y trouvais.

On attendait au retour de Fontainebleau M. de Lorraine pour rendre au
roi son hommage lige du duché de Bar et de ses autres terres mouvantes
de la couronne. M\textsuperscript{me} la Duchesse devait venir avec lui,
et Monsieur les défraya à Paris et leur donna, au Palais-Royal,
l'appartement de M. et de M\textsuperscript{me} la duchesse de Chartres.
Nul embarras pour M\textsuperscript{me} de Lorraine, qui conservait son
rang de petite-fille de France. Il n'y en devait pas avoir non plus pour
M. de Lorraine. Ses pères, ducs de Lorraine comme lui, ont été bien des
fois à la cour de France sans difficultés.

Charles Ier, duc de Lorraine, fut fait connétable de France après la
mort ou plutôt le massacre du connétable d'Armagnac, le 12 juin 1418,
dans Paris, par le parti de Bourgogne. Il est vrai qu'il n'en jouit pas
longtemps, pour avoir été institué par cette terrible Isabeau de
Bavière, femme de Charles VI, qui, dans un intervalle de sa triste
maladie, le destitua à Bourges en avril 1423\footnote{C'est évidemment
  par erreur que Saint-Simon attribue à Charles VI, mort en 1422, la
  nomination de J. Stuart, comte de Buckan ou Boucan, qui n'eut lieu
  qu'en avril 1423-1424. Ce fut Charles VII qui destitua Charles de
  Lorraine et le remplaça par J. Stuart.}, et donna l'épée de connétable
à J. Stuart, comte de Boucan et de Douglas, qui fut tué le 17 août
suivant à la bataille de Verneuil au Perche contre les Anglais. Le comte
de Richemont fut fait connétable en sa place\,; il était fils, frère et
oncle des ducs de Bretagne, et le fut lui-même après eux en 1457, et
voulut conserver l'épée de connétable avec laquelle il avait acquis tant
de gloire, et mourut duc de Bretagne et connétable de France en décembre
1458 dans son château de Nantes, portant lors le nom d'Artus III. René
II, duc de Lorraine, fut fait grand chambellan, 7 août 1486, par Charles
VIII, qui avait alors seize ans, à la place du comte de Longueville,
fils du célèbre bâtard d'Orléans, destitué et ses terres confisquées
pour avoir pris le parti du duc d'Orléans, depuis roi Louis XII, contre
M\textsuperscript{me} de Beaujeu, sœur du roi, sa tutrice et gouvernante
du royaume. Le duc de Lorraine ne demeura pas longtemps grand chambellan
de France. Il se ligua avec le même duc d'Orléans contre le roi, et
Philippe de Bade, marquis d'Hochberg et comte de Neuchâtel, fut pourvu
en sa place de cet office de la couronne en 1491.

Sans aller si loin, Louis XIII et le roi son fils ont vu Charles IV, duc
de Lorraine, plus d'une fois en leur cour et y faire des séjours, et la
duchesse Nicole a passé à Paris ses dernières années. La planche était
donc faite, et il n'y avait qu'à la suivre. On y peut ajouter que le
père du duc de Lorraine a été aussi à Paris et à la cour, mais il s'y
arrêta peu, quoique assez pour continuer les exemples et régler
celui-ci. Mais cela même était ce qui incommodait les cadets de sa
maison établis en France, qui, tirant leurs prétentions de leur
naissance, avaient grand intérêt de relever leur aîné, et grande
facilité par Monsieur, entièrement abandonné au chevalier de Lorraine,
jusqu'au point où je l'ai remarqué au mariage de M\textsuperscript{me}
de Lorraine. Des gens qui avaient osé vouloir élever leur aîné jusqu'en
compétence de M. le duc de Chartres n'étaient pas pour s'accommoder de
celle des princes du sang, et ceux-ci encore moins pour la souffrir.
Jamais aucun duc de Lorraine ne leur avait disputé, pas même le père de
celui-ci, beau-frère de l'empereur et à la tête de son armée, aux deux
princes de Conti, volontaires dans la même armée, auquel l'électeur de
Bavière qui y servait ne disputait pas. Le second de ces princes était
vivant et existant à la cour, et cet électeur était frère de
M\textsuperscript{me} la Dauphine alors vivante, et gendre de
l'empereur. On n'avait pas oublié encore comment le fameux
Charles-Emmanuel, duc de Savoie, gendre de Philippe II, roi d'Espagne,
et qui fit tant de figure en Europe, avait vécu avec les princes du
sang, ni le célèbre mot d'Henri IV là-dessus. Charles-Emmanuel l'était
venu trouver à Lyon pour arrêter ses armes, après avoir séjourné
longtemps à sa cour à Paris, dans l'espérance de le tromper sur la
restitution du marquisat de Saluces. Il se trouva qu'un matin, venant au
lever d'Henri IV, le prince de Condé et lui qui venaient par différents
côtés, se rencontrèrent en même temps à la porte de la chambre où le roi
s'habillait. Ils s'arrêtèrent l'un pour l'autre\,: Henri IV, qui les
vit, éleva la voix et dit au prince de Condé\,: «\,Passez, passez, mon
neveu, M. de Savoie sait trop ce qu'il vous doit. » Le prince de Condé
passa, et M. de Savoie tout de suite et sans difficulté après lui.

Ces considérations firent proposer un biais qui comblait les vues et les
prétentions des Lorrains contre les princes du sang, et ce biais fut
l'incognito parfait de M. de Lorraine, qui aplanissait et voilait tout
en même temps. Mais cet incognito était aussi parfaitement ridicule\,;
incognito tandis que M\textsuperscript{me} la duchesse de Lorraine n'y
pouvait être\,; incognito, et être publiquement logé, traité et défrayé
par Monsieur dans le Palais-Royal aux yeux de toute la France,
incognito, venant exprès pour un acte dans lequel il fallait qu'il fût
publiquement connu et à découvert\,; incognito enfin, sans cause ni
prétexte, puisque ses pères avaient été publiquement à la cour et à
Paris, et son père même\,: aussi prirent-ils un habile détour pour le
faire passer. Monsieur, en le proposant au roi, ne manqua pas de bien
faire les honneurs de son gendre, de l'assurer qu'il était bien éloigné
de disputer rien aux princes du sang\,; que, venu pour son hommage et
ayant son pays enclavé et comme sous la domination du roi, il ne pouvait
songer qu'à lui plaire et à obéir sans réserve à tout ce qu'il lui
plairait de lui commander\,; mais que lui, Monsieur, croyait lui devoir
faire faire la réflexion qu'ayant donné le rang de prince du sang à ses
enfants naturels, il ne voudrait peut-être pas exiger pour eux les mêmes
déférences de M. de Lorraine que pour les princes du sang\,; qu'il
répugnerait à sa générosité, en étant le maître de l'y obliger, et que
ne l'y obligeant pas, cela mettrait une différence entre eux qui ne leur
serait pas avantageuse. Ce propos humble et flatteur, qui dans le fond
n'avait que la superficie, éblouit le roi et le toucha si bien qu'il
consentit à l'incognito, moyennant lequel nulles visites actives ni
passives pour M. de Lorraine, et nuls honneurs dus ou prétendus. Tout
allait à la petite-fille de France, son épouse, et se confondait sous
son nom\,; après quoi ils demeuraient sur leurs pieds, avec cet
incognito, en liberté de l'expliquer avec tous les avantages qu'ils s'en
étaient bien proposés. Ce grand point gagné, tout le reste leur fut
facile.

Le vendredi 20 novembre, Monsieur et Madame allèrent à Bondy au-devant
de M. et de M\textsuperscript{me} de Lorraine, qui tous deux se mirent
sur le devant de son carrosse. On remarqua avec scandale que M. le duc
de Chartres était à la portière. On débita que le devant lui faisait
mal. Cela aurait pu s'éviter, mais ce n'était pas le compte de la maison
de Lorraine, qui fit en sorte que M\textsuperscript{me} la duchesse de
Chartres demeurât à Versailles, avec laquelle il n'eût pas été si aisé
de bricoler. Ils furent à l'Opéra, dans la loge de Monsieur, qui retint
à souper toutes les princesses de la maison de Lorraine, avec d'autres
dames. Le lendemain samedi, Monsieur amena M. de Lorraine à Versailles.
Ils arrivèrent un moment avant midi dans le salon. Nyert, valet de
chambre en quartier, avertit le roi qui était au conseil et qui avait la
goutte. Il se fit aussitôt rouler par lui dans sa chaise. Il n'y avait
dans le salon qu'eux trois, et la porte du cabinet était demeurée
ouverte, d'où les ministres les voyaient. M. de Lorraine embrassa les
genoux du roi baissé fort bas, et fut reçu fort gracieusement, mais sans
être embrassé. La conversation dura un bon quart d'heure, pendant
laquelle Monsieur alla causer une fois ou deux à la porte du cabinet
avec les ministres, pour laisser M. de Lorraine seul avec le roi.
Monsieur lui demanda ensuite permission que le lord Carlingford et un ou
deux hommes principaux de M. de Lorraine pussent entrer et lui faire la
révérence. Alors le duc de Gesvres, premier gentilhomme de la chambre en
année, M. le maréchal de Lorges, capitaine des gardes en quartier, et
quelques principaux courtisans, entrèrent avec les gens de M. de
Lorraine, mais aucun de sa maison, et je n'ai pu en découvrir la raison.
Monsieur demanda ensuite au roi s'il trouvait bon qu'il fît voir son
petit appartement à M. de Lorraine, à qui il nomma les ministres en
passant dans le cabinet du conseil. Du petit appartement ils entrèrent
dans la grande galerie, où ils furent assez longtemps, et où M. de
Lorraine vit M\textsuperscript{me} la duchesse de Bourgogne qui revenait
de la messe, mais sans l'approcher. De là Monsieur le mena dîner à
Saint-Cloud, où M\textsuperscript{me} de Lorraine ne put se trouver,
parce que la fièvre l'avait prise. Seignelay, maître de la garde-robe,
alla le lendemain matin savoir de ses nouvelles de la part du roi, qui
rapporta que ce n'était rien, et qu'elle viendrait à Versailles le mardi
suivant. M\textsuperscript{me} la duchesse de Chartres l'avait été voir
de Versailles le jour de son arrivée, et MM. les ducs d'Anjou et de
Berry l'allèrent voir le dimanche après dîner. Elle leur donna des
fauteuils où ils s'assirent, et elle prit un tabouret, comme de raison.
Mgr le duc de Bourgogne ni Monseigneur n'y furent point\,; on laissa
aller les cadets comme par galanterie. Le père et le fils étaient ce
jour-là à Meudon, ce que je remarque pour la courte distance de Paris,
d'où leur visite eût été plus aisée s'ils l'avaient voulu faire. Le
mardi, M\textsuperscript{me} de Lorraine devait venir à Versailles dîner
chez M\textsuperscript{me} la duchesse de Chartres, puis aller chez le
roi, etc., et M\textsuperscript{me} la duchesse de Bourgogne après
l'avoir vue, c'est-à-dire reçue, aller à l'Opéra à Paris. Mais la petite
vérole qui parut rompit les voyages. M. le duc de Chartres le vint dire
au roi. Monsieur, M. de Lorraine, ni personne ne la vit que Madame, qui
s'enferma presque seule avec elle, et M\textsuperscript{me} de
Lenoncourt, dame d'atours de M\textsuperscript{me} de Lorraine, seule
dame qu'elle eût amenée, qui gagna la petite vérole, et qui en fut fort
mal.

J'achèverai de suite pour ne point interrompre la narration du voyage de
M. de Lorraine, quoique ce fît ici le lieu de le faire pour raconter ce
qui m'arriva, ce que je ferai après. Le mercredi 25 novembre, jour
marqué pour l'hommage, Monsieur amena M. de Lorraine à Versailles, qui
en mettant pied à terre s'en fut attendre chez M. le Grand, et Monsieur
monta tout droit chez le roi. M. le duc de Chartres ne vint point avec
eux. Monsieur avait eu soin de l'éviter pour plaire au chevalier de
Lorraine. Un peu après que Monsieur fut chez le roi, Monsieur envoya
dire à M. de Lorraine d'y venir\,: c'était vers les trois heures après
midi. Il fut suivi de tous ceux de ses sujets qui l'avaient accompagné
dans son voyage, et passa toujours entre une double haie de voyeur et de
curieux de bas étage. Il traversa les salles des gardes sans qu'ils
fissent aucun mouvement, non plus que pour le dernier particulier. Le
roi l'attendait dans le salon, qui était lors entre sa chambre et le
cabinet du conseil, et qui depuis est devenu sa chambre. Il était dans
son fauteuil le chapeau sur la tête, M. le maréchal de Lorges derrière
lui, au milieu de M. le chancelier et du duc de Gesvres, en l'absence de
M. de Bouillon, grand chambellan, qui était à Évreux\,; Mgr le duc de
Bourgogne debout et découvert, un peu en avant de M. le chancelier, mais
sans le couvrir\,; M. le duc d'Anjou de même, de l'autre côté, sans
couvrir le duc de Gesvres, qui avait derrière lui Nyert, premier valet
de chambre du roi. M. le duc de Berry, Monsieur, M. le duc de Chartres,
les princes du sang et les deux bâtards étaient tous en rang, faisant le
demi-cercle, avec force courtisans derrière eux, et après eux. Aucun duc
que les deux que je viens de nommer, parce qu'ils étaient en fonction de
leurs charges et nécessaires, ni aucun prince étranger. Les secrétaires
d'État étaient derrière M. le chancelier et les princes du même côté.
Monseigneur ne se soucia pas de voir la cérémonie.

M. de Lorraine trouva fermée la porte de la chambre du roi qui entre
dans le salon, et l'huissier en dedans. Un de la suite de M. de Lorraine
gratta, l'huissier demanda\,: «\,Qui est-ce\,?» Le gratteur répondit\,:
«\,C'est M. le duc de Lorraine,\,» et la porte demeura fermée. Quelques
instants après, même cérémonie. La troisième fois le gratteur
répondit\,: «\,C'est M. le duc de Bar\,; » alors l'huissier ouvrit un
seul battant de la porte. M. de Lorraine entra, et de la porte, puis du
milieu de la chambre, enfin assez près du roi, il fit de très profondes
révérences. Le roi ne branla point, et demeura couvert sans faire aucune
sorte de mouvement. Le duc de Gesvres alors suivi de Nyert, mais ayant
son chapeau sous le bras, s'avança deux ou trois pas, et prit le
chapeau, les gants et l'épée de M. de Lorraine qu'il lui remit, et le
duc de Gesvres tout de suite à Nyert qui demeura en place, mais fort en
arrière de M. de Lorraine, et le duc de Gesvres se remit en la place où
il était auparavant. M. de Lorraine se mit à deux genoux sur un carreau
de velours rouge bordé d'un petit galon d'or qui était aux pieds du roi
qui lui prit les mains jointes entre les deux siennes. Alors M. le
chancelier lut fort haut et fort distinctement la formule de l'hommage
lige\footnote{Voy. à la fin du volume la note sur l'hommage lige et
  l'hommage simple.} et du serment, auxquels M. de Lorraine acquiesça,
et dit et répéta ce qui était de forme, puis se leva, signa le serment
avec la plume que Torcy lui présenta un peu à côté du roi, où Nyert lui
présenta son épée qu'il remit, puis lui rendit son chapeau dans lequel
étaient ses gants, et se retira. Pendant ce moment le roi s'était levé
et découvert, et tous les princes du sang et les deux bâtards
demeurèrent en leurs places. M. de Lorraine retourné vers le roi, Sa
Majesté se couvrit, le fit couvrir en suite, et en même temps les
princes du sang et les deux bâtards se couvrirent aussi. Après être
demeurés quelque peu de temps en conservation ainsi debout et rangés, le
roi se découvrit et passa dans son cabinet, où après moins de demi-quart
d'heure il fit appeler M. de Lorraine. Monsieur demeura dans le salon,
et M. de Lorraine demeura seul avec le roi une bonne demi-heure. Il
trouva Monsieur qui l'attendait dans le salon, qui tout de suite le
ramena à Paris, où le lendemain Torcy alla lui faire signer un écrit de
tout le détail de la cérémonie, et de sa prestation de foi et hommage
lige\,; et lui en délivra une copie signée de lui et de Pontchartrain.

Le jeudi, lendemain de l'hommage, Monsieur mena M. de Lorraine à
Meudon\,; ils y arrivèrent au sortir de table. Monseigneur les promena
fort par sa maison, après quoi ils s'en retournèrent à Paris sur les
quatre heures. Le samedi suivant, M. de Lorraine alla seul dîner à
Versailles chez M. le Grand, puis voir la grande écurie dont le comte de
Brionne lui fit les honneurs, et revint de là à Paris. Le lundi d'après,
Monsieur mena M. de Lorraine à Marly, où le roi venait d'arriver au
sortir de table, qui lui fit voir la maison et les jardins. Monsieur,
qui était enrhumé, demeura dans le salon. Le roi rentra des jardins dans
son cabinet avec M. de Lorraine, où il fut quelque temps seul avec lui,
qui sur la fin prit congé. En sortant de son cabinet le roi parla
quelque temps à milord Carlingford, connu à Vienne sous le nom de
général Taff, et qui a été gouverneur de M. de Lorraine\,; puis reçut
les révérences de ceux qui la lui avaient faites en arrivant, et
retourna se promener, puis revint à Versailles, et Monsieur ramena M. de
Lorraine à Paris, à qui le roi envoya une tenture de tapisserie de
l'histoire d'Alexandre, de vingt-cinq mille écus. Monsieur avait prié le
roi, qui lui voulait faire un présent, de lui donner une tapisserie
plutôt que toute autre chose. Le mardi 1er décembre, Monseigneur qui
était à Meudon y donna à dîner à Mgr et à M\textsuperscript{me} la
duchesse de Bourgogne et à leur suite. M\textsuperscript{me} la
princesse de Conti y dîna aussi, et il les mena tous à l'Opéra. Monsieur
y était dans sa loge en haut avec M. de Lorraine. Il l'amena en celle de
Monseigneur, où il ne fut qu'un moment, pour prendre congé de lui et de
Mgr et de M\textsuperscript{me} la duchesse de Bourgogne, chez qui
pourtant il n'avait point été, ce qui parut assez bizarre, et s'en alla
aussitôt après avec Monsieur. Il partit la nuit suivante en poste avec
sa suite pour s'en retourner en Lorraine. On fit doubler les chevaux
partout, et ce qu'il y eut encore de rare, fut que le roi en paya toute
la dépense, et malgré lui, par complaisance pour Monsieur. Quelque
abandonné qu'il fût au chevalier de Lorraine, M. de Lorraine commençait
à lui peser beaucoup pour la dépense et pour la liberté. Il s'en aperçut
ou on l'en fit apercevoir, et c'est ce qui hâta son départ. Il ne laissa
pas de se trouver importuné de l'assiduité de tous ceux de sa maison
auprès de lui, d'aucun desquels il ne parut faire cas, que de M. le
Grand et du chevalier de Lorraine\,; et il lui échappa plus d'une fois
de dire, qu'il ne savait à qui en voulaient tous ces petits princes de
l'obséder continuellement.

Le dimanche, 20 décembre, comme le roi sortait du sermon, il rencontra
Monsieur qui allait au-devant de lui, et qui l'accompagna chez lui. Ils
y trouvèrent Madame et M\textsuperscript{me} de Lorraine, et ils furent
assez longtemps tous quatre, seuls dans le cabinet du roi. Monsieur,
Madame et M\textsuperscript{me} de Lorraine allèrent de là chez
M\textsuperscript{me} la duchesse de Chartres, et Monsieur mena après
M\textsuperscript{me} de Lorraine chez M. le Grand, qui avait la goutte,
et ensuite à Paris. Le roi n'a pas voulu qu'elle vît Monseigneur, ni
Mgrs ses petits-fils ni M\textsuperscript{me} la duchesse de
Bourgogne\,; ni même qu'elle les rencontrât, à cause de la petite
vérole, qu'ils n'avaient point eue. Le samedi, le roi seul alla dîner à
Marly, où Monsieur, Madame, et M\textsuperscript{me} de Lorraine vinrent
de Paris dîner avec lui. Il fut remarqué que Monsieur ni
M\textsuperscript{me} de Chartres n'y étaient point. On évita tant qu'on
put que les belles-sœurs se trouvassent ensemble. Monsieur, pour faire
sa cour au chevalier de Lorraine, Madame, parce qu'elle regardait son
gendre comme un prince allemand, et qui, par conséquent, pouvait tout
prétendre. On se contenta qu'avant partir, M. de Lorraine allât chez M.
le duc de Chartres, et lui fit force protestations qu'il ne lui était
jamais entré dans la pensée de lui disputer rien. C'était donc à dire
que cela se pouvait imaginer. C'était aux princes du sang à y faire le
commentaire. En tout cas, la petite vérole de M\textsuperscript{me} de
Lorraine ne vint pas mal à propos, et les Lorrains eurent grand sujet
d'être contents de Monsieur et de Madame, et de s'applaudir de leur tour
d'adresse d'avoir mis les bâtards en jeu, pour esquiver nettement les
princes du sang, et parer tout par l'incognito. M\textsuperscript{me} de
Lorraine prit congé du roi après dîner, qui retourna à Versailles, et
Monsieur, Madame, et M\textsuperscript{me} de Lorraine à Paris, qui en
partit le lendemain lundi, pour retourner en Lorraine. Elle en marqua
une impatience qui alla jusqu'à l'indécence. Apparemment qu'elle voulut
profiter de sa petite vérole, et ne pas demeurer ici assez longtemps
pour se trouver en état de remplir des devoirs qui l'auraient
embarrassée.

\hypertarget{chapitre-xxi.}{%
\chapter{CHAPITRE XXI.}\label{chapitre-xxi.}}

1699

~

{\textsc{Bassesses et noirceur étrange du duc de Gesvres à mon égard.}}
{\textsc{- Duc de Gesvres, méchant dans sa famille, fait un trait cruel
au maréchal de Villeroy.}} {\textsc{- Origine de la conduite des
ambassadeurs, à leur première audience, par ceux des maisons de
Lorraine, Savoie et Longueville, et à leur entrée, par des maréchaux de
France.}} {\textsc{- Origine du chapeau aux audiences de cérémonie des
ambassadeurs, qui ne s'étend nulle part ailleurs.}} {\textsc{- Mort de
M\textsuperscript{me} de Marsan.}} {\textsc{- Le nonce Delfini fait
cardinal\,; son mot sur l'Opéra.}} {\textsc{- Mariage de Coigny et de
M\textsuperscript{lle} du Bordage.}} {\textsc{- Silence imposé par le
roi aux bénédictins et aux jésuites sur une nouvelle édition des
premiers de saint Augustin.}} {\textsc{- Exécution de
M\textsuperscript{me} Ticquet, pour avoir fait assassiner son mari,
conseiller au parlement.}} {\textsc{- Mort du fils unique de Guiscard.}}
{\textsc{- Mort de Barin.}}

~

Je viens maintenant à ce qui m'arriva de ce voyage. Il était certain que
le grand chambellan, et en son absence le premier gentilhomme de la
chambre du roi en année, devait prendre l'épée, le chapeau et les gants
de M. de Lorraine allant rendre son hommage. Les prendre en ce cas-là
c'est dépouiller le vassal des marques de dignité en présence de son
seigneur et non pas le servir, et ce qui le montre c'est que le premier
gentilhomme de la chambre ne les garde ni ne les rend. Toute sa fonction
n'est que de dépouiller le vassal, et c'est le premier valet de chambre
qui les reçoit du premier gentilhomme de la chambre dans l'instant qu'il
les a étés au vassal, et c'est ce même premier valet de chambre qui les
rend au vassal après son hommage. Cela se passa ainsi en 1661 à
l'hommage du duc de Lorraine, Charles IV, grand-oncle de celui-ci, et il
se trouve même que le connétable de Richemont de la race royale, et qui
mourut duc de Bretagne, prit l'épée, les gants et le chapeau du duc de
Bretagne, son neveu, s'étant trouvé présent à son hommage. Cela ne peut
s'entendre autrement, et fut en effet entendu de la sorte sans nuage ni
détour. Néanmoins, à l'adresse avec laquelle la maison de Lorraine a su
tirer des avantages de tout, et des choses les plus fortuites et les
plus indifférentes en faire des distinctions, des prétentions, des
prérogatives, je voulus éviter jusqu'aux riens les plus décidés, pour ne
leur laisser aucune prise et profiter de la conjoncture du monde la plus
naturelle.

Le duc de Gesvres, qui était en année, ne servait plus les soirs quand
le marquis de Gesvres son fils et son survivancier l'en pouvait
soulager. Il se portait bien, il était à Versailles, il était donc tout
simple de lui laisser la fonction. Le duc de Gesvres avait fait toute sa
vie profession d'être ami particulier de mon père, et le venait voir
fort souvent jusqu'à sa mort. Depuis il m'accabla d'amitiés\,; et toutes
ses années me procurait toutes les sortes d'entrées dont le premier
gentilhomme de la chambre peut favoriser. Il me venait voir, à
quatre-vingts ans qu'il avait, avec une politesse et les manières les
plus propres à donner de la confiance. J'y avais toujours répondu avec
tous les soins, les égards et le respect dû à son âge et à ses avances,
et, au peu d'accès qu'il donnait auprès de lui, la tendresse qu'il me
témoigna toujours était tout à fait singulière.

Je crus donc pouvoir en user avec lui en confiance, et lui faire
remarquer l'avantage que les Bouillon pourraient vouloir prendre de
l'absence affectée de M. de Bouillon, et qu'un duc et pair eût fait la
fonction. J'ajoutai, qu'aucun duc sans fonction absolument nécessaire,
comme le premier gentilhomme de la chambre et le capitaine des gardes en
quartier, qui était mon beau-père, ni pas un prince étranger ne devant
se trouver à l'hommage, parce qu'aussitôt après M. de Lorraine se
couvrirait et qu'eux demeureraient découverts, c'était une autre raison
de laisser la fonction au marquis de Gesvres. J'assaisonnai cela de
toutes les excuses et de tous les respects bienséants à mon âge. Il m'en
parut satisfait et goûter ce que je lui proposais. Il en raisonna avec
moi, et il convint que le roi ne trouvant pas mauvais l'absence de M. de
Bouillon, qui n'avait point de survivancier, c'était une raison de ne
pas trouver mauvaise non plus la sienne, ayant un survivancier accoutumé
à le remplacer tous les soirs. Il me témoigna qu'il sentait bien toutes
les raisons que je lui venais de dire, qu'il tâcherait de laisser la
fonction à son fils, mais qu'il fallait qu'il en parlât au roi. Il
ajouta\,: «\,Voyez-vous, avec l'homme à qui j'ai affaire (c'était le
roi), il faut que je me mette bas, bas, bas comme cela (montrant de la
main) pour m'élever haut après.\,» En cela il n'avait pas tort et le
connaissait bien. Je me retirai louant sa prudence et flattant bien mon
vieillard, content de tout, pourvu que son fils fît la fonction.

Je vis là-dessus M\textsuperscript{me} de Noailles et le duc de Béthune,
ancien ami du duc de Gesvres, qui lui avait parlé depuis moi et qui n'en
avait pas été content. Je commençai à soupçonner l'humeur fantasque de
ce vieillard, à laquelle le servile surnageait toujours. Plusieurs
mesures me manquèrent. Je crus que le pis qu'il pouvait m'arriver de lui
parler encore serait de ne pas réussir, et dans cette confiance je monte
chez lui, et je le trouve s'habillant. Je fais sortir ses valets, je lui
parle, il me répond froidement que le roi lui a dit que c'était sa
fonction et qu'il la devait faire\,; qu'il n'avait pu répliquer parce
qu'à l'instant sa chaise avait roulé, le menant au degré pour aller se
promener à Marly\,; cela ne m'étonna point. Je lui répondis que, la
fonction étant pour l'après-dînée, il aurait encore le temps au lever du
roi où il s'en allait, ou ailleurs, de parler, et finalement, après
quelques disputes toujours très mesurées et respectueuses de ma part,
froides mais polies de la sienne et qui semblait désirer ce que je
souhaitais, je conclus par lui dire qu'il n'y avait donc de parti à
prendre que de s'en aller à Paris au sortir du lever comme pour quelque
affaire pressée, ou de faire le malade, et que puisque le roi trouvait
bon que M. de Bouillon se tînt à Évreux sans l'être ni le faire, le roi
ne trouverait pas plus mauvais qu'il le fit s'il ne croyait pas qu'il le
fût. Tout fut inutile, son parti était pris. Je descendis chez le duc de
Béthune\,; je ne trouvai que son fils, à qui je contai ce que je venais
de faire et de voir. Il me rassura sur ce que M. de Chevreuse en devait
parler au roi à l'issue de son lever. En effet il réussit, et le roi dit
publiquement tout haut au marquis de Gesvres, dans son cabinet, allant
donner l'ordre, que ce serait lui qui servirait à l'hommage au lieu de
son père. Tout le monde l'entendit et le débita sur-le-champ. Le duc de
Gesvres, qui l'avait oui comme les autres, laissa sortir tout le monde,
puis harangua si bien le roi qu'il consentit qu'il fît la fonction.
Voilà bien de la bassesse et de la friponnerie gratuites, mais ce n'est
encore rien.

Deux jours après je fus averti par la comtesse de Roucy qu'il y avait
grande rumeur contre moi au Palais-Royal\,; que Madame avait parlé fort
aigrement de moi à la comtesse de Beuvron\,; et que la chose était à un
point que j'y devais mettre ordre. J'allai trouver la comtesse de
Beuvron, qui me conta que le duc de Gesvres, non content de faire la
fonction de l'hommage, avait fait sa cour au roi à mes dépens, et lui
avait raconté d'une manière burlesque tous les pas que j'avais faits
auprès de lui pour l'en empêcher, jusqu'à lui vouloir faire jouer une
apoplexie, de quoi il s'était très bien gardé, à son âge et de sa
taille, de peur que l'apoplexie ne se vengeât et de mourir comme
Molière\,; qu'il avait ajouté à cela, sur son compte, toutes les
prostitutions qui se peuvent proférer, et qu'il n'avait surtout rien
oublié pour me sacrifier d'une manière complète\,; qu'au partir de là il
était allé trouver M\textsuperscript{me} d'Armagnac, quoique sans
liaison avec elle ni avec M. le Grand, qu'il lui avait fait la même
histoire, et qu'il l'avait ensuite répétée à tout ce qu'il avait
rencontré\,; que cela était revenu à Monsieur, à qui on avait ajouté que
j'avais tenu quantité de propos sur la petitesse de la souveraineté et
du rang de M. de Lorraine\,; que Monsieur était dans une colère horrible
et en parlait à mille gens\,; que Madame, pour être plus retenue, n'en
était pas moins dangereuse, et que je ferais bien d'apaiser des gens
avec qui on ne peut avoir raison.

Une si énorme perfidie me fut un coup de foudre, et je n'imaginais
personne assez gratuitement méchant pour vouloir perdre dans l'esprit du
roi le fils de son ancien ami, qu'il avait toujours accablé d'amitiés et
de caresses, qui y avait toujours répondu par toutes sortes de soins et
de respects, qui, dans ce dont il s'agissait, ne lui avait rien dit qui
pût lui déplaire, et qu'il n'eût pas même montré de goûter, et qui par
l'entière disproportion d'âge, de figure et d'établissement, ne pouvait
en mille ans être en son chemin, ni d'aucun des siens. Quand je fus
revenu du premier étourdissement d'une si infâme scélératesse, je
remerciai la comtesse de Beuvron, et je la priai de rendre compte à
Madame de la véritable raison qui m'avait fait agir, qui était l'absence
de M. de Bouillon\,; que je ne pouvais trouver indécente dans un duc et
pair une fonction qu'avait faite un connétable, prince du sang, et mort
depuis duc de Bretagne\,; et que, pour les propos qu'on m'attribuait, je
la suppliais de ne pas ajouter foi à ce que des gens, ou ennemis, ou
curieux de faire leur cour, pouvaient lui avoir rapporté. J'ajoutai que
j'irais lui dire moi-même les mêmes choses, si elle l'avait agréable, et
qu'elle trouvât bon que ce fût en particulier, dans son cabinet.
Ensuite, j'allai chez M\textsuperscript{me} de Maré\,; elle était ma
parente, amie de tout temps de mon père et de ma mère, et la mienne, de
plus, dès ma première jeunesse. Elle avait été gouvernante des enfants
de Monsieur, avec et après la maréchale de Grancey, sa mère\,; elle
l'était de ceux de M. le duc de Chartres, et de tous temps intimement
bien avec Monsieur. Elle me faisait chercher partout\,; elle me dit les
mêmes choses que la comtesse de Beuvron m'avait apprises, et plus de
noirceurs encore du duc de Gesvres. Je lui contai toute l'histoire, à
laquelle elle n'eut rien à répondre que de me quereller d'amitié de
m'être fié à un fou et à un méchant homme, pour mon ami que je le
crusse. Elle se chargea volontiers, pour Monsieur, des mêmes choses dont
la comtesse de Beuvron s'était chargée pour Madame, mais je ne lui
demandai rien pour M\textsuperscript{me} de Lorraine, qu'elle me dit
être furieuse\,: ce n'était qu'un oiseau de passage, et rien du tout
d'ailleurs. Monsieur et Madame, qui s'étaient déchaînés à leur aise,
parurent satisfaits de ce qui leur fut dit de ma part, et n'en
désirèrent rien davantage. Restait le roi, de bien loin le plus
important sur les impressions qu'il pouvait prendre. Le procès de M. de
Luxembourg, l'excuse de la princesse d'Harcourt à la duchesse de Rohan,
mon affaire avec M. le Grand, tout cela que j'avais si vivement mené me
faisait craindre d'avoir trop souvent raison. M. de Beauvilliers ne fut
pas d'avis que je fisse sur celle-ci aucune démarche auprès du roi, de
peur de tourner en sérieux ce que le roi pouvait n'avoir pris qu'en
bouffonnerie, mais d'être attentif à la manière plus froide ou ordinaire
avec laquelle le roi me traiterait, et différer à prendre mes mesures
là-dessus. Le conseil en effet fut très bon\,; le roi me traita à
l'ordinaire, et je demeurai en repos.

Ce vieux Gesvres était le mari le plus cruel d'une femme de beaucoup
d'esprit, de vertu et de biens, qui se sépara de lui, et le père le plus
dénaturé d'enfants très honnêtes gens qui fut jamais. L'abbé de Gesvres
était depuis quelques années camérier d'honneur d'Innocent XI, et
tellement à son gré, qu'il l'allait faire cardinal lorsque l'éclat entre
lui et le roi fit appeler tous les François sur le démêlé des
franchises. L'abbé de Gesvres y perdit tout, mais revint de bonne grâce.
Le roi, qui en fut touché, lui donna en arrivant de plein saut
l'archevêché de Bourges, qui venait de vaquer par la mort du frère de
Châteauneuf, secrétaire d'État. Le duc de Gesvres, en furie, alla
trouver le roi, lui dit rage de son fils, et fit tout ce qui lui fut
possible pour empêcher cette grâce. Le marquis de Gesvres, il l'a
traité, lui et sa femme, comme des nègres, toute sa vie, au point que le
roi y est souvent entré par bonté. Ses équipages étaient superbes en
chevaux, en harnais, en voitures, en livrées qui se renouvelaient sans
cesse, et ses écuries pleines des plus rares chevaux de monture, sans en
avoir jamais monté un depuis plus de trente ans\,; son domestique
prodigieux\,; ses habits magnifiques et ridicules pour son âge. Quand on
lui parlait de ses grands revenus, du mauvais état de ses affaires
malgré sa richesse, du désordre de sa maison, et de l'inutilité et de la
folie de ses dépenses, il se mettait à rire et répondait qu'il ne les
faisait que pour ruiner ses enfants. Il disait vrai, et il y réussit
complètement\,; mais ce n'était pas seulement sa famille qu'il
persécutait gratuitement.

Il fit, cette même année, un tour au maréchal de Villeroy à le tuer.
Tous deux étaient venus de secrétaires d'État, et tous deux avaient eu
des pères qui avaient fait une grande et extraordinaire fortune. Un jour
que le petit couvert était servi, et que le roi était encore chez aime
de Maintenon, où il allait souvent les matins, les jours qu'il n'avait
point de conseil, comme les jeudis et les vendredis (et qu'elle n'avait
point là de Saint-Cyr à aller dès le matin, comme à Versailles), les
courtisans étaient autour de la table du roi, à l'attendre, et M. de
Gesvres, pour le servir. Le maréchal de Villeroy arriva avec ce bruit et
ces airs qu'il avait pris de tout temps, et que sa faveur et ses emplois
rendaient plus superbes\,; je ne sais si cela impatienta ce vieux
Gesvres plus qu'à l'ordinaire, mais dès qu'il le vit arriver, derrière
un coin du fauteuil du roi où il se mettait toujours\,: «\,Monsieur le
maréchal, se prit-il à lui dire tout d'un coup, la table et le fauteuil
entre-deux, il faut avouer que vous et moi sommes bien heureux.\,» Le
maréchal, étonné d'un propos que rien n'amenait, en convint avec un air
modeste, et, secouant sa tête et sa perruque, voulut le rompre en
parlant à quelqu'un\,; mais l'autre, qui n'avait pas si bien commencé
pour rien, continue, l'apostrophe pour se faire écouter, admire la
fortune du Villeroy qui épouse une Créqui, et de son père qui épouse une
Luxembourg, et de là, des charges, des gouvernements, des dignités, des
biens sans nombre\,; et les pères de ces gens-là des secrétaires
d'État\,: «\,Arrêtons-nous là, monsieur le maréchal, s'écria-t-il,
n'allons pas plus loin\,; car, qui étaient leurs pères, à ces deux
secrétaires d'État\,? de petits commis, et commis eux-mêmes\,; et de qui
venaient-ils\,? le vôtre, d'un vendeur de marée aux halles, et le mien,
d'un porteballe et peut-être de pis. Messieurs, s'adressant à la
compagnie tout de suite, est-ce que je n'ai pas raison de trouver notre
fortune prodigieuse, à M. le maréchal et à moi\,? N'est-il pas vrai
donc, monsieur le maréchal, que nous sommes bien heureux\,?» Puis à
regarder, à se pavaner et à rire. Le maréchal eût voulu être mort,
beaucoup mieux encore l'étrangler, mais que faire à un homme qui, pour
vous dire une cruauté, s'en dit à lui-même le premier\,? Tout le monde
se tut et baissa la vue\,; il y en eut plus d'un qui ne fut pas fâché de
regarder le maréchal du coin de l'œill, et de voir ses grandes manières
si plaisamment humiliées. Le roi vint et finit le spectacle et
l'embarras, mais il ne fit que suspendre. Ce fut la matière de la
conversation de plusieurs jours, et le divertissement de la malignité et
de l'envie si ordinaires à la cour.

Cette aventure, quelle qu'elle fût, ne pouvait me servir de leçon de ne
me pas fier à un si méchant homme. Il pouvait avoir cru se parer de sa
modestie par un discours qui au fond n'apprenait rien que tout le monde
ne sût, et la jalousie de la faveur et de l'éclat du maréchal de
Villeroy pouvait l'avoir excité à lui dire à brûle-pourpoint des vérités
si fâcheuses à entendre. J'étais à mille lieues de tout ce que ce
pernicieux vieillard pouvait désirer ou envier, et je ne crois pas qu'un
autre en ma place eût pu se défier d'une scélératesse aussi gratuite et
aussi complète. Mais il faut achever ce qui regarde l'absence affectée
de M. de Bouillon.

Ce n'était point le service de l'hommage qui en éloigna M. de
Bouillon\,; je l'ai expliqué. Mais accoutumé à se couvrir aux audiences
des ambassadeurs, depuis que les félonies héréditaires de ses pères,
depuis que la faveur d'Henri IV leur eût valu Sedan, avaient acquis le
rang de prince à son père au lieu de lui coûter la tête, il ne voulut
pas se trouver à une cérémonie où les princes du sang, les bâtards et M.
de Lorraine se couvriraient, et où il demeurerait découvert, et c'est ce
qui empêcha la maison de Lorraine de s'y trouver et tous les autres qui
ont cet avantage. Mais pour entendre cette différence aux mêmes
personnes de se couvrir aux audiences des ambassadeurs, et de ne se
couvrir jamais en aucune autre occasion, il faut remonter à l'origine de
cette distinction.

Anciennement tout le monde était couvert devant nos rois à l'ordinaire
de la vie, et dans les cérémonies par conséquent, et quand autour du roi
quelqu'un avalait son chaperon\footnote{Le chaperon était une coiffure
  en usage principalement sous les règnes de Jean, Charles V et Charles
  VI\,; elle était en drap, bordée de Fourrures avec une longue queue
  qui retombait par derrière. \emph{Avaler} est un vieux mot synonyme de
  \emph{descendre ou abaisser}. On n'ôtait pas toujours son chaperon en
  parlant aux princes. Monstrelet rapporte que la reine Isabeau, ou
  Isabelle de Bavière, haïssait Jean Torel, parce qu'en lui parlant
  \emph{il ne levait point son chaperon}.}, les plus près du roi lui
faisaient place, parce que c'était une marque qu'il voulait parler au
roi. Le changement des chaperons en bonnets, puis en toques, altéra peu
à peu cet usage et l'abolit à la fin, tellement que personne ne se
couvrit plus devant le roi à l'ordinaire de la vie, ni dans les
cérémonies, hors celles où cela fut ou réservé ou marqué, comme au
sacre, aux pompes funèbres, aux cérémonies de l'ordre, et alors il ne
s'agissait point d'être prince, mais seulement d'avoir l'office qui
faisait qu'on était couvert, comme les pairs et les officiers de la
couronne au sacre et au lit de justice, tout le monde aux convois des
pompes funèbres, et tous les chevaliers au chapitre et au festin de
l'ordre. Les ambassadeurs étaient reçus et accompagnés par des
chambellans du roi à leur entrée et à leur audience, et cela a duré
jusqu'à la puissance des Guise et à leurs projets. Comme M. de Guise fut
le premier qui fit ajouter à la formule de son serment de pair ces
paroles à la suite des autres si différentes\,: \emph{Et comme un bon
conseiller de cour souveraine}, pour flatter le parlement et la
magistrature, ce qui a été ôté longtemps depuis\,; comme il fut le
premier homme non seulement de sa dignité et de son état, mais de
quelque distinction, qui ait été marguillier d'honneur de sa paroisse
pour s'attirer la bourgeoisie, au delà de laquelle cette marguillerie
n'avait jamais passé\,; aussi dans ces mêmes desseins voulut-il gagner
les puissances étrangères et s'en attacher les ambassadeurs. Comme il
pouvait des choses assurément plus importantes, il mit en usage de
conduire à l'audience de cérémonie ceux des premières têtes couronnées,
c'est-à-dire du pape, de l'empereur et des rois d'Espagne et
d'Angleterre, sous prétexte de sa charge de grand chambellan, et de les
présenter au roi. Eux se trouvèrent bien plus honorés d'être menés par
lui que par des chambellans, et cette conduite leur donnait occasion de
civilités qui introduisaient visite, commerce et affaires. De M. de
Guise l'usage par ces mêmes raisons s'en étendit peu à peu à ses
enfants, à ses frères, puis à ses cousins, d'abord pour le suppléer,
dans la suite comme une distinction qu'ils avaient acquise par l'usage,
et comme un honneur dont les ambassadeurs ne voulurent plus se départir.
De l'un à l'autre MM. de Nemours, si unis aux Guise leurs frères
utérins, voulurent partager cet avantage. Ils n'y trouvèrent point de
difficulté de leur part, puis M. de Longueville et les ambassadeurs,
accoutumés à être menés par des princes de la maison de Lorraine, se le
trouvèrent également bien par ceux de la maison de Savoie et par une
autre maison bien inférieure, mais qui ne cédait rien à ces deux-là en
avantages. C'est ce qui a fait que longues années après, MM. de Bouillon
et de Rohan, ayant obtenu les mêmes distinctions que MM. de Lorraine
avaient usurpées pendant la Ligue, et qu'ils ont bien su se conserver
depuis, et qui ont été étendues à MM. de Savoie, etc., ils n'ont pu
néanmoins atteindre à celle de mener les ambassadeurs à l'audience, qui
ont fort bien su dire que le rang qui leur avait été donné ne les
rendait pas princes, et qu'ils ne se départiraient point d'en avoir de
véritables et non de factices pour conducteurs. Quand la chose fut bien
établie, et que la maison de Lorraine se vit en état de tout
entreprendre, arrivée qu'elle fut par les dignités et les offices de
l'État qu'elle sut si bien faire valoir contre les princes du sang, et
que, pièce à pièce, et de conjonctures en conjonctures, et d'occasion en
occasion, elle fut venue à bout de se former un rang par naissance, et
des distinctions différentes de celles des rangs de l'État, elle imagina
de faire accompagner les ambassadeurs à leur entrée par des maréchaux de
France, pour marquer par là leur supériorité sur les officiers de la
couronne. Il y avait alors très peu de ducs qui ne fussent pas princes
du sang ou de maison souveraine, et on n'avait point encore vu de
maréchaux de France ducs. Il n'y en a eu que bien depuis que cette
conduite aux entrées a été établie. Longtemps encore depuis, les
maréchaux de France qui étaient ducs n'y étoient pas employés. À la fin
ils l'ont été aussi comme à une fonction attachée à leur office de
maréchal, comme tels et non comme ducs, et insensiblement ç'a été un
nouveau degré de distinction pour les princes à qui la conduite à
l'audience est demeurée. Mais pour cet avantage ils n'avaient pas celui
de se couvrir\,; l'ambassadeur seul jouissait de cet honneur, et le
prince qui le menait à l'audience y assistait découvert. Quelque
entreprenants que se soient montrés les Guise, jamais ils n'ont imaginé
de se couvrir devant les rois qu'ils maîtrisaient, et dont ils étaient
sur le point d'usurper la couronne. Cet usage ne s'introduisit que sous
Henri IV, et en voici l'occasion.

Après l'entière chute de la Ligue et la paix de Vervins, il vint un
ambassadeur d'Espagne en France, qui était grand d'Espagne. Il alla
trouver le roi à Monceaux, où il était avec peu de monde, et il
l'accompagna dans les jardins que le roi avait fait faire, et qu'il se
plut à lui montrer. Dans les commencements de la promenade, le roi se
couvrit. L'ambassadeur, accoutumé à se couvrir en même temps que le roi
d'Espagne se couvrait, se couvrit aussi. Henri IV le trouva fort
mauvais. Il ne voulut pourtant rien marquer à l'ambassadeur, mais jetant
les yeux autour de soi, il commanda à M. le Prince, à M. de Mayenne et à
M. d'Épernon de se couvrir, qui étaient les seuls grands qui de hasard
se trouvèrent à cette promenade. De là M. de Mayenne obtint de se
couvrir aux audiences des ambassadeurs\,; à plus forte raison M. le
Prince et l'heureux duc d'Épernon aussi, par la fortune de s'être trouvé
là en troisième avec eux. Avec M. de Mayenne, ceux de sa maison qui
conduisaient les ambassadeurs à l'audience se couvrirent, et une fois
couverts, s'y couvrirent toujours menant ou non les ambassadeurs. Sur
cet exemple, les enfants de M. d'Épernon se couvrirent de même, parce
que cet honneur vint pour eux tous de la même origine à Monceaux. Les
princes des maisons de Savoie et de Longueville, égalés en tout aux
Lorrains, se couvrirent de même, et par conséquent les cardinaux
supérieurs à tous en rang, et les princes du sang, quand il y en eut en
âge, autres que M. le Prince. Telle est l'origine de ce qui s'appelle le
\emph{chapeau}, et ce chapeau, de si grand hasard pour M. d'Épernon, lui
valut, et à ses fils et à son petit-fils, le rang et les honneurs des
princes étrangers, quelque peu bien qu'il fût dans le goût et les bonnes
grâces d'Henri IV. Mais ce roi et ses successeurs à qui ce chapeau était
échappé, comme je viens de l'expliquer, ont été continuellement jaloux
de ne pas le laisser étendre au delà des audiences de cérémonie des
ambassadeurs, et jamais en aucune autre occasion ils n'ont permis aux
princes étrangers de se couvrir, et c'est pour cela aussi qu'aucun d'eux
ne se trouve aux audiences publiques des souverains que le roi fait
couvrir, ni en aucune autre où autre que lui puisse être couvert, ni les
cardinaux non plus qu'eux. Ils ont essayé plus d'une fois d'obtenir
cette extension d'honneur. Les ducs aussi ne se trouvent jamais nulle
part où d'autres se couvrent, excepté le premier gentilhomme de la
chambre en année, et le capitaine des gardes en quartier, par le service
nécessaire de leurs charges. On a vu que quand ils sont mandés par le
roi à une audience, comme il arriva à celle du cardinal Chigi, légat
\emph{a latere}, que les princes étrangers ne s'y couvrent point\,;
ainsi on n'en répétera rien. Mais voilà assez d'explication sur cette
matière, il est temps de reprendre les événements qui ont fini cette
année.

M\textsuperscript{me} de Marsan mourut avant le départ de Paris de
M\textsuperscript{me} la duchesse de Lorraine. C'était une nièce
paternelle de MM. de Matignon, et de plus sœur de M\textsuperscript{me}
de Matignon, femme altière, impérieuse, de peu d'esprit, et parfaitement
gâtée par la place, la splendeur, l'autorité et l'étrange hauteur de
Seignelay, son premier mari, et par le rang et la naissance du second.
Elle était en couche de son second fils. La nourrice fit je ne sais quoi
qui lui déplut\,; la colère la transporta, la couche s'arrêta, il n'y
eut jamais moyen de la sauver\,: elle mourut et ne fut regrettée de
personne, ni des siens que, par crédit, et après par rang, elle avait
toujours traités avec beaucoup d'humeur et de hauteur, ni de son mari
qu'elle tenait de court, et qui demeurait riche usufruitier d'une partie
de ses biens.

Le nonce Delfini fut fait cardinal dans une promotion de nonces et
d'Italiens. Le courrier de M. de Monaco devança celui du pape. Le roi
crut avoir ses raisons pour lui faire une faveur singulière\,; il lui
écrivit un billet de sa main pour le lui apprendre et s'en réjouir avec
lui. Dès qu'il l'eut reçu, il s'en vint à Versailles remercier lui-même,
et débarqua chez Torcy. Comme le cas était extraordinaire, Torcy le mena
chez M\textsuperscript{me} de Maintenon où le roi était déjà, et le fit
avertir. Le roi les fit entrer. M\textsuperscript{me} la duchesse de
Bourgogne qui s'y trouva et M\textsuperscript{me} de Maintenon lui
firent là leur compliment, le tout dura fort peu. Le courrier du pape
arriva enfin le soir, et lui apporta sa calotte. Il sut assez vivre pour
la mettre dans sa poche et de demeurer ainsi, depuis le dimanche qu'à
son retour d'avoir remercié le roi, il trouva le courrier arrivé,
jusqu'au mercredi matin, jour de l'hommage, qu'il eut audience
particulière du roi dans son cabinet, auquel il présenta sa calotte pour
la recevoir de sa main. Le roi la lui rendit, à la différence de ses
sujets à qui il la met sur la tête. Ce nonce avait beaucoup d'esprit, et
en avait bien aussi la physionomie. Je n'ai jamais vu deux si petits
yeux ni qui disent tant. Il était galant, et peut-être quelque chose de
plus\,; il aimait à se divertir, et allait fort souvent à l'Opéra. Le
roi, qui était alors plus austère qu'il n'a été depuis dans sa dévotion,
en fut scandalisé, et lui fit insinuer avec adresse que ce n'était pas
l'usage ici que les évêques ni les prêtres allassent aux spectacles\,;
il fut sourd et ne fit pas semblant de comprendre. Enfin le roi le lui
fit dire de sa part. Le bon Delfin, glissant sur la conscience, et
passant à côté de l'usage, se confondit en remerciements de la bonté
avec laquelle le roi avait soin de sa fortune, et répondit qu'il n'avait
jamais compté d'en faire aucune en France, mais bien en Italie, où
l'Opéra et les spectacles n'étaient obstacle à rien, et y retourna tout
de plus belle. Le roi, le voyant arrivé en effet au but malgré l'Opéra,
voulut peut-être effacer la petite amertume de l'avis par l'agrément du
billet, et ne pas renvoyer à Rome un cardinal mécontent.

Coigny, mestre de camp du Royal-Étranger, qui longtemps depuis a fait
une si belle fortune, épousa en ce temps-ci M\textsuperscript{lle} du
Bordage, du nom de Montboucher, fille de qualité de Bretagne, très
jolie, et encore plus vertueuse et plus sainte toute sa vie. Toute sa
famille était huguenote. On les rattrapa comme ils étaient à la
frontière pour se retirer en Hollande. Son père se convertit comme il
put, et fut tué devant Philippsbourg. Le roi mit le fils au collège, et
la fille chez M\textsuperscript{me} de Miramion, où ils abjurèrent. Le
fils eut un régiment que le roi lui donna pour rien de bonne heure. Il
était bien fait, avec bien de l'esprit, aimant la bonne compagnie et
encore plus la liberté et le jeu par-dessus tout, où il a passé sa vie
sans se marier, a peu servi et peu paru à la cour. Leur mère était
Goyon-Matignon, fille du marquis de La Moussaye, et d'une sœur de MM. de
Bouillon et de Turenne et de M\textsuperscript{me}s de La Trémoille, de
Duras et de Roye. M\textsuperscript{lle} du Bordage était ainsi nièce
maternelle de M. de Quintin, mari sans enfants de la Montgomery qui se
remaria à Mortagne, de laquelle j'ai parlé à cette occasion.

L'année finit par les holà que le roi mit entre les jésuites, qui en
eurent apparemment besoin, puisqu'ils le firent parler, et les
bénédictins. Ces derniers avaient donné depuis peu une belle édition de
saint Augustin, dont la morale n'est pas celle des jésuites. Pour
l'étouffer ils employèrent leur égide ordinaire qui les a toujours si
bien servis. Le livre, selon eux, était tout janséniste\,; ils
l'attaquèrent. Les bénédictins répondirent, ils s'échauffèrent fort de
part et d'autre. Les jésuites, a bout de preuves et de raisons, mais non
d'injures et d'assertions plus que hardies, ne purent venir à bout de
ternir cette édition, ni de la faire supprimer. À ce défaut qui leur fut
amer, ils eurent au moins le crédit de faire cesser le combat quand ils
se virent les plus faibles, par une défense de la part du roi aux uns et
aux autres de plus écrire ni parler en aucune sorte sur cette édition.
Ce fut Pontchartrain qui l'écrivit aux uns et aux autres. Les jésuites
eurent bientôt après le déplaisir de voir cette édition solennellement
approuvée à Rome.

Il faut réparer les oublis quand on s'en aperçait\,; d'autres matières
m'ont emporté. Les premiers jours d'avril, Ticquet, conseiller au
parlement, et même de la grand'chambre, fut assassiné chez lui, et s'il
n'en mourut pas, ce ne fut pas la faute du soldat aux gardes et de son
portier qui s'étaient chargés de l'exécution et qui le laissèrent, le
croyant mort, sur du bruit qu'ils entendirent. Ce conseiller, qui en
tout était un fort pauvre homme, s'était allé plaindre l'année
précédente au roi, à Fontainebleau, de la conduite de sa femme avec
Montgeorges, capitaine aux gardes fort estimé, à qui le roi défendit de
la plus voir. Cela donna du soupçon contre lui et contre la femme, qui
était belle, galante, hardie, et qui prit sur le haut ton ce qu'on eu
voulut dire. Une femme fort de mes amies et des siennes lui conseilla de
prendre le large, et lui offrit de quoi le faire, prétendant qu'en
pareil cas on se défend mieux de loin que de près. L'effrontée s'en
offensa contre elle et contre plusieurs autres amis, qui avec les mêmes
offres lui donnèrent même conseil. En peu de jours la trace fut trouvée,
le portier et le soldat reconnus par Ticquet, arrêtés et unis à la
question, auparavant laquelle M\textsuperscript{me} Ticquet fut assez
folle pour s'être laissé arrêter, et n'être pas déjà en pays de sauveté.
Elle eut beau nier, elle eut aussi la question, et avoua tout.
Montgeorges avait des amis qui le servirent si bien qu'il ne fut fait
aucune mention juridique de lui. La femme condamnée à perdre la tête et
ses complices à être roués, Ticquet vint avec sa famille pour se jeter
aux pieds du roi et demander sa grâce. Le roi lui fit dire de ne se pas
présenter devant lui, et l'exécution fut faite à la Grève, le mercredi
17 juin après midi. Toutes les fenêtres de l'hôtel de ville, toutes
celles de la place et des rues qui y conduisent depuis la Conciergerie
du palais où elle était, furent remplies de spectateurs hommes et
femmes, et de beaucoup de nom, et de plusieurs de distinction. Il y eut
même des amis et des amies de cette malheureuse qui n'eurent pas honte
et horreur d'y aller. Dans les rues la foule était à ne pouvoir
passer\,; en général on en avait pitié et on souhaitait sa grâce, et
c'était avec cela à qui l'irait voir mourir. Et voilà le monde si peu
raisonnable et si peu d'accord avec soi-même\,!

Tout à la fin de l'année, Guiscard perdit son fils unique, de la petite
vérole, à Vienne. Il l'avait envoyé voyager en ce temps de paix, ce qui
rendit sa sœur une riche héritière.

Barin mourut aussi. Il était premier maître d'hôtel de Monsieur.
C'était, je pense, un homme d'assez peu, mais de très bonne mine, et
fort grand et bien fait, quoique déjà vieux, ce qui lui avait fort servi
auprès des dames. Il avait de l'esprit, du sens, de l'adresse, de
l'intrigue, de la conduite, de l'honneur, et un grand attachement et une
grande fidélité pour ses amis. Il avait été fort avant dans les affaires
de Mademoiselle, de M. de Lauzun et de M\textsuperscript{me} de
Montespan, et j'en ai vu quantité de lettres fort curieuses à M. de
Lauzun sur tout cela vers la fin de sa prison\footnote{Lauzun avait été
  enfermé à Pignerol en 1671\,; il en sortit en 1681, en renonçant au
  duché d'Aumale et au comté d'Eu que lui avait donnés Mademoiselle.}.
Les ministres d'alors en faisaient cas, et il a toujours été dans le
monde sur un bien meilleur pied que son état. Il n'était point marié, et
mourut fort peu riche, rangé et tout à fait désintéressé, et longtemps
avant sa mort assez retiré, et fort homme de bien.

\hypertarget{chapitre-xxii.}{%
\chapter{CHAPITRE XXII.}\label{chapitre-xxii.}}

1700

~

{\textsc{Année 1700. Le roi ne paye plus les dépenses que les courtisans
font à leurs logements.}} {\textsc{- Exil de M\textsuperscript{me} de
Nemours.}} {\textsc{- Porte sainte du grand jubilé ouverte par le
cardinal de Bouillon.}} {\textsc{- Dispute de Torcy et des ambassadeurs
pour leurs carrosses aux entrées.}} {\textsc{- Delfini, nonce et
cardinal, s'en va sans présents pour n'avoir pas voulu visiter les
bâtards.}} {\textsc{- Archevêque de Paris officie à la Chapelle avec sa
croix.}} {\textsc{- Altesse refusée à M. de Monaco avec éclat.}}
{\textsc{- Cardinaux français à Rome.}} {\textsc{- Gualterio nonce en
France.}} {\textsc{- Grandes couronnes ont le choix de leurs nonces.}}
{\textsc{- Mort de M\textsuperscript{me} Tambonneau la mère.}}
{\textsc{- Mort, fortune et famille de M\textsuperscript{me} de
Navailles.}} {\textsc{- Mort de Lavocat.}} {\textsc{- Mort de
M\textsuperscript{me} de Maulevrier.}} {\textsc{- Mort de Biron père.}}
{\textsc{- Mort du chevalier de Villeroy.}} {\textsc{- Mort
d'Hauterive.}} {\textsc{- Cossé, duc de Brissac.}} {\textsc{- Mort du
cardinal Casanata.}} {\textsc{- Quatre-vingt mille livres à M.
d'Elbœuf.}} {\textsc{- Cent mille à M\textsuperscript{me} de Montespan,
qui achète Oiron.}}

~

L'année 1700 commença par une réforme. Le roi déclara qu'il ne ferait
plus la dépense des changements que les courtisans feraient dans leurs
logements. Il en avait coûté plus de soixante mille livres depuis
Fontainebleau. On croit que M\textsuperscript{me} de Mailly en fut
cause, qui depuis trois ou quatre ans avait fait changer le sien tous
les ans. Cela fut plus commode parce qu'avec les gens des bâtiments, on
faisait ce qu'on voulait chez soi sans en demander la permission au
roi\,: mais d'autre part tout fut aux dépens de chacun.

M\textsuperscript{me} de Nemours fut exilée en sa maison de Coulommiers
en Brie, qui est magnifique. Torcy lui en porta l'ordre du roi, auquel
elle obéit avec une fermeté qui approcha fort de la hauteur. Elle avait
mis un gouverneur à Neuchâtel dont on n'était pas content, et qu'on
disait un brouillon, c'est-à-dire qu'il la servait à sa mode, et point à
celle de la cour. On voulut donc qu'elle le changeât, et par la même
raison elle n'en voulut rien faire. On ouvrit ses lettres à ce
gouverneur, et on y trouva choses qui déplurent, et qui la firent
chasser. Être souveraine d'une belle terre, et sujette d'un grand roi,
sont deux choses difficiles à accorder quand on se sent et qu'on veut
faire ce qu'on est.

Le cardinal de Bouillon, devenu sous-doyen du sacré collège, eut le
plaisir d'ouvrir la porte sainte du grand jubilé du renouvellement du
siècle, par l'infirmité du cardinal Cibo, doyen. Il en fit frapper des
médailles, et faire des estampes et des tableaux. On ne peut marquer un
plus grand transport de joie, ni se croire plus honoré et plus grand de
cette fonction, qu'il ne devait pourtant à aucun choix\,: ce lui fut une
consolation après l'affaire de M. de Cambrai qui lui avait causé tant
d'amertume. C'est ainsi que les gens si glorieux se montrent souvent
bien petits. Jamais homme ne se montra tant l'un et l'autre.

Nos secrétaires d'État, parvenus à pas de géant où ils en sont, ne se
contentèrent pas des succès domestiques\,; ils en voulurent essayer
d'étrangers, qui ne leur réussirent pas si bien, parce que les étrangers
ne dépendent point d'eux. Le secrétaire d'État qui a le département des
affaires étrangères envoie son carrosse aux entrées des ambassadeurs\,;
il ne dispute pas de sa personne la préséance à un ambassadeur qui a la
main\footnote{L'expression \emph{avoir la main} signifie avoir la droite
  et la place d'honneur.} chez les princes du sang. Mais tout modeste
que frit Torcy, son carrosse s'était doucement coulé entre le dernier
des princes du sang, et ceux d'Erino et de Ferreiro, derniers
ambassadeurs de Venise et de Savoie. Le successeur d'Erino y prit garde
de plus près, et ne le voulut pas souffrir\,; le successeur de Ferreiro
l'imita, et dit que son maître ne lui pardonnerait jamais s'il faisait
la moindre chose du monde moins que l'ambassadeur de Venise. Torcy
n'envoya point son carrosse. Cette tentative, ainsi manquée presque
aussitôt qu'aperçue et tournée en prétention, fut rejetée dans la suite
par tous les autres ambassadeurs, et finalement les choses revinrent
dans l'ordre. Torcy renvoya son carrosse aux autres entrées, et il ferma
la marche le pénultième de tous, suivi seulement de celui de
l'introducteur des ambassadeurs.

Il y eut une autre difficulté de différente espèce, et qui mortifia le
roi. On a vu ci-devant comme il fit singulièrement merveilles au nonce
Delfin sur son chapeau. Il avait amené peu à peu tous les ambassadeurs à
visiter MM. du Maine et de Toulouse comme les princes du sang, et sans
différence aucune. Le nonce Cavallerini, prédécesseur de celui-ci, et
fait cardinal en France comme lui, se laissa aller à les visiter de
même. Il en fut tancé, et si mal reçu à son retour à Rome, que Delfin
n'osa l'imiter. Les cardinaux, accoutumés à l'usurpation générale dont
ils jouissent partout, croyaient être fort descendus depuis les
cardinaux de Richelieu et Mazarin, de traiter d'égal avec les princes du
sang, et de leur donner la main chez eux, ce qui n'était pas du temps de
ces deux premiers ministres. La donner aux bâtards du roi, et en acte de
cérémonie, leur parut monstrueux. On négocia un mois durant, sans le
pouvoir fléchir ainsi, quoiqu'on fût d'ailleurs fort content de lui
pendant sa nonciature, il ne put avoir ni audience de congé, ni même
audience secrète, ni lettres de récréance, et il fut privé du présent de
dix-huit mille livres en vaisselle d'argent, qu'on a coutume de faire
aux nonces-cardinaux à leur départ, et il s'en alla sans dire adieu à
personne.

Autre tracasserie\,: le cardinal de Bouillon, absent et grand aumônier,
était en disgrâce de l'affaire de M. de Cambrai\,; l'archevêque de
Paris, au contraire, était en faveur. La Chapelle, qui se prétend
exempte de la juridiction de l'ordinaire\footnote{L'ordinaire est
  l'évêque diocésain.}, ne voulait pas souffrir la croix de
l'archevêque, ni l'archevêque officier à la Chapelle sans cette marque
de sa juridiction. M. de Paris venait d'avoir l'ordre, et le roi le fit
officier à la Chandeleur avec sa croix, à la messe de l'ordre.

En voici une de plus de conséquence\,: on a vu ailleurs l'origine d'hier
de la princerie de M. de Monaco, et de sa prétention de l'altesse, et
combien cette chimère l'isola à Rome, et y nuisit aux affaires du roi
par les entraves qu'elle mit au commerce le plus nécessaire de
l'ambassadeur. Lassé de la résistance, il imagina de refuser
l'\emph{excellence} à qui il la devait qui ne lui donnerait pas
l'\emph{altesse}, et par là, fit qu'aucun d'eux ne le vit plus, jusqu'au
duc Lanti et au prince Vaïni, dont la France avait fait la moderne et
légère élévation. Ce qui est difficile à comprendre est comment le roi
le souffrit à son ambassadeur, et comment il préféra la fantaisie toute
nouvelle éclose d'un homme qui n'était ni favori ni ministre intérieur,
au succès de ses affaires qui en reçurent des entraves continuelles.
Cette situation des deux hommes chargés des affaires à Rome, l'un comme
cardinal, l'autre comme ambassadeur, hâta le départ de nos cardinaux. La
santé du pape avait fort menacé, et leur avait fait ordonner de se tenir
prêts. Elle était devenue moins mauvaise, et ils n'étaient plus pressés
de partir, lorsque cet incident fit prendre le parti de les envoyer à
Rome. Mais il n'en partit que deux, Estrées et Coislin. Le premier était
parent proche de M. de Savoie, dont la mère était fille du duc de
Nemours, beau-frère aîné de notre exilée à Coulommiers et de la fille du
duc de Vendôme, bâtard d'Henri IV et de la belle Gabrielle, sœur du
maréchal d'Estrées, père du cardinal. Il s'était toujours tenu en grande
liaison avec Madame Royale. Il s'arrêta à Turin en passant\,; mais il y
avait déjà quelque temps que M. de Savoie, ennuyé de la hauteur des
cardinaux, n'en voulait plus voir aucun, tellement qu'il ne vit le
cardinal d'Estrées que chez M\textsuperscript{me} sa mère et chez
M\textsuperscript{me} sa femme. Le cardinal Le Camus n'était point
rentré en grâce depuis sa promotion à l'insu du roi, et que sans sa
permission il eût pris la calotte à Grenoble et se fût contenté de le
mander au roi. Il n'eut jamais depuis la permission de sortir de son
diocèse, que pour aller à Rome à la mort des papes. Encore ne l'eut-il
pas d'aller au premier conclave, qui arriva depuis qu'il fut cardinal,
et fut obligé de demeurer à Grenoble. Le cardinal Bonzi, tout à fait
tombé de tête et de santé, ne fut pas en état d'y penser, et le cardinal
de Fürstemberg, sucé jusqu'aux moelles par sa nièce, et qui était revenu
très précipitamment du dernier conclave, dans la peur d'être enlevé une
seconde fois par les Impériaux, eut permission de demeurer. On avait
affaire de lui à la cour, et de ne le pas séparer de cette nièce qui le
gouvernait, qui n'aurait pu le suivre à Rome avec bienséance.
M\textsuperscript{me} de Soubise avait ses raisons pour les laisser
ensemble, et ne les pas laisser écarter.

Le nonce Delfin fut relevé ici par Gualterio, vice-légat d'Avignon, que
le roi préféra dans une liste de cinq sujets que le pape lui proposa.
C'est un usage, tourné en espèce de droit que l'empereur et le roi ont
ainsi le choix des nonces que Rome leur envoie. Je pense que le roi
d'Espagne l'a aussi. Gualterio, homme de beaucoup d'esprit, s'était
gouverné dans sa vice-légation de manière à se rendre agréable au roi,
dans la vue de cette nonciature, dont on ne sort point qu'avec le
chapeau. Mailly, archevêque d'Arles, qui tout éloigné qu'il était de la
pourpre y pensait dès avant d'être évêque, comme je crois l'avoir dit,
avait profité de la position d'Arles pour lier des commerces sourds à
Rome et amitié avec ce vice-légat. Les mêmes raisons lui firent désirer
de la liaison entre lui et moi depuis qu'il fut déclaré nonce. Elle se
fit et se tourna depuis en véritable estime et amitié de part et
d'autre, qui se retrouvera en plus d'un endroit dans la suite. C'est ce
qui m'a fait étendre sur sa nomination.

La vieille Tambonneau, tante maternelle de M. de Noailles mourut. J'en
ai suffisamment parlé à l'occasion de la mort de la mère de M. de
Noailles. J'ajouterai qu'en ses dernières années, elle s'était retirée
aux Enfants-Trouvés, et que là même elle fut suivie par ses amis, et
visitée de la meilleure compagnie de la cour et de la ville qui avait
accoutumé de la voir chez elle. Elle avait plus de quatre-vingts ans\,;
elle n'avait jamais fait grand cas de son mari ni de son fils
l'ambassadeur en Suisse\,; elle ne l'appelait jamais que Michaut\,; il
ne la voyait guère que les matins, ni sa femme non plus, qui était une
autre intrigante qui ne valait pas sa belle-mère, et qui aurait voulu
l'imiter. La bonne femme ne voulait point mêler ce bagage-là avec la
bonne compagnie dont sa maison était toujours remplie.

M\textsuperscript{me} de Navailles mourut le même jour, 14 février\,:
son nom était Baudéan, et son père s'appelait le comte de Neuillant,
était gouverneur de Niort et frère cadet de M. de Parabère, chevalier de
l'ordre en 1633, et gouverneur de Poitou. Il laissa sa femme veuve assez
longtemps, qui s'appelait Tiraqueau et qui était l'avarice même. Je ne
puis dire par quelle raison, ou hasard, M\textsuperscript{me} de
Maintenon, revenant jeune et pauvre fille d'Amérique, où elle avait
perdu père et mère, tomba en débarquant à la Rochelle chez
M\textsuperscript{me} de Neuillant qui démeurait en Poitou. Elle ne put
se résoudre à lui donner du pain sans en tirer quelque service\,: elle
la chargea donc de la clef de son grenier pour donner le foin et
l'avoine par compte, et l'aller voir manger à ses chevaux. Ce fut elle
qui la mena à Paris, et qui, pour s'en défaire, la maria à Scarron. Elle
retourna chez elle en Poitou. Son fils unique fut tué, sans alliance, à
la bataille de Lens. M\textsuperscript{me} de Navailles était sa fille
aînée, et la cadette épousa le comte de Froulay, grand maréchal des
logis de la maison du roi en 1650, quatorze ans après ce mariage,
chevalier de l'ordre en 1664, et mort à soixante-dix ans en 1671, et
elle en 1678. M. de Froulay, ambassadeur à Venise, l'évêque du Mans, le
bailli de Froulay, sont petits-fils de ce mariage, et cousins issus
germains cadets du comte de Tessé, fils du maréchal de Tessé,
petits-fils des deux frères. Ces deux sœurs étaient filles d'honneur de
la reine régente, et l'aînée devait {[}être{]} et a été en effet fort
riche. M. de Navailles s'était entièrement attaché au cardinal Mazarin
et commandait sa compagnie de chevau-légers, car il avait en petit une
maison militaire comme le roi. Navailles était homme de qualité de
Gascogne, de ces gens de l'ancienne roche, pleins d'honneur, de valeur
et de fidélité à toute épreuve, comme il le montra bien au cardinal
Mazarin dans les temps les plus critiques de sa vie. C'était lui qui
avait le secret de ses retraites, de ses adresses, de ses chiffres dans
tous ses deux éloignements, et qui avec grand péril demeura dans son
attachement à visage découvert, que rien ne put ébranler, et le canal le
plus sûr du cardinal. Cette conduite, qui, quelque décrié que fût le
cardinal, lui fit beaucoup d'honneur, lui valut aussi la confiance
entière et toute la faveur du cardinal et de la reine, auprès de qui il
l'avait toujours laissé dans ses retraites. Il aima mieux que son père,
qui n'avait jamais vu la cour, fût duc à brevet que lui. Il le fut après
sa mort, et par degrés il devint capitaine des gens d'armes de la garde,
gouverneur de Bapaume, puis du Havre-de-Grâce, et de la Rochelle et pays
d'Aunis, capitaine général, général de l'armée d'Italie et en Catalogne
avec succès, ambassadeur plénipotentiaire vers les princes d'Italie,
chevalier de l'ordre en 1661, enfin maréchal en 1675. Il servit beaucoup
sous M. le Prince, qui l'estimait fort, et il mourut gouverneur de M. le
duc de Chartres, 5 février 1685, n'y ayant pas été deux ans, et n'en
ayant que soixante-cinq. C'était un grand homme, maigre, jaune, poli,
qui ne lais soit pas d'avoir des dits et des naïvetés étranges, et qui
était ignorant. Il fut un jour étrangement rabroué par M. le Prince qui
était fort en peine en Flandre du cours exact d'un ruisseau que ses
cartes ne marquaient point, à qui, pour y suppléer, il alla chercher une
mappemonde. Une autre fois, étant allé voir M. Colbert à Sceaux qui le
promena partout, il ne loua jamais que la chicorée de son potager, et
lorsqu'à l'occasion des huguenots on parlait de la difficulté de changer
de religion, il assura que si Dieu lui avait fait la grâce de le faire
naître Turc, il le serait demeuré. C'était un homme fort propre à
inspirer la vertu et la piété par son exemple, mais qui ne l'était à
être gouverneur de M. de Chartres que par sa décoration, qui flattait
extrêmement Monsieur.

M\textsuperscript{me} de Navailles, depuis son mariage en 1651, était
souvent en Guyenne. La maréchale de Guébriant, nommée dame d'honneur de
la reine à son mariage, étant morte en allant joindre la cour à
Bordeaux, M\textsuperscript{me} de Navailles qui était dans ses terres
fut mise en sa place, où personne ne convenait plus qu'elle au cardinal
Mazarin et à la reine mère. C'était une femme d'esprit et qui avait
conservé beaucoup de monde, malgré ses longs séjours en province, et
d'autant de vertu que son mari. La reine eut des filles d'honneur, et
les filles d'honneur avec leurs gouvernante et sous-gouvernante sont
dans l'entière dépendance de la dame d'honneur. Le roi était jeune et
galant. Tant qu'il n'en voulut point à la chambre des filles,
M\textsuperscript{me} de Navailles ne s'en mit pas en peine\,; mais elle
avait l'œil ouvert sur ce qui la regardait. Elle s'aperçut que le roi
commençait à s'amuser, et bientôt après elle apprit qu'on avait
secrètement percé une porte dans leur chambre, qui donnait sur un petit
degré par lequel le roi y montait la nuit, et que le jour cette porte
était cachée par le dossier d'un lit. Elle tint sur cela conseil avec
son mari. Ils mirent la vertu et l'honneur d'un côté\,; la colère du
roi, la disgrâce, le dépouillement, l'exil de l'autre\,; ils ne
balancèrent pas. M\textsuperscript{me} de Navailles prit si bien son
temps, pendant le jeu et le souper de la reine, que la porte fut
exactement murée, et qu'il n'y parut pas. La nuit le roi, pensant entrer
par ce petit degré, fut bien étonné de ne trouver plus de porte. Il
tâte, il cherche, il ne comprend pas comment il s'est mépris, et
découvre enfin qu'elle est devenue muraille. La colère le saisit, il ne
doute point que ce ne soit un trait de M\textsuperscript{me} de
Navailles, et qu'elle ne l'a pas fait sans la participation de son mari.
Du dernier, il ne put l'éclaircir que par la connaissance qu'il avait
d'eux\,; mais pour la porte, il s'en informa si bien qu'il sut
positivement que c'était M\textsuperscript{me} de Navailles qui l'avait
fait murer. Aussitôt il leur envoie demander la démission de toutes
leurs charges, et ordre de s'en aller chez eux en Guyenne (c'était en
juin 1664), et en va faire ses plaintes à la reine mère dont il les
savait fort protégés. La reine mère, qui avait un grand crédit sur le
roi, l'employa tout entier pour parer ce coup. Tout ce qu'elle put
obtenir ce fut de leur sauver le gouvernement de la Rochelle et du pays
d'Aunis, et de les y faire envoyer\,; mais tout le reste sauta. M. de
Saint-Aignan acheta le Havre, M. de Chaulnes, les chevau-légers de la
garde, et M\textsuperscript{me} de Montausier fut dame d'honneur, sans
quitter sa place de gouvernante de Mgr le Dauphin. Les suites ont fait
voir que le roi se connaissait bien en gens, et qu'il n'en pouvait
choisir une plus commode, malgré toute la morale et la vertu de l'hôtel
de Rambouillet et l'austérité de M. de Montausier. L'exil ne fut pas
long. La reine mourut tout au commencement de 1666, et en mourant elle
demanda au roi son fils le retour et le pardon de M. et de
M\textsuperscript{me} de Navailles, qui ne put la refuser. Le mari est
devenu neuf ans depuis maréchal de France, et, quoique simple duc à
brevet, n'a jamais porté le titre de maréchal, ni sa femme de maréchale.
Elle parut le reste de sa vie fort rarement et des moments à la cour.
M\textsuperscript{me} de Maintenon ne pouvait lui refuser des
distinctions et des privances, mais rares et momentanées. Le roi se
souvenait toujours de sa porte, et elle du foin et de l'avoine de
M\textsuperscript{me} de Neuillant\,; les années ni la dévotion n'en
avaient pu amortir l'amertume.

M\textsuperscript{me} de Navailles est la dernière femme à qui j'ai vu
conserver le bandeau qu'autrefois les veuves portaient toute leur vie.
Il n'avait rien de commun avec le deuil, qui ne se portait que deux
ans\,; aussi ne le porta-t-elle pas davantage, mais toujours ce petit
bandeau qui finissait en pointe vers le milieu du front. Quand elle
venait à Versailles, c'était toujours avec une considération marquée de
toute la cour, tant la vertu se fait respecter, et le roi lui faisait
toujours quelque honnêteté, mais froide. Il n'y aurait qu'à la louer
s'il n'y avait pas mille contes plus étranges et plus plaisants les uns
que les autres de son avarice, trop nombreux à rapporter. M. de
Navailles ne laissa que trois filles. Il avait marié la seconde à
Rothelin qui fut tué à\ldots.\footnote{Henri d'Orléans, marquis de
  Rothelin, guidon des gens d'armes du roi, mourut le 19 septembre 1691
  des suites des blessures qu'il avait reçues au combat de Leuze.}, et
qui a laissé des enfants. Pompadour épousa par amour la troisième, dont
il n'a eu que M\textsuperscript{me} de Courcillon\,; et l'aînée, depuis
la mort du père, fut la troisième femme de M. d'Elbœuf, dont elle eut
M\textsuperscript{me} de Mantoue. Tout cela avant ce dernier mariage
logeait à l'hôtel de Navailles, où faute de pavé on s'embourbait dans la
cour, quoique M\textsuperscript{me} de Navailles fût fort comptée et
visitée. Ses gens mouraient de faim et ses filles aussi, dont l'aînée,
qui se mêlait tant, qu'elle pouvait de la dépense, grappillait dessus
pour se donner un morceau en cachette avec ses sœurs quand leur mère
était couchée. M. et M\textsuperscript{me} de Navailles étaient
extrêmement des amis de mon père.

Un bon homme, mais fort ridicule, mourut en même temps. Ce fut un M.
Lavocat, maître des requêtes, frère de M\textsuperscript{me} de Pomponne
et de M\textsuperscript{me} de Vins, qui avait des bénéfices et beaucoup
de bien, qui allait partout, qui avait eu toute sa vie la folie du beau
monde, et de ne rien faire qu'être amoureux des plus belles et des plus
hautes huppées, qui riaient de ses soupirs et lui faisaient des tours
horribles.

C'était avec cela un grand homme, maigre, jaune comme un coing, et qui
l'avait été toute sa vie, et qui tout vieux qu'il était voulait encore
être galant.

Une femme de vertu et d'un vrai mérite mourut en même temps, veuve de
Maulevrier, chevalier de l'ordre, frère de MM. Colbert et de Croissy.
Elle était sœur de M\textsuperscript{me} de Vaubrun, Beautru en son nom,
et fille de Serrant, autrefois chancelier de Monsieur. Elle laissa un
fils, gendre du comte de Tessé, dont j'aurai occasion de parler, un
autre fils, et une fille mariée à Médavy, mort chevalier de l'ordre, et
enfin maréchal de France, sans enfants.

Biron, qui si longtemps depuis a fait une fortune complète en biens et
en honneurs, et qui l'a toute sa vie attendue dans la plus dure
indigence, perdit un père obscur, qui, après la mort de sa femme, qui
était Cossé tante paternelle de la maréchale de Villeroy, épousa une
servante avec laquelle il acheva de se confiner et n'en eut point
d'enfants.

Le chevalier de Villeroy se noya dans la capitane\footnote{La galère
  capitane était celle qui portait le commandant de la flotte.} de
Malte, qui coula à fond en attaquant un bâtiment turc de quatorze pièces
de canon. Spinola était le général, qui se sauva seul avec le chevalier
de Saint-Germain et deux matelots\,; tout le reste fut noyé. Ce
chevalier de Villeroy était beau et bien fait, et n'avait nulle envie de
faire ses caravanes\,; mais le maréchal de Villeroy qui ne voulait qu'un
aîné, qui destinait le second à l'Église pour en faire un archevêque de
Lyon, et qui avait fort gaillardement marié une fille en Portugal et
cloîtré les autres, força ce troisième fils à partir et eut tout lieu de
s'en repentir. J'avais été élevé avec lui et avec l'abbé son frère, qui
ne le valait pas à beaucoup près. Cela fit le raccommodement de la
famille, brouillée depuis l'affaire, que j'ai racontée en son temps, qui
obligea la princesse d'Harcourt, par ordre du roi, à demander
publiquement pardon à la duchesse de Rohan-Chabot, à Versailles, chez
M\textsuperscript{me} la chancelière, dans laquelle M. le Grand avait
voulu donner le change au roi sur M\textsuperscript{me} de Saint-Simon,
à qui j'expliquai le fait, dont M. le Grand essuya pour lui et pour
M\textsuperscript{me} d'Armagnac une petite réprimande, qui l'outra
d'autant plus qu'il était fort accoutumé à tout le contraire.
M\textsuperscript{me} d'Armagnac, faute de mieux, s'en prit à elle-même
pour piquer son frère, et dégoisa sur sa propre naissance d'une manière
fort fâcheuse. Ils ne s'étaient pas vus depuis. La réconciliation était
d'autant plus difficile que le maréchal de Villeroy était
personnellement ami intime de M. le Grand et du chevalier de Lorraine,
et fort aussi de M. de Marsan, et qu'il y mettait une dose de
subordination fort à leur goût et fort peu de celui de la maréchale.
Cette triste occasion fit entremettre des amis communs pour que, sans
parler plus de ce qui s'était passé, le maréchal et la maréchale
voulussent bien recevoir la visite de M. le Grand et de
M\textsuperscript{me} d'Armagnac. Ils se raccommodèrent en effet, et
furent aussi bien depuis que jamais\,; mais pour les belles-sœurs, qui
n'eurent en aucun temps que des bienséances réciproques, cela ne les
réchauffa pas plus qu'à l'ordinaire.

Ils perdirent en même temps un fort honnête homme, brave et autrefois
beau et bien fait, mais qui n'était pas fait pour être leur beau-frère.
Il s'appelait M. d'Hauterive. Son nom était Vignier, comme la mère de M.
de Noyon-Tonnerre, et ces Vignier n'avaient aucune naissance. Celui-ci
avait servi avec réputation et avait été cornette des chevau-légers de
la reine mère. La sœur du maréchal de Villeroy, aînée de
M\textsuperscript{me} d'Armagnac, veuve en premières noces du dernier de
la maison de Tournon, en secondes, du duc de Chaulnes, frère aîné de
celui qui a été ambassadeur à Rome, etc., et gouverneur de Bretagne,
puis de Guyenne, s'amouracha de ce M. d'Hauterive, et l'épousa
publiquement malgré toute sa famille, qui ne l'a jamais voulu voir
depuis.

Hauterive se conduisit avec tant d'égards et de respects avec le
maréchal de Villeroy et M. et M\textsuperscript{me} d'Armagnac, qu'au
bout de quelque temps ils voulurent bien le voir, et l'ont toujours bien
traité toute sa vie. Toute sa vie aussi il fut galant, jusque dans sa
vieillesse. Il y a lieu de juger qu'il en mourut. Il se trouva fort mal
après avoir mis une paire de gants, et mourut brusquement avec des
symptômes qui persuadèrent qu'il en avait été empoisonné. Il était mal
avec sa femme depuis assez longtemps, qui vivait fort obscure.

Cossé enfin termina ses affaires et fut reçu duc et pair au parlement,
bien servi par la liaison qui était entre le maréchal de Villeroy et le
premier président Harlay. Je ne répète rien de cette affaire que j'ai
expliquée à l'occasion de la mort du duc de Brissac, mon beau-frère,
frère de la maréchale de Villeroy et cousin germain de celui-ci.

Rome perdit en Casanata un de ses plus illustres cardinaux, par sa
piété, par sa doctrine, par le nombre et le choix des livres qu'il
ramassa, et par le bien qu'il fit aux lettres. Il légua sa bibliothèque
à la Minerve, à Rome, la rendit publique et y joignit tout ce qui était
nécessaire pour l'entretenir et la rendre utile. Il mourut
bibliothécaire de l'Église, dans la vingt-troisième année de son
cardinalat.

M. d'Elbœuf attrapa assez adroitement quatre-vingt mille livres du
roi\,; il lui proposa de séparer l'Artois de son gouvernement de
Picardie, et de lui permettre de vendre, et qu'il en trouvait cent mille
écus. Le roi qui ne voulut ni de cette nouveauté ni du premier venu pour
gouverneur d'Artois, qui ne pouvait être autre puisqu'il en voulait bien
donner cent mille écus, mais qui toute sa vie avait eu du faible pour M.
d'Elbœuf, crut y gagner que de lui donner cette gratification en le
refusant de la vente, et sûrement M. d'Elbœuf n'y perdit pas.

Presque en même temps, le roi envoya cent mille francs à
M\textsuperscript{me} de Montespan pour lui aider à faire l'acquisition
d'Oiron. Ce présent ne fut pas gratuit. M\textsuperscript{me} de
Montespan était déjà dans la pénitence, elle avait renvoyé au roi depuis
quelque temps un parfaitement beau fil de perles qu'elle en avait eu, et
qu'il donna encore augmenté à M\textsuperscript{me} la duchesse de
Bourgogne\,; il était alors de vingt et une perles admirables, et valait
cent cinquante mille livres. M\textsuperscript{me} de Montespan, entre
autres réparations, s'appliquait à former du bien à d'Antin. Elle aurait
pu mieux choisir qu'Oiron, beau château et beau parc à la vérité en
Poitou, et qui avait fait la demeure et les délices des ducs de
Roannais\,; mais cette terre relevait de celle de Thouars avec une telle
dépendance, que toutes les fois qu'il plaisait au seigneur de Thouars il
mandait à celui d'Oiron qu'il chasserait un tel jour dans son voisinage,
et qu'il eût à abattre une certaine quantité de toises des murs de son
parc pour ne point trouver d'obstacles en cas que la chasse s'adonnât à
y entrer. On comprend que c'est un droit si dur qu'on ne s'avise pas de
l'exercer\,; mais on comprend aussi qu'il se trouve des occasions où on
s'en sert dans toute son étendue, et alors que peut devenir le seigneur
d'Oiron\,?

\hypertarget{chapitre-xxiii.}{%
\chapter{CHAPITRE XXIII.}\label{chapitre-xxiii.}}

1700

~

{\textsc{Force bals à la cour.}} {\textsc{- Bal de M. le Prince\,;
quatre visages.}} {\textsc{- Malice cruelle de M. le Prince à un bal à
Marly.}} {\textsc{- Ordre des bals chez le roi.}} {\textsc{- Bal de la
chancellerie.}} {\textsc{- M. de Noirmoutiers\,; ses mariages.}}
{\textsc{- La Bourlie hors du royaume.}} {\textsc{- Dettes de jeu de
M\textsuperscript{me} la Duchesse payées par le roi.}} {\textsc{-
Langlée.}} {\textsc{- Acquisition de l'hôtel de Guise.}} {\textsc{- Abbé
de Soubise passe adroitement chanoine de Strasbourg\,; ses progrès.}}
{\textsc{- Cardinal de Fürstemberg\,; sa famille.}} {\textsc{- Comtesse
de Fürstemberg.}} {\textsc{- Coadjutoreries de Strasbourg.}} {\textsc{-
Conduite et disgrâce du cardinal de Bouillon\,; sa désobéissance.}}
{\textsc{- Mariage d'une fille du duc de Rohan avec le comte de La
Marck\,; sa naissance et sa fortune.}} {\textsc{- Mariage du prince
d'Isenghien avec M\textsuperscript{lle} de Fürstemberg.}} {\textsc{-
Mariage du duc de Berwick avec M\textsuperscript{lle} Bockley.}}

~

Dès avant la Chandeleur jusqu'au carême, ce ne fut que bals et plaisirs
à la cour. Le roi en donna à Versailles et à Marly, mascarades
ingénieuses, entrées, espèces de fêtes qui amusèrent fort le roi, sous
le prétexte de M\textsuperscript{me} la duchesse de Bourgogne. Il y eut
des musiques et des comédies particulières chez M\textsuperscript{me} de
Maintenon. Monseigneur donna aussi des bals, et les principales
personnes se piquèrent d'en donner à M\textsuperscript{me} la duchesse
de Bourgogne. M. le Prince, dans son appartement composé de peu de
pièces et petites, trouva moyen de surprendre la cour par la fête du
monde la plus galante, la mieux entendue et la mieux ordonnée. Un bal
paré, des masques, des entrées, des boutiques de tout pays, une
collation dont la décoration fut charmante\,; le tout sans refuser
personne de la cour, et sans foule ni embarras.

Une femme, depuis fort de mes amies, et qui, quoique bien jeune,
commençait à pointer par elle-même à la cour, qui y figura tôt après, et
qui y serait parvenue apparemment aux situations les plus flatteuses, si
la petite vérole ne l'eût emportée quelques années après, y essuya une
triste aventure. Le comte d'Évreux lui avait plu\,; à peine
commençait-on à s'en apercevoir. Un masque entra vers le milieu du bal
avec quatre visages de quatre personnes de la cour\,; celui du comte
d'Évreux en était un, et tous quatre en cire parfaitement ressemblants.
Ce masque était couvert d'une robe ample et longue qui dérobait sa
taille, et avait dans cette enveloppe le moyen de tourner ces visages
tout comme il voulait avec facilité et à tous moments. La singularité de
la mascarade attira tous les yeux sur lui. Il se fit force commentaires
sur les quatre visages, et il ne fut pas longtemps sans être pris à
danser. En ce premier menuet il tourna et retourna ses visages et en
divertit fort la compagnie. Quand il l'eut achevé, voilà mon démon qui
s'en va faire la révérence à cette pauvre femme, en lui présentant le
visage du comte d'Évreux. Ce n'est pas tout, il dansait bien et était
fort maître de sa danse, tellement qu'il eut la malice de si bien faire
que quelques tours et retours qu'il fit en ce menuet, ce même visage
tourna toujours si à point, et avec tant de justesse, qu'il fut toujours
vis-à-vis de la dame avec qui il dansait. Elle était cependant de toutes
les couleurs\,; mais, sans perdre contenance, elle ne songea qu'à couper
court. Dès le deuxième tour elle présente la main\,; le masque fait
semblant de la prendre, et d'un autre temps léger s'éloigne et fait un
autre tour. Elle croit au moins à celui-là être plus heureuse\,; point
du tout, même fuite et toujours ce visage sur elle. On peut juger quel
spectacle cela donna, les personnes les plus éloignées en pied, d'autres
encore plus reculées debout sur les bancs, pourtant point de huée. La
dame était grande dame, grandement apparentée, et de gens en place et en
crédit. Enfin elle en eut pour le triple au moins d'un menuet ordinaire.
Ce masque demeura encore assez longtemps, puis trouva le moyen de
disparaître sans qu'on s'en aperçût. Le mari masqué vint au bal dans ce
temps-là\,; un de ses amis en sortait, je crois pour l'attendre\,; il
lui dit qu'il y avait un flot de masques, qu'il ferait bien de laisser
sortir s'il ne voulait étouffer, et le promena en attendant dans la
galerie des Princes. À la fin il s'ennuya et voulut entrer\,; il vit le
masque à quatre visages, mais quoiqu'il en fût choqué, il n'en fit pas
semblant, et son ami lui avait sauvé le menuet. Cela fit grand bruit,
mais n'empêcha pas le cours des choses qui dura quelque temps. Ce qui
est fort rare, c'est que ni devant ni depuis, il n'a été question de
personne avec elle, quoique ce fût un des plus beaux visages de la cour,
et qui sérieuse à un cercle ou à une fête, défaisait toutes les autres
femmes et même plus belles qu'elle.

Un des bals de Marly donna encore une ridicule scène. J'en nommerai les
acteurs, parce que la conduite publique ne laisse rien à apprendre. M.
et M\textsuperscript{me} de Luxembourg étaient à Marly. On manquait
assez de danseurs et de danseuses, et cela fit aller
M\textsuperscript{me} de Luxembourg à Marly, mais avec grand'peine,
parce qu'elle vivait de façon qu'aucune femme ne voulait la voir. On en
était là encore quand le désordre était à un certain point\,; maintenant
on est malheureusement revenu de ces délicatesses. M. de Luxembourg
était peut-être le seul en France qui ignorât la conduite de sa femme,
qui vivait aussi avec lui avec tant d'égards, de soins et d'apparente
amitié, qu'il n'avait pas la moindre défiance d'elle. Par même raison de
faute de gens pour danser, le roi fit danser ceux qui en avaient passé
l'âge, entre autres M. de Luxembourg. Il fallait être masqué\,; il
était, comme on a vu, fort des amis de M. le Duc et de M. le prince de
Conti, et fort bien aussi avec M. le Prince, qui était l'homme du monde
qui avait le plus de goût pour les fêtes, les mascarades et les
galanteries. Il s'adressa donc à lui pour le masquer. M. le Prince,
malin plus qu'aucun singe, et qui n'eut jamais d'amitié pour personne, y
consentit pour s'en divertir et en donner une farce à toute la cour\,;
il lui donna à souper, puis le masqua à sa fantaisie.

Ces bals de Marly, rangés ou en masque, étaient toujours comme à
Versailles un carré long. Le fauteuil du roi, ou trois quand le roi et
la reine d'Angleterre y étaient, ce qui arrivait souvent, et des deux
côtés sur la même ligne la famille royale, c'est-à-dire jusqu'au rang de
petit-fils de France inclusivement. Quelquefois par dérangement, au
milieu du bal, M\textsuperscript{me} la Duchesse, et
M\textsuperscript{me} la princesse de Conti s'approchaient sous prétexte
de causer avec quelqu'un à côté ou derrière, et s'y mettaient aux
dernières places. Les dames, les titrées les premières et sans mélange,
puis les autres, occupaient les deux côtés longs à droite et à gauche\,;
et vis-à-vis du roi les danseurs, princes du sang et autres\,; et les
princes du sang qui ne dansaient pas, avec les courtisans derrière les
dames\,; et quoique en masque, tout le monde d'abord à visage découvert,
le masque à la main. Quelque temps après le bal commencé, s'il y avait
des entrées ou des changements d'habits, ceux et celles qui en étaient
en différentes troupes avec un prince ou une princesse sortaient, et
alors on revenait masqué, et on ne savait en particulier qui étaient les
masques. J'étais, moi surtout et plusieurs de nous, demeuré tout à fait
brouillé avec M. de Luxembourg. Je venais d'arriver, et j'étais déjà
assis lorsque je vis par derrière force mousseline plissée, légère,
longue et voltigeante, surmontée d'un bois de cerf au naturel sur une
coiffure bizarre, si haut qu'il s'embarrassa dans un lustre. Nous voilà
tous bien étonnés d'une mascarade si étrange, à nous demander avec
empressement, qui est-ce et à dire qu'il fallait que ce masque-là fût
bien sûr de son front pour l'oser parer ainsi, lorsque le masque se
tourne et nous montre M. de Luxembourg. L'éclat de rire subit fut
scandaleux. Le hasard fit qu'un moment après, il vint s'asseoir entre M.
le comte de Toulouse et moi, qui aussitôt lui demanda où il avait été
prendre cette mascarade. Le bon seigneur n'y entendit jamais finesse, et
la vérité est aussi qu'il était fort éloigné d'être fin en rien. Il prit
bénignement les rires, qui ne se pouvaient contenir, comme excités par
la bizarrerie de sa mascarade, et raconta fort simplement que c'était M.
le Prince à qui il s'était adressé, chez qui il avait soupé, et qui
l'avait ajusté ainsi\,; puis se tournant à droite et à gauche se faisait
admirer et se pavanait d'être masqué par M. le Prince. Un moment après
les dames arrivèrent et le roi aussitôt après elles. Les rires
recommencèrent de plus belle, et M. de Luxembourg à se présenter de plus
belle aussi à la compagnie avec une confiance qui ravissait. Sa femme,
toute connue qu'elle fût, et qui ne savait rien de cette mascarade, en
perdit contenance, et tout le monde à les regarder tous deux et toujours
à mourir de rire. M. le Prince en arrière du service, qui est des
charges qui se placent derrière le roi, regardait par la chatière et
s'applaudissait de sa malice noire. Cet amusement dura tout le bal, et
le roi, tout contenu qu'il était toujours, riait aussi, et on ne se
laissait point d'admirer une invention si cruellement ridicule, ni d'en
parler les jours suivants.

Il n'y avait soir qu'il n'y eût bal. M\textsuperscript{me} la
chancelière en donna un à la chancellerie, qui fut la fête la plus
galante et la plus magnifique qu'il fût possible. Le chancelier y reçut
à la portière Monseigneur, les trois princes ses fils et
M\textsuperscript{me} la duchesse de Bourgogne sur les dix heures du
soir, puis s'alla coucher au château. Il y eut des pièces différentes
pour le bal paré, pour les masques, pour une collation superbe, pour des
boutiques de tout pays, Chinois, Japonais, etc., qui vendaient des
choses infinies et très recherchées pour la beauté et la singularité,
mais qui n'en recevaient point d'argent\,: c'étaient des présents à
M\textsuperscript{me} la duchesse de Bourgogne et aux dames. Une musique
à sa louange, une comédie, des entrées. Rien de si bien ordonné et de si
superbe, de si parfaitement entendu\,; et la chancelière s'en démêla
avec une politesse, une galanterie et une liberté, comme si elle n'eût
eu rien à faire. On s'y divertit extrêmement, et on sortit après huit
heures du matin. M\textsuperscript{me} de Saint-Simon qui suivit
toujours M\textsuperscript{me} la duchesse de Bourgogne, et c'était
grande faveur, et moi, fûmes les dernières trois semaines sans jamais
voir le jour. On tenait rigueur à certains danseurs de ne sortir du bal
qu'en même temps que M\textsuperscript{me} la duchesse de Bourgogne, et
m'étant voulu sauver un matin à Marly, elle me consigna aux portes du
salon\,; nous étions plusieurs de la sorte. Je fus ravi de voir arriver
les Cendres, et j'en demeurai un jour ou deux étourdi, et
M\textsuperscript{me} de Saint-Simon à bout ne put fournir le mardi
gras. Le roi joua aussi chez M\textsuperscript{me} de Maintenon, avec
quelques dames choisies, au brelan et à petite prime, quelquefois au
reversi, les jours qu'il n'y avait point de ministres, ou que leur
travail était court, et cet amusement se prolongea un peu dans le
carême.

M. de Noirmoutiers épousa, ce carnaval-ci, la fille d'un président en la
chambre des comptes, qui s'appelait Duret de Chevry. Il était veuf dès
1689 de la veuve de Bermont, conseiller au parlement de Paris, fille de
La Grange-Trianon, président aux requêtes du palais, qu'il avait épousée
au commencement de 1688, et n'eut point d'enfants de l'une ni de
l'autre. Il était de la maison de La Trémoille et son trisaïeul était
frère du premier duc de La Trémoille et du baron de Royan et d'Olonne,
de manière que le duc de La Trémoille, gendre du duc de Créqui, et lui
étaient petits-fils des cousins germains. Il était frère de la célèbre
princesse des Ursins, de M\textsuperscript{me} de Royan, mère de la
duchesse de Châtillon, de la duchesse Lanti et de l'abbé de La
Trémoille, auditeur de rote\footnote{Tribunal séant à Rome. Dangeau
  explique, dans son \emph{Journal}, à la date du 19 août 1686, la
  destination et l'organisation de ce tribunal\,: «\,La rote est un
  tribunal qui juge les causes importantes de l'État ecclésiastique et
  quelques autres qui y viennent, par appel, des États catholiques de
  l'Europe. Ce tribunal se compose de douze juges qu'on nomme auditeurs.
  Il y a un Français, deux Espagnols, un Allemand\,; les autres huit
  sont Italiens. Pour juger les causes, ces douze auditeurs se partagent
  en trois bureaux\,; chacun est composé de quatre auditeurs. Quand une
  cause a été jugée par un de ces bureaux, on la porte devant le second
  et ensuite devant le troisième, et l'affaire n'est point jugée
  définitivement qu'il n'y ait trois sentences conformes, et qu'elle
  n'ait passé comme roulée par ces trois petits bureaux, c'est ce qui
  fait que tout le corps de ces juges entre lesquels on fait ainsi
  rouler les causes, se nomme en italien \emph{la rota}.\,»}, mort
cardinal. Il était beau, bien fait, agréable, avec beaucoup d'esprit et
d'envie de se distinguer et de s'élever. Il n'avait pas vingt ans,
lorsque allant trouver la cour à Chambord, la petite vérole l'arrêta à
Orléans, sortit bien, et comme il touchait à la guérison, sortit une
deuxième fois et l'aveugla. Il en fut si affligé qu'il demeura vingt ans
et plus sans vouloir que personne le vit, enfermé à se faire lire. Avec
beaucoup d'esprit et de mémoire, il n'était point distrait et n'avait
que cet unique amusement qui le rendit fort savant en toutes sortes
d'histoires. Le comte de Fiesque, son ami de jeunesse, alla enfin loger
avec lui, et le tourmenta tant, qu'il le força à souffrir quelque
compagnie. De l'un à l'autre il eut bientôt du monde, et sa maison
devint un réduit du meilleur et du plus choisi par l'agrément de sa
conversation, et peu à peu par la sûreté que l'on reconnut dans son
commerce, et dans la suite par la bonté solide de ses conseils. C'était
un esprit droit, qui avait une grande justesse et une grande facilité à
concevoir et à s'énoncer. Il eut, sans sortir de chez lui, les amis les
plus considérables par leurs places et par leur état\,; il se mêla d'une
infinité de choses et d'affaires\,; et sans jamais faire l'important, il
le devint en effet, et sa maison un tribunal dont l'approbation était
comptée, et où on était flatté d'être admis. Le prodige fut que, quoique
pauvre, il se bâtit une maison charmante à Paris, vers le bout de la rue
Grenelle, qu'il en régla la distribution et les proportions, et en gros
et en détail les dégagements, les commodités, et jusqu'aux ornements,
aux glaces, aux corniches, aux cheminées, et au tact choisit des étoffes
pour les meubles en lui en disant les couleurs. Il était fils du marquis
de Noirmoutiers, qui intrigua tant dans les troubles de la minorité et
de la jeunesse du roi, et qui en tira un brevet de duc avec le
gouvernement du Mont-Olympe.

La Bourlie, frère de Guiscard, avait quitté après avoir servi longtemps,
et s'était retiré dans une terre vers les Cévennes, où il se mit à vivre
avec beaucoup de licence. Vers ce temps-ci il fut volé chez lui, il en
soupçonna un domestique, et sans autre façon lui fit de son autorité
donner en sa présence une cruelle question. Cela ne put demeurer si
secret que les plaintes n'en vinssent. Il y allait de la tête\,; La
Bourlie sortit du royaume, où il fit d'étranges personnages jusqu'à sa
mort, qui le fut encore plus, mais dont il n'est pas temps de parler.

M\textsuperscript{me} la Duchesse dont le roi avait payé les dettes il
n'y avait pas longtemps, qui se montaient fort haut, à des marchands et
en toutes sortes de choses, n'avait pas osé parler de celles du jeu qui
allaient à de grosses sommes. Ses dettes augmentaient encore\,; elle se
trouvait tout à fait dans l'impuissance de les payer, et par là même
dans le plus grand embarras du monde. Ce qu'elle craignait le plus était
que M. le Prince, et surtout M. le Duc, ne le sût. Dans cette extrémité,
elle prit le parti de s'adresser à son ancienne gouvernante, et de lui
exposer son état au naturel dans une lettre avec une confiance qui
attira sa toute-puissante protection. Elle n'y fut pas trompée,
M\textsuperscript{me} de Maintenon eut pitié de sa situation, et obtint
que le roi payât ses dettes, ne lui fit point de réprimandes et lui
garda le secret. Langlée, espèce d'homme fort singulier dans une cour,
fut chargé, de dresser tous les états de ses dettes avec elle, de
toucher les payements du roi, et de les faire ensuite à ceux à qui
M\textsuperscript{me} la Duchesse devait, qui en peu de semaines se
trouva quitte, sans que personne de ceux qu'elle craignait sût les
dettes ni l'acquittement.

Sans aller plus loin, disons un mot de ce Langlée. C'était un homme de
rien, de vers Mortagne au Perche, dont le père s'était enrichi et la
mère encore plus. L'un avait acheté une charge de maréchal des logis de
l'armée pour se décorer, qu'il n'avait jamais faite\,; l'autre avait été
femme de chambre de la reine mère, fort bien avec elle, intrigante qui
s'était fait de la considération et des amis, et qui avait produit son
fils de bonne heure parmi le grand monde, où il s'était mis dans le jeu.
Il y fut doublement heureux, car il y gagna un bien immense, et ne fut
jamais soupçonné de la moindre infidélité. Avec très peu ou point
d'esprit, mais une grande connaissance du monde, il sut prêter de bonne
grâce, attendre de meilleure grâce encore, se faire beaucoup d'amis et
de la réputation à force de bons procédés. Il fut des plus grosses
parties du roi du temps de ses maîtresses. La conformité de goût
l'attacha particulièrement à Monsieur, mais sans dépendance et sans
perdre le roi de vue, et il se trouva insensiblement de tout à la cour
de ce qui n'était qu'agréments et futile, et qui n'en est pas une des
moindres parties à qui sait bien en profiter. Il fut donc de tous les
voyages, de toutes les parties, de toutes les fêtes de la cour, ensuite
de tous les Marlys et lié avec toutes les maîtresses, puis avec, toutes
les filles du roi, et tellement familier avec elles, qu'il leur disait
fort souvent leurs vérités. Il était fort bien avec tous les princes du
sang, qui mangeaient très souvent à Paris chez lui, où abondait la plus
grande et la meilleure compagnie. Il régentait au Palais-Royal, chez M.
le Grand et chez ses frères, chez le maréchal de Villeroy, enfin chez
tous les gens en première place. Il s'était rendu maître des modes, des
fêtes, des goûts, à tel point, que personne n'en donnait que sous sa
direction, à commencer par les princes et les princesses du sang, et
qu'il ne se bâtissait ou ne s'achetait point de maison qu'il ne présidât
à la manière de la monter, de l'orner et de la meubler.

Il avait été sur ce pied-là avec M. de Louvois, avec M. de Seignelay,
avec le maréchal d'Humières\,; il y était avec M\textsuperscript{me} de
Bouillon, avec la duchesse du Lude, en un mot avec tout ce qui était le
plus distingué et qui recevait le plus de monde. Point de mariages dont
les habits et les présents n'eussent son choix, ou au moins son
approbation. Le roi le souffrait, cela n'allait pas à plus\,; tout le
reste lui était soumis, et il abusait souvent de l'empire qu'il
usurpait. À Monsieur, aux filles du roi, à quantité de femmes, il leur
disait des ordures horribles, et cela chez elles, à Saint-Cloud, dans le
salon de Marly. Il entrait encore, et était entré toute sa vie dans
quantité de secrets de galanterie. Son commerce était sûr, et il n'avait
rien de méchant, était obligeant même et toujours porté à servir de sa
bourse ou de ses amis, et n'était mal avec personne. Il était assez vêtu
et coiffé comme Monsieur, il en avait aussi fort la taille et le
maintien\,; mais il n'était pas, comme de raison, à beaucoup près si
paré, et moins gros. Il était fort bien et fort familier avec
Monseigneur. Il avait tout un côté du visage en paralysie, et à force de
persévérance à Vichy, où il s'était bâti une maison, il put n'y plus
retourner, et n'eut plus du tout d'apoplexie. Sa sœur avait épousé
Guiscard\,; elle logeait avec lui, et Guiscard où bon lui semblait. Ils
s'aimaient et s'estimaient peu l'un l'autre\,; mais Langlée était fort
riche, et tout aussi éloigné de se marier, par conséquent fort ménagé
par sa sœur qu'il aimait et par son beau-frère. Une espèce comme
celle-là dans une cour y est assez bien, pour deux c'en serait beaucoup
trop. Finalement les personnes les plus sérieuses et les plus
importantes, et les moins en commerce avec lui, et celles-là étaient en
petit nombre, le ménagement, et il n'y avait qui que ce fût qui se
voulût attirer Langlée.

Tandis que tout était cet hiver en bals et en divertissements, la belle
M\textsuperscript{me} de Soubise, car elle l'était encore, et l'était
fort utilement toujours, travaillait à des choses plus sérieuses. Elle
venait d'acheter l'immense hôtel de Guise à fort grand marché, que le
roi lui aida fort à payer. Elle en avait tiré une autre faveur qui ne
fut qu'une semence\,: c'était sa protection pour faire passer les
preuves de son fils pour être chanoine de Strasbourg. La mère de M. de
Soubise était Avaugour des bâtards de Bretagne\,; cela n'était déjà pas
trop bon pour un chapitre allemand, où la bâtardise est abhorrée, de
sorte qu'aucun prince du sang sorti par femme de M\textsuperscript{me}
de Montespan, ni aucune princesse du sang venue d'elle n'entrevoit dans
pas un chapitre d'Allemagne. Mais ce n'était pas là le pis. C'est que la
mère de cette Avaugour, par conséquent grand'mère de
M\textsuperscript{me} de Soubise, était Fouquet, non des Fouquet du
surintendant (et le réconfort en eût été médiocre), mais propre fille de
ce cuisinier, auparavant marmiton, après portemanteau d'Henri IV, qui, à
force d'esprit, d'adresse, de le bien servir dans ses plaisirs, le
servit dans ses affaires, devint M. de La Varenne, et fut compté le
reste de ce règne, où il s'enrichit infiniment, le même qui après la
mort d'Henri IV se retira à la Flèche, qu'il partageait avec les
jésuites, qu'il avait plus que personne fait rappeler et rétablir, et
dont j'ai raconté la mort singulière à propos du mariage d'un de ses
descendants avec une fille de Tessé. Cette La Varenne était donc la
bisaïeule de l'abbé de Soubise. Comment la compter parmi les seize
quartiers à prouver\,? comment la sauter\,? Cette difficulté n'était pas
médiocre. On ne fit ni l'un ni l'autre.

Camilly, fin Normand, de beaucoup d'esprit et d'adresse, était grand
vicaire de Strasbourg et de ces sous-chanoines sans preuves, et Labatie,
qui n'avait ni moins d'esprit, de souplesse et d'industrie, se trouvait
lieutenant de roi de Strasbourg, et tous deux gens vendus à leurs vues,
à la cour et à tout faire. Par le conseil de la comtesse de Fürstemberg,
de laquelle je parlerai après, M\textsuperscript{me} de Soubise se livra
à eux, mais avec le roi en croupe, qui leur fit parler à l'oreille en
maître et en amant\,; car, bien que le commerce fini, il le demeura
toute sa vie, ou en usa comme s'il l'eût encore été. Ces deux hommes
firent si bien que les preuves tombèrent à des commissaires, bons
Allemands, grossiers, ignorants, et fort aisés à tromper\,; on les
étourdit du grand nom de MM. de Rohan\,; on les éblouit de leurs
dignités et de leurs établissements\,; on les accabla de leur rang de
prince étranger, et on les mit aisément hors de tout doute sur les
preuves, qu'on ne leur présenta que comme une cérémonie dont personne
n'était dispensé, et dont l'abbé de Soubise avait moins besoin d'être
dispensé que personne.

Ces Avaugour prennent très franchement le nom de Bretagne. MM. de Rohan
ont épousé plusieurs filles ou sœurs des ducs de Bretagne\,; on ne le
laissa pas ignorer aux commissaires qui ne se doutèrent point de la
totale différence de cette dernière Bretagne-ci\,: et, quant à sa mère,
on la leur donna effrontément pour être d'une ancienne maison de La
Varenne en Poitou, depuis longtemps éteinte, avec qui ni les Avaugour ni
les Rohan n'eurent jamais aucune alliance. Par ces adresses, ou plutôt
hardiesses, l'abbé de Soubise passa haut à la main, fut admis et reçu
dans le chapitre, et, sa brillante Sorbonne achevée, y alla faire ses
stages, y déployer ses agréments et ses charmes, et capter le chapitre
et tout ce qui est à Strasbourg. Ce grand pas toutefois n'était que le
premier échelon et le fondement indispensable de la grandeur où la belle
dame destinait un fils, en la fortune duquel le roi ne se croyait pas
moins intéressé qu'elle, et qu'il désirait par d'autres détours égaler à
MM. du Maine et de Toulouse\,: il ne s'agissait donc de rien moins que
de lui assurer l'évêché de Strasbourg.

Quelle que fit la bonne volonté du roi pour M\textsuperscript{me} de
Soubise, il se trouvait des obstacles à cette affaire, qui furent
peut-être autant surmontés par la conjoncture que par la seule faveur.
L'abbé d'Auvergne était depuis longtemps chanoine de Strasbourg, il y
avait fait de longs séjours, il avait mis un de ses frères dans ce
chapitre\,: depuis que le cardinal de Bouillon était à Rome, il lui en
avait obtenu la première dignité, qui est celle de grand prévôt, et le
cardinal lui-même s'y était fait chanoine. L'abbé d'Auvergne était
prêtre-coadjuteur de Cluni, et son oncle, pour l'avancer, n'avait pas
trouvé au-dessous de sa vanité de le faire grand vicaire de l'archevêque
de Vienne Montmorin, et de lui en faire faire les fonctions dans ce
diocèse\,; enfin il était beaucoup plus avancé en années, en
établissements, en ancienneté à Strasbourg que l'abbé de Soubise\,; mais
il s'en fallait bien que sa réputation fût entière\,; ses mœurs étaient
publiquement connues pour être celles des Grecs, et son esprit pour ne
leur ressembler en aucune sorte. La bêtise décelait sa mauvaise
conduite, son ignorance parfaite, sa dissipation, son ambition, et ne
présentait pour la soutenir qu'une vanité basse, puante, continuelle,
qui lui attirait le mépris autant que ses mœurs, qui éloignait de lui
tout le monde, et qui le jetait dans des panneaux et des ridicules
continuels. Son frère aussi bête, plus obscur avec beaucoup moins de
monde et fort jeune, ne pouvait suppléer à rien, et le cardinal, par sa
conduite, approfondissait de plus en plus sa disgrâce.

Au contraire, tout riait à l'abbé de Soubise, dont l'extérieur montrait
qu'il était le fils des plus tendres amours. Il se distingua sur les
bancs de Sorbonne, et bien instruit et bien aidé par son habile mère, il
se dévoua toute cette célèbre école par ses manières. On lui crut assez
de fond pour hasarder de le faire prieur de Sorbonne, place passagère
qui oblige à quantité d'actes publics dont il est très difficile de se
tirer par le seul secours d'autrui. Il y brilla, et, par le soin qu'il
avait eu de se gagner la Sorbonne, les éloges allèrent encore fort au
delà du mérite. Il y en eut beaucoup du roi dans ses discours publics,
qui ne lui déplurent pas, et il sortit de cet emploi avec une réputation
extraordinaire, que son talent de se faire aimer lui acquit pour la plus
grande partie. À ces applaudissements de capacité, M\textsuperscript{me}
de Soubise y en voulut joindre d'autres encore plus importants, et pour
cela elle le mit à Saint-Magloire, séminaire alors autant à la mode
qu'il y a été peu depuis. Il était conduit par ce que les pères de
l'Oratoire avaient de meilleur dans leur congrégation, alors solidement
brillante en savoir et en piété. La Tour, leur général, était dans la
première considération que ses sermons, sa direction, sa capacité, la
sagesse de sa conduite et l'art de gouverner qu'il possédait éminemment,
lui avaient acquise, et qui, jointe à sa probité, rendaient son
témoignage d'un grand poids. Dès l'arrivée de M. de Paris dans ce grand
siège, M\textsuperscript{me} de Soubise lui avait fait sa cour\,; elle
avait toujours fort ménagé les Noailles, ennemis nés des Bouillon, avec
qui ils avaient des procès immortels et piquants pour la mouvance de
leurs principales terres de la vicomté de Turenne, où ces derniers
avaient prodigué leurs hauteurs. M. de Paris avait une attention
particulière sur Saint-Magloire\,: c'était son séminaire favori\,; il
aimait et estimait l'Oratoire, et avait toute confiance au P. de La
Tour. Il était dans l'apogée de son crédit, et sur les avancements
ecclésiastiques, l'estime du roi et la liaison intime de
M\textsuperscript{me} de Maintenon, en partageaient, du moins alors, la
confiance entre lui et le P. de La Chaise. Ce dernier ni sa société
n'avaient pas été négligés\,; M\textsuperscript{me} de Soubise en savait
trop pour ne mettre pas de son côté un corps aussi puissant, et quand il
lui plaît aussi utile\,; et le P. de La Chaise et les principaux
bonnets, semant toujours pour recueillir, ne demandèrent pas mieux que
de servir son fils qu'ils voyaient en état d'aller rapidement à tout, et
de devenir en état de le leur rendre avec usure.

Tout était donc pour l'abbé de Soubise, et toutes les avenues de la
fortune saisies de toutes parts. Il sortit du séminaire comme il avait
fait de dessus les bancs. De là, une merveille de savoir\,; d'ici, un
miracle de piété et de pureté de mœurs. Oratoire, jésuites, Sorbonne, P.
de La Tour, P. de La Chaise, M. de Paris s'écriaient à l'envi. Ils
ravissaient la mère et ne plaisaient guère moins au roi, à qui on avait
grand soin que rien n'échappât des acclamations sur l'abbé de Soubise,
dont la douceur, la politesse, l'esprit, les grâces, le soin et le
talent de se faire aimer, confirmait de plus en plus une réputation si
établie. Les choses, amenées à ce point, parurent en maturité à
M\textsuperscript{me} de Soubise, et la situation du cardinal de
Bouillon la hâtait. Il s'agissait de pouvoir disposer du cardinal de
Fürstemberg, qui avait deux neveux dans le chapitre de Strasbourg, et de
lui faire vouloir avec chaleur un coadjuteur que les prélats n'admettent
que bien difficilement, et de plus un coadjuteur étranger.

Fürstemberg était un homme de médiocre taille, grosset, mais bien pris,
avec le plus beau visage du monde, et qui à son âge l'était encore\,;
qui parlait fort mal français\,; qui, à le voir et à l'entendre à
l'ordinaire, paraissait un butor, et qui, approfondi et mis sur la
politique et les affaires, à ce que j'ai ouï dire aux ministres et à
bien d'autres de tous pays, passait la mesure ordinaire de la capacité,
de la finesse et de l'industrie. Il a tant fait de bruit en Europe qu'il
est inutile de chercher à le faire connaître\,; il faut se rabattre à
l'état où il s'était réduit\,; en pensions du roi ou en bénéfices, il
jouissait de plus de sept cent mille livres de rente, et il mourait
exactement de faim, sans presque faire aucune dépense ni avoir personne
à entretenir. Il faut entrer dans quelques détails de sa famille. Son
père servit toute sa vie avec réputation et commanda les armées
impériales avec succès après avoir commandé l'aile gauche à la bataille
de Leipzig. Il mourut en 1635 et laissa nombre d'enfants d'Anne, fille
de Jean-Georges, comte de Hohenzollern, que l'empereur Ferdinand II fit
prince de l'empire en 1623. Son fils aîné, mort en 1662, ne laissa
qu'une fille, unique héritière de Berg-op-Zoom par sa mère, et cette
fille de Hohenzollern porta Berg-op-Zoom en mariage au comte d'Auvergne
et était la mère de l'abbé d'Auvergne dont je viens de parler\,; en
sorte que cette comtesse d'Auvergne était fille du frère aîné de la mère
du cardinal de Fürstemberg, qui se trouvait ainsi cousin germain de
cette comtesse d'Auvergne qui venait de mourir, et oncle à la mode de
Bretagne de l'abbé d'Auvergne, compétiteur de l'abbé de Soubise pour
Strasbourg, lequel abbé de Soubise n'avait ni parenté, ni alliance, ni
liaison aucune, par lui ni par aucun de sa famille, avec le cardinal de
Fürstemberg.

Ce cardinal, qui étant évêque de Metz avait succédé à son frère aîné
évêque de Strasbourg, eut un autre frère, que l'empereur fit prince de
l'empire, auquel je reviendrai après, et, entre autres sœurs\,:
Élisabeth, mère du comte Reicham, chanoine de Strasbourg, dans les
ordres, à qui le roi donna des abbayes, et qui était coadjuteur de
l'abbaye de Stavelo du cardinal de Fürstemberg son oncle\,;
Marie-Françoise, mariée à un palatin de Neubourg, puis à un marquis de
Bade, grand'mère de la feue reine de Sardaigne et de
M\textsuperscript{me} la Duchesse\,; et Anne-Marie mariée, en 1651 à
Ferdinand-Charles, comte de Lowenstein, père et mère de
M\textsuperscript{me} de Dangeau. Herman Egon, comte, puis fait prince
de Fürstemberg et de l'empire, pour lui et ses descendants, en 1654, et
ses frères seulement à vie, fut grand maître de la maison de Maximilien,
électeur de Bavière, et son premier ministre, ainsi que de l'électeur de
Cologne, frère de Maximilien. Il mourut en 1674, et laissa entre autres
enfants\,: le prince de Fürstemberg, marié à Paris à la fille de Ligny,
maître des requêtes, dont il n'eut que trois filles, la laissa et s'en
alla en Allemagne, où le roi de Pologne le fit gouverneur général de son
électorat de Saxe, où il est mort en 1711\,; le comte Ferdinand, mort à
Paris, brigadier, en 1696, à trente-cinq ans, sans alliance\,;
Emmanuel-François Egon, tué devant Belgrade, en 1686, à vingt-cinq ans,
sans enfants de Catherine-Charlotte, comtesse de Wallenwoth, veuve de
François-Antoine comte de La Marck, mère du comte de La Marck dont je
parlerai bientôt, et qui longues années depuis s'est distingué par ses
ambassades dans le Nord et en Espagne, et est devenu chevalier de
l'ordre en 1724, et grand d'Espagne en 1739. M\textsuperscript{me} de
Dangeau avait un frère abbé de Murbach, que le cardinal de Fürstemberg,
frère de sa mère, lui avait cédé, qu'on appelait le P. de Murbach, qui
était aussi chanoine de Strasbourg, et qui, après que nous eûmes perdu
Tournai, en a été évêque, tellement que le cardinal de Fürstemberg avait
les fils de ses deux sœurs et le petit-fils du frère de sa mère, qui
était l'abbé d'Auvergne, chanoines de Strasbourg et fort en état d'être
coadjuteurs ou successeurs de l'évêché.

On prétendait que le cardinal de Fürstemberg, fort amoureux de cette
comtesse de La Marck, la fit épouser à son neveu, qui avait alors
vingt-deux ou vingt-trois ans au plus, pour la voir plus commodément à
ce titre. On prétend encore qu'il avait été bien traité\,; et il est
vrai que rien n'était si frappant que la ressemblance, trait pour trait,
du comte de La Marck au cardinal de Fürstemberg, qui, s'il n'était pas
son fils, ne lui était rien du tout. Il était destiné à l'Église, déjà
chanoine de Strasbourg, lorsque la fortune de M\textsuperscript{me} de
Soubise et de son fils lui fit prendre l'épée, par la mort de son frère
aîné en 1697, et se défaire de son canonicat et de ses autres bénéfices.

L'attachement du cardinal pour la comtesse de Fürstemberg avait toujours
duré. Il ne pouvait vivre sans elle\,; elle logeait et régnait chez
lui\,: son fils, le comte de La Marck, y logeait aussi, et cette
domination était si publique que c'était à elle que s'adressaient tous
ceux qui avaient affaire au cardinal. Elle avait été fort belle, et en
avait encore à cinquante-deux ans de grands restes, mais grande et
grosse, hommasse comme un Cent-Suisse habillé en femme, hardie,
audacieuse, parlant haut et toujours avec autorité, polie pourtant et
sachant vivre. Je l'ai souvent vue au souper du roi, et souvent le roi
chercher à lui dire quelque chose. C'était au dedans la femme du monde
la plus impérieuse, qui gourmandait le cardinal qui n'osait souffler
devant elle, qui en était gouverné et mené à baguette, qui n'avait pas
chez lui la disposition de la moindre chose, et qui, avec cette
dépendance, ne pouvait s'en passer. Elle était prodigue en toutes sortes
de dépenses\,; des habits sans fin, plus beaux les uns que les autres\,;
des dentelles parfaites en confusion, et tant de garnitures et de linge
qu'il ne se blanchissait qu'en Hollande\,; un jeu effréné où elle
passait les nuits chez elle et ailleurs, et y faisait souvent le tour du
cadran\,; des parures, des pierreries, des joyaux de toutes sortes.
C'était une femme qui n'aimait qu'elle, qui voulait tout, qui ne se
refusait rien, non pas même, disait-on, des galanteries, que le pauvre
cardinal payait comme tout le reste. Avec cette conduite elle vint à
bout de l'incommoder si bien qu'il fallut congédier la plupart de sa
maison, et aller épargner six à sept mois de l'année à la Bourdaisière,
près de Tours, qu'elle emprunta d'abord de Dangeau et qu'elle acheta
après à vie. Elle vivait dans cette détresse pour avoir de quoi se
divertir à Paris le reste de l'année, lorsque M\textsuperscript{me} de
Soubise songea tout de bon à la coadjutorerie pour son fils.

Elle avait rapproché de loin la comtesse, et je n'ai pas vu que personne
se soit inscrit en faux, ni même récrié contre ce qui se débita d'abord
à l'oreille, et qui fit après grand fracas, qu'elle avait donné beaucoup
d'argent à la comtesse pour s'assurer d'elle, et par elle du cardinal.
Ce qui est certain, c'est que, outre les prodigieuses pensions que le
cardinal tirait du roi, toujours fort bien payées, il toucha en ce
temps-ci une gratification de quarante mille écus, qu'on fit passer pour
promise depuis longtemps. M\textsuperscript{me} de Soubise, s'étant
assurée de la sorte de la comtesse et du cardinal, scella son affaire,
et, les faisant remercier par le roi à l'oreille, et tout de suite, fait
envoyer ordre au cardinal de Bouillon de demander au pape, au nom du
roi, une bulle pour faire assembler le chapitre de Strasbourg pour élire
un coadjuteur avec future succession, et un bref d'éligibilité pour
l'abbé de Soubise.

Cet ordre fut un coup de foudre pour le cardinal de Bouillon, qui ne
s'attendait à rien moins. Il ne put soutenir de se voir échapper cette
magnifique proie, qu'il croyait déjà tenir par tant d'endroits. Il lui
fut encore plus insupportable d'en être le ministre. Le dépit le
transporte et l'aveugle assez pour s'imaginer, qu'en la situation si
différente où M\textsuperscript{me} de Soubise et lui sont auprès du
roi, il lui fera changer une résolution arrêtée, et rompre l'engagement
qu'il a pris. Il dépêche au roi un courrier, lui mande qu'il n'y a pas
bien pensé, lui met en avant des scrupules, comme s'il eût été un grand
homme de bien, et par ce même courrier écrit aux chanoines de Strasbourg
une lettre circulaire pleine de fiel, d'esprit et de compliments. Il
leur mandait que le cardinal de Fürstemberg était aussi en état de
résider que jamais (c'était à dire qu'il n'y avait jamais résidé et
qu'on s'en passerait bien encore), que l'abbé de Soubise était si jeune
qu'il y avait de la témérité à s'y fier, et qu'un homme qu'on mettait en
état sitôt de n'avoir plus à craindre ni à espérer, se gâtait bien vite,
et il leur faisait entendre, comme il l'avait fait au roi, que le
cardinal de Fürstemberg, gouverné comme il l'était par sa nièce, n'était
gagné au préjudice de ses neveux que par le gros argent qu'elle avait
touché de M\textsuperscript{me} de Soubise. Il est vrai qu'il envoya ces
lettres à son frère le comte d'Auvergne, pour ne les faire rendre
qu'avec la permission du roi. Ce n'était pas qu'il pût l'espérer, mais
pour le leurrer de cet hommage, et cependant en faire glisser assez pour
que l'effet n'en fût pas perdu, et protester après qu'il ne savait pas
comment elles étaient échappées. Ces lettres firent un fracas
épouvantable.

J'étais chez le roi le mardi 30 mars, lorsqu'à la fin du souper je vis
arriver M\textsuperscript{me} de Soubise menant la comtesse de
Fürstemberg, et se poster toutes deux à la porte du cabinet du roi. Ce
n'était pas qu'elle n'eût bien le crédit d'entrer dedans si elle eût
voulu, et d'y faire entrer la comtesse, mais comme l'éclat était public
et qu'on ne parlait d'autre chose que du marché pécuniaire et des
lettres du cardinal de Bouillon, elles voulurent aussi un éclat de leur
part. Je m'en doutai dès que je les vis, ainsi que bien d'autres, et je
m'approchai aussitôt pour entendre la scène. M\textsuperscript{me} de
Soubise avait l'air tout bouffi, et la comtesse, de son naturel
emportée, paraissait furieuse. Comme le roi passa elles l'arrêtèrent\,;
M\textsuperscript{me} de Soubise dit deux mots d'un ton assez bas, puis
la comtesse, haussant le sien, demanda justice de l'audace du cardinal
de Bouillon, dont l'orgueil et l'ambition, non contente de résister à
ses ordres, la déshonorait elle et le cardinal son frère, qui avait si
utilement servi le roi, par les calomnies les plus atroces, et qui
n'épargnait pas M\textsuperscript{me} de Soubise elle-même. Le roi
l'écouta et lui répondit avec autant de grâces et de politesse pour
elle, que d'aigreur qu'il ne ménagea pas sur le cardinal de Bouillon,
l'assura qu'elle serait contente, et passa.

Ces dames s'en allèrent, mais ce ne fut pas sans montrer une colère
ardente, et qui est en espérance de se venger. M\textsuperscript{me} de
Soubise était d'autant plus piquée, que le cardinal de Bouillon
apprenait au roi un manège et des simonies que sûrement il ignorait, et
qui l'auraient empêché de consentir à cette affaire, s'il s'en fût
douté, bien loin de la protéger. Elle craignait donc des retours de
scrupules, et qu'ils ne se portassent à éclairer de trop près les
marchés qu'elle avait mis en mouvement à Strasbourg pour l'élection. Les
mêmes Camilly et Labatie, qui l'avaient si lestement servie pour faire
passer son fils chanoine avec cet orde\footnote{Vieux mot, synonyme de
  \emph{sale} et \emph{ignoble}. Il ne faut pas oublier que ce mot
  s'applique à La Varenne, que Saint-Simon a traité plus haut de
  \emph{marmiton}.} quartier\footnote{On appelait \emph{quartiers}, en
  terme de blason, une partie de l'écu qui contenait les armes d'une des
  branches de la famille, soit du côté paternel, soit du côté maternel.
  Il fallait plusieurs quartiers de noblesse pour entrer dans certains
  chapitres. Saint-Simon désigne par cet orde quartier, \emph{la fille
  de La Varenne bisaïeule de l'abbé de Soubise}.} de La Varenne, furent
encore ceux qu'elle employa pour emporter la coadjutorerie. Ni l'un ni
l'autre n'étaient scrupuleux. Camilly avait déjà eu une bonne abbaye du
premier service, il espérait bien un évêché du second. Il n'y fut pas
trompé, et Labatie de placer un nombre d'enfants utilement et
honorablement, comme il arriva.

Pendant qu'ils préparaient les matières à Strasbourg, le cardinal de
Bouillon se conduisait à Rome par sauts et par bonds, mit tous les
obstacles qu'il put aux bulles que le roi demandait, et lui écrivit une
deuxième lettre là-dessus plus folle encore que la première. Elle mit le
comble à la mesure. Pour réponse, il reçut ordre par un courrier de
partir de Rome sur-le-champ, et de se rendre droit à Cluni ou à Tournus,
à son choix, jusqu'à nouvel ordre. Le commandement de revenir parut si
cruel au cardinal de Bouillon, qu'il ne put se résoudre à obéir. Il
était sous-doyen du sacré collège\,; Cibo, doyen, décrépit, ne sortait
plus de son lit. Pour être doyen, il faut être à Rome lorsque le décanat
vaque, et opter soi-même les évêchés unis d'Ostie et de Velletri au
consistoire affectés au doyen, ou comme quelques-uns ont fait opter le
décanat en retenant l'évêché qu'ils avaient déjà. Le cardinal de
Bouillon manda donc au roi, parmi force soumissions à ses ordres, l'état
exagéré du cardinal Cibo\,; qu'il ne pouvait croire qu'il le voulût
priver du décanat, ni ses sujets de l'honneur et de l'avantage d'un
doyen français\,; que dans cette persuasion il allait demander au pape
un bref pour lui assurer le décanat en son absence\,; qu'il partirait
dans l'instant qu'il l'aurait obtenu, et qu'en attendant il allait faire
prendre les devants à tous ses gens, et se renfermer comme le plus petit
particulier dans le noviciat des jésuites, sans aucun commerce avec
personne que pour son bref. Il se conduisit en effet de la sorte, et
demanda ce bref qu'il se doutait bien qu'il n'obtiendrait pas, mais dont
il espérait faire filer assez longtemps l'espérance et les prétendues
longueurs pour atteindre à la mort le cardinal Cibo, ou à celle du pape
même, qui menaçait ruine depuis longtemps. Laissons-le pour un temps
dans ses ruses, qui lui devinrent funestes, pour ne pas trop interrompre
la suite des événements.

M\textsuperscript{me} de Soubise fut si bien servie à Strasbourg, et
l'autorité du roi appuya si bien à l'oreille l'argent qui fut répandu,
que l'abbé de Soubise fut élu tout d'une voix coadjuteur de Strasbourg.
Le rare fut que ce fut en présence de l'abbé d'Auvergne, qui comme grand
prévôt du chapitre, dit la messe du Saint-Esprit avant l'élection. La
colère du roi fit peur aux Bouillon\,; leur rang et leur échange, encore
informe et non enregistré au parlement, ne tenaient qu'à un bouton\,;
ils virent de près l'affaire sans ressource, et ils tâchèrent à se
sauver de la ruine de leur frère par cette bassesse.

En même temps, et je ne sais si ce fut une des conditions du marché,
M\textsuperscript{me} de Soubise, toujours mal avec le duc de Rohan son
frère, s'était raccommodée avec lui, et en avait fait tous les pas pour
faire le mariage de sa fille aînée avec le comte de La Marck, fils de la
comtesse de Fürstemberg, qui n'avait quoi que ce fût en France où il
s'était mis dans le service, colonel d'un des régiments que le roi
entretenait fort chèrement au cardinal de Fürstemberg, desquels il lui
laissait la disposition, et dont tout le médiocre bien était en
Westphalie sous la main de l'empereur. Ces Allemands ne se mésallient
pas impunément\,; celui-ci sentit ce qu'il en coûte par une triste
expérience\,; il ne la voulait pas aggraver. Sa mère le voulait marier,
et un étranger qui n'a rien en France, et peu sous une coupe étrangère
et souvent ennemie, n'était pas un parti aisé à établir. Le duc de Rohan
ne comptait ses filles pour rien et ses cadets pour peu de chose\,; en
donnant aussi peu qu'il voulut, il fut aisé à persuader, et le mariage
fut bâclé de la sorte.

Voilà l'état du comte de La Marck. Il était de la maison des comtes de
La Marck, dont une branche a longtemps possédé Clèves et Juliers par le
mariage de l'héritière, et le cadet de cette branche a figuré ici avec
le duché de Nevers, le comté d'Eu, etc., qui par deux filles héritières
passèrent\,: Nevers à un Gonzague, frère du duc de Mantoue, Eu au duc de
Guise. Une autre branche eut Bouillon, Sedan, etc., dont deux maréchaux
de France, d'autres capitaines des Cent-Suisses, un premier écuyer de la
reine, chevalier de l'ordre, parmi les gentilshommes\,; et l'héritière
de Sedan, par laquelle Henri IV fit la fortune du vicomte de Turenne, si
connu depuis sous le nom de maréchal de Bouillon, qui n'en eut point
d'enfants, et en garda les biens par la même protection d'Henri IV qui
s'en repentit bien après. La dernière branche, et la seule qui subsiste,
fut celle de Lumain. Le grand-père du comte de La Marck, dont il s'agit
ici, étant veuf d'une Hohenzollern avec un fils qui lui survécut mais
qui n'eut point d'enfants, s'était remarié fort bassement, et de ce
deuxième mariage vint le père du comte de La Marck, à qui il en coûta
bon pour se faire réhabiliter à la succession de son frère du premier
lit, et à la dignité de comte. Cette branche de Lumain, dont le chef se
rendit célèbre sous le nom de Sanglier d'Ardennes que sa férocité lui
valut, et qui tua Louis de Bourbon, évêque de Liège, et jeta son corps
du haut du pont dans la Meuse, était déjà séparée lorsque Clèves et
Juliers entrèrent dans une branche leur aînée, et plus encore de celle
qui a eu Bouillon et Sedan. Ils n'étaient que barons de Lumain, lorsque
le grand-père de notre comte de La Marck prit, avant sa mésalliance, le
nom et le rang dans l'empire de comte de La Marck, à la mort
d'Henri-Robert de La Marck, comte de Maulevrier, chevalier de l'ordre,
et premier écuyer de la reine, qui mourut le dernier de sa branche,
toutes les autres étant éteintes depuis longtemps\,; tellement que lors
de ce mariage du comte de La Marck avec la fille du duc de Rohan, il n'y
avait plus que lui et son frère cadet de la maison de La Marck.

Le cardinal de Fürstemberg fit un autre mariage presque en même temps
d'une des trois filles, que son neveu, le gouverneur général de
l'électorat de Saxe, avait, avec le prince d'Isenghien, et qu'il avait
laissées à Paris avec sa femme.

Le duc de Berwick, qui depuis la mort de sa femme, avait été se promener
ou se confesser à Rome, devint amoureux de la fille de
M\textsuperscript{me} Bockley, une des principales dames de la reine
d'Angleterre à Saint-Germain. Il n'avait qu'un fils de la première.

\hypertarget{chapitre-xxiv.}{%
\chapter{CHAPITRE XXIV.}\label{chapitre-xxiv.}}

1700

~

{\textsc{Traité de partage de la monarchie d'Espagne.}} {\textsc{-
Harcourt revient d'Espagne et y laisse Blécourt.}} {\textsc{- Recherche
et gain des gens d'affaires.}} {\textsc{- Desmarets\,; ma liaison avec
lui.}} {\textsc{- Loteries.}} {\textsc{- Mort de Châteauneuf\,; ses
charges de secrétaire d'État et de greffier de l'ordre données à son
fils en épousant M\textsuperscript{lle} de Mailly, et le râpé de l'ordre
au chancelier.}} {\textsc{- Cauvisson lieutenant général de Languedoc
par M. du Maine.}} {\textsc{- Noailles, archevêque de Paris, fait
cardinal.}} {\textsc{- Abbé de Vaubrun exilé.}} {\textsc{- Ruses et
opiniâtre désobéissance du cardinal de Bouillon, qui devient doyen et
que le roi dépouille.}} {\textsc{- Argent à Mgr le duc de Bourgogne.}}
{\textsc{- Cent mille livres à Mansart.}} {\textsc{- Détails de
l'assemblée du clergé.}} {\textsc{- Jésuites condamnés par la Sorbonne
sur la Chine.}} {\textsc{- P. de La Rue, confesseur de
M\textsuperscript{me} la duchesse de Bourgogne, au lieu du P. Le Comte,
renvoyé.}} {\textsc{- Rage du P. Tellier.}} {\textsc{- Jésuites
affranchis pour toujours des impositions du clergé.}} {\textsc{-
Pelletier va visiter les places et ports de l'Océan.}} {\textsc{- M. de
Vendôme retourne publiquement suer la vérole.}} {\textsc{- Mort de la
duchesse d'Uzès.}} {\textsc{- Mariage du duc d'Albemarle avec
M\textsuperscript{lle} de Lussan.}} {\textsc{- M\textsuperscript{me}
Chamillart, pour la première femme de contrôleur général, admise dans
les carrosses et à manger avec M\textsuperscript{me} la duchesse de
Bourgogne.}} {\textsc{- L'évêque de Chartres gagne son procès contre son
chapitre de la voix du roi unique.}} {\textsc{- M. de Reims cède la
présidence de l'assemblée générale du clergé à M. de Noailles.}}
{\textsc{- Comte d'Albert cassé.}} {\textsc{- Étrange embarras de M. le
prince de Conti avec M. de Luxembourg.}} {\textsc{-
M\textsuperscript{me} de Villacerf admise dans les carrosses et à manger
avec M\textsuperscript{me} la duchesse de Bourgogne.}} {\textsc{- Dons
pécuniaires à M. le prince de Conti, à M. de Duras et à Sainte-Maure.}}
{\textsc{- Fiançailles de La Vrillière et de M\textsuperscript{lle} de
Mailly, et leur mariage.}} {\textsc{- P. Martineau confesseur de M. le
duc de Bourgogne à la place du feu}}

~

P. Valois. --- Mort de Le Nôtre. --- Mort de Labriffe, procureur
général. --- D'Aguesseau, avocat général, fait procureur général en sa
place.

Le traité de partage de la monarchie d'Espagne commençait à faire grand
bruit en Europe. Le roi d'Espagne n'avait point d'enfants ni aucune
espérance d'en avoir. Sa santé, qui avait toujours été très faible,
était devenue très mauvaise depuis deux ou trois ans, et il avait été à
l'extrémité depuis un an à plusieurs reprises. Le roi Guillaume qui,
depuis les succès de son usurpation, avait fort augmenté son crédit par
la confiance de tous les alliés de la grande alliance qu'il avait ourdie
contre la France, et dont il avait été l'âme et le chef jusqu'à la paix
de Ryswick, et qui se l'était depuis conservée sur le même pied,
entreprit de pourvoir de façon à cette vaste succession que, lorsqu'elle
s'ouvrirait, elle ne causât point de guerre. Il n'aimait ni la France ni
le roi, et dans la vérité il était payé pour les bien haïr\,; il en
craignait l'agrandissement\,; il venait d'éprouver, par l'union de toute
l'Europe contre elle dans une guerre de dix ans, quelle puissance
c'était, après toutes celles dont ce règne n'avait été qu'un tissu plein
de conquêtes. Malgré les renonciations de la reine, il n'osa espérer que
le roi vit passer toute cette immense succession sans en tirer rien\,;
il avait vu, par les conquêtes de la Franche-Comté et d'une partie de la
Flandre, le peu de frein de ces renonciations. Il songea donc à un
partage que l'appât de le recueillir en paix, et sous la garantie des
puissances principales, prit faire accepter au roi, et qui fût tel en
même temps qu'il n'augmentât pas sa puissance, ne fût qu'un
arrondissement léger vers des frontières bien assurées, et que ce qu'il
aurait de plus fût si éloigné, que la difficulté de le conserver le tint
toujours en brassière et ses successeurs après lui. En même temps, il
voulut bien assurer les bords de la mer du côté de l'Angleterre, et
mettre ses chers Hollandais à l'abri de la France, et partager
l'empereur si grandement, qu'il eût lieu de s'en contenter et de ne pas
regretter une totalité qu'il n'avait pas la puissance d'espérer contre
la France. Il ne destinait donc à celle-ci, pour ainsi parler, que des
rognures. Ce fut pour cela qu'il s'en voulut assurer d'abord, comme la
prévoyant la plus difficile à se contenter de ce qu'il lui voulait
offrir, et que sûr, s'il le pouvait, de son acceptation, il n'eût à
présenter à l'empereur que la plus riche et la plus grande partie avec
un nom qui pouvait passer pour le tout, et que la tentation d'une si
ample monarchie, sans coup férir, le consolât du reste et la lui fit
promptement accepter.

Son plan arrêté fut donc de donner à l'archiduc, deuxième fils de
l'empereur, l'Espagne et les Indes avec les Pays-Bas et le titre de roi
d'Espagne\,; le Guipuscoa à la France, parce que l'aridité et la
difficulté de cette frontière est telle qu'elle était demeurée en paix
de tout ce règne, au milieu de toutes les guerres contre l'Espagne\,;
Naples et Sicile, dont l'éloignement et le peu de revenu était plutôt un
embarras et un sauve-l'honneur qu'un accroissement, et dont la
conservation tiendrait à l'avenir la France en bride avec les puissances
maritimes\,; la Lorraine, qui était un arrondissement très sensible,
mais qui ne portait pas la France au delà d'où elle était, et qui en
temps de guerre ne la soulageait que d'une occupation qui ne lui coûtait
rien à faire\,; et pour dédommagement, le Milanais à M. de Lorraine, qui
y gagnait les trois quarts de revenu et d'étendue, et, d'esclave de la
France par l'enclavement de la Lorraine, de devenir un prince puissant
et libre en Italie, et qui ferait compter avec lui.

Le roi d'Angleterre fit donc d'abord cette proposition au roi, qui, las
de la guerre, et dans un âge et une situation qui lui faisaient goûter
le repos, disputa peu et accepta. M. de Lorraine n'était ni en intérêt
ni en état de ne pas consentir au changement de pays que l'Angleterre
avec la Hollande lui proposèrent d'une part, et le roi de l'autre qui
lui envoya Callières. Cela fait, il fut question de l'empereur. Ce fut
où tout le crédit et l'adresse du roi d'Angleterre échoua\,: l'empereur
voulait la succession entière\,; il se tenait ferme sur les
renonciations du mariage du roi\,; il ne pouvait souffrir de voir la
maison d'Autriche chassée d'Italie (et elle l'était entièrement par le
projet du roi d'Angleterre qui donnait à la France les places maritimes
de Toscane que l'Espagne tenait, connues sous le nom de \emph{gli.
Presidii}). Pressé par Villars envoyé du roi, par l'Angleterre, par la
Hollande, qui avaient signé le traité, et qui lui faisaient entendre
qu'ils se joindraient contre lui s'il s'opiniâtrait dans le refus d'un
si beau partage, il se tint ferme à répondre qu'il était inouï, et
contre tout droit naturel et des gens, de partager une succession avant
qu'elle fût ouverte\,; et qu'il n'entendrait jamais à rien là-dessus
pendant la vie du roi d'Espagne, chef de sa maison, et qui lui était si
proche. Cette résistance, et plus encore l'esprit de cette résistance,
divulgua bientôt le secret qui devait durer jusqu'à la mort du roi
d'Espagne, qui fut averti par l'empereur et pressé de faire un testament
en faveur de l'archiduc et de sa propre maison.

Le roi d'Espagne jeta les hauts cris comme si on l'eût voulu dépouiller
de son vivant, et son ambassadeur en fit un tel bruit en Angleterre, et
en des termes si peu respectueux, jusqu'à nommer le roi d'Angleterre le
roi Guillaume, que ce prince lui fit dire de sortir en quatre jours
d'Angleterre, ce qu'il exécuta et se retira en Flandre. Mais l'empereur,
quoique mécontent du roi d'Angleterre, le voulait ménager dans ce qu'il
n'était pas le point principal, pour ne se brouiller pas absolument avec
lui. Il s'offrit entre lui et le roi d'Espagne, et fit en sorte que ce
mécontentement accessoire se raccommoda et que l'ambassadeur d'Espagne
retourna à Londres.

Harcourt eut à essuyer à Madrid toutes les plaintes et les clameurs\,;
elles furent au point que, sur le compte qu'il rendit de tous les
désagréments qu'il essuyait et de l'inutilité où il se voyait par cette
découverte, il eut permission de revenir. Il laissa Blécourt, son
parent, qu'il avait mené avec lui, et qui était homme ferme et capable
d'affaires, quoiqu'il n'eût fait toute sa vie d'autre métier que celui
de la guerre, et qui, en l'absence d'ambassadeur, servit très bien et
très dignement avec le caractère d'envoyé du roi.

L'empereur cependant ne pensait qu'à fortifier son parti en Espagne. La
reine sa belle-sœur y était toute-puissante\,; elle avait fait chasser
les plus grands seigneurs et les principaux ministres qui ne ployaient
pas sous elle. Par sa faveur la Berlips était l'objet de l'envie
universelle\,; elle prenait à toutes mains et vendait les plus grands
emplois. Un de ses enfants avait été fait par le roi d'Espagne
archimandrite de Messine, qui est un bénéfice de quatre-vingt-dix mille
livres de rente, par la mort d'un frère du duc de Lorraine\,; et le
prince de Hesse-Darmstadt, vice-roi de Catalogne et colonel des
Allemands dont elle avait rempli Madrid. Quoiqu'elle eût réduit Harcourt
à la plus honteuse solitude après avoir éprouvé tout le contraire, elle
ne laissa pas de lui détacher l'amirante, avec des propositions fortes
pour elle, et des espérances pour un des fils de Monseigneur. Harcourt,
qui voulait voir plus clair, et qui, avec raison, se défiait de la sœur
de l'impératrice, battit froid, et, disant toujours qu'il ne pouvait
écrire en France tant qu'il ne verrait que du vague, ne laissait pas de
le faire et de se flatter de la plus grande fortune s'il pouvait
réussir. Mais en France on était content du traité de partage\,; il
était signé\,; la sœur de l'impératrice y était trop suspecte, et
l'amirante à Harcourt pour le moins autant\,; par la mauvaise réputation
de sa foi, et par son attachement héréditaire à la maison d'Autriche, et
très particulier à la reine\,; Harcourt eut donc ordre de ne plus rien
écouter, qui en fut au désespoir, et qui de dépit s'éloigna de Madrid,
et ne songea plus qu'à s'amuser avec son domestique et à tirer des
lapins, en attendant son retour, dont bientôt après il reçut la
permission

Cette position si jalouse fit mettre toutes choses en œuvre pour
recouvrer de l'argent, et se tenir en bonne posture et prêt à tout
événement. On commença par une recherche sourde des gens d'affaires,
dont les profits avaient été immenses pendant la dernière guerre.
Chamillart obtint à grand'peine permission du roi de se servir de
Desmarets pour cette opération. Il figura assez dans la suite pour qu'il
ne soit pas inutile de le faire connaître dès à présent. C'était un
grand homme, très bien fait, d'un visage et d'une physionomie agréable
qui annonçait la sagesse et la douceur, qui étaient les deux choses du
monde qu'elle tenait le moins. Son père était trésorier de France à
Soissons, qui était riche dans son état, fils d'un manant, gros
laboureur d'auprès de Noyon, qui s'était enrichi dans la ferme de
l'abbaye d'Orcamp qu'il avait tenue bien des années, après avoir labouré
dans son jeune temps. Son fils, le trésorier de France, avait épousé une
sœur de M. Colbert, longtemps avant la fortune de ce ministre, qui
depuis prit Desmarets son neveu dans ses bureaux, et le lit après
intendant des finances. C'était un homme d'un esprit net, lent et
paresseux, mais que l'ambition et l'amour du gain aiguillonnaient, en
sorte que M. de Seignelay, son cousin germain, l'avait pris en aversion,
parce que M. Colbert le lui donnait toujours pour exemple. Il lui fit
épouser la fille de Bechameil, secrétaire du conseil, qui devint après
surintendant des finances et affaires de Monsieur quand il chassa
Boisfranc, beau-père du marquis de Gesvres. Desmarets, élevé et conduit
par son oncle, en avait appris toutes les maximes et tout l'art du
gouvernement des finances. Il en avait pénétré parfaitement toutes les
différentes parties, et comme tout lui passait par les mains, personne
n'était instruit plus à fond que lui des manèges des financiers, du gain
qu'ils avaient fait de son temps, et par ces connaissances de celui
qu'ils pouvaient avoir fait depuis.

Tout à la fin de la vie de M. Colbert, on s'avisa de faire à la Monnaie
une quantité de petites pièces d'argent de la valeur de trois sous et
demi pour la facilité du commerce journalier entre petites gens.
Desmarets avait acquis plusieurs terres, entre autres Maillebois, et
l'engagement du domaine de Châteauneuf en Timerais\footnote{Le Timerais
  ou Thimerais faisait autrefois partie du Perche. Il est compris
  maintenant dans le département d'Eure-et-Loir.} dont cette terre
relevait et quantité d'autres sortes de biens. Il avait fort embelli le
château bâti par d'O, surintendant des finances d'Henri III et d'Henri
IV. Il en avait transporté le village d'un endroit à un autre pour orner
et accroître son parc, qu'il avait rendu magnifique. Ces dépenses si
fort au-dessus de son patrimoine, de la dot de sa femme et du revenu de
sa place, donnèrent fort à parler. Il fut accusé ensuite d'avoir
énormément pris sur la fabrique de ces pièces de trois sous et demi. Le
bruit en parvint à la fin à M. Colbert, qui voulut examiner, et qui
tomba malade de la maladie prompte dont il mourut. Preuves, doutes ou
humeur, je n'assurerai lequel des trois, mais ce qui est vrai, c'est que
de son lit il écrivit au roi contre son neveu, qu'il pria d'ôter des
finances, et à qui il donna les plus violents soupçons contre lui.
Colbert mort, et Pelletier, contrôleur général de la façon de M. de
Louvois, à qui et à M. Le Tellier il était intimement attaché de toute
sa vie, le roi lui donna ordre de chasser Desmarets, et de lui faire une
honte publique. C'était bouillir du lait à une créature de Louvois. Il
manda Desmarets, et prit son moment à une audience publique. Là, au
milieu de tous les financiers qui rampaient et tremblaient huit jours
auparavant devant lui, et de tout ce qui se présenta là pour parler au
contrôleur général, il appela Desmarets, et tout haut pour que tout ce
qui était là n'en perdit pas une parole\,: «\,Monsieur Desmarets, lui
dit-il, je suis fâché de la commission dont je suis chargé pour vous. Le
roi m'a commandé de vous dire que vous êtes un fripon\,; que M. Colbert
l'en a averti\,; qu'en cette considération, il veut bien vous faire
grâce, mais qu'entre ci et vingt-quatre heures vous vous retiriez dans
votre maison de Maillebois sans en sortir ni en découcher, et que vous
vous défassiez de votre intendance des finances, dont le roi a
disposé.\,» Desmarets, éperdu, voulut pourtant ouvrir la bouche, mais
Pelletier tout de suite la lui ferma par un\,: «\,Allez-vous-en,
monsieur Desmarets, je n'ai pas autre chose à vous dire,\,» et lui
tourna le dos. La lettre de M. Colbert mourant au roi ferma la bouche à
toute sa famille, tellement que Desmarets, dénué de toute sorte de
protections, n'eut qu'à signer la démission de sa place et s'en aller à
Maillebois.

Il y fut les quatre ou cinq premières années sans avoir la liberté d'en
découcher, et il y essuya les mépris du voisinage, et les mauvais
procédés d'une menue noblesse qui se venge avec plaisir, sur
l'impuissance, de l'autorité dure qu'elle avait exercée dans le temps de
sa fortune. Mon père était ami de M. Colbert, de M. de Seignelay et de
toute leur famille\,; il connaissait peu Desmarets, jeune homme à son
égard. La Ferté, où mon père passait souvent la fin des automnes, se
trouvait à quatre lieues de Maillebois. La situation de Desmarets lui
fit pitié. Coupable ou non, car rien n'avait été mis au net, il trouva
que sa chute était bien assez profonde, sans se trouver encore mangé des
mouches dans le lieu de son exil\,; il l'alla voir, lui fit amitié, et
déclara qu'il ne verrait pas volontiers chez lui ceux qui chercheraient
à lui faire de la peine\,: un reste de seigneurie palpitait encore en ce
temps-là. Mon père, toute sa vie honnête et bienfaisant, était fort
respecté dans le pays. Cette déclaration changea en un moment la
situation de Desmarets dans la province\,: il lui dut tout son repos et
la considération qui succéda au mépris et à la mauvaise volonté qu'il
avait éprouvée. Mon père même alla trop loin dans les suites, car il
s'engagea dans des procès de mouvance à la prière de Desmarets, qui lui
coûtèrent à soutenir et qu'il perdit. Dès que Desmarets eut permission
de sortir de sa maison sans découcher, il vint dîner à la Ferté, sitôt
que mon père y fut. Il n'oublia rien, ni M\textsuperscript{me}
Desmarets, pour témoigner à mon père et à ma mère leur attachement et
leur reconnaissance. Il eut enfin permission de faire à Paris des tours
courts, puis allongés et réitérés, enfin liberté d'y demeurer en
n'approchant pas de la cour. Il continua la même amitié avec moi, et moi
avec lui, après la mort de mon père, et elle fut telle qu'on en verra
bientôt une marque singulière. Desmarets était en cet état lorsque
Chamillart obtint à grand'peine la permission de se servir de ses
lumières, et de le faire travailler à la recherche des gens d'affaires,
etc., qui, par compte fait et arrêté avec eux, se trouvèrent avoir gagné
depuis 1689 quatre-vingt-deux millions. On s'abstient de réflexions sur
un si immense profit en moins de dix ans, et sur la misère qu'il
entraîne nécessairement, sur qui a tant gagné et qui a tant perdu, sans
parler d'une autre immensité d'une autre sorte de gain et de perte, qui
sont les frais non compris dans ces quatre-vingt-deux millions.

Il fut proposé d'attirer la cupidité publique par des loteries\,; il
s'en fit de plusieurs façons en quantité. Pour leur donner plus de
crédit et de vogue, M\textsuperscript{me} la duchesse de Bourgogne en
fit une de vingt mille pistoles\,; elle et ses dames et plusieurs autres
de la cour firent les billets. Hommes et femmes, depuis Monseigneur
jusqu'à M. le comte de Toulouse, les cachetèrent, et les diverses façons
qu'on leur donna firent l'amusement du roi et de toutes ces personnes.
On y garda toutes les mesures les plus soupçonneuses pour y conserver
une parfaite fidélité. Elle fut tirée avec les mêmes précautions devant
toutes les personnes royales et autres distinguées qui y furent admises.
Le gros lot tomba à un garde du roi de la compagnie de Lorges\,; il
était de quatre mille louis.

Châteauneuf, secrétaire d'État, fort affligé du refus de sa survivance,
et fort tombé de santé, s'en alla prendre les eaux de Bourbon, et pria
le roi de trouver bon que Barbezieux signât pour lui en son absence. Il
était naturel que ce fût Pontchartrain, mais ces deux branches ne
s'étaient jamais aimées, comme on l'a pu voir plus haut, et j'ai ouï
plus d'une fois le chancelier reprocher à La Vrillière le vol de la
charge de son père par son bisaïeul, et fort médiocrement en
plaisanterie. Châteauneuf était un homme d'une prodigieuse grosseur
ainsi que sa femme, fort peu de chose, bon homme et servant bien ses
amis. Il avait le talent de rapporter les affaires au conseil de
dépêches mieux qu'aucun magistrat, du reste la cinquième roue d'un
chariot, parce qu'il n'avait aucun autre département que ses provinces,
depuis qu'il n'y avait plus de huguenots. Sa considération était donc
fort légère, et sa femme, la meilleure femme du monde, n'était pas pour
lui en donner. Peu de gens avaient affaire à lui, et l'herbe croissait
chez eux. En passant chez lui à Châteauneuf, en revenant de Bourbon,
dont il avait fait un des plus beaux lieux de France, il y mourut
presque subitement.

Il en vint un courrier à son fils pour le lui apprendre, qui arriva à
cinq heures du matin. Il ne perdit point le jugement\,; il envoya
éveiller la princesse d'Harcourt et la prier instamment de venir chez
lui sur l'heure. La surprise où elle en fut à heure si indue l'y fit
courir. La Vrillière lui conta son malheur, lui ouvrit sa bourse à une
condition c'est qu'elle irait sur-le-champ au lever de
M\textsuperscript{me} de Maintenon lui proposer son mariage pour rien
avec M\textsuperscript{lle} de Mailly, moyennant la charge de son père,
et d'écrire au roi avant d'aller à Saint-Cyr, pour lui faire rendre sa
lettre au moment de son réveil. La princesse d'Harcourt, dont le métier
était de faire des affaires depuis un écu jusqu'aux plus grosses sommes,
se chargea volontiers de celle-là. Elle la fit sur-le-champ et le vint
dire à La Vrillière\,; il la renvoya à la comtesse de Mailly, qui, sans
biens et chargée d'une troupe d'enfants, garçons et filles, y avait déjà
consenti quand Châteauneuf tenta vainement la survivance. En même temps
La Vrillière s'en va chez le chancelier, l'avertit de ce qu'il venait de
faire avec la princesse d'Harcourt, et l'envoie chez le roi pour lui
demander la charge en cadence de M\textsuperscript{me} de Maintenon. Le
chancelier fit demander à parler au roi avant que personne fût entré. Le
roi venait de lire la lettre de M\textsuperscript{me} de Maintenon, et
accorda sur-le-champ la charge, à condition du mariage, et l'un et
l'autre fût déclaré au lever du roi.

La Vrillière était extrêmement petit, assez bien pris dans sa petite
taille. Son père, pour le former, l'avait toujours fait travailler sous
lui, et il en était venu à y tout faire. Tous ces La Vrillière, depuis
le bonhomme La Vrillière, grand-père de celui-ci, avaient toujours été
extrêmement des amis de mon père. Blaye par la Guyenne était de leur
département. Cette amitié s'était continuée avec moi. Je tirai d'eux
plusieurs services importants pour mon gouvernement\,; je fus ravi que
la charge fût demeurée à La Vrillière. Il eût été bien à plaindre sans
cela\,: d'épée ni de robe, il n'avait pris aucun de ces deux chemins\,:
à la cour, sans charge quelle figure y eût-il pu faire\,? C'était un
homme sans état et sans consistance. Sa future ne fut pas si aise que
lui\,: elle n'avait pas douze ans. Elle se mit à pleurer et à crier
qu'elle était bien malheureuse\,; qu'on lui donnât un homme pauvre, si
l'on voulait, pourvu qu'il fût gentilhomme, et non pas un petit
bourgeois pour faire sa fortune\,; elle était en furie contre sa mère et
contre M\textsuperscript{me} de Maintenon. On ne pouvait l'apaiser, ni
la faire taire, ni faire qu'elle ne fît pas la grimace à La Vrillière et
à toute sa famille, qui accoururent la voir, et sa mère. Ils le
sentirent tous bien, mais le marché était fait, et trop bon pour eux
pour le rompre. Ils espérèrent que c'était enfance qui passerait, mais
ils l'espérèrent vainement\,; jamais elle ne s'est accoutumée à être
M\textsuperscript{me} de La Vrillière, et souvent elle le leur a montré.

Le roi fit en ce même temps un autre beau présent\,: Cauvisson mourut en
Languedoc, dont il était un des trois lieutenants généraux\,; il n'avait
qu'une fille unique qu'il avait mariée à son frère. Son fils avait été
tué peu après son mariage avec la sœur de Biron, dont il n'avait point
eu d'enfants, et M\textsuperscript{me} de Nogaret, sa veuve, était dame
du palais de M\textsuperscript{me} la duchesse de Bourgogne, et
intimement amie de M\textsuperscript{me} de Saint-Simon et de moi.
Cauvisson, frère et gendre, demandait la charge. C'était une fort
vilaine figure d'homme, mais avec beaucoup d'esprit, de lecture et de
monde, aimé et mêlé avec tout le meilleur et le plus brillant de la cour
dès qu'il y revenait, car il était souvent en Languedoc, où son frère
passait sa vie. Il avait été capitaine aux gardes et avait quitté.
C'était le grief. M. du Maine, gouverneur de la province, demanda la
charge pour lui. Cela dura quelques jours. Le roi, qui voulut suivre sa
maxime de refuser tout à ceux qui avaient quitté le service, et qui ne
manquait aucune occasion d'élever M. du Maine, et de relever son crédit,
remplit ces deux vues. Il donna la charge à M. du Maine, pour en
disposer en faveur de qui il voudrait. Il la donna à Cauvisson, qui de
la sorte la tint de lui et point du roi.

Une autre grâce plus importante fut la nomination au cardinalat que le
roi donna à l'archevêque de Paris, qui n'en avait fait aucune démarche.
Mais son frère et M\textsuperscript{me} de Maintenon firent tout pour
lui. On ne le sut que par les lettres de Rome. Il n'attendit pas deux
mois la pourpre depuis sa nomination. Le pape avait résolu de faire la
promotion des couronnes dès qu'il y aurait trois chapeaux vacants. Le
cardinal Maldachini mourut le troisième, et aussitôt, c'est-à-dire le 28
juin, il arriva un courrier de M. de Monaco, qui apporta la nouvelle que
le pape avait fait le cardinal de Noailles pour la France, le cardinal
de Lamberg, évêque de Passau, pour l'empereur, et le cardinal Borgia
pour l'Espagne. Le courrier du pape ne fit pas diligence, tellement que
ce ne fut que le 1er juillet, qu'au retour de sa promenade de Marly, le
roi trouva le nouveau cardinal qui l'attendait à Versailles dans son
appartement, qui lui présenta sa calotte. Le roi la lui mit sur la tête
avec force gracieusetés.

Cette promotion fut une cuisante douleur pour le cardinal de Bouillon,
de voir un Noailles paré comme lui de la pourpre, et un de ceux qui
étaient en lice contre M. de Cambrai, et qui l'avaient vaincu. Il venait
d'éprouver un coup de fouet plus personnel, mais qui lui fut peut-être
moins sensible.

L'abbé de Vaubrun avait été exilé à Serrant, en Anjou, chez son
grand-père maternel. Il était frère de la duchesse d'Estrées, et fils
unique de Vaubrun, tué lieutenant général à cette belle et mémorable
retraite que fit M. de Larges devant les Impériaux, après la mort de M.
de Turenne. Il avait pris le petit collet pour se cacher. Il était tout
à fait nain, en avait la laideur et la grosse tête, et il s'en fallait
pour le moins un pied que ses courtes jambes tortues ne fussent égales.
Avec cela beaucoup d'esprit et de la lecture, mais un esprit dangereux
tout tourné à la tracasserie et à l'intrigue. Il était accusé avec cela
de l'avoir fort mauvais, d'être peu sûr dans le commerce, et de se
livrer à tout pour être de quelque chose. Sa figure ne l'empêchait pas
d'attaquer les dames ni d'en espérer les faveurs, et de se fourrer comme
que ce fût partout où il pouvait trouver entrée. Ennuyé de l'obscurité
où il languissait, il obtint par MM. d'Estrées l'agrément de la charge
de lecteur du roi, que le baron de Breteuil lui vendit quand il acheta
celle d'introducteur des ambassadeurs, après la mort de Bonnœill, et ce
vilain et dangereux escargot se produisit à la cour et chercha à s'y
accrocher\,; il fit une cour basse aux Bouillon, il fut admis chez
eux\,; le cardinal de Bouillon le reconnut bientôt pour ce qu'il était.
Il lui fallait de tels pions pour jeter en avant\,; il se trouva son
espion, son agent, son correspondant dans toute sa conduite à Rome, et
d'un coup de pied il fut chassé.

Malgré tant de revers, le cardinal de Bouillon persévéra dans sa
résolution de ne pas perdre le décanat. Il amusa le roi tant qu'il put
d'une obéissance d'un ordinaire à l'autre, dès qu'il aurait son bref.
N'en pouvant cacher le refus, il fit semblant de partir et alla jusqu'à
Caprarole, où il s'arrêta, fit le malade, et dépêcha un courrier au P.
de La Chaise, pour le prier de rendre au roi une lettre par laquelle il
lui demandait la permission de demeurer à Rome, sans voir personne,
jusqu'à la mort du cardinal Cibo, lui remontrait la prétendue importance
que le décanat n'échappât pas aux Français, et ajoutait qu'il attendrait
ses ordres à Caprarole, qui est une magnifique maison du duc de Parme, à
huit lieues de Rome, à faire des remèdes dont sa santé avait, disait-il,
grand besoin. Il avait pris le parti de s'adresser au P. de La Chaise,
parce que M. de Torcy lui avait enfin mandé que le roi lui avait défendu
d'ouvrir aucunes de ses lettres, ni de lui en rendre aucunes de lui. Les
jésuites lui étaient de tout temps entièrement dévoués, et il espéra de
la voix touchante et accréditée du confesseur. Mais il trouva cette
porte aussi fermée que celle de M. de Torcy, et le P. de La Chaise lui
manda qu'il avait reçu les mêmes défenses. Il avait offert en même temps
la démission de son canonicat de Strasbourg. Comme on n'en avait aucun
besoin, elle fut refusée, et un nouvel ordre d'obéir et de partir
sur-le-champ lui fut renvoyé par un nouveau courrier.

Tous ces divers prétextes, les courriers du cardinal de Bouillon,
chargés de faire peu de diligence, ceux du roi retenus par le cardinal
le plus qu'il pouvait, tirèrent tant de long qu'il parvint à atteindre
ce qu'il désirait. Le cardinal Cibo mourut à Rome le 21 juillet. Le
cardinal de Bouillon, qui n'en était qu'à huit lieues, à Caprarole,
averti de son extrémité, alla à Rome la veille de sa mort, et dépêcha un
courrier par lequel il manda au roi qu'il avait reçu son dernier ordre
de partir, mais que l'extrémité du cardinal Cibo l'avait fait retourner
à Rome pour opter le décanat et partir vingt-quatre heures après,
persuadé que le roi ne trouverait pas mauvais un si court délai à lui
obéir par l'importance de conserver le décanat à un François. Cela
s'appelait se moquer du roi et de ses ordres, et être doyen malgré lui.
Aussi le roi en témoigna-t-il sa colère le jour même qu'il reçut cette
nouvelle, en parlant à Monsieur et à M. de Bouillon, quoique avec bonté
pour lui\,; cependant la mauvaise santé du pape empêcha qu'il ne pût
tenir le consistoire, et par conséquent le cardinal de Bouillon d'opter
l'évêché d'Ostie, tant qu'enfin le roi, ne pouvant plus souffrir une si
longue dérision de ses ordres, envoya ordre à M. de Monaco, son
ambassadeur, de lui commander de sa part de donner la démission de sa
charge de grand aumônier, d'en quitter le cordon bleu, et de faire ôter
les armes de France de dessus son palais, et de défendre à tous les
Français de le voir, et d'avoir aucun commerce avec lui.

M. de Monaco, qui haïssait le cardinal de Bouillon, surtout pour avoir
traversé sa prétention d'altesse, exécuta cet ordre fort volontiers,
après l'avoir concerté avec les cardinaux d'Estrées, Janson et
Coislin\,; le cardinal répondit qu'il recevait avec respect les ordres
du roi et ne s'expliqua pas davantage. Quoiqu'il dût bien s'attendre
qu'à la fin la bombe crèverait, il en parut accablé\,; mais comme il n'a
voit pu se résoudre à obéir sur le départ et perdre le décanat, il ne le
put encore sur la démission de sa charge\,; il se crut si grand d'être
doyen du sacré collège qu'il ne pensa pas au-dessus de lui de commencer
avec éclat une lutte avec le roi, qu'il n'avait jusqu'alors soutenue
qu'à la sourdine, et sous le masque des adresses et des mensonges. Mais
il faut encore interrompre ici cette matière qui arriérerait trop sur
les autres.

Au mariage de Mgr le duc de Bourgogne, le roi lui avait offert de lui
augmenter considérablement ses mois. Ce prince, qui s'en trouva assez,
le remercia et lui dit que si l'argent lui manquait il prendrait la
liberté de lui en demander. En effet, s'étant trouvé court en ce
temps-ci, il lui en demanda. Le roi le loua fort, et d'en demander quand
il en avait besoin, et de lui en demander lui-même sans mettre de tiers
entre eux\,; il lui dit d'en user toujours avec la même confiance et
qu'il jouât hardiment, sans craindre que l'argent lui manquât, et qu'il
n'était de nulle importance d'en perdre à des personnes comme eux. Le
roi se plaisait à la confiance, mais il n'aimait pas moins à se voir
craint, et lorsque des gens timides qui avaient à lui parler se
déconcertaient devant lui et s'embarrassaient dans leurs discours, rien
ne faisait mieux leur cour et n'aidait plus à leur affaire.

Il donna aussi cent mille francs à Mansart, qui fit son fils conseiller
au parlement.

L'archevêque de Reims présida à l'assemblée du clergé qui se tient de
cinq en cinq ans. L'archevêque d'Auch, Suze\footnote{L'archevêque d'Auch
  était alors Armand-Anne-Tristan de La Baume de Suze, qui gouverna ce
  diocèse de 1684 à 1703.}, lui fut adjoint, et tous deux firent si bien
qu'il n'y eut point d'évêques présidents avec eux, quoique la dernière
assemblée eût ordonné qu'il y aurait deux évêques avec deux
archevêques\,; ils eurent onze provinces pour eux qui l'emportèrent sur
les cinq autres. M. de Reims, dans sa harangue au roi à l'ouverture,
aurait pu se passer de nommer l'archevêque de Cambrai, dont les amis et
même les indifférents furent scandalisés\,; il proposa aussi à
l'assemblée d'insérer dans son procès-verbal copie de ceux des
assemblées provinciales tenues à l'occasion de sa condamnation, ce qui
fut fait en conséquence de pareils exemples. Elle fit aussi une
commission de six évêques, et de six du second ordre, à la tête desquels
fut M. de Meaux, pour examiner plusieurs livres, la plupart d'auteurs
jésuites, sur la morale, qui fut accusée d'être fort relâchée. M. d'Auch
ouvrit cet avis, qui passa à la pluralité de dix provinces contre six.
Il s'éleva une dispute dans ce bureau entre le premier et le second
ordre qui y prétendait la voix délibérative. Le premier ne lui voulut
reconnaître que la consultative, parce qu'il s'agissait, non d'affaires
temporelles, mais de doctrines, et, après quelques débats assez forts,
cela passa ainsi en faveur du premier ordre, et la fin de cette affaire
fut la condamnation de cent vingt propositions extraites de ces livres
par l'assemblée, en suite du beau rapport que lui en fit M. de Meaux.

Cette assemblée se tint à Saint-Germain quoique le roi d'Angleterre
occupât le château. M. de Reims y tenait une grande table et avait du
vin de Champagne qu'on vanta fort. Le roi d'Angleterre, qui n'en buvait
guère d'autre, en entendit parler et en envoya demander à l'archevêque,
qui lui en envoya six bouteilles. Quelque temps après, le roi
d'Angleterre, qui l'en avait remercié, et qui avait trouvé ce vin fort
bon, l'envoya prier de lui en envoyer encore. L'archevêque, plus avare
encore de son vin que de son argent, lui manda tout net que son vin
n'était point fou et ne courait point les rues, et ne lui en envoya
point. Quelque accoutumé qu'on fût aux brusqueries de l'archevêque,
celle-ci parut si étrange qu'il en fut beaucoup parlé, mais il n'en fut
autre chose.

Les disputes de la Chine commençaient à faire du bruit sur les
cérémonies de Confucius et des ancêtres, etc., que les jésuites
permettaient à leurs néophytes et que les missions étrangères
défendaient aux leurs\,; les premiers les soutenaient purement civiles,
les autres qu'elles étaient superstitieuses et idolâtriques. Ce procès
entre eux a eu de si terribles suites qu'on en a écrit des mémoires fort
étendus et des questions et des faits, et on en a des histoires
entières. Je me contenterai donc de dire ici que les livres que les PP.
Tellier et Le Comte avaient publiés sur cette matière furent déférés à
la Sorbonne par les missions étrangères, et qu'après un long et mûr
examen, ils furent fortement condamnés, tellement que le roi, alarmé que
la conscience de M\textsuperscript{me} la duchesse de Bourgogne fût
entre les mains du P. Le Comte, qu'elle goûtait fort et la cour aussi,
le lui ôta, et pour un sauve-l'honneur les jésuites l'envoyèrent à Rome
et publièrent que de là, après s'être justifié, il retournerait à la
Chine. La vérité fut qu'il alla à Rome, mais qu'il ne s'y justifia ni ne
retourna aux missions. On fit essayer plusieurs jésuites à
M\textsuperscript{me} la duchesse de Bourgogne, qui aurait bien voulu ne
se confesser à pas un. Elle avait eu à Turin, la seule cour catholique
qu'ils ne gouvernent pas et qui se tient en garde contre eux et les
tient bas, un confesseur qui était barnabite, et un fort saint homme et
fort éclairé. Elle eût bien voulu pouvoir choisir dans le même ordre,
mais le roi voulut un jésuite\,; et, après en avoir essayé plusieurs,
elle s'en tint au P. de La Rue, si connu par ses sermons et par d'autres
endroits.

Cette affaire mortifia cruellement les jésuites, d'autant plus que cette
même affaire leur bâtait mal à Rome, et remplit le P. Tellier d'une rage
qui devint bien funeste dans la suite. Les jésuites, ainsi pincés sur
leur morale d'Europe et d'Asie, s'en revanchèrent en attendant d'autres
conjonctures sur le temporel, et firent si bien par le roi, auprès de
l'assemblée, qu'ils furent pour toujours affranchis des taxes et des
impositions du clergé. Ils alléguèrent la pauvreté de leur maison
professe et les besoins de leurs collèges. Ils ne parlaient pas de leurs
ressources\,; le roi témoigna désirer qu'il ne fût rien imposé sur eux
pour ce que le clergé lui paye, et l'assemblée qui les avait malmenés
d'ailleurs ne voulut pas, en résistant là-dessus, témoigner de passion
contre eux. Les jésuites firent une protestation contre la censure de la
Sorbonne, laquelle publia une réponse fort vive à la protestation, de
manière que les esprits de part et d'autre demeurèrent fort aigres.

Pelletier, conseiller d'État, qui avait été longtemps intendant de
Flandre, et qui avait été fort connu du roi, parce qu'il y avait eu
nécessairement la confiance et la commission de beaucoup de
dispositions, pour les conquêtes de ce pays-là, avait eu à la mort de
Louvois l'intendance des fortifications de toutes les places, ce qui lui
donnait toutes les semaines un travail tête à tête avec le roi. Cela ne
laissait pas d'être plaisant d'un homme de robe de décider de
l'importance des places, du choix de leurs ouvrages, du mérite même
militaire et de la fortune du corps des ingénieurs, tandis que Vauban
avait acquis en ce genre la première réputation de l'Europe, et que le
roi n'ignorait pas que ce ne fût à lui qu'il ne dût tous les succès de
tous les sièges qu'il avait faits en personne et de la plupart de ceux
qu'il avait fait faire, et qu'il eût pour lui l'estime et l'amitié qu'il
méritait. C'était aussi l'homme entre tous à choisir pour l'envoyer
visiter toutes les places et les ports de l'Océan, qu'on voulait mettre
en état de ne rien craindre\,; mais c'était le règne de la robe pour
tout, et ce fut Pelletier qui fut chargé de cette commission.

M. de Vendôme prit une autre fois congé publiquement du roi et des
princes et princesses pour s'aller remettre entre les mains des
chirurgiens. Il reconnut enfin qu'il avait été manqué, que son
traitement serait long, et il s'en alla à Anet travailler au
recouvrement de sa santé, qui ne lui réussit pas mieux que la première
fois. Mais il rapporta, celle-ci, un visage sur lequel son état demeura
encore plus empreint que la première fois.

M\textsuperscript{me} d'Uzès, fille unique du prince de Monaco, mourut
de ce mal\,; c'était une femme de mérite et fort vertueuse, peu heureuse
et qui méritait un meilleur sort. Son mari était un homme obscur, qui ne
voyait personne que des gueuses et qui s'en tira mieux qu'elle, qui fut
fort plainte et regrettée. Ses enfants périrent du même mal et elle n'en
laissa point.

M\textsuperscript{me} du Maine fit un mariage de la faim et de la
soif\,: ce fut celui de M\textsuperscript{lle} de Lussan, fille de
Lussan, chevalier de l'ordre, qui était à M. le Prince, et de la dame
d'honneur de M\textsuperscript{me} la Princesse, avec le duc
d'Albemarle, bâtard du roi d'Angleterre et d'une comédienne. Il était
chef d'escadre et n'avait rien vaillant\,: M\textsuperscript{lle} de
Lussan, quoique unique, n'avait guère davantage. M\textsuperscript{me}
du Maine, qui s'en était coiffée, lit accroire au bâtard qu'il en était
amoureux, et que, par le crédit de M. du Maine, il aurait tout à souhait
en l'épousant. C'était bien l'homme le plus stupide qui se pût trouver.
Il se maria donc sur ces belles espérances, logé et nourri chez M. du
Maine, où il fila le parfait amour. Elle fut assise comme duchesse du
roi d'Angleterre, que le roi traitait bien en tout, car d'ailleurs les
ducs et les duchesses d'Angleterre n'ont point de rang en France.

Le roi, dont le goût croissait chaque jour pour Chamillart, lui fit une
grâce que Pontchartrain ni aucun autre contrôleur général n'avait osé
espérer\,: ce fut de faire entrer M\textsuperscript{me} Chamillart dans
les carrosses de M\textsuperscript{me} la duchesse de Bourgogne et
manger avec elle. Sa fille eut le même honneur, sous prétexte de la
charge de grand maître des cérémonies qu'avait eue son mari, et par là
la porte de Marly leur fut ouverte et de tous les agréments de la cour.
La vérité est que, dès que les femmes des secrétaires d'État y étaient
parvenues, celles des contrôleurs généraux pouvaient bien valoir autant.

Le roi fit presque en même temps ce qu'il n'a pas fait cinq ou six fois
dans sa vie. Le chapitre de Chartres, tout à fait indépendant de son
évêque, avait toute l'autorité dans la cathédrale, où l'évêque ne
pouvait officier sans sa permission que très peu de jours marqués dans
l'année, ni jamais y dire la messe basse\,; il avait un grand territoire
où étaient un grand nombre de paroisses qui lui faisait un petit diocèse
à part, où l'évêque ne pouvait rien, et quantité d'autres droits fort
étranges, directement contraires à toute hiérarchie. Godet des Marais,
évêque de Chartres, et qui en faisait très assidûment et très
religieusement tous les devoirs, se trouvait barré en mille choses. Dans
la position intime où il se trouvait avec le roi et
M\textsuperscript{me} de Maintenon, il essaya de faire entendre raison à
son chapitre sur des droits si abusifs, sans l'avoir pu induire à
entendre à aucune sorte de modération\,; il espéra de sa patience, et de
temps en temps revint à la charge et toujours sans aucun succès. Lassé
enfin, il crut devoir user, pour le rétablissement d'un meilleur ordre,
de la conjoncture où il était\,: il attaqua son chapitre en justice, où
il sentait bien qu'il ne réussirait pas, mais le procès engagé, il le
fit évoquer pour être jugé par le roi lui-même.

Un bureau de conseillers d'État avec un maître des requêtes, rapporteur,
travailla contradictoirement sur cette affaire, et lorsqu'elle fut
instruite, ce bureau entra au conseil des dépêches\footnote{Voy. sur le
  conseil des dépêches la noie à la fin du t. Ier.} où le rapporteur la
rapporta. L'usurpation était si ancienne, si confirmée par les papes,
par les rois, par un usage non interrompu, que tous ceux qui étaient à
ce conseil, convenant de la difformité de l'usurpation et du désordre,
furent pourtant d'avis de maintenir le chapitre en tout. Le roi leur
laissa tout dire tant qu'ils voulurent, sans montrer ni impatience ni
penchant. Tout le monde ayant achevé d'opiner\,: «\,Messieurs, leur
dit-il, j'ai très bien entendu l'affaire et vos opinions à tous, mais
votre avis n'est pas le mien, et je trouve la religion, la raison, le
bon ordre et la hiérarchie si blessés par les usurpations du chapitre,
que je me servirai en cette occasion, contre ma constante coutume, de
mon droit de décision, et je prononce en tout et partout en faveur de
l'évêque de Chartres.\,» L'étonnement fut général, tous se
regardèrent\,; M. le chancelier, qui n'aimait pas M. de Chartres, fort
sulpicien, fit quelques représentations. Le roi l'écouta, puis lui dit
qu'il persistait, le chargea de dresser l'arrêt conformément aux
conclusions de M. de Chartres, et lui ordonna de plus de lui apporter
l'arrêt le lendemain, qui fut une défiance qui dut peiner le chancelier.

Malgré une volonté si rare et si marquée, le chancelier, ou piqué, ou
plein du droit du chapitre, ou craignant qu'en certaines affaires le roi
s'accoutumât à l'exercice de ce droit, osa adoucir l'arrêt en faveur du
chapitre. Le roi écouta encore ses raisons, puis raya lui-même l'arrêt,
et se le fit apporter le lendemain tel en tout qu'il l'avait ordonné. Ce
fut un grand dépit au chancelier, qui ne le put cacher à l'évêque de
Chartres lorsqu'il l'alla voir. Ce prélat, qui avec les défauts d'un
homme nourri et pétri de Saint-Sulpice, était un grand et saint évêque,
se contenta d'avoir vaincu et remis les choses dans l'ordre naturel et
dans la règle sans user de son arrêt après l'avoir fait signifier, et ne
songea qu'à regagner l'amitié de son chapitre, dont cette modération et
l'estime qu'il ne pouvait lui refuser facilita fort le retour. Ce prélat
était fort loin d'être janséniste ni quiétiste, comme on a vu\,; mais,
d'autre part, il n'aimait point les jésuites, les tenait de court et
bas, et partageait fort avec le P. de La Chaise la distribution des
bénéfices sans en prendre pour soi ni pour les siens. Malheureusement,
comme je l'ai dit ailleurs, ses choix ne furent pas bons\,; il infecta
l'épiscopat d'ignorants, entêtés, ultramontains, barbes sales de
Saint-Sulpice et de tous gens de bas lieu et du plus petit génie, ce qui
n'a été que trop suivi depuis.

L'archevêque de Reims, ravi de présider l'assemblée du clergé lors fort
bien composée, y brilla par sa doctrine, par sa capacité, par sa
dépense. Il était fort bien avec le roi et fort soutenu de Barbezieux,
son neveu, qui tirait de sa place une grande autorité. Dans les
commencements, le prélat contraignait son naturel brutal comme sont tous
ceux de sa famille, et plus que qui que ce soit les bourgeois
\emph{porphyrogénètes}\footnote{On donnait le nom de
  \emph{porphyrogénètes}, ou nés dans la pourpre, aux fils des empereurs
  byzantins nés depuis l'avènement de leur père au trône.}, c'est-à-dire
nés dans toute la considération et le crédit d'un long et puissant
ministère\,; mais peu à peu l'homme revient à son naturel. Celui-ci bien
ancré, ce lui semblait, dans l'assemblée, s'y contraignit moins et de
l'un à l'autre se permit tant de brutalités et d'incartades qu'il la
banda entièrement contre lui\,; il y reçut tant de dégoûts et y essuya
tant de refus de choses que le moindre de l'assemblée eût fait approuver
s'il les eût proposées, qu'il se détermina au remède du monde le plus
honteux et dont il fit le premier exemple. M. de Paris était devenu
cardinal depuis l'ouverture de l'assemblée, et depuis peu de jours le
roi lui avait donné le bonnet apporté par l'abbé de Barrière, camérier
d'honneur du pape. S'il l'eût été avant l'ouverture, la présidence lui
pouvait être offerte et acceptée. C'eût été un dégoût pour M. de Reims,
l'ancien des archevêques députés, mais moindre par la qualité de
diocésain, jointe à celle de cardinal, dans le cardinal de Noailles\,;
mais de se le mettre, à la moitié et plus de l'assemblée, sur la tête,
cela ne s'était jamais pratiqué. C'est pourtant ce que fit l'archevêque
de Reims, qui lui-même y fit entrer le roi, en lui avouant qu'il ne
trouvait plus qu'obstacles personnels à tout ce qu'il était à propos de
faire, tellement que le cardinal de Noailles présida tout le reste de
l'assemblée, et M. de Reims n'y fit plus de rien que de sa présence en
second. Avec son siège, sa pourpre, sa faveur, sa douceur, ses mœurs, sa
piété et son savoir, il gouverna toute l'assemblée sans peine, et s'y
acquit beaucoup de réputation.

C'était un homme fort modeste, et continuellement résidant à Châlons, où
il n'y avait pas occasion de faire montre de sa capacité en affaires ni
en doctrines. Un air de béatitude que sa physionomie présentait, avec un
parler gras, lent et nasillard, la faisait volontiers prendre pour
niaise, et sa simplicité en tout pour bêtise. La surprise fut grande,
quand par des discours sur-le-champ, et sur des matières de doctrine ou
d'affaires qui, naissant dans les séances, ne pouvaient laisser aucun
soupçon de la préparation la plus légère, on reconnut un grand fonds
d'érudition d'une part, de capacité de l'autre, d'ordre et de netteté en
tous les deux, avec le même style de ses mandements et de ses écrits
contre M. de Cambrai, et sur d'autres matières de doctrine, et sans
sortir de sa simplicité ni de sa modestie. On vit cet homme, qui à Paris
comme à Châlons se contentait de son bouilli avec deux petites et
grossières entrées, servi splendidement et délicatement, et, l'occasion
passée, retourner tout court à son petit ordinaire, en gardant toujours
ses officiers pour s'en servir quand il était nécessaire. Jamais grand
seigneur ni cardinal qui, sans sortir d'aucune bienséance, fût moins
l'un et l'autre, et jamais ecclésiastique plus prêtre ni plus évêque
qu'il le fut toujours.

Le roi ordonna que les comtes d'Uzès et d'Albert, accusés de duel contre
les comtes de Rantzau, Danois, et de Schwartzenberg, Autrichien, se
remettraient à la Conciergerie\,; ils prirent le large. Barbezieux
envoya courre après son beau-frère, qui sur sa parole se remit\,; le
comte d'Albert ne revint que longtemps après dans la même prison. Il fut
cassé pour sa désobéissance, et le roi voulut que Monseigneur disposât
de son régiment de dragons qu'il avait. À la fin ils sortirent l'un et
l'autre, mais le comte d'Albert, avec tout le crédit de M. de Chevreuse,
et la belle action qu'il avait faite de s'être jeté dans Namur à travers
les assiégeants et d'y être entré à la nage, son épée entre ses dents,
ne put jamais être rétabli. Il était plus que bien avec
M\textsuperscript{me} de Luxembourg, Rantzau aussi\,; cela fit la
querelle, dont la raison fut sue de tout le monde et fit un étrange
bruit. M. le prince de Conti me conta en revenant de Meudon qu'il
n'avait jamais été si embarrassé, ni n'avait tant souffert en sa vie. Il
était, comme on l'a vu, ami intime de feu M. de Luxembourg et l'était
demeuré de même de celui-ci. À Meudon on ne parlait que de ce combat et
de sa cause.

M. de Luxembourg était le seul qui l'ignorât. Il la demandait à tout le
monde, et, comme on peut croire, personne ne la lui voulut apprendre\,;
lui aussi ne comprit jamais ce secret, et alla à maintes reprises à M.
le prince de Conti pour le savoir, avec des presses et des instances à
le mettre au désespoir. Il en sortit pourtant sans le lui dire, et il
m'assura qu'il n'avait jamais été si aise de sortir de Meudon et de la
fin du voyage, pour éviter M. de Luxembourg jusqu'à ce qu'il n'en fût
plus question.

Le roi, pressé par M\textsuperscript{me} la duchesse de Bourgogne, bonne
et facile, permit l'entrée de ses carrosses et de manger avec elle, à
M\textsuperscript{me} de Villacerf, qui était Saint-Nectaire et femme de
son premier maître d'hôtel, sur l'exemple de M\textsuperscript{me} de
Chamarande, quoique M\textsuperscript{me} de Villacerf la mère, en
pareille place et femme d'un homme bien plus accrédité et considéré,
n'eût jamais osé y prétendre\,; mais aussi d'elle, elle n'était rien.

Il donna aussi à M. le prince de Conti dix-huit mille livres
d'augmentation de pension, et à M. de Duras vingt mille livres
d'augmentation d'appointements de son gouvernement de Franche-Comté.
Monseigneur donna aussi deux mille louis à Sainte-Maure, qui lui fit
représenter par M\textsuperscript{me} la princesse de Conti l'embarras
où il était d'avoir beaucoup perdu au jeu.

Les fiançailles de La Vrillière avec M\textsuperscript{lle} de Mailly
avaient été faites quinze jours après la déclaration de son mariage avec
M\textsuperscript{lle} de Mailly, en présence du roi et de toute la
cour, dans le grand cabinet de M\textsuperscript{me} la duchesse de
Bourgogne, où le contrat avait été signé par le droit de fille de la
dame d'atours\,; dès qu'elle eut douze ans accomplis ils se marièrent\,;
la chancelière donna à dîner à la noce. Ils couchèrent dans
l'appartement de la comtesse de Mailly, où M\textsuperscript{me} la
duchesse de Bourgogne s'en amusa tout le jour. Le roi avait donné la
charge de greffier de l'ordre à La Vrillière qu'avait son père, et le
râpé au chancelier. Le premier en avait grand besoin pour le parer un
peu.

Le P. Valois, jésuite célèbre, mais meilleur homme que ceux-là ne le
sont d'ordinaire, mourut d'une longue maladie de poitrine qui ne
l'empêcha point d'aller presque jusqu'à la fin. Il était confesseur des
enfants de France. Le P. de La Chaise en fit la fonction quelque temps,
et le P. Martineau remplit après cette place. Le P. Valois était un de
ceux qui avaient tenu pour M. de Cambrai. C'était un homme doux,
d'esprit et de mérite, qui fut et qui mérita d'être regretté.

Le Nôtre mourut presque en même temps, après avoir vécu
quatre-vingt-huit ans dans une santé parfaite, {[}avec{]} sa tête et
toute la justesse et le bon goût de sa capacité\,; illustre pour avoir
le premier donné les divers dessins de ces beaux jardins qui décorent la
France, et qui ont tellement effacé la réputation de ceux d'Italie qui,
en effet, ne sont plus rien en comparaison, que les plus fameux maîtres
en ce genre viennent d'Italie apprendre et admirer ici. Le Nôtre avait
une probité, une exactitude, et une droiture qui le faisait estimer et
aimer de tout le monde. Jamais il ne sortit de son état ni ne se
méconnut, et fut toujours parfaitement désintéressé. Il travaillait pour
les particuliers comme pour le roi, et avec la même application\,; ne
cherchait qu'à aider la nature, et à réduire le vrai beau aux mains de
frais qu'il pouvait\,; il avait une naïveté et une vérité charmante. Le
pape pria le roi de le lui prêter pour quelques mois. En entrant dans la
chambre du pape, au lieu de se mettre à genoux, il courut à lui.
«\,Eh\,! Bonjour, lui dit-il, mon révérend père, en lui sautant au cou
et l'embrassant et le baisant des deux côtés. Eh\,! que vous avez bon
visage, et que je suis aise de vous voir et en si bonne santé\,!» Le
pape, qui était Clément X, Altieri, se mit à rire de tout son cœur. Il
fut ravi de cette bizarre entrée, et lui fit mille amitiés.

À son retour le roi le mena dans ses jardins de Versailles, où il lui
montra ce qu'il y avait fait depuis son absence. À la colonnade il ne
disait mot. Le roi le pressa d'en dire son avis\,: «\,Eh bien\,! sire,
que voulez-vous que je vous dise\,? d'un maçon vous avez fait un
jardinier (c'était Mansart)\,; il vous a donné un plat de «\,son
métier.\,» Le roi se tut et chacun sourit\,; et il était vrai que ce
morceau d'architecture, qui n'était rien moins qu'une fontaine et qui la
voulait être, était fort déplacé dans un jardin. Un mois avant sa mort,
le roi, qui aimait à le voir et à le faire causer, le mena dans ses
jardins, et, à cause de son grand âge, le fit mettre dans une chaise que
des porteurs roulaient à côté de la sienne, et Le Nôtre disait là\,:
«\,Ah\,! mon pauvre père, si tu vivais et que tu pusses voir un pauvre
jardinier comme moi, ton fils, se promener en chaise à côté du plus
grand roi du monde, rien ne manquerait à ma joie.\,» Il était intendant
des bâtiments et logeait aux Tuileries, dont il avait soin du jardin,
qui est de lui, et du palais. Tout ce qu'il a fait est encore fort
au-dessus de tout ce qui a été fait depuis, quelque soin qu'on ait pris
de l'imiter et de travailler d'après lui le plus qu'il a été possible.
Il disait des parterres qu'ils n'étaient bons que pour les nourrices
qui, ne pouvant quitter leurs enfants, s'y promenaient des yeux et les
admiraient du deuxième étage. Il y excellait néanmoins comme dans toutes
les parties des jardins, mais il n'en faisait aucune estime, et il avait
raison, car c'est où on ne se promène jamais.

Labriffe, procureur général, mourut bientôt après, d'une longue maladie,
du chagrin dans lequel il vécut dans cette charge, des dégoûts et des
brocards dont le premier président Harlay l'accabla. J'ai assez parlé de
ce magistrat, à propos du procès de préséance de M. de Luxembourg, pour
n'avoir rien à y ajouter. D'Aguesseau, avocat général, eut sa charge.
C'est lui aussi dont j'ai parlé à la même occasion, et qui longtemps
depuis a fait une si grande et si triste fortune.

\hypertarget{chapitre-xxv.}{%
\chapter{CHAPITRE XXV.}\label{chapitre-xxv.}}

1700

~

{\textsc{Arrêt du conseil, à faute de mieux, qui dépouille le cardinal
de Bouillon.}} {\textsc{- Cardinal de Coislin fait grand aumônier.}}
{\textsc{- Évêque de Metz premier aumônier en titre.}} {\textsc{-
Conduite du cardinal de Bouillon.}} {\textsc{- Réflexion sur les
cardinaux français.}} {\textsc{- Mort du duc de Glocester.}} {\textsc{-
Le Vassor.}} {\textsc{- Mesures sur l'Espagne.}} {\textsc{- Paix du Nord
en partie.}} {\textsc{- Voyage de Fontainebleau.}} {\textsc{-
Zinzendorf, envoyé de l'empereur, mange avec Monseigneur.}} {\textsc{-
M\textsuperscript{me} de Verue\,; ses malheurs, sa fuite de Turin en
France.}} {\textsc{- Jugement en faveur de la Bretagne, de sa propre
amirauté contre l'amirauté de France.}} {\textsc{- Acquisition de sceaux
par M. du Maine.}} {\textsc{- Mort de M\textsuperscript{lle} de Condé.}}
{\textsc{- D'Antin quitte le jeu solennellement et le reprend ensuite.}}
{\textsc{- Mort de M. de la Trappe.}} {\textsc{- Mort du pape Innocent
XII (Pignatelli).}}

~

M. le cardinal de Bouillon, toujours dans Rome, attendant un consistoire
pour y opter le décanat et l'évêché d'Ostie, continuait à porter
l'ordre, et en bon français à se moquer du roi. Il prétendait très
faussement que sa charge de grand aumônier était office de la couronne,
comme force autres choses, et que conséquemment en ne donnant point de
démission, elle ne pouvait lui être ôtée sans lui faire son procès, dont
sa pourpre le mettait à l'abri. Le roi, enfin, excédé d'une
désobéissance si poussée et si éclatante, ordonna au parlement de lui
faire son procès, mais, quand on voulut y travailler, tant d'obstacles
se présentèrent qu'il en fallut quitter le dessein. On y suppléa par un
arrêt du conseil rendu en présence du roi, le dimanche 12 septembre, qui
ordonna la saisie de tous les biens laïques et ecclésiastiques du
cardinal de Bouillon, en partageant les derniers en trois portions, pour
les réparations, les aumônes et la confiscation, et tous les biens
laïques confisqués, et cet arrêt fut envoyé à tous les intendants des
provinces pour le faire exécuter sur-le-champ et à la rigueur. Le même
jour les provisions de la charge de grand aumônier furent envoyées au
cardinal de Coislin, à Rome, et celles de premier aumônier expédiées à
l'évêque de Metz, son neveu, qui n'en avait que la survivance. Le roi
chargea Pontchartrain de porter cette triste nouvelle au duc de
Bouillon, et de lui dire que c'était avec déplaisir qu'il était obligé
d'en venir là. Le désespoir du cardinal fut extrême en apprenant cet
arrêt, et sa charge donnée au cardinal de Coislin qui n'osa la refuser.
L'orgueil l'avait toujours empêché de croire qu'on en vint à cette
extrémité avec lui. Il ne donna point sa démission qu'on ne lui
demandait plus et dont on n'avait plus que faire. Son embarras fut
l'ordre\,; M. de Monaco le fit avertir que, s'il ne le quittait, il
avait ordre de le lui aller arracher du cou. S'il avait pu espérer
quelque suite embarrassante d'une démarche si forte contre un cardinal,
il n'eût pas mieux demandé, mais sa fureur un peu rassise lui laissa
voir toute sa faiblesse, et toute la folie de prétendre garder malgré le
roi l'ordre qu'il n'en avait reçu que comme la marque d'une charge qu'il
lui avait ôtée, et dont il avait revêtu un autre cardinal actuellement
aussi dans Rome. Il quitta donc les marques de l'ordre, mais ce qu'il
fit de pitoyable est qu'il porta un cordon bleu étroit avec la croix
d'or au bout, sous sa soutane, et qu'il tâchait de fois à autres de
laisser entrevoir un peu de ce bleu, entre le haut de sa soutane et son
porte-collet.

Je ne puis m'empêcher d'admirer ici la manie d'avoir des cardinaux en
France, et de mettre des sujets en état de faire compter avec eux,
d'attenter tout ce que bon leur semble, et de narguer impunément les
rois et les lois. Le roi avait senti au commencement de son règne le
poids insultant de cette pourpre, jusque dans sa capitale, par le
cardinal de Retz, qui, après tout ce qu'il avait commis, força enfin à
lui faire un pont d'or, et à se faire recevoir avec toutes sortes de
distinctions et d'avantages. Les dernières années du même règne furent
marquées au même coin par le cardinal de Bouillon. Si nos rois ne
souffraient point de cardinaux en France, et s'ils donnaient leur
nomination à des Italiens, ils s'attacheraient les premières maisons et
les principaux sujets de Rome par cette espérance, et ceux qu'ils
nommeraient, étant du pays, dans leurs familles et parmi leurs amis, au
fait de jour à jour de tout ce qui se passe à Rome, y serviraient bien
plus utilement qu'un cardinal français, qui est longtemps à se mettre au
fait de cette carte, qui y est toujours considéré comme en passant et
qui ne peut jamais acquérir l'amitié, la confiance, ni la facilité de
manège et d'industrie d'un naturel du pays. Ce cardinal italien n'a
point d'amis ni de famille en France qui le soutienne s'il vient à
mécontenter. Il est donc bien plus attentif à bien faire qu'un Français,
qui ne parvient pas là sans de bons appuis, ou qui tout au plus s'en
console en retournant chez lui parmi les siens, où, quoi qu'il ait fait,
il nage dans les biens, dans les plus grands honneurs, et jouit de
toutes les distinctions, de toute la considération, et de tous les
ménagements, pour soi et pour les siens, qui en sont une suite
nécessaire. On ne craint plus un Italien qui, avec la confiance de la
cour qui l'a élevé, perd tout son relief à Rome et tombe dans le mépris,
et dont l'exemple apprend à son successeur à éviter une disgrâce qui
remplit de dégoûts tout le reste d'une vie.

Pour les conclaves, les Italiens se trouvent tout portés et tout
instruits des intérêts des brigues et des menées, et à portée de serrer
la mesure avant l'arrivée des étrangers, s'ils voient jour à faire leur
coup, au lieu qu'il faut bien du temps à ceux qui arrivent pour se
mettre au fait dont ils ne peuvent être instruits que par les autres qui
les abusent bien souvent, et, le conclave fini, n'ont plus grande hâte
que de s'en retourner. Un Italien, au contraire, qui a contribué à une
exaltation, et qui n'a d'autre demeure que Rome, profite pour la
couronne qu'il sert de la bienveillance qu'il s'est acquise du pape et
de sa famille, et susceptible qu'il est pour la sienne de toutes les
petites grâces de la prélature de Rome, et lié et instruit comme il
l'est à fond dans cette cour, ses vues sont bien plus justes et plus
animées, et mieux secondées de son adresse et de ses amis, pour procurer
un pape, qui convienne, et dont l'amitié, influant sur les siens,
devienne aussi utile à la couronne. Il se contente de quelques bonnes
abbayes\,; il ne lui faut pas quarante ou cinquante mille livres de
rente, comme à nos cardinaux qui se croient pauvres et maltraités à
moins de trois cent mille livres de rente\,; et comme tout est de
proportion, et que les cardinaux italiens ne sont pas riches, jusqu'à
s'accommoder de deux cents écus de pension, il est en biens fort
au-dessus de tous les autres pour peu qu'il ait quelques abbayes
considérables, et a plus de crédit et de moyen que les nôtres, à les
prendre régulières à la décharge de notre clergé, et comme il n'a point
de voyages à faire, il n'y en a point à lui payer, comme à nos
cardinaux. Il n'a rien en France à demander pour les siens, et sa
fortune de ce côté-là se borne à lui-même. Il est plus souple avec notre
ambassadeur, parce qu'il est sans appui à la cour que son service, et
leur concert n'est point sujet aux jalousies, parce que, bien loin
d'espérer l'emporter sur lui comme nos cardinaux, c'est de son union
avec lui que dépendent ses succès dans les affaires, et de son
témoignage la satisfaction et la considération qu'il se propose de
mériter. Par là notre clergé devient indépendant de la cour de Rome\,;
il n'a plus de tentations de nourrir ses espérances par sa mollesse et
le sacrifice des droits de l'épiscopat, de ceux du roi et de la couronne
et des libertés de notre Église. Pour un chapeau qu'un de nos prélats
attrape par ses souplesses et sa dépendance de Rome, un grand nombre
d'autres suivent la même route pour une espérance qui se diffère, qui
les anime au lieu de les rebuter, et qui pourtant ne s'accomplit jamais.

Cette ambition, coupée par la racine, rendrait la cour de Rome bien
moins entreprenante, et bien plus mesurée, préviendrait ses pratiques
par le confesseur, par les jésuites et par les autres réguliers dont
elle dispose, et délivrerait des embarras d'avoir à lui résister. Elle
n'aurait plus d'espérance en celle des ministres et des favoris pour
leurs proches. Le cardinalat, qui est une grande illustration pour les
gens nouveaux, est toujours un grand avantage pour les autres, qui
trouvent des avancements et des préférences par la considération d'un
cardinal leur parent qui les pousse, et dont la riche bourse supplée à.
leurs besoins. C'est ce qui rend les gens en place si mesurés avec Rome,
qu'ils savent irréconciliable pour les moindres oppositions qu'elle
rencontre. Ceux même qui n'ont encore personne en maturité pour songer
au cardinalat n'en veulent pas devenir obstacles, et par tous ces
ménagements Rome entreprend et réussit toujours\,; au lieu que si aucun
François ne pouvait jamais parvenir à la pourpre, tous n'auraient plus
les yeux tournés que vers le roi, parce qu'ils n'espéreraient rien que
de lui, et que tout autre avancement, grandeur, richesse, leur serait
absolument interdit. Mais voilà assez inutilement raisonné, puisque nos
rois sont complices contre eux-mêmes, et que rien ne les corrige de
fournir des armes contre leur personne et contre leur couronne, et que
leurs plus grands dons sont pour ceux qui s'affranchissent de leur
dépendance et de l'autorité de toutes les lois.

Le roi d'Angleterre perdit le duc de Glocester, héritier présomptif de
ses couronnes, depuis que son usurpation avait passé en lui. Il avait
onze ans, et était fils unique de la princesse de Danemark, sœur puînée
de père et de mère de la défunte reine, femme du roi Guillaume, et
n'avait ni frères ni sœurs. Son précepteur était le docteur Burnet,
évêque de Salisbury, qui eut le secret de l'affaire de l'invasion, et
qui passa en Angleterre avec le prince d'Orange à la révolution, dont il
a laissé une très frauduleuse histoire, et beaucoup d'autres ouvrages où
il n'y a pas plus de vérité ni de bonne foi.

Le sous-précepteur était le fameux Vassor, auteur de l'\emph{Histoire de
Louis XIII}, qui se ferait lire avec encore plus de plaisir, s'il y
avait mis moins de rage contre la religion catholique, et de passion
contre le roi et contre beaucoup de gens\,: cela près, elle est
excellente et vraie\,; il faut qu'il ait été singulièrement bien informé
des anecdotes qu'il raconte et qui échappent à presque tous les
historiens. J'y ai trouvé par exemple la journée des Dupes précisément
comme mon père me l'a racontée, qui y a fait un personnage si principal
et si intime, et plusieurs autres endroits curieux qui n'ont pas moins
d'exactitude. Cet auteur a tant fait de bruit qu'il vaut bien la peine
que j'en dise quelque chose. Il était prêtre de l'Oratoire, fort
appliqué à l'étude, et fort bien dans sa congrégation\,; d'ailleurs
homme de bas lieu. Personne ne s'y défiait de lui, et il était même
considéré comme un homme dont les mœurs étaient sans reproches, dont
l'esprit et le savoir faisait honneur à l'Oratoire, et qui était pour y
occuper les premières places avec le temps.

La surprise fut donc extrême lorsque, durant la tenue d'une assemblée
générale, le P. de La Chaise témoigna beaucoup d'aigreur aux supérieurs
principaux d'une résolution qu'ils avaient crue entièrement secrète. Le
soupçon n'en put tomber crue sur le P. Le Vassor, qui la savait par la
confiance qu'on avait en lui. On prit un temps qu'il n'était point à sa
chambre pour y entrer. Les mêmes supérieurs y visitèrent ses papiers. Sa
table même le trahit\,; il y avait laissé des lettres de lui et à lui,
des mémoires et d'autres choses qui firent la plus complète preuve de sa
trahison, et que, depuis qu'il avait pris le collet de l'Oratoire, il
n'avait cessé d'y être l'espion des jésuites. Cet honnête homme, revenu
dans sa chambre, jette les yeux sur sa table, et la voit fort déchargée
de papiers\,: il la visite, et voit ce qui lui manque. Le voilà éperdu.
Il cherche partout dans un reste de désir, plutôt que d'incertitude, de
les avoir déplacés lui-même, mais la recherche n'est pas achevée que ces
mêmes supérieurs viennent lui en ôter la peine. La fureur d'être
découvert succéda à l'inquiétude\,: il fit son paquet, se retira et
allongea dès le lendemain son collet. Désespéré, il va au P. de La
Chaise lui demander une abbaye, et lui exposer l'accablement de son
état. Un espion, devenu inutile, ne porte pas grand mérite avec soi. La
découverte qui le déshonorait retombait à plomb sur les jésuites, qui ne
furent pas pressés de récompenser son imprudence. Outré de désespoir, de
honte, de faim, et d'une attente de bénéfice qui devenait un surcroît de
douleur, il fut se jeter à la Trappe. Les vues qui l'y portèrent
n'étaient pas droites, aussi n'eurent-elles aucunes bénédictions\,: en
peu de jours sa vocation se trouva desséchée. Il s'en alla à l'abbaye de
Perseigne\,; il en loua le logis abbatial, et y demeura quelques mois.
Il y eut cent prises avec les moines. Leur jardin n'était séparé du sien
que par une forte haie. Les poules des moines la franchissaient\,; il
s'en prit aux moines, tant qu'un jour il attrapa le plus de leurs poules
qu'il put, leur coupa le bec et les ergots avec un couperet, et les jeta
aux moines pardessus la haie. Cette cruauté est si marquée, que je l'ai
voulu rapporter. Une retraite si hargneuse, et dont Dieu n'était pas
l'objet, ne put durer.

Il retourna à Paris faire un dernier effort pour avoir de quoi vivre en
récompense de son crime. Il n'en put venir à bout. De rage et de faim il
passa en Hollande, se fit protestant, et se mit à vivre de sa plume.
Elle le fit bientôt connaître. Sa qualité de prosélyte, quoique pour
l'ordinaire méprisée dans ces pays-là, et avec grande raison, se trouva
appuyée d'esprit, de savoir, de talent, d'un beau génie. Un homme chassé
de l'Oratoire, pour y avoir été espion des jésuites, fit espérer
d'apprendre bien des choses de lui. Tout cela ensemble lui procura des
connaissances, des amis, des protecteurs. Il fut connu de réputation en
Angleterre, il y espéra plus de fortune qu'en Hollande, il y passa
recommandé par ses amis. Burnet le reçut à bras ouverts. Son
\emph{Histoire de Louis XIII} délecta la haine contre la religion
catholique et contre le roi, et Burnet le fit connaître au roi
d'Angleterre, et l'obtint pour sous-précepteur sous lui du duc de
Glocester. Il était difficile de le faire instruire par deux autres
aussi grands ennemis des catholiques et de la France, et rien ne
convenait mieux aux sentiments du roi Guillaume pour l'éducation de son
successeur. Portland, entièrement dégoûté, s'était tout à fait retiré
auprès de la Haye, et le roi d'Angleterre essuyait tant de dégoûts du
parlement qu'on l'appelait publiquement roi de Hollande, et stathouder
d'Angleterre.

Quelque crédit qu'il eût à Vienne il n'y put jamais faire goûter le
traité de partage, et après bien des délais, l'empereur crut répondre
bien modérément de déclarer à la France, à l'Angleterre et à la
Hollande, et par là à toute l'Europe, qu'étant le plus proche parent du
roi d'Espagne, il ne pouvait durant sa vie entrer en aucun traité
touchant sa Succession, et donna ordre à une levée de trente mille
hommes dans ses pays héréditaires. Bientôt après, Blécourt déclara au
roi d'Espagne que s'il prenait dans aucun de ses États des troupes de
l'empereur, sous prétexte de recrues, d'achat, ou de quelque autre que
ce frit, le roi le regarderait et le prendrait comme une infraction à la
paix. Le conseil d'Espagne répondit au nom du roi d'Espagne qu'il avait
assez de troupes, et en assez bon état, pour n'en pas prendre
d'étrangères dont il n'avait aucun besoin, et qu'on pouvait s'assurer
qu'en aucun cas il n'en prendrait de l'empereur. La même déclaration
avait été faite sur la réception de l'archiduc, dans aucun des États du
roi d'Espagne où on avait soupçonné que l'empereur le voulait envoyer.
Sur l'assurance du conseil d'Espagne, Blécourt déclara à ce même conseil
que, pourvu que cela fût bien observé, le roi n'entreprendrait rien sur
les États du roi d'Espagne pendant sa vie\,; la même déclaration fut
faite à Vienne, et l'empereur s'engagea à n'envoyer point de troupes
dans les États d'Espagne moyennant la même assurance du roi. Castel dos
Rios avait souvent des audiences du roi, et une fort longue depuis peu
où il voulut être tête à tête avec le roi, sans Torcy, à qui même il ne
voulut pas dire ni devant ni après le sujet de cette audience, dont il
parut sortir fort content. Ce secret fut une chose tout à fait hors
d'usage, ainsi que ce tête-à-tête sans le ministre des affaires
étrangères. En même temps de cette levée de l'empereur et de cette
déclaration en Espagne, le roi signa un acte avec force menus princes de
l'empire, par lequel il s'engageait à ne point reconnaître un neuvième
électeur, en conséquence des traités de Westphalie.

Le roi de Danemark le signa aussi, mais je ne sais pourquoi, puisqu'il
s'était engagé à l'empereur de n'employer pas la voie de fait. Il venait
enfin de faire la paix avec la maison d'Holstein et le jeune roi de
Suède, qui avait passé en personne dans l'île de Seeland, forcé ses
retranchements, pris bien des lieux et menacé tellement Copenhague et
les restes de la flotte danoise battue par celle de Suède, que
l'empereur et le roi d'Angleterre s'entremirent fort à propos pour
arrêter tant de progrès. Celle du roi de Pologne durait toujours contre
l'électeur de Brandebourg, en laquelle les Polonais ne voulurent prendre
aucune part, et avec lesquels le roi eut de fâcheuses affaires à démêler
et avec les Suédois.

Le roi alla le 23 septembre à Fontainebleau\,; le roi et la reine
d'Angleterre y arrivèrent le 28, et y demeurèrent jusqu'au 12 octobre
avec toutes sortes d'attentions du roi et de respects de toute la cour
pour eux, comme toutes les autres années. M. de Beauvilliers, qu'une
très mauvaise santé avait fait aller à Bourbon, en revint à
Fontainebleau le 4 octobre, avec assez de succès.

On remarqua que le comte de Zinzendorf ayant suivi Monseigneur à la
chasse du loup le 1er octobre, Monseigneur qui, au retour de ces
chasses, nommait assez souvent plusieurs des plus distingués qui y
avaient été pour manger avec lui dans son appartement, y retint cet
envoyé de l'empereur\,; quatre jours après, le roi donna ses ordres pour
une grande augmentation de troupes.

Parmi tant de choses importantes qui préparaient les plus grands
événements, il en arriva un fort particulier, mais dont la singularité
mérite le court récit. Il y avait bien des années que la comtesse de
Verue vivait à Turin, maîtresse publique de M. de Savoie. Elle était
fille du duc de Luynes et de sa seconde femme qui était aussi sa tante,
sœur de père de sa mère la fameuse duchesse de Chevreuse. Le nombre
d'enfants de ce second lit du duc de Luynes, qui n'était pas riche,
l'avait engagé à se défaire de ses filles comme il avait pu. La plupart
étaient belles, celle-ci l'était fort, et fut mariée toute jeune en
Piémont en 1683, et n'avait pas quatorze ans lorsqu'elle y alla. Sa
belle-mère était dame d'honneur de M\textsuperscript{me} de Savoie\,;
elle était veuve et fort considérée. Le comte de Verue était tout jeune,
beau, bien fait, riche, de l'esprit, et fort honnête homme. Elle aussi
avait beaucoup d'esprit, et, dans la suite, un esprit suivi, appliqué,
tout tourné à gouverner. Ils s'aimèrent fort et passèrent quelques
années dans ce bonheur.

M. de Savoie, jeune aussi et qui voyait souvent la jeune Verue par la
charge de la douairière, la trouva à son gré\,; elle s'en aperçut et le
dit à son mari et à sa belle-mère, qui se contentèrent de la louer et
qui n'en firent aucun compte. M. de Savoie redoubla de soins, et donna
des fêtes, contre sa coutume et son goût. La jeune Verrue sentit que
c'était pour elle, et fit tout ce qu'elle put pour ne s'y pas trouver\,;
mais la vieille s'en fâcha, la querella, lui dit qu'elle voulait faire
l'importante, et que c'était une imagination que lui donnait son
amour-propre. Le mari, plus doux, voulut aussi qu'elle fût de ces fêtes,
et {[}dit{]} que sûr d'elle, quand bien même M. de Savoie en serait
amoureux, il ne convenait ni à son honneur ni à sa fortune qu'elle
marquât rien. M. de Savoie lui fit parler. Elle le dit à son mari et à
sa belle-mère, et fit toutes les instances possibles pour aller à la
campagne passer du temps. Jamais ils ne le voulurent et ils commencèrent
à la rudoyer si bien, que, ne sachant plus que devenir, elle fit la
malade, se fit ordonner les eaux de Bourbon, et manda au duc de Luynes,
à qui elle n'avait osé écrire sa dure situation, qu'elle le conjurait de
se trouver à Bourbon, où elle avait à l'entretenir des choses qui lui
importaient le plus sensiblement, parce qu'on ne lui permettait pas
d'aller jusqu'à Paris. M. de Luynes s'y rendit en même temps qu'elle,
conduite par l'abbé de Verue, frère du père de son mari, qu'on appelait
aussi l'abbé Scaglia, du nom de sa maison. Il avait de l'âge, il avait
passé par des emplois considérables et par des ambassades, et devint
enfin ministre d'État. M. de Luynes, grand homme de bien et d'honneur,
frémit au récit de sa Lille du double danger qu'elle courait par l'amour
de M. de Savoie, et par la folle conduite de la belle-mère et du mari\,;
il pensa à faire aller sa fille à Paris pour y passer quelque temps,
jusqu'à ce que M. de Savoie l'eût oubliée, ou se fût pris ailleurs. Rien
n'était plus sage ni plus convenable, et que le comte de Verrue vînt
chez lui voir la France et la cour à son âge, dans un {[}temps{]} de
paix en Savoie. Il crut qu'un vieillard important et rompu dans les
affaires, comme l'était l'abbé de Verue, entrerait dans cette vue et la
ferait réussir. Il lui en parla avec cette force, cette éloquence et
cette douceur qui lui était naturelle, que la sagesse et la piété dont
il était rempli devaient rendre encore plus persuasive, mais il n'avait
garde de se douter qu'il se confessait au renard et au loup, qui ne
voulait rien moins que dérober sa brebis. Le vieil abbé était devenu fou
d'amour pour sa nièce. Il n'avait donc garde de s'en laisser séparer. La
crainte du duc de Luynes l'avait retenu en allant à Bourbon. Il avait eu
peur qu'il ne sût son désordre. Il s'était contenté de se préparer les
voies par tous les soins et les complaisances possibles, mais le duc de
Luynes, éconduit et retourné à Paris, le vilain vieillard découvrit sa
passion, qui, n'ayant pu devenir heureuse, se tourna en rage. Il
maltraita sa nièce tant qu'il put, et au retour à Turin il n'oublia rien
auprès de la belle-mère et du mari pour la rendre malheureuse. Elle
souffrit encore quelque temps, mais la vertu cédant enfin à la démence
et aux mauvais traitements domestiques, elle écouta enfin M. de Savoie
et se livra à lui pour se délivrer de persécutions. Voilà un vrai roman.
Mais il s'est passé de notre temps au vu et au su enfin de tout le
monde.

L'éclat fait, voilà tous les Verue au désespoir, et qui n'avaient qu'à
s'en prendre à eux-mêmes. Bientôt la nouvelle maîtresse domina
impérieusement toute la cour de Savoie, dont le souverain était à ses
pieds avec des respects comme devant une déesse. Elle avait part aux
grâces, disposait des faveurs de son amant, et se faisait craindre et
compter par les ministres. Sa hauteur la fit haïr. Elle fut
empoisonnée\,; M. de Savoie lui donna d'un contre-poison exquis, qui
heureusement se trouva propre au poison qu'on lui avait donné. Elle
guérit, sa beauté n'en souffrit point, mais il lui en resta des
incommodités fâcheuses, qui pourtant n'altérèrent pas le fond de sa
santé. Son règne durait toujours. Elle eut enfin la petite vérole\,; M.
de Savoie la vit et la servit durant cette maladie comme aurait fait une
garde, et quoique son visage en eût souffert, il ne l'en aima pas moins
après. Mais il l'aimait à sa manière. Il la tenait fort enfermée, parce
qu'il aimait lui-même à l'être, et bien qu'il travaillât souvent chez
elle avec ses ministres, il la tenait fort de court sur ses affaires. Il
lui avait beaucoup donné, en sorte que, outre les pensions, les
pierreries belles et en grand nombre, les joyaux et les meubles, elle
était devenue riche. En cet état elle s'ennuya de la gêne où elle se
trouvait, et médita une retraite. Pour la faciliter, elle pressa le
chevalier de Luynes, son frère, qui servait dans la marine avec
distinction, de l'aller voir. Pendant son séjour à Turin, ils
concertèrent leur fuite, et l'exécutèrent après avoir mis à couvert et
en sûreté tout ce qu'elle put.

Ils prirent leur temps que M. de Savoie était allé, vers le 15 octobre,
faire un tour à Chambéry, et sortirent furtivement de ses États avant
qu'il en eût le moindre soupçon, et sans qu'elle lui eût même laissé une
lettre. Il le manda ainsi à Vernon, son ambassadeur ici, en homme
extrêmement piqué. Elle arriva sur notre frontière avec son frère, puis
à Paris, où elle se mit d'abord dans un couvent. La famille de son mari
ni la sienne n'en surent rien que par l'événement. Après avoir été reine
en Piémont pendant douze ou quinze ans, elle se trouva ici une fort
petite particulière. M. et M\textsuperscript{me} de Chevreuse ne la
voulurent point voir d'abord. Gagnés ensuite par tout ce qu'elle fit de
démarches auprès d'eux, et par les gens de bien qui leur firent un
scrupule de ne pas tendre la main à une personne qui se retire du
désordre et du scandale, ils consentirent à la voir. Peu à peu d'autres
la virent, et quand elle se fut un peu ancrée, elle prit une maison, y
fit bonne chère, et comme elle avait beaucoup d'esprit de famille et
d'usage du monde, elle s'en attira bientôt, et peu à peu elle reprit les
airs de supériorité auxquels elle était si accoutumée, et à force
d'esprit, de ménagements et de politesse, elle y accoutuma le monde. Son
opulence dans la suite lui fit une cour de ses plus proches et de leurs
amis, et de là elle saisit si bien les conjonctures qu'elle s'en fit une
presque générale, et influa beaucoup dans le gouvernement, mais ce temps
passe celui de mes Mémoires. Elle laissa `t Turin un fils fort bien
fait, et une fille, tous deux reconnus par M. de Savoie sur l'exemple du
roi. Le fils mourut sans alliance\,; M. de Savoie l'aimait fort et ne
pensait qu'à l'agrandir. La fille épousa le prince de Carignan, qui
devint amoureux d'elle. C'était le fils unique de ce fameux muet, frère
aîné du comte de Soissons, père du dernier comte de Soissons et du
fameux prince Eugène\,; ainsi M. de Carignan était l'héritier des États
de M. de Savoie s'il n'avait point eu d'enfants. M. de Savoie aimait
passionnément cette bâtarde, pour qui il en usa comme le roi avait fait
pour M\textsuperscript{me} la duchesse d'Orléans. Ils vinrent grossir
ici la cour de M\textsuperscript{me} de Verue après la mort du roi, et
piller la France sans aucun ménagement.

Le roi jugea à Fontainebleau un très ancien procès entre l'amirauté de
France et la province de Bretagne, qui prétendait avoir la sienne à part
indépendante en tout de celle de France, et elle en avait joui jusqu'à
présent. C'est ce qui avait mis par les prises, pendant les guerres, les
gouverneurs de Bretagne si à leur aise, et qui avait donné moyen à M. de
Chaulnes d'y vivre si grandement et d'y répandre tant de biens. Dès que
M. le comte de Toulouse eut ce gouvernement, le roi prit la résolution
de juger cette question. Les parties dès longtemps averties pour
instruire l'affaire, Valincourt, secrétaire général de la marine, agit
pour l'amirauté, et Bénard-Rézé, évêque de Vannes, Sévigné et un du
tiers état pour la Bretagne comme députés de la province. M. le comte de
Toulouse demeura neutre comme sans intérêt, parce qu'il avait l'un et
l'autre. Le roi donna un conseil extraordinaire, un jeudi matin, dans
lequel entrèrent Mgr le duc de Bourgogne qui avait voix depuis quelque
temps, les ministres, les secrétaires d'État, le contrôleur général, et
les deux conseillers au conseil royal des finances, qui étaient Pomereu
et d'Aguesseau\,: ce dernier était chargé du rapport. Monsieur y était
aussi. La province gagna en plein tout ce qu'elle prétendait, et fut
heureuse de ne se trouver point de partie puissante en tête, et qu'au
contraire le roi ne fût pas fâché de la favoriser pour y faire aimer et
accréditer M. le comte de Toulouse.

En même temps M. du Maine acheta, des héritiers de M. de Seignelay, la
belle et délicieuse maison de Sceaux, où M. Colbert et beaucoup plus M.
de Seignelay avaient mis des sommes immenses. Le prix fut de neuf cent
mille livres qui allèrent bien à un million avec les droits, et
si\footnote{\emph{Et si} est une vieille locution correspondant à
  \emph{et cependant}.} les héritiers en conservent beaucoup de meubles,
et pour plus de cent mille francs\footnote{Saint-Simon emploie souvent,
  comme on l'a déjà remarqué, les mots \emph{livres} et \emph{francs},
  sans y attacher un sens distinct.} de statues dans les jardins. Aux
dépenses prodigieuses de M\textsuperscript{me} du Maine, on peut
présumer que M. du Maine n'aurait pas été en état de faire une telle
acquisition sans les bontés ordinaires du roi pour lui.

M\textsuperscript{lle} de Condé mourut à Paris le 24 octobre d'une
longue maladie de poitrine, qui la consuma moins que les chagrins et les
tourments qu'elle essuya sans cesse de M. le Prince, dont les caprices
continuels étaient le fléau de tous ceux sur qui il les pouvait exercer,
et qui rendirent cette princesse inconsolable de ce que deux doigts de
taille avaient fait préférer sa cadette pour épouser M. du Maine, et
sortir de sous ce cruel joug. Tous les enfants de M. le Prince étaient
presque nains, excepté M\textsuperscript{me} la princesse de Conti,
l'aînée de ses filles, quoique petite. M. le Prince et
M\textsuperscript{me} la Princesse étaient petits, mais d'une petitesse
ordinaire\,; et M. le Prince, le héros, qui était grand, disait
plaisamment que si sa race allait toujours ainsi en diminuant elle
viendrait à rien. On en attribuait la cause à un nain que
M\textsuperscript{me} la Princesse avait eu longtemps chez elle\,; et il
était vrai que, outre toute la taille et l'encolure, M. le Duc et
M\textsuperscript{me} de Vendôme en avaient tout le visage. Celui de
M\textsuperscript{lle} de Condé était beau, et son âme encore plus
belle\,; beaucoup d'esprit, de sens, de raison, de douceur, et une piété
qui la soutenait dans sa plus que très triste vie. Aussi fut-elle
vraiment regrettée de tout ce qui la connaissait.

M. le Prince envoya Lussan, chevalier de l'ordre et premier gentilhomme
de sa chambre, à ma mère pour la prier de lui faire l'honneur, en
qualité de parente (ce furent ses termes), d'accompagner le corps de
M\textsuperscript{lle} de Condé, que M\textsuperscript{lle} d'Enghien,
qui a depuis été M\textsuperscript{me} de Vendôme, conduirait aux
Carmélites du faubourg Saint-Jacques, où elle avait choisi sa sépulture.
Ma mère qui n'allait guère, et qui, non plus que mon père jusqu'à sa
mort, ni moi non plus, n'avait aucune liaison avec l'hôtel de Condé, ne
put qu'accepter, et se rendit en mante dans son carrosse à six chevaux à
l'hôtel de Condé, chez M\textsuperscript{lle} d'Enghien. La duchesse de
Châtillon, jadis M\textsuperscript{lle} de Royan, dont j'ai parlé à
propos de mon mariage, était l'autre conviée. Comme on sortit, elle prit
le devant sur ma mère qui n'avait garde de s'y attendre. Elle crut que
c'était une faute d'attention de jeunesse, mais comme ce fut pour monter
en carrosse, la duchesse de Châtillon y entra encore la première, et se
voulut placer à côté de M\textsuperscript{lle} d'Enghien. Ma mère, sans
monter, témoigna sa surprise à M\textsuperscript{lle} d'Enghien, et la
supplia de lui faire rendre sa place ou de trouver bon qu'elle s'en
retournât. M\textsuperscript{me} de Châtillon répondit qu'elle savait
bien qu'elle était de beaucoup son ancienne et qu'elle la devait
précéder, mais qu'en cette occasion la parenté devait décider, et
qu'elle était plus proche. Ma mère, toujours froidement mais avec un air
de hauteur, lui répondit qu'elle pardonnait cet égarement à sa jeunesse
et à son ignorance, qu'il était là question de rang et non de proximité,
qu'en tout cas elle se trouverait embarrassée d'en prouver plus que
celle de mon père. La vérité était qu'elles étaient fort éloignées
toutes les deux, si même il y en avait de M\textsuperscript{me} de
Châtillon, dont le mari ne venait point du connétable de Montmorency, et
qui était bien éloignée de la grand'mère de M. le Prince, le héros.

Desgranges qui gagnait le carrosse où il allait entrer, averti de cette
dispute, accourut et la termina en disant qu'il n'y avait point de
difficulté pour l'ancienne duchesse, tellement que
M\textsuperscript{lle} d'Enghien pria M\textsuperscript{me} de Châtillon
de passer sur le devant, et ma mère monta et se mit au derrière. Comme
les carrosses se mirent en marche, Desgranges, avec soupçon par ce qui
venait d'arriver, mit la tête à la portière et vit le carrosse de
M\textsuperscript{me} de Châtillon qui coupait celui de ma mère. Il cria
pour arrêter et descendit pour aller lui-même mettre les carrosses en
ordre, et fit précéder celui de ma mère. Depuis cela la duchesse de
Châtillon, ni son cocher, n'osèrent plus rien entreprendre, mais elle
grommelait tout bas à côté de M\textsuperscript{me} de Lussan.

Je ne puis comprendre où elle avait pris cette fantaisie, dont après
elle fut honteuse, et fit faire des excuses à ma mère sur cette
imagination de proximité, que nous sûmes après que M. de Luxembourg
lui-même avait trouvée fort ridicule, quoique nous ne nous vissions
point encore en ce temps-là, ni de bien des années depuis.

Le lendemain de la cérémonie, M. de Lussan vint remercier ma mère, de la
part de M. le Prince, de l'honneur qu'elle lui avait fait, s'informer si
elle n'en était point incommodée, et lui témoigner son déplaisir de
l'incident si peu convenable qui était arrivé, excusant
M\textsuperscript{lle} d'Enghien sur sa jeunesse, de la part de M. le
Prince, et sur son affliction de n'y avoir pas mis ordre à l'instant. Il
ajouta les excuses de M. le Prince de n'être pas venu lui-même chez
elle, sur ce qu'il avait été obligé d'aller à Fontainebleau pour les
visites, et qu'il ne manquerait pas de s'acquitter de ce devoir-là à son
retour. Si je m'étends sur tous ces compliments, et si je les ai si
correctement retenus, ce n'est pas fatuité, la vanité y serait déplacée.
Mais les façons des princes du sang ont tellement changé depuis, que je
n'ai pas voulu omettre ce contraste d'un premier prince du sang, qui
était plus éloigné qu'aucun de ses devanciers de donner à personne plus
qu'il ne devait, et qui plus que pas un d'eux en est demeuré en reste.
Pour achever donc ceci, la déclaration du roi d'Espagne fit aller ma
mère à Versailles au retour de Fontainebleau, où elle n'allait pas
souvent. Elle rencontra M. le Prince, qui dès qu'il l'aperçut traversa
tout ce grand salon qui est devant cette petite pièce qui mène à la
grande salle des gardes, vint à elle, lui dit qu'il mourait de honte de
la rencontrer sans avoir encore été chez elle lui témoigner sa
reconnaissance de l'honneur qu'elle lui avait fait, et de là toutes
sortes de compliments. Huit ou dix jours après, il la vint voir à Paris,
la trouva, et recommença les compliments. Il y demeura une demi-heure,
et ne voulut jamais que ma mère passât au delà de quelques pas hors de
la porte du lieu où elle l'avait reçu. Il ne faut pas oublier que ce fut
un gentilhomme ordinaire du roi qui alla de sa part faire les
compliments à l'hôtel de Condé, et que, trois mois auparavant, Souvré,
maître de la garde-robe, y avait été les faire, sur la mort d'un enfant
au maillot de M\textsuperscript{me} du Maine.

D'Antin, pour un homme d'autant d'esprit et aussi versé à la cour, fit
en ce temps-ci une bien ridicule démarche. M\textsuperscript{me} de
Montespan, comme on l'a vu plus haut, entre autres pratiques de
pénitence, travaillait à lui former des biens, mais elle ne voulait pas
travailler en l'air. Il était de toute sa vie dans le plus gros jeu, et
faisait toutes sortes d'autres dépenses\,; elle voulait donc qu'il se
réglât et qu'il quittât le jeu, parce que cela n'est pas possible à un
homme qui joue. Elle lui promit une augmentation de douze mille livres
par an à cette condition, mais elle voulut le lier, et lui pour la
satisfaire ne trouva point de lien plus fort que de prier M. le comte de
Toulouse de dire au roi de sa part qu'il ne jouerait de sa vie. La
réponse du roi fut sèche. Il demanda au comte de Toulouse qu'est-ce que
cela lui faisait que d'Antin jouât ou non. On le sut, et le courtisan,
qui n'est pas bon, en fit beaucoup de risées. Ce fut le serment d'un
joueur\,; il ne put renoncer pour longtemps aux jeux de commerce, puis
il les grossit, enfin il se remit aux jeux de hasard, et à peine quinze
ou dix-huit mois furent-ils passés, qu'il joua de plus belle, et a
depuis continué. Lorsqu'il fit faire cette belle protestation au roi, il
avoua qu'il avait gagné six ou sept cent mille livres au jeu, et tout le
monde demeura persuadé qu'il avait bien gagné davantage.

J'éprouvai à Fontainebleau une des plus grandes afflictions que je pusse
recevoir, par la perte que je fis de M. de la Trappe. Attendant un soir
le coucher du roi, M. de Troyes me montra une lettre qui lui en
annonçait l'extrémité. J'en fus d'autant plus surpris que je n'en avais
point reçu de là depuis dix ou douze jours, et qu'alors sa santé était à
l'ordinaire. Mon premier mouvement fut d'y courir, mais les réflexions
qu'on me fit faire sur cette disparate m'arrêtèrent. J'envoyai
sur-le-champ à Paris prendre un médecin fort bon, nommé Audri, que
j'avais mené à Plombières, qui partit aussitôt, mais qui en arrivant ne
trouva plus M. de la Trappe en vie. Ces Mémoires sont trop profanes pour
rapporter rien ici d'une vie aussi sublimement sainte, et d'une mort
aussi grande et aussi précieuse devant Dieu. Ce que je pourrais dire
trouvera mieux sa place parmi les Pièces, page 5\footnote{Voy., sur les
  pièces auxquelles renvoie Saint-Simon, t. Ier, note.}. Je me
contenterai de rapporter ici que les louanges furent d'autant plus
grandes et plus prolongées, que le roi fit son éloge en public\,; qu'il
voulut voir des relations de sa mort\,; et qu'il en parla plus d'une
fois aux princes ses petits-fils, en forme d'instruction. De toutes les
parties de l'Europe, on parut sensible à l'envi à une si grande perte\,;
l'Église le pleura et le monde même lui rendit justice. Ce jour si
heureux pour lui et si triste pour ses amis fut le 26 octobre, vers midi
et demi, entre les bras de son évêque, et en présence de sa communauté,
à près de soixante-dix-sept ans, et de quarante ans de la plus
prodigieuse pénitence. Je ne puis omettre néanmoins la plus touchante et
la plus honorable marque de son amitié. Étant couché par terre sur la
paille et sur la cendre, pour y mourir comme tous les religieux de la
Trappe, il daigna se souvenir de moi de lui-même, et chargea l'abbé de
la Trappe de me mander de sa part, que, comme il était bien sûr de mon
affection pour lui, il comptait bien que je ne doutais pas de toute sa
tendresse. Je m'arrête tout court\,; tout ce que je pourrais ajouter
serait ici trop déplacé.

Le pape était mort le 27 septembre, après avoir longtemps menacé d'une
fin prochaine. C'était un grand et saint pape, vrai pasteur et vrai père
commun, tel qu'il ne s'en voit plus que bien rarement sur la chaire de
Saint-Pierre, et qui emporta les regrets universels, comblé de
bénédictions et de mérite. Il s'appelait Antoine Pignatelli, d'une
ancienne maison de Naples dont il était archevêque, lorsqu'il fut élu le
12 juillet 1691, près de six mois après la mort d'Alexandre VIII,
Ottoboni, auquel il ressembla si peu. Il était né en 1615, et avait été
inquisiteur à Malte, nonce à Florence, en Pologne et à Vienne, enfin
maître de chambre de Clément X, Altieri, et d'Innocent XI, Odescalchi,
qui le fit cardinal, en septembre 1681, en l'honneur duquel il prit le
nom d'Innocent XII.

On verra bientôt pourquoi je me suis étendu sur ce pape, dont la mémoire
doit être précieuse à tout Français, et singulièrement chère à la maison
régnante. Le cardinal de Noailles eut ordre de partir\,; le même ordre
fut envoyé au cardinal Le Camus, et il eut pour son voyage la même somme
que ses confrères. Le cardinal de Bouillon entra au conclave avec les
autres\,; il avait quitté l'ordre, et comme il était là en lieu où les
cardinaux d'Estrées, Janson et Coislin ne pouvaient éviter de se trouver
avec lui aux scrutins et aux autres fonctions publiques de l'intérieur
du conclave, il en prit le temps pour essayer de leur persuader de
quitter l'ordre aussi\,; et prétendit qu'ils étaient tous engagés par
une bulle de ne porter l'ordre d'aucun prince. C'était s'en aviser bien
tard, après trente années qu'il l'avait porté comme grand aumônier,
après le neveu d'un pape, et qu'il l'avait porté et vu porter à tant de
cardinaux dans Rome, et à toutes les fonctions. Aussi ne fut-il pas
écouté, et ce vernis qu'il jetait au dehors retomba sur lui à sa
confusion.

\hypertarget{note-i.-hommage-lige-et-hommage-simple.}{%
\chapter{NOTE I. HOMMAGE LIGE ET HOMMAGE
SIMPLE.}\label{note-i.-hommage-lige-et-hommage-simple.}}

L'hommage était la cérémonie dans laquelle un vassal se reconnaissait
l'\emph{homme} de son seigneur, et prêtait serment de fidélité entre ses
mains. On distinguait deux espèces d'hommage\,: l'hommage simple ou
franc, et l'hommage lige, par lequel le vassal se liait plus étroitement
à son seigneur. Pour l'hommage simple, le vassal se tenait debout,
gardait son épée et ses éperons, pendant que le chancelier lisait la
formule du serment. Le vassal se bornait à répondre, quand la lecture
était terminée\,: \emph{Voire} (\emph{verum}, c'est vrai). Celui qui
devait l'hommage lige ne gardait ni éperons, ni baudrier, ni épée. Il
fléchissait le genou devant son seigneur, mettait les mains dans les
siennes, et prononçait la formule suivante, \emph{qui} nous a été
conservée par Bouteiller, dans sa \emph{Somme rurale\,:} «\,Sire, je
viens à votre hommage et en votre foi, et deviens votre homme de bouche
et de mains. Je vous jure et promets foi et loyauté envers tous et
contre tous, et garder votre droit en mon pouvoir.\,»

L'hommage rendu par un noble était souvent terminé par un baiser. De là
l'expression \emph{devenir l'homme de bouche et de mains}, que l'on
trouve dans la formule précédente. Si le vassal, lorsqu'il se présentait
pour rendre hommage, ne trouvait pas son seigneur à son logis, il devait
accomplir certaines formalités, qui variaient avec les coutumes. D'après
les lois de quelques contrées, il devait heurter trois fois à la porte
du manoir seigneurial et appeler trois fois. Si l'on n'ouvrait pas, il
heurtait l'\emph{huis} (la porte) ou le verrou de la porte, et récitait
la formule de l'hommage comme si le seigneur eût été présent. La
formalité de l'hommage était alors regardée comme légalement accomplie.

\hypertarget{note-ii.-labbuxe9-dalbret-et-labbuxe9-le-tellier.}{%
\chapter{NOTE II. L'ABBÉ D'ALBRET ET L'ABBÉ LE
TELLIER.}\label{note-ii.-labbuxe9-dalbret-et-labbuxe9-le-tellier.}}

Saint-Simon dit que lorsque l'abbé d'Albret, plus tard cardinal de
Bouillon, soutint ses thèses en Sorbonne, l'abbé Le Tellier était déjà
coadjuteur de l'archevêque de Reims. Il y a dans ce récit une erreur
chronologique. L'abbé d'Albret soutint ses thèses le 29 février 1664, et
ce fut plus de quatre ans après, le 30 mai 1668, que l'abbé Le Tellier
devint coadjuteur de l'évêque-duc de Langres, et ensuite de
l'archevêque-duc de Reims. Comme ces faits ont une certaine importance
dans le récit de Saint-Simon, et que d'ailleurs on ne connaît qu'assez
imparfaitement ces détails, je citerai deux passages du \emph{Journal}
inédit d'Olivier d'Ormesson, qui fixent avec la dernière précision
l'ordre chronologique. L'auteur raconte d'abord les incidents de la
thèse de l'abbé d'Albret.

«\,Le vendredi 29 février 1664, l'après-dînée, je fus en Sorbonne à
l'acte de M. le duc d'Albret\footnote{Emmanuel-Théodose de La Tour, né
  en 1644, mort en 1715. Il fut nommé cardinal en 1669, et porta depuis
  cette époque le nom de cardinal de Bouillon.}, neveu de M. de Turenne.
M. l'archevêque de Paris présidait\footnote{L'archevêque de Paris était
  alors Hardouin de Péréfixe, dont Saint-Simon parle à l'occasion de
  cette soutenance.}. Le répondant se couvrait quelquefois comme étant
prince, et la chose avait été ainsi résolue en Sorbonne, dont les jeunes
bacheliers de condition étaient fort offensés, et avaient fait ligue
entre eux de ne point disputer. J'ai su depuis que l'abbé de Marillac
seul, des bacheliers de condition, avait disputé, M. le premier
président l'ayant voulu absolument pour obliger M. de Turenne\,; que les
autres lui avaient fait reproche\,; que l'abbé Le Tellier s'était le
plus signalé, ayant dit beaucoup de choses fort désobligeantes.\,»

~

{\textsc{On voit que l'abbé Le Tellier n'était encore promu à cette
époque à aucune dignité ecclésiastique. Ce fut seulement le 30 mai
1668}}

~

\footnote{«\,Le mercredi 30 mai. M. l'abbé Le Tellier est coadjuteur de
  M. l'évêque et duc de Langres. » --- \emph{Journal d'Olivier
  d'Ormesson}.} , comme l'atteste le même Journal, qu'il devint
coadjuteur de l'évêque-duc de Langres, et quelques jours plus tard de
l'archevêque-duc de Reims.

«\,Le jeudi 14 juin 1668, dit Olivier d'Ormesson, je fus faire mes
compliments à M. l'abbé Le Tellier sur la coadjutorerie de l'archevêché
de Reims. Il en témoignait une joie très grande, comme d'un
établissement très élevé et beaucoup au delà de ses espérances. Il y
avait longtemps que l'on ménageait cette coadjutorerie avec le cardinal
Antoine\footnote{Antonio Barberini, archevêque-duc de Reims.}, et l'on
croit que celle de Langres a fait réussir la seconde, parce que M. Le
Tellier ayant obtenu l'agrément, de M. le cardinal Antoine, il le dit au
roi, et marqua que la coadjutorerie de Reims était un même titre de
duché que Langres\,; une plus grande dignité, étant archevêché\,; et
néanmoins qu'il ne désirait l'une plus que l'autre que parce que celle
de Reims n'était qu'à deux journées de Paris, et celle de Langres
beaucoup plus éloignée\,; et ainsi, sans faire une grande différence de
ces deux grâces, le roi lui accorda sur-le-champ celle de Reims. Tout le
monde considère cette grâce comme trop considérable pour M. l'abbé Le
Tellier, à son âge, etc.\,; et que c'était un effet et de la bonne
fortune de M. Le Tellier, et de la puissance que les trois ministres ont
sur le roi\footnote{Les trois ministres principaux étaient alors Le
  Tellier, Colbert et de Lyonne. Louvois n'avait pas encore le titre de
  secrétaire d'État.}\,; car ils font chacun tout ce qu'ils veulent pour
leur intérêt.\,»

\hypertarget{note-iii.-note-de-mm.-de-dreux-nancruxe9-et-de-dreux-bruxe9zuxe9-uxe9tablissant-que-m.-de-dreux-uxe9tait-de-grande-et-ancienne-maison.}{%
\chapter{NOTE III. NOTE DE MM. DE DREUX-NANCRÉ ET DE DREUX-BRÉZÉ,
ÉTABLISSANT QUE M. DE DREUX ÉTAIT DE GRANDE ET ANCIENNE
MAISON.}\label{note-iii.-note-de-mm.-de-dreux-nancruxe9-et-de-dreux-bruxe9zuxe9-uxe9tablissant-que-m.-de-dreux-uxe9tait-de-grande-et-ancienne-maison.}}

\begin{enumerate}
\def\labelenumi{\Roman{enumi}.}
\setcounter{enumi}{1999}
\tightlist
\item
  de Dreux-Nancré et de Dreux-Brézé ont adressé la réclamation
  suivante\,:
\end{enumerate}

M. le duc de Saint-Simon mentionne, année 1699, que «\,M. Dreux devint
le marquis de Dreux aussitôt après son mariage avec
M\textsuperscript{lle} de Chamillart, et lorsqu'il fut pourvu de la
charge de grand maître des cérémonies, prenant (sans prétexte de terre)
le titre de marquis, etc.\,»

Cependant la terre de Brézé avait été érigée en marquisat dès l'année
1685, pour Thomas de Dreux, père de celui qui, le premier de sa famille,
fut grand maître des cérémonies\,; cette érection fut enregistrée à la
chambre des comptes le 23 juillet suivant, et au parlement le 5 août
1686.

Des titres originaux, \emph{dont l'extrait se trouve déposé aux archives
du royaume}, remontent par tous les degrés de filiation jusqu'à Pierre
de Dreux, écuyer, seigneur de Ligueil, vivant en 1406.

Une donation faite le 7 juillet 1472, par Thomas de Dreux, écuyer
seigneur de Ligueil, à Simon de Dreux, écuyer, son fils aîné et
principal héritier, établit que le susdit Thomas de Dreux donne, etc.,
ainsi qu'il a reçu \emph{japieça} (depuis longtemps) de feu Pierre de
Dreux, son père, vivant, écuyer, seigneur de Ligueil, et de son oncle
messire Simon de Dreux, chevalier, maître de \emph{l'houstel} du roi.
Cet acte fut fait en présence dudit Thomas de Dreux et de son oncle,
messire Jean de Garquesalle, grand maître de l'écurie du roi Louis XI.

Les titres de M. de Dreux avaient été vérifiés au parlement de Bretagne
le 13 juin 1669, ainsi que par l'assemblée des commissaires généraux,
tenue à Paris les 28 janvier 1700 et 22 mars 1703.

\hypertarget{note-iv.-note-de-m.-le-marquis-de-saumery-relative-uxe0-johanne-de-la-carre-de-saumery-son-ancuxeatre.}{%
\chapter{NOTE IV. NOTE DE M. LE MARQUIS DE SAUMERY, RELATIVE À JOHANNE
DE LA CARRE DE SAUMERY, SON
ANCÊTRE.}\label{note-iv.-note-de-m.-le-marquis-de-saumery-relative-uxe0-johanne-de-la-carre-de-saumery-son-ancuxeatre.}}

Arrière-petit-fils de M. de Saumery, attaqué par M. le duc de
Saint-Simon, j'ai cru que mon honneur m'imposait la loi de rendre
publique ma justification quand l'injure l'était.

J'ignore quels étaient les motifs de M. de Saint-Simon, mais la manière
dont est tracé le portrait de Saumery fait assez voir combien il y a mis
de partialité. Des traditions de famille, l'estime générale dont
jouissait le marquis de Saumery, la confiance dont l'honora Mgr le duc
de Bourgogne, son intimité avec les ducs de Chevreuse et de
Beauvilliers, l'un son neveu, l'autre son cousin germain\,; enfin,
l'amitié du vertueux Fénelon, suffisent pour faire apprécier à leur
juste valeur toutes les assertions dont il est l'objet.

Les couleurs employées pour peindre M\textsuperscript{me} de
Saumery\footnote{Marguerite-Charlotte de Montlezun de Besmeaux, était
  fille du marquis de Montlezun de Besmeaux, des seigneurs de Projan,
  comtes de Champagne, issus des seigneurs de Saint-Lary, puînés des
  comtes de Pardillac, sortis des comtes d'Astarac, cadets des ducs
  héréditaires de Gascogne\,; il descendait de mâle en mâle au
  vingt-deuxième degré de génération de Garcie Sanche, surnommé le
  Corbé, troisième duc héréditaire de Gascogne, qui unit à son duché le
  comté de Bordeaux en 904.} ne sont pas moins noires, et cependant
cette femme si dissolue, selon M. de Saint-Simon, fut la meilleure des
mères. Elle fut toujours l'amie de sa nièce, la vertueuse duchesse de
Beauvilliers\footnote{Henriette-Louise Colbert, duchesse de
  Beauvilliers.}, et l'affection que lui portait sa cousine de
Navailles\footnote{Suzanne de Baudéan Parabère, duchesse de Navailles.},
de toutes les femmes de la cour la plus universellement révérée, ne se
démentit jamais. Enfin la pieuse reine Marie Leczinska la traita
toujours avec une bienveillance particulière.

Quant à la naissance, possédant des preuves matérielles, il m'est facile
de rétablir les faits. Ces preuves ne sont point fondées sur des
généalogies, faites souvent avec peu de scrupule\,; elles sont établies
sur des actes et sur les registres des églises de Chambord et de
Huisseau-sur-Cosson, paroisse dans laquelle est situé le château de
Saumery\,; tout le monde peut donc, sur les originaux mêmes, s'assurer
de l'inexactitude des assertions de M. de Saint-Simon. Je parlerai peu
de ma famille avant son établissement en France, mon intention n'étant
point de faire ici une généalogie, mais seulement de prouver que le
premier des Johanne qui se fixa dans le Blaisois n'a jamais été et n'a
jamais pu être valet d'Henri IV. S'il fût effectivement né dans l'état
de domesticité, je ne serais pas assez vil pour chercher à prouver le
contraire. L'homme qui, de cette condition, se serait élevé aux emplois
de premier président de la chambre des comptes de Blois et conseiller
d'État, eût eu plus de mérite sans doute que celui qui ne les obtint
peut-être que par sa naissance.

La maison de Johanne de La Carre de Saumery est originaire de Béarn\,;
elle possédait de toute ancienneté dans la ville de Mauléon un hôtel
noble, appelé de Johanne et de Mauléon, duquel dépendaient des fiefs,
terres, etc. Les armoiries propres de Johanne sont de gueules au lion
d'or. (Voir, pour vérifier les armes, \emph{l'Histoire des grands
officiers de la couronne}, tome IX, page 92, promotion des chevaliers de
l'ordre du Saint-Esprit du 31 décembre 1585, au nom \emph{Giraud de
Mauléon}, et à celui de \emph{Johanne} dans le \emph{Nobiliaire de
France}.)

Vers l'an 1566, Arnault de Johanne, seigneur de Johanne de Mauléon,
ayant épousé Gartianne de La Carre, sœur de Ménault de La Carre,
aumônier du roi, et nièce de Bernard de Ruthie, abbé de Pontlevoy, nommé
grand aumônier de France le 1er juillet 1552, les armes de Johanne
furent écartelées de celles de La Carre, qui sont partie au premier
d'azur à trois faces d'or, au deuxième de sable à trois coquilles
d'argent posées en pal.

Le seul des seigneurs de Johanne qui quitta le Béarn, pour se fixer en
France, fut Arnault II\footnote{Il est qualifié noble écuyer seigneur
  dans vingt-deux actes de l'église de Huisseau, depuis le 3 avril 1580
  jusqu'au 26 avril 1619, et, selon les mêmes registres, il était déjà
  en possession de la seigneurie de Saumery avant le 8 janvier 1590.},
bisaïeul de M. de Saumery. Il fut appelé en 1579, très jeune encore, par
son oncle l'abbé de La Carre, résidant à se terre des Veaux, paroisse de
Cour-Cheverny, non loin du château de Saumery, qu'il acquit d'Antoine de
Laudières, gentilhomme de la maison du roi\,; ledit abbé de La Carre,
ayant acquis la seigneurie de Saumery le 13 avril 1583, laissa cette
terre à son neveu Arnault, lequel devint dès lors seigneur de Saumery,
et ajouta à son nom celui de La Carre\footnote{Le nom et les armes de La
  Carre furent ajoutés à ceux de Johanne, en conséquence d'un acte
  expédié par le lieutenant général du bailliage et gouvernement de La
  Carre en Soule, le 23 juillet 1613, par lequel Jean d'Arbède, chef de
  la maison de La Carre, autorisa son neveu, Arnault, à ajouter ses
  armes et son nom aux armes et nom de Johanne (inventorié en original
  au château de Saumery, par Bourreau, notaire de Blois, le \emph{5}
  juin 1709 et jours suivants, coté 263).}. Il fut pourvu de la charge
de premier président de la chambre des comptes de Blois à l'âge de
vingt-six ans, sur la démission de Merry de Vic, qui fut garde des
sceaux de France\footnote{Voy., preuves de l'antiquité de la chambre des
  comptes de Blois.}. Nommé conseiller d'État le 27 avril 1616, il prêta
le serment entre les mains du chancelier de Sillery, le 29 du même mois.

Arnault il épousa en 1592 Cyprienne de Rousseau de Villerussien, fille
de Claude de Rousseau de Villerussien, écuyer du roi\footnote{Voy.
  Bernier, \emph{Histoire de Blois}, page 631.}\,; il testa par acte du
25 mai 1631\footnote{Voy., la sentence arbitrale entre MM. de Colbert de
  Saumery et de Menars du 3 septembre 1671, déposée le même jour chez de
  Beauvais, notaire au Châtelet de Paris.}.

Son fils François de Johanne, chevalier, seigneur de Saumery, etc.,
capitaine des chasses du comte de Blois, conseiller d'État, gouverneur
du château royal de Chambord, premier gentilhomme de la chambre de S. A.
R. Gaston de France, duc d'Orléans, frère de Louis XIII, naquit au
château de Saumery le 23 novembre 1593, et fut baptisé dans la chapelle
le 24 du même mois\footnote{Voy., registres de l'église paroissiale de
  Huisseau.}. Il épousa, en 1618, Charlotte de Martin de Villiers, fille
de Daniel de Martin, écuyer, seigneur de Villiers\footnote{Elle était
  arrière-petite-fille de Christophe de Martin, écuyer, seigneur de
  Villeneuve, et d'Anne Compaing de Fresnay. De cette maison était Marie
  Compaing, fille de Nicolas Compaing, seigneur de Fresnay, chancelier
  de Navarre, qui épousa, le 29 avril 1593. Leclères II, baron de
  Juigné, d'où sont descendus MM de Juigné, actuellement vivants.}. De
ce mariage est né Jacques de Johanne de La Carre, chevalier, marquis,
seigneur de Saumery, écuyer de S. A. R. Gaston de France, duc d'Orléans,
mestre de camp du régiment d'Orléans, gouverneur de Chambord, maréchal
des camps et armées du roi, grand maître des eaux et forêts, gouverneur
de Blois (première provision en date du 15 février 1650) et conseiller
d'État. Il fut baptisé dans la chapelle du château de Saumery le 23
octobre 1623\footnote{Voy., registres de Huisseau.}. Il épousa, le 6
février 1650, Catherine de Charron de Nozieux, fille de Jacques Charron
de Nozieux et sœur de M\textsuperscript{me} de Colbert.

C'est de ce mariage qu'est issu Jacques-François de Johanne de La Carre,
marquis de Saumery, sur la personne et sur l'extraction duquel le duc de
Saint-Simon a répandu tant d'erreurs. La mère de M. de Saumery n'était
point \emph{une petite bourgeoise de Blois\,;} la maison de Charron
était noble. Elle est alliée passivement à celles de Colbert, de La
Poupelinière, du Gué de Bagnols, de Longueu, de Castellane, de Novejan,
de Lastic\,; et de ces alliances sont descendus, de la maison de
Charron, les Montmorency-Fosseux, les Talleyrand-Périgord, les
Chabannes-Curton, les Duchillau, les d'Albert de Luynes, les Mortemart
et la dernière duchesse de Gesvres, née du Guesclin.

Si, avant d'assurer un fait, M. de Saint-Simon avait bien voulu faire
des recherches, il se serait facilement convaincu que non seulement le
brevet de gouverneur de Chambord, mais encore ceux des autres maisons
royales, portaient anciennement pour souscriptions\,: jardinier,
concierge, capitaine, etc.

Vivant sans ambition, ce ne sont point des titres que je viens
revendiquer, c'est la vérité seule que je veux et devais rétablir. Cette
tâche, pénible pour un homme qui n'est et ne veut pas être en évidence,
une fois remplie, je dois rendre hommage à la loyauté de M. le marquis
de Saint-Simon, pair de France, etc., auquel j'ai adressé ma
réclamation, et qui m'autorise à faire insérer cette note.

\end{document}
